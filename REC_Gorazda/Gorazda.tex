\begin{recengenv}{Marcin Gorazda}
	{Believable world of economic models}
	{Believable world of economic models}
	{Łukasz Hardt, \textit{Economics Without Laws: Towards a New Philosophy of Economics}, Palgrave Macmillan, Cham, 2017,
		pp.220.}
	
	



\subsubsection{Introduction}
Philosophy of economics is a young discipline, especially in Poland, which is still looking for its proper place within
the scientific world and its definition. What has been contemplated so far under this brand is a mixture of the
methodology of economics, the ontology of economic phenomena and ethical commitments in economics. Hardt’s book
subscribes itself to this club and constitutes a fascinating example of such an analysis. The book is outstanding for
many reasons (which are developed further below), but one must be emphasized at the very beginning. As far as the
reviewer is aware it is the first book by Polish economist and philosopher in the field of philosophy of economist
published in English by the distinguished scientific publisher. Therefore, regardless of some minor reservations below,
Łukasz Hardt definitely did a great a job as a representative of Polish thinkers. 

Hardt’s book focuses mainly on ontological and methodological issues in economics, especially the part economic laws and
economic models play in the explanation of economic phenomena. As economists usually present their ideas in the form of
various models, their methodological and ontological status is definitely of utmost importance. However, it needs to be
underlined that the analysis is done from the specific point of view, namely scientific realism which is in opposition
to scientific constructivism or instrumentalism and which still seems to dominate the contemporary philosophy of
science.  Even if the author in some places declares that he intends to find the third way somewhere in between, his
initial stance, used terminology and style of reasoning constitutes a clear hallmark and sometimes a burden, not easy
to be neglected. 

There is a general idea in the book’s background, which I believe is shared commonly by the philosophers of economics,
that there seems to be a crucial distinction between the exactness of the so-called natural sciences and inexactness of
the social sciences, economics including. On the phenomenal level, the distinction may be observed by the apparent
inability of economics to formulate reliable predictions. For some thinkers, if the specific science is unable to
predict future outcomes, it undermines its scientific character. And as those predictions are usually formulated based
on the past regularities which are then reformulated into the universal laws of nature, the natural question is about
the status of those laws. Is it possible to practice any science without the concept of scientific laws? What would be
the construction of scientific theory if there be no laws and how such a theory may explain? The book is a courageous
attempt at answering those hard philosophical questions. 

Briefly, Hardt claims that it is not about laws in economics but about tendencies, models based on \textit{ceteris
normalibus} assumptions, mechanisms, believable worlds which are never entirely true but for some reasons might be
considered believable and therefore may have certain explanatory power and be informative for policymakers.

\enlargethispage{-1\baselineskip}

Hardt’s account is a strong case against law-centrism in science and especially in economics as well as against
dogmatism and fundamentalism. According to him, the economists’ efforts concentrate and should concentrate on models’
construction and their further exploration and those efforts may eventually bring them about the reliable insight into
the essentials of the modelled economic realm. 

\subsubsection{The composition and the content of the book}
Reading the book, one has to acknowledge that its composition is compliant to its content. The chapters are logically
sequenced what makes the line of reasoning transparent and comprehensible. The introduction gives us an overview of the
book’s content, and brief information about the main theses defended. The first chapter is an in-depth review of the
thoughts of four well known classical economists on the nature of economic law. From this perspective, Hardt
scrutinises the works of Adam Smith, David Ricardo, John Stuart Mill and Alfred Marshall, and this scrutiny is itself
an achievement worth reading. There were thousands of pages written on those worldly philosophers, but I do not recall
myself any attempt at reconstructing their methodological and ontological views on the economic laws (except Mill whose
concept of tendencies is broadly discussed). However, the conclusions of that chapter are in the end, a bit untoward.
As we learn from the introduction and the book’s title, Hardt is trying to persuade the reader that this law-centrism
in economics was declining throughout the centuries. So I would expect that the chosen thinkers were drifting from the
concept of universal economic laws toward the theoretical models. It is not precisely the case. Even Smith, who as
Hardt admits, used the idea of universal laws at least on the sematic level is presented like the contemporary
theorist, whose references to the universal laws are due to his fresh, underdeveloped awareness of the scientific
methodology.  We read:


\begin{myquoterev}
	[\mydots] Smith’s claims suggesting the existence of universal economic laws are a testimony of his desire to build a
deductive economic theory and his more cautious assertions about the working of the real markets expresses his
empirical orientation
\parencite[p.30]{hardt_economics_2017}.
% \label{ref:RNDwvxc7jkftR}(Hardt, 2017, p.30).
\end{myquoterev}
And further

\begin{myquoterev}
[\mydots] it is certain that his economic laws are not universal natural laws and their nature is more complicated
\parencite[p.31]{hardt_economics_2017}.
%\label{ref:RNDmAffQKsxYM}(Hardt, 2017, p.31).
\end{myquoterev}

\enlargethispage{-.5\baselineskip}

In Hardt’s interpretation, Ricardo is an abstract model’s constructor, Mill, an author of the concept of laws as
tendencies and Marshal is a denialist of the ``possibility of all-encompassing knowledge about the economic phenomena''.
He concludes that ``classical economists, together with Marshall, kept their conclusions regarding economic reality
separated from their purely theoretical claims''. In the light of the next chapter, ``The Demise of Laws in Economics'' it
makes the whole story more complicated. The reader may be puzzled. If even the most classical economists were not the
proponents of ``universal regularities that are omnitemporally and omnispatially true characterised by a high level of
necessity'' so who was? The answer is offered in the next chapter, which is supposed firstly to ``sketch the history of
the process of the demise of law-centrism in the philosophy of science'' and secondly to present other approaches to
scientific explanations which do not refer to the scientific laws with special emphasise of ceteris paribus laws. In
this chapter, we also find an astonishing commentary on natural law tradition in economics. The author finds two
sources of law-centrism. The first is David Hume and his concept of causation as a regular concomitance between events
and their effects. The second is the neo-positivism, and especially Hempel’s account of explanation in science wherein
the law of nature is a necessary component of sound explanatory reasoning. However, the central part of the chapter
includes the discussion with those ``traditional'' accounts and presentation of other views, the author’s including.
Here, after remarks on models, we learn what the idea of the science without law means:

\begin{myquoterev}
It is not to erase the notion of laws from the fabric of science, but rather to define their role in a very specific
way, precisely, as statements being always true only in models used in their construction
\parencite[p.78]{hardt_economics_2017}.
%\label{ref:RND2lliVQaVuL}(Hardt, 2017, p.78).
\end{myquoterev}

Models in science and specifically in economics are crucial not only because they are the modern way of practising
economics but mainly because models are theory-creators and in them the capacities manifest themselves. Models,
however, have certain constraints which can be translated into the CP-laws, which are nevertheless explanatory and are
expected to survive the transition to the world. 

Chapter fourth is dedicated to the causal explanations in economics, and it constitutes a pretty good, though a
necessarily subjective summary of what remarkable has been written on the subject matter so far. Again the necessary
point of reference is the Humean concept of causation, which is outlined but mostly criticised. Undoubtedly he gave
rise to the set of regularity theories of causation, and probability view, which are still prevalent in contemporary
economics, both being discussed in the consecutive sections. The special place is reserved for Nancy Cartwright’s
account of capacities
\cite{cartwright_dappled_1999},
%\label{ref:RNDXu10TYyjs3}(Cartwright, 1999),
which is understandable as the author declares that,
although ``\mydots it seems that finding the appropriate theory of causation is impossible and we should accept the coexistence
of various philosophies of causation [\mydots] it does not mean that all of them are equally valid—for me the most
promising ones are approaches that are metaphysically rich; for example the Cartwright’s approach''
\parencite[p.118]{hardt_economics_2017}.
%\label{ref:RNDin7H4veSHJ}(Hardt, 2017, p.118).
In the last sections, we find an overview of the interventionist account
and causation in econometrics. 

Chapter five is probably the most important in the book, as it presents the author’s original idea of economic models as
believable worlds. This concept is so peculiar that it requires more in-depth analysis provided in the sperate section
below.

%\enlargethispage{-.1\baselineskip}

In the last chapter, the so-called distinctively mathematical explanations are discussed as the possible new sort of
explanation in economics, which seems to be already applied but is also promising for the future. The statement that
mathematic dominates economics is trivial. It is extensively used in many different fields. The focus in the chapter is
however on the mathematical explanation in the ``external'' sense, where according to the author mathematics is not only
the tool of deduction besides the causes and scientific laws but it explains by itself. It constitutes the decisive
element of explanation. The concept is not easy to be comprehended, so Hardt gives us examples. The Schelling’s
checkboard model of segregation is considered to be a distinctively mathematical explanation while Hal Varian’s model
of sale and price distortion is not. The main difference is that in Schelling’s model there is no causal inference and
no references to the economic laws, and the empirical interpretation (or application) is given ex-post, while in
Varian’s model the causal chain is fundamental. The conclusion is that the causal–mechanistic explanation in economics
is more typical, while Schelling’s case is rather exceptional. 

\subsubsection{Some critical remarks on laws and models}
Although the book presents an interesting position in the contemporary philosophy of economics, and its composition,
comprehensive approach to the discussed topics and references to ideas of well-known economists and philosophers make
it excellent reading for anyone interested in the subject matter, it has certain shortcomings. They lie mainly in those
parts of the book which constitute the core of Hardt’s account and which are therefore the most challenging.  

Let me start from the leading thesis that the concept of economics founded on scientific laws is obsolete and no longer
sustainable. Scientific laws are not in the centre of economic explanation. This leading thesis makes sense only in
light of the very peculiar understanding of scientific laws, namely ``universal regularities that are omnitemporally and
omnispatially true characterised by a high level of necessity''. Although such a definition is quite often a point of
departure for further discussion, it is usually considered to be a counterexample for presenting the main problems of
philosophy of science and thus naïve. I could hardly name a philosopher who really believes that any scientific law
meets this characteristic. Apparently, after reading the excellent historical chapter of Hardt’s book, I guess that he
could hardly name such an economist too. Is Hume the right candidate? I doubt. His concept of causation based on
temporal and spatial contiguity and especially his uniformity principle may resemble Hardt’s definition of the law of
nature, but one has to remember that both are constructs of our mind,  customs, belief that the future will be like the
past. Empiricist has nothing to say on the ontology of those phenomena, and the causal chain is always hypothetical as
based on fallible features of our mind
\parencite{hume_treatise_2000}.
%\label{ref:RNDgCfw5zFZf9}(Hume, 2000).
Better candidates seem to be the
neo-positivists like Hempel
\parencite{hempel_studies_1948}
%\label{ref:RNDgDJCOUayAj}(Hempel and Oppenheim, 1948)
and Carnap
\parencite*{carnap_logical_1967}.
%\label{ref:RNDnuhNCZiVns}(1967).
They might have believed in the universal laws of science and its cognizability, and
at least they used this concept in the explanatory reasoning, and they were trying to work out the logic of induction
which could provide the scientist with the method of sound reasoning from the observable repetitive phenomena to the
universal inductive generalisation. The problem is that they failed, remaining an interesting counterexample rather for
contemporary philosophy of science and moreover, they have never dedicated their work to social sciences like
economics, and their respective impact on this field was rather weak. Hardt knows it. In the chapter devoted to
scientific laws, most of the discussion is about the ideas which were trying to soften this demanding and impractical
definition of scientific law. On the other hand, if we relax the definition of laws of nature and acknowledge that they
are certain generalisations of observed regularities strongly dependent on assumed or observed constraints and context,
then the leading thesis seems unproven. 

A few words must be dedicated to Hardt’s concept of scientific models. It is not clear what models for Hardt are, but it
seems that he rejects the mathematical approach, wherein models are considered to be an interpretation of particular
language expressed with the use of predicates, constants, variables, functions and relations
\parencite{margaris_first_1990}.
%\label{ref:RNDd0TBcMbEPL}(Margaris, 1990).
It reads:

\begin{myquoterev}
[\mydots] logical positivism gave rise to the syntactic view of theories according to which a given theory is a set of
sentences in an axiomatized system of first order logic. In such an approach, there should be no role for other
consituents of science, including models . [\mydots] according to the syntactic view of theories, ‘a model is just a system
of semantic rules that interpret the abstract calculus and the study of a model amounts to scrutinizing the semantics
of a scientific language’. So, models are not independent entities, since they are largely defined by theories. No
theories (including laws of these theories), no models
\parencite[p.71]{hardt_economics_2017}.
%\label{ref:RNDeKOoqy83Yq}(Hardt, 2017, p.71).
\end{myquoterev}

It is a pity. Both first-order logic and models defined therein constitute an useful pattern to which we can always
refer. Rejecting this pattern may be the reason why Hardt perceives the difference between above criticised semantic
approach and his account. In the same section below, he quotes Giere and adds his comment which may give us a hint of
that:

\begin{myquoterev}
What have traditionally been interpreted as laws of nature thus turns to be merely statements describing the behaviour
of theoretical models''. So here the focus is on models but not as systems of semantics rules (syntactic views on
theories) but rather as being constitutive parts of theories. No models, no theories one could say (in syntactic
approach it is the other way round)
\parencite[p.72]{hardt_economics_2017}.
%\label{ref:RNDECtyz7BF9H}(Hardt, 2017, p.72).
\end{myquoterev}

This idea that models are instruments of theory creation returns in other chapters and sections of the book and has its
source in detachment of models from its mathematical pattern. It makes him forget that the main feature of the function
of interpretation is to preserve the truth value of sentences and that the function is reflexive. So these two
statements, no model, no theory and no theory, no model are rather equivalent. It would not be equivalent only in case
the author determined the features of the interpretation function in a way declining from its mathematical pattern. He
probably did, but without the reference to the pattern, it is not clear what was his intention. I deduce it from the
fact that in his account the isomorphism is gradable and respectively the truth value of the sentences in the model and
modelled domain are not entirely preserved after the ``transition''. Surprisingly within these sentences, we also find
laws of nature, and they play a remarkable part as we learn that  ``models are specified by laws''.  Therefore he claims
that science without laws ``is not to erase the notion of laws from the fabric of science, but rather to define their
role in a very specific way, precisely as statements being always true in models used in their construction''.  If this
interpretation is correct, so having in mind the fact that no one supports a very restrictive definition of a
scientific law, we have to conclude that that the concept of laws has not been erased from the fabric of science, as
they still constitute the foundation of a theories and models, and the perverse title of the book is a rhetorical trick
rather than an expression of a serious claim. Moreover, if the level of model’s isomorphism is gradable what means that
depending on this level some sentences in a model are true and some false, and if among those statements are scientific
laws, how is it possible be that they are always true in a model? Either they are tautologies true by assumptions and
applied inference rules or we have another test of this truth value. If the latter is the case, we should find an
answer in the chapter on models as believable worlds. They are indeed supposed to include the test of the truth value.
Hardt’s proposal is worth quoting:

\begin{myquoterev}
It was shown that models explain by producing theoretical insights (laws) that are always true within models but they
are just beliefs if claimed to accurately describe the real world. Thus such beliefs are more credible if the target is
close enough to the model’s structure
\parencite[p.161]{hardt_economics_2017}.
%\label{ref:RND9hoO4b7tbR}(Hardt, 2017, p.161).
\end{myquoterev}

In his concept of models as believable worlds, the important element is a ``theoretical insight'' produced by the model.
Laws are part of that insight, and they are, again true within a model. Here it is more expressly stated that they are
``true'' as they are either assumed or deduced according to the assumed rules of inference. If they are not, the model is
inconsistent and useless. But they are also beliefs about the real world, and credibility of those beliefs is gradable.
So instead of external test for truth value, we need a test for the level of credibility. We need the reliable (or
workable) criteria; otherwise models with outstanding theoretical insight, inherently true (consistent) might be
entirely detached from any economic realm and at most explains the meanders of the researcher’s mind. Here the
criterium is the closeness of the target to the model’s structure. So, how to assess the ``closeness''? In the chapter,
we encounter several attempts at clarification this term or its replacements, like ``similarity''. We learn that to be a
believable world, the model must meet the requirement of the mechanism in Woodward’s terms
\parencite{woodward_what_2002}.
%\label{ref:RNDnSeS1zWnlr}(Woodward, 2002).
So any model which is based on, e.g. data analysis but do not present any
mechanistic interpretation is by definition excluded. We also learn that it must refer to the ``essential explaining
items (including mechanism)'', but we do not know how to distinguish between essentials and non-essentials. But the
problem is correctly formulated. Hardt asks, what if we have multiple models fulfilling the above conditions? Which one
is closer or more similar to the target? And he proposes an answer: ``\mydots one must check as to what extent the theories
brought upon by models survive the transition from the world of the model to the real world'', period. Further reading
does not give us any further insight into the puzzling process of ``transition''. In reference to the exemplary model of
price distortion by Varian, Hardt only remarks: ``\mydots what is needed is a systematic empirical investigation into the
applicability of the model’s theoretical claims to a particular domain''
\parencite[p.154]{hardt_economics_2017}.
%\label{ref:RNDVQCseTAFgx}(Hardt, 2017, p.154).
I would say that the job is at least unfinished. The crucial element of his account, the criteria of various model
discriminations are not explained. The problem is shifted step by step to the consecutive, vogue terms: similarity,
closeness, transition survival. 

\subsubsection{Is the book worth reading? }
Although I could not resist myself from the above critical remarks, I still maintain that the book is worth reading, for
at least three reasons:

\begin{enumerate}
\item It is very informative. The reader can learn a lot about various ideas of philosophers of science (especially
economics) mostly from the realistic camp and those ideas are presented clearly and comprehensively. They are supported
with numerous, well-chosen examples from economics. So not only economists can learn some philosophy but also
philosophers can learn some economics. Anyway, the author is an economist in the first place. 
\item It is inspiring. The fact that I allow myself to present the above critical remarks is a visible sign that Hardt’s
ideas make us think about them thoroughly and sometimes makes us reconsider our position or become more aware of it. 
\item It is a good and pleasant reading, demanding in certain sections but definitely not dull. 
\end{enumerate}


\autorrec{Marcin Gorazda}

\subsubsection{Bibliography}\nopagebreak[4]
\end{recengenv}
