\begin{artplenv}{Filip Lubiński}
	{Rola państwa i~prawa w~systemie Adama Smitha}
	{Rola państwa i~prawa w~systemie Adama Smitha}
	{Rola państwa i~prawa w~systemie Adama Smitha}
	{Uniwersytet Warszawski\label{lub-start}}
	{The role of state and law\\in the system of Adam Smith}
	{The dominant contemporary interpretation of Adam Smith's thought endeavors to show his anti-state attitude.
		According to this interpretation, Smith, as the father of economics would also be an opponent of state interference in
		the activity of private entities. The purpose of this work is a~comprehensive presentation of the social system
		described by a~Scottish philosopher. This task is accomplished by drawing attention to Smith's views on the state and
		the law, as well as on his participation in the creation of an area of knowledge known then as political economy.
		
		In order to reconcile the seemingly inconsistent views presented in Smith's most famous works, the main subject of
		this analysis are the less frequently studied \textit{Lectures on Jurisprudence}. This book, which offers Smith's views
		on the role of the state and law, is the key to the correct reading of the thesis put in \textit{Theory of Moral
			Sentiments} and \textit{An Inquiry into the Nature and Causes of the Wealth of Nations}. The social system emerging
		from these works referred to by Smith as the civil government, can in retrospect be regarded as the progenitor of
		liberal democracy. The state and the law play a~vital role in the economic life of this system.}
	{classical economics, etatism, liberalism, state, law, Adam Smith.}
	

\begin{footnotesize}
\begin{flushright}
	The more improved any society is and the greater length the severall\\
	means of supporting the inhabitants are carried, the greater will be\\
	the number of their laws and regulations necessary to maintain justice\\
	and prevent infringements of the right of property.\\
	Adam Smith, \textit{Lectures on Jurisprudence}
	
\end{flushright}	
\end{footnotesize}




\section*{Wstęp}
\lettrine[loversize=0.13,lines=2,lraise=-0.05,nindent=0em,findent=0.2pt]%
{Z}{}naczenie twórczości Adama Smitha dla rozwoju nauk społecznych jest powszechnie znane. Należy on do autorów, których
dzieła nie tylko roznieciły dyskusję nad społeczeństwem w~ramach ośrodków akademickich, ale i~odcisnęły piętno na
społecznych i~międzynarodowych przemianach następnych dziesięcioleci
\parencite[s.~12]{milgate_after_2009}.
%\label{ref:RND1WL5POjFQ5}(Milgate, Stimson, 2009, s.~12).
Podczas swojej wieloletniej kariery dydaktycznej wykładał on szeroki zakres przedmiotów -- literaturę angielską,
retorykę, filozofię moralną, prawo oraz ekonomię polityczną
\parencite[s.~126]{roncaglia_wealth_2005}.
%\label{ref:RNDAE89MTXX6x}(Roncaglia, 2005, s.~126).
Jednakże zarówno w~kwestii jego dorobku naukowego, jak i~wywartego przezeń wpływu na proces nauczania, ostatnia
dziedzina wydaje się przesłaniać znaczenie pozostałych. Wszystko za sprawą jego życiowego dzieła, \textit{Badania nad
naturą i~przyczynami bogactwa narodów}, uznawanego powszechnie za pracę konstytutywną dla nauki ekonomii politycznej
(\cite{smith_badania_2007} [1776]).
%\label{ref:RNDezUk1QStk4}(Smith, 2007 [1776]).

Dysproporcja w~uwadze poświęconej poszczególnym obszarom twórczości Smitha pozostaje niepokojąca z~przynajmniej
kilku powodów. Przede wszystkim dlatego, że przyczyniła się ona do rozpowszechnienia błędnego
przekonania o~autonomiczności ekonomii jako dziedziny wiedzy, pomimo jej zakorzenienia w~innych naukach społecznych.
To przekonanie
doprowadziło do oddzielnego badania trzech sfer nowoczesności -- rynku, społeczeństwa i~państwa, kierujących się jakoby
inną logiką i, co za tym idzie, wymagających odmiennego podejścia
\parencite[s.~19]{wallerstein_analiza_2007}.
%\label{ref:RNDxEklb6Rjec}(Wallerstein, 2007, s.~19).
Niebezpieczeństwo związane z~próbą udzielenia ekonomii maksymalnej autonomii nie jest dziś dla nikogo tajemnicą. Już na
początku XIX wieku, gdy na Uniwersytecie w~Oxfordzie rozważano utworzenie stopnia profesorskiego w~dziedzinie ekonomii,
budziło to głosy sprzeciwu niechętnych wpuszczeniu w~progi uniwersytetu nauki ,,tak skorej do przywłaszczania sobie
cudzych zdobyczy''
\parencite[s.~54]{schumacher_male_1981}.
%\label{ref:RNDq88dfFYNSw}(Schumacher, 1981, s.~54).

Kolejnym powodem do niepokoju jest poświęcanie niewielkiej uwagi wcześniejszym pracom Smitha, przede wszystkim
\textit{Teorii uczuć moralnych}
(\cite{smith_teoria_1989} [1759]).
%\label{ref:RNDuvnqXhGdwl}(Smith, 1989 [1759]).
Prowadzi to często do uznawania sfery
ekonomii za nie tylko nieobejmowaną przez etykę, lecz wręcz znajdującą się ponad nią. Ignoruje się w~ten sposób fakt,
iż ekonomia i~etyka pozostawały aż do końca XIX wieku w~nierozerwalnym związku
\parencite[s.~235]{rawls_wyklady_2010},
%\label{ref:RNDifRkgFWC39}(Rawls, 2010, s.~235),
a~akceptacja kluczowej dla dominującej współcześnie szkoły ekonomii teorii racjonalności ,,zobowiązuje do
przyjęcia kontrowersyjnych zasad moralnych''
\parencites[s.~50]{hausman_etyka_2017}{turek_2019}.
%\label{ref:RNDHRnH8e5vNp}(Hausman, i~in., 2017, s.~50). 

Smith zdawał sobie sprawę z~powiązań pomiędzy naukami społecznymi. Zagadnień z~zakresu etyki i~ekonomii nie
rozpatrywał on w~oderwaniu od siebie
\parencite[s.~130]{soll_reckoning:_2014}.
%\label{ref:RNDc7lad85KL1}(Soll, 2014, s.~130).
Dobrym tego świadectwem może być
lista autorytetów, na które powołuje się w~\textit{Bogactwie narodów}: Francis Hutcheson, Dugald Stewart,
Voltaire a~także Beccaria, Grocjusz i~Puffendorf -- wielkie postaci ówczesnego świata nauk etycznych i~prawnych
\parencite[s.~654]{boorstin_discoverers:_1983}.
%\label{ref:RNDYwU1DwRdXl}(Boorstin, 1983, s.~654).
Z~drugiej strony badacze podejmujący próbę zestawienia poglądów
Smitha na teorię ekonomii ze stworzonym przez niego systemem etycznym stwierdzają istnienie tak zwanego \textit{Das
Adam Smith Problem} -- sprzeczności pomiędzy \textit{Bogactwem narodów} a~\textit{Teorią uczuć moralnych}
\parencites[s.~215–216]{sedlacek_ekonomia_2012}{turek_2019}.
%\label{ref:RNDVh9gEUItJS}(Sedláček, 2012, s.~215–216).
To budzące powszechną konsternację zagadnienie skłania do
przyjrzenia się w~tej pracy nieukończonemu dziełu Adama Smitha -- traktatowi na temat państwa i~prawa, który
wraz z~resztą jego niewydanych prac został zgodnie z~testamentem zniszczony po śmierci autora
\parencite[s.~140]{buchan_adam_2008}.
%\label{ref:RNDsGGYAWrkxh}(Buchan, 2008, s.~140). 

Pomimo ostatniej woli Smitha treść jego ,,teorii ogólnych zasad prawa i~rządu'' została zachowana dzięki
zapisom z~wykładów na uniwersytecie w~Glasgow w~latach 1762--1763 i~1763--1764
(\cite{smith_lectures_1982} [1763]).
%\label{ref:RNDDaJwKBJe61}(Smith, 1982 [1763]).
Analiza tez wysuwanych przez Smitha na temat roli państwa i~prawa w~zestawieniu z~poglądami wyrażonymi w~jego
opublikowanych dziełach stanowi główne zadanie poniższej pracy. Celem jest nie tylko rekonstrukcja systemu
polityczno-prawnego Adama Smitha, ale także zestawienie tego systemu z~resztą jego dorobku intelektualnego, co pozwala
dostrzec spójność pomiędzy teoriami wysuniętymi w~dwóch wydanych za życia autora dziełach. Ponowna interpretacja
tekstów Smitha przywraca mu także należne miejsce wśród tych, którzy z~należytą ostrożnością spoglądali na rodzący się
na ich oczach system społeczno-gospodarczy
\parencite[s.~83]{ferguson_wielka_2017}.
%\label{ref:RNDsLlADmbRUC}(Ferguson, 2017, s.~83).


Moment wydania \textit{Badań nad naturą i~przyczynami bogactwa narodów} (\hyperlink{smith-bib}{1776}) uchodzi za historyczną
cezurę -- wyznacza narodziny nowoczesnej ekonomii politycznej stanowiącej od tamtego czasu obszar coraz bardziej aktywnych badań
\parencite[s.~199]{broadie_history_2009}.
%\label{ref:RNDxlNYKcaRLz}(Broadie, 2009, s.~199).
Rozważanie definiującego tę naukę -- jak to określił Lionel
Robbins -- podstawowego problemu ekonomicznego
\parencite[s.~15]{robbins_essay_1932}
%\label{ref:RNDnxuF1ebG8p}(Robbins, 1932, s.~15)
trwało już od czasów antycznych,
jednakże to Adam Smith wprowadził w~te badania jakość, która zapewniła mu honorowe miejsce pośród wszystkich
ekonomistów
\parencite[s.~XII]{gordon_economic_1975}.
%\label{ref:RNDXwmhFbTJ5B}(Gordon, 1975, s.~XII).
Źródła podstawowych różnic między pracą
Smitha a~wcześniejszą twórczością z~zakresu ekonomii da się odnaleźć w~szerokich badaniach, których podejmował się on w~ramach
swojej pracy akademickiej. Praca na uniwersytecie pozwalała mu opisywać zjawiska gospodarcze z~odpowiednią
bezstronnością i~obiektywizmem, czego brakowało chociażby krytykowanym przez niego merkantylistom,
będącym w~większości ludźmi interesu. Również jego postawie naukowca zawdzięczamy
przedstawioną w~jego dziełach szeroką wizję występujących w~społeczeństwie współzależności
\parencite[s.~111–112]{landreth_historia_1998}.
%\label{ref:RNDQbbqHBuaQg}(Landreth, Colander, 1998, s.~111–112).

Pomimo znaczenia jakie przypisuje się \textit{Badaniom nad naturą i~przyczynami bogactwa narodów}, jest to
jedna z~książek, które najmocniej ucierpiały na skutek zaniknięcia praktyki lektury tekstów źródłowych, szczególnie widocznego
w programie nauczania ekonomii
\parencite[s.~58]{blaug_teoria_1994}.
%\label{ref:RNDwb8UrvF6xd}(Blaug, 1994, s.~58).
Niektórzy z~badaczy tematu posuwają się
nawet do stwierdzenia, iż jest to książka najczęściej cytowana z~najrzadziej czytanych
\parencite[s.~67]{heilbroner_worldly_1999}.
%\label{ref:RNDhU0CYVU7xS}(Heilbroner, 1999, s.~67).
Prowadzi to do łatwych do przewidzenia efektów -- subtelna myśl
Smitha zostaje w~debacie zwulgaryzowana, a~on sam utrwalony w~historii jako radykalny leseferysta
\parencites[s.~16–17]{skousen_big_2007}[s.~152]{landreth_historia_1998}[s.~261]{rasmussen_pragmatic_2014}.
%\label{ref:RNDllaqGb90E5}(Skousen, 2007, s.~16–17; Landreth, Colander, 1998, s.~152; Rasmussen, 2014, s.~261).
Zjawisko
to najlepiej opisuje cytat ze wspomnień wybitnego polskiego ekonomisty: ,,Czytałem ponownie klasyków -- okazało się, że
Adam Smith jest o~wiele bardziej postępowy niż wynikałoby to z~karykatury, jaką czyni z~niego
liberalizm''
\parencite[s.~39]{dowbor_rozbita_2005}.
%\label{ref:RNDuVHaEp23gh}(Dowbor, 2005, s.~39).
Czymże jest jednak
,,liberalizm'', któremu zszarganie dobrego imienia Smitha zarzuca Ladislau Dowbor? Czy Smith niesłusznie zaliczany jest
do klasyków tej doktryny
\parencite[s.~38]{gray_liberalizm_1994}?
%\label{ref:RNDk1CfNtw5Df}(Gray, 1994, s.~38)?
Podejmując próbę odpowiedzi na te pytania,
wynikłe z~pierwotnej recepcji \textit{Badań nad naturą i~przyczynami bogactwa narodów}, warto jest jednocześnie wykazać
aktualność prac Smitha w~kontekście współczesnych sporów ekonomicznych i~politycznych.

Interpretacja poglądów Smitha przeszła pomiędzy 1776 a~1817 rokiem wyraźną transformację. ,,Adam Smith znajdujący
się w~każdym przypadku konfliktu interesów pomiędzy bogatymi a~biednymi, pomiędzy silnymi i~słabymi, bez wyjątku po
stronie tych drugich''
\parencite[s.~223]{menger_kleinere_1935}
%\label{ref:RNDsAFNZfRokB}(Menger, 1935, s.~223)
zaczął być uznawany za ,,ewangelistę pracodawców
XIX wieku'' -- przemianę tę najlepiej opisała w~swoim dziele \textit{Economic Sentiments} Emma Rothschild
\parencite*[s.~113]{rothschild_economic_2002}.
%\label{ref:RNDs0eUTXaI1v}(Rothschild, 2002, s.~113).
Ostrożność wypowiedzi Smitha i~przeinaczenia, których dopuszczono
się w~trakcie debaty nad reformą systemu terminatorskiego, doprowadziły do powstania trwającego po dziś dzień wizerunku
Smitha jako leseferysty. W~rzeczywistości należy przypisywać mu głęboką świadomość roli państwa i~prawa w~gospodarce
oraz społeczeństwie, którą udowadniał w~swoich opublikowanych dziełach, a~w~pełni przedstawił w~ramach \textit{Lectures
on Jurisprudence}. Smith nigdy nie był zwolennikiem państwa pełniącego rolę ,,stróża nocnego'' (określenie to powstało
zresztą dopiero jako inwektywa użyta przez Lassalle'a
\parencite[s.~87]{sawer_ethical_2003}).
%\label{ref:RNDPvkhPhffGM}(Sawer, 2003, s.~87)).

Jednak czy jest sens wracać do źródeł ekonomii współcześnie, gdy większość z~jej kluczowych zagadnień uważa się już
za rozstrzygnięte? Historia tej nauki pełna jest zwrotów, a~nawet najmocniej ugruntowane teorie mogą utrzymywać swoje
uprzywilejowane miejsce tylko tak długo, aż nie pojawią się inne, choć w~małym stopniu lepiej tłumaczące sfery
ekonomicznej aktywności
\parencite[s.~33]{blaug_teoria_1994}.
%\label{ref:RNDIzKebFVZYU}(Blaug, 1994, s.~33).
Twierdzę, że spostrzeżenia prekursorów nie tylko
pozostają aktualne, ale i~zawierają w~sobie elementy, które nie doczekały się właściwego docenienia lub zostały
zapomniane. Dowodem na to może być przeżywająca aktualnie rozkwit krytyka teorii przewagi komparatywnej w~handlu
międzynarodowym
\parencite{reinert_how_2008}.
%\label{ref:RNDCycJKXS8KA}(Reinert, 2008).
Zawarte w~niej argumenty prócz tego, że przemawiają za przyjrzeniem
się ponownie teorii przewag bezwzględnych Smitha, wywodzą się z~poczynionych już przez niego spostrzeżeń 
(\cite{smith_badania_2007} [1776], t.~1, s.~11--12).
%\label{ref:RNDtOabovEKbm}(Smith, 2007 [1776], t. 1, s.~11-12).
Twierdzenia te (zapewne dzięki Smithowi) były dobrze znane intelektualistom następnych pokoleń
\parencite[s.~343]{tolstoj_anna_1986}.
%\label{ref:RND8shRFO5Ofw}(Tolstoj, 1986, s.~343).

W następnych częściach tej pracy omówione zostaną kluczowe w~kontekście rozważań na temat roli państwa i~prawa
fragmenty dzieł pozostawionych przez Adama Smitha. Całościowa analiza dorobku intelektualnego Smitha pozwala na
zrekonstruowanie jego systemu filozoficznego oraz, co w~kontekście tej pracy najważniejsze, znaczenia, jakie przypisywał
on instytucji państwa. Analiza poglądów szkockiego filozofa na zagadnienia etyczne związane z~etatyzmem, jak i~na jego
wpływ na funkcjonowanie gospodarki prowadzi do wniosków w~znacznym stopniu odbiegających od współczesnej interpretacji
myśli Smitha.

\section{Relacja pomiędzy państwem a~gospodarką}
Dziedzina, za narodziny której przyjęło się uważać wydanie największego dzieła Adama Smitha, od początku nosiła miano
\textit{ekonomii} \textit{politycznej}. Nie uważał on państwa za balast ciążący na barkach ludzkiej aktywności
ekonomicznej, a~w~swojej pracy będącej w~istocie traktatem na temat rozwoju dobrobytu postrzegał ,,dobre rządzenie'' jako
jeden ze środków koniecznych do jego osiągnięcia
(\cite{smith_badania_2007} [1776], t.~1, s.~16, a~także t.~2, s.~7).
%\label{ref:RNDnQbUYFullV}(Smith, 2007 [1776], t. 1, s.~16, a~także t. 2, s.~7).
Celem tej części pracy jest przeanalizowanie zadań, które dzieła Smitha stawiają przed państwem w~zakresie
funkcjonowania gospodarki.

Należy podkreślić, że rola państwa i~prawa nie pozostaje w~systemie Smitha wyłącznie korygująca -- ma ona wymiar
strategiczny i~przybiera postać radykalnie etatystycznych rozwiązań, co przejawia się chociażby w~określaniu jawnie
godzących w~swobodę handlu Aktów Nawigacyjnych ,,najmądrzejszą ze wszystkich ustaw handlowych Anglii''
(\cite{smith_badania_2007} [1776], t.~2, s.~51).
%\label{ref:RNDSfqMVn36ju}(Smith, 2007 [1776], t.~2, s.~51).
Smith wspierał także strategię Wielkiej Brytanii polegającą
na zwiększaniu eksportu poprzez rozbudowę państwowego stocznictwa
\parencite[s.~185–186]{beattie_false_2010}.
%\label{ref:RNDNSdWnZqXWk}(Beattie, 2010, s.~185–186).
Państwo bowiem nie tylko chroni własność, ale jako reprezentacja zorganizowanej społeczności ma względem prawa
własności charakter pierwotny
(\cite{smith_lectures_1982} [1763], s.~23),
%\label{ref:RNDRS9jKbzBoD}(Smith, 1982 [1763], s.~23),
a~poprzez regulacje ma umożliwić
korzystanie z~niego w~ramach procesu będącego źródłem dobrobytu -- konkurencji
(\cite{smith_badania_2007} [1776], t. 1, s.~74).
%\label{ref:RND8sqtcJZ1ob}(Smith, 2007 [1776], t. 1, s.~74).
Rywalizacja w~systemie rynkowym rozumiana była jednak u~swych początków w~sposób nieco odmienny
niż współcześnie.

,,Niewidzialna ręka'' rynku jest bez wątpienia najpopularniejszą metaforą w~dorobku
Smitha, a~zapewne i~jedną z~najsłynniejszych w~historii nauk społecznych
\parencite[s.~104]{cremaschi_metaphors_2002}.
%\label{ref:RND56kR0uPFkj}(Cremaschi, 2002, s.~104).
Jednak wszystko
wskazuje na to, że określenie użyte tylko raz na przestrzeni ponad tysiąca stron \textit{Bogactwa narodów} miało
pierwotnie inne znaczenie niż to, które mu się zwykle przypisuje. Zasługę wykazania tego należy przyznać Emmie
Rothschild, która analizując działalność Smitha jako wykładowcy literatury angielskiej i~autora dzieła na temat
historii astronomii, stwierdziła, że miał on do ukrytego pod tą metaforą procesu zgoła sceptyczny stosunek i~nie przypisywał jej
takiego znaczenia, jak się współcześnie uważa
\parencite[s.~116–153]{rothschild_economic_2002}.
%\label{ref:RNDX0KZpqEQRj}(Rothschild, 2002, s.~116–153).
Sama idea, iż
podążanie za interesem własnym przyczyni się do wspólnego dobra, jest o~wiele starsza od \textit{Bogactwa narodów} i~da
się ją znaleźć już u~Monteskiusza
\parencite[s.~50]{montesquieu_o_2009}.
%\label{ref:RNDStjVF4fTz2}(Montesquieu, 2009, s.~50).
Przy czym zaznaczyć należy, że
myślano w~ten sposób o~dobrobycie w~sensie społecznym, a~nie ekonomicznym. Utwierdza nas to w~przekonaniu o~politycznym
charakterze dzieł Smitha, wynikającym z~tego, że u~początków systemu kapitalistycznego argumenty formułowane za jego
wprowadzeniem miały przede wszystkim etyczny, a~nie efektywnościowy charakter
\parencite[s.~37–45]{hirschman_namietnosci_1997}.
%\label{ref:RNDkxQbrekKjf}(Hirschman, 1997, s.~37–45).
Liczono na to, że podyktowana wyłącznie interesem ekonomicznym wymiana doprowadzi w~efekcie do wzrostu
społecznej tolerancji i~usunie odziedziczone w~schedzie po epoce feudalnej antagonizmy
\parencite[s.~23–36]{muller_mind_2003}.
%\label{ref:RNDYmmEjf3eVb}(Muller, 2003, s.~23–36).

Smith został zapamiętany w~dużej mierze jako krytyk instytucji, którym wytykał ich wątpliwe podstawy etyczne bądź
katastrofalne ekonomiczne efekty. Należy jednak zaznaczyć, że choć uważany jest on za przeciwnika etatyzmu, jego pisma
stanowią reprymendę przede wszystkim dla instytucji znajdujących się poza centralnym ośrodkiem władzy -- gildii,
korporacji, kościołów, rad miejskich i~uprzywilejowanych kompanii kupieckich
\parencite[s.~108]{rothschild_economic_2002}.
%\label{ref:RNDTxyZWr558R}(Rothschild, 2002, s.~108).
To w~nich w~pierwszej kolejności Smith doszukuje się źródła ekonomicznych porażek Anglii, a~nie w
ustawodawstwie państwowym, które może, a~nawet powinno, wprowadzać obowiązek dobroczynności przyczyniającej się do
ogólnego dobrobytu państwa
(\cite{smith_teoria_1989} [1759], s.~118--119).
%\label{ref:RNDW6eBaQF4K0}(Smith, 1989 [1759], s.~118-119).
Obowiązek ten miał być
egzekwowany poprzez redystrybucję i~podatki o~charakterze proporcjonalnym
(\cite{smith_badania_2007} [1776], t.~2, s.~500).
%\label{ref:RNDpQhGODGDa8}(Smith, 2007 [1776], t. 2, s.~500).
Nowoczesny rząd ma za zadanie chronić obywateli od ,,gwałtów państwa
feudalnego'', zapewniając im bezpieczeństwo, bez którego niemożliwe jest prawidłowe działanie sił konkurencji
(\cite{smith_badania_2007} [1776], t.~1, s.~315).
%\label{ref:RNDWm3VFowt3n}(Smith, 2007 [1776], t.~1, s.~315).
Jak już wspomniano, konkurencja ta pojmowana jest u
Smitha inaczej niż współczesna absolutna swoboda działania podmiotów gospodarczych. Państwo ma zadanie aktywnego
równoważenia sił rynkowych, gdyż w~interesie kupców leży zawsze ograniczanie konkurencji, co pozostaje w~sprzeczności z
interesem publicznym
(\cite{smith_badania_2007} [1776], t.~1, s.~292).
%\label{ref:RND2jnRYZ0Sx9}(Smith, 2007 [1776], t. 1, s.~292).

Aby podsumować uwagi na temat państwa i~prawa poczynione przez Smitha w~\textit{Bogactwie narodów}, warto przyjrzeć
się umieszczonemu w~nim katalogowi obowiązków panującego. Jak stwierdza sam autor, składa się on ,,jedynie z~trzech''
pozycji. Są to zapewnienie bezpieczeństwa militarnego, organizacja wymiaru sprawiedliwości oraz obowiązek utworzenia
pewnych instytucji publicznych, których ,,ustanowienie i~utrzymywanie nie może nigdy leżeć w~interesie jednostki lub
niewielkiej liczby jednostek, a~to dlatego, że dochód z~nich nie pokryje nigdy kosztów jednostce lub małej grupie
jednostek, choć koszty jakie poniosło jakieś wielkie społeczeństwo może często pokryć z~nadwyżką''
(\cite{smith_badania_2007} [1776], t.~2, s.~340).
%\label{ref:RNDtabhicNFdT}(Smith, 2007 [1776], t. 2, s.~340).
Z~pewnością nie chodzi tu wyłącznie o~instytucje wpisujące
się w~standardowe zadania państwa, w~rodzaju szerzenia oświaty, któremu Smith osobno poświęcił sporo miejsca
(\cite{smith_badania_2007} [1776], t.~2, s.~450--451).
%\label{ref:RND91wqVhsg2n}(Smith, 2007 [1776], t.~2, s.~450-451).
W~powyższym cytacie opisano rozwiązania dotyczące
wszelkich aspektów życia, w~których jedynie za sprawą publicznej interwencji możliwe jest podniesienie sumy
indywidualnych korzyści. Rozwój nauki ekonomii doprowadził do odkrycia licznych sytuacji tego rodzaju. Wielokrotnie
były to rozwiązania nieoczywiste, jak chociażby publiczne dopłaty do pensji najsłabiej zarabiających,
finansowane z~podatków powszechnych, które zasugerował laureat ekonomicznej Nagrody Nobla o~poglądach libertariańskich Edmund Phelps
\parencite{phelps_placa_2013}.
%\label{ref:RNDa9f9c5iU52}(Phelps, 2013).
Część tę należy więc podsumować stwierdzeniem, że przedstawiona przez Smitha
rola państwa w~gospodarce znacznie odbiega od współczesnej interpretacji jego myśli. Rządowe interwencje ekonomiczne
mają wedle teorii wyłożonych w~\textit{Bogactwie narodów} kluczowe znaczenie dla powszechnego dobrobytu. 

\section{Etyka indywidualnych uprawnień}
Jeśli metaforę przewodnią \textit{Bogactwa narodów} stanowi ,,niewidzialna ręka'' to w~przypadku \textit{Teorii uczuć
moralnych} rolę tę pełni bez wątpienia ,,bezstronny obserwator''
(\cite{smith_teoria_1989} [1759], s.~163).
%\label{ref:RNDkXOuGroqna}(Smith, 1989 [1759], s.~163).
Autor przyznał tej koncepcji duże znaczenie i~umieścił ją również w~rozważaniach z~zakresu innych dziedzin, nie tylko
etyki
(\cite{smith_lectures_1982} [1763], s.~104).
%\label{ref:RND3u7VBrgecb}(Smith, 1982 [1763], s.~104).
Jest to istotne, ponieważ bywa on interpretowany jako
apologeta egoizmu, zdaniem którego dbanie o~własny interes stanowi najwyższą cnotę. W~tej części zostanie wykazane,
że u~podstaw wyłożonego w~dziełach Smitha systemu filozoficznego leży nie subiektywizm, a~etyka oparta na
przekonaniu o~istnieniu przyrodzonych każdej jednostce uprawnień.

Smith nie przecenia efektów troski o~siebie i~zauważa, że ,,droga prowadząca do majątku i~ta, która prowadzi do
cnoty, czasami prowadzą w~zupełnie różne strony''
(\cite{smith_teoria_1989} [1759], s.~92).
%\label{ref:RNDo3EelCAtEP}(Smith, 1989 [1759], s.~92).
Bezstronny
obserwator nie jest relatywistą, ale nie należy też do żadnych stronnictw -- kierują nim wartości wynikające
wprost z~przyrodzonej człowiekowi godności
(\cite{smith_teoria_1989} [1759], s.~227--228).
%\label{ref:RNDCU5LDHfghn}(Smith, 1989 [1759], s.~227-228).
Jak zostanie wykazane,
jest to efekt przyjmowanej przez Smitha koncepcji jednostkowych uprawnień
(\cite{smith_lectures_1982} [1763], s.~8).
%\label{ref:RNDBHV5ZarCoy}(Smith, 1982 [1763], s.~8). 

Autor \textit{Teorii uczuć moralnych} daje się poznać wielokrotnie jako obrońca tłamszonych, wykraczający poza
prywatną perspektywę. Pomimo niebezpieczeństw z~tym związanych krytykuje m.in. krucjaty, procesy
heretyków i~niewolnictwo
(\cite{smith_badania_2007} [1776], t.~1, s.~461).
%\label{ref:RNDkKZBWJ1Bu0}(Smith, 2007 [1776], t.~1, s.~461).
To właśnie w~kontekście niewolnictwa
najłatwiej zauważyć za Jerzym Szackim, że ,,związek pomiędzy Smithem-ekonomistą a~Smithem-filozofem jest jak
najbardziej ścisły, a~jego socjologii należy szukać w~obu sferach jego działalności naukowej''
\parencite[s.~124]{szacki_historia_1983}.
%\label{ref:RNDC8VLy7vrzx}(Szacki, 1983, s.~124).
Smith krytykuje niewolnictwo nie tylko z~przyczyn etycznych. W~całym
jego wywodzie dominuje myśl, iż najefektywniejszą formę współpracy ekonomicznej stanowi ta oparta na sprawiedliwej
wymianie i~wzajemności. Pozbawiona tych elementów i~nieuregulowana prawnie relacja pana i~niewolnika nie jest w~stanie
osiągnąć optymalnych efektów. Opisana powyżej zależność stanowi obszar badań na nowo odkryty we współczesnej ekonomii
instytucjonalnej
\parencite{acemoglu_dlaczego_2014}.
%\label{ref:RNDvpwxutjCAN}(Acemoglu, Robinson, 2014).

Koncepcja sprawiedliwości, która musi odnaleźć swój wyraz w~prawnym uregulowaniu, odgrywa, jak się przekonamy, ważną
rolę w~systemie Smitha. Stanowiący główną oś rozważań \textit{Bogactwa narodów} mechanizm rynkowy ,,będzie sprzyjał
harmonii, ale jedynie w~otoczeniu odpowiedniego systemu prawnego i~instytucjonalnego''
\parencite[s.~83]{blaug_teoria_1994}.
%\label{ref:RNDuJ77RGCwoa}(Blaug, 1994, s.~83).
Ma to duże znaczenie w~kontekście wielokrotnie wspominanego przez Smitha rozwarstwienia
społecznego -- ,,szacunek dla możnych, zgodnie z~tym najbardziej może przynieść szkody,
gdy występuje w~nadmiarze; współoddźwięk wobec
nędzarzy, gdy jest niewystarczający''
(\cite{smith_teoria_1989} [1759], s.~336).
%\label{ref:RNDu4UprONNkk}(Smith, 1989 [1759], s.~336).
Niebezpieczeństwo
wynikające z~tej zależności zrozumiemy najlepiej, kiedy przyjrzymy się zjawisku nazywanemu ,,zmęczeniem współczucia''.
Położenie ludzi biednych jest często tak trudne, że przytłacza obserwatorów i~sprawia, iż wypierają je ze świadomości,
a w~efekcie ,,ludzie na dole drabiny społecznej mają wiele empatii dla ludzi z~jej szczytu, ale odwrotna zależność
praktycznie nigdy nie występuje''
\parencite[s.~95]{graeber_utopia_2016}.
%\label{ref:RNDNWyWu164di}(Graeber, 2016, s.~95).
Dlatego też Smith nie pokłada
wiary w~możliwości rozwiązania problemów społecznych poprzez charytatywność i~opowiada się za państwowymi rozwiązaniami
(\cite{smith_lectures_1982} [1763], s.~50).
%\label{ref:RNDLkHuKp1To6}(Smith, 1982 [1763], s.~50).

Jak zostało to zaznaczone na początku rozważań dotyczących \textit{Teorii uczuć moralnych}, to nie interes
własny, lecz obiektywna ocena prowadzi do przyznania czynowi pozytywnej klasyfikacji etycznej. Warto też zaznaczyć, iż
Smith jawnie krytykował libertyńskie systemy etyczne, które jego zdaniem zamazywały granicę między cnotą i~przywarą
(\cite{smith_teoria_1989} [1759], s.~463).
%\label{ref:RND6UVhftNekE}(Smith, 1989 [1759], s.~463).
W~\textit{Teorii uczuć moralnych} przykładem poddanym krytyce
jest koncepcja Bernarda Mandeville'a wyrażona w~jego \textit{Bajce o~pszczołach}
\parencite[s.~186–188]{mandeville_bajka_1957}.
%\label{ref:RNDhsq78lJp7f}(Mandeville, 1957, s.~186–188).
Jak wynika z~przedstawionych tu poglądów Smitha, nie uważał on etyki za kategorię subiektywną.
W~swoich dziełach uzasadniał, że istnieją liczne działania, które zarówno państwo, jak i~jednostki powinny podejmować ze
względu na przysługujące każdemu przyrodzone uprawnienia.

 Należy tu także dodać, że teoria libertyńska, choć odmienna na poziomie filozoficznym od poglądów Smitha, jest do
nich całkiem podobna w~kwestii uznania możliwości powiązania interesów indywidualnych z~powszechnym dobrobytem za
pomocą odpowiednich interwencji państwa. Mandeville wykazuje, że nawet instynkty w~rodzaju tych, które stanowią temat
\textit{Skromnej obrony domów publicznych}, mogą w~odpowiedniej siatce instytucjonalnej przyczynić się do społecznych
korzyści
\parencite{mandeville_skromna_2016}.
%\label{ref:RNDKGiGp2VVGh}(Mandeville, 2016).
Jak ujął to w~swojej korespondencji -- ,,osobiste wady mogą obracać
się w~korzyści publiczne przez zręczne rządy utalentowanego polityka''
\parencite[ s.~37]{mandeville_letter_1953}.
%\label{ref:RNDFxtQ9hz1QF}(Mandeville, 1953, s.~37).

\section{Sprawiedliwość, równość i~własność}
Sam opis przedmiotu \textit{Lectures on Jurisprudence} wskazuje na to, że pomimo charakteru dydaktycznego zawierały
one normatywny przekaz: ,,jurysprudencja jest teorią zasad, którymi kierować powinien się rząd cywilny'' -- informuje nas
Smith na samym początku swoich wykładów
(\cite{smith_lectures_1982} [1763], s.~5).
%\label{ref:RNDscYVjbho0w}(Smith, 1982 [1763], s.~5).
Zasady te
wynikają z~wartości mających z~kolei swoje źródło w~opisanych w~poprzedniej części tej pracy, przysługujących zdaniem Smitha
wszystkim ludziom naturalnych uprawnieniach
(\cite{smith_lectures_1982} [1763], s.~8--9).
%\label{ref:RND2YqiGCGi5H}(Smith, 1982 [1763], s.~8-9).
Ta część stanowi
rekonstrukcję wartości leżących u~podstaw systemu społecznego, który wyłania się z~pozostawionego przez Smitha dorobku
intelektualnego.

 Należy zaznaczyć, że indywidualność nie stanowi jedynego źródła ludzkich uprawnień -- jest nim nie tylko bycie
osobą, ale i~członkiem rodziny, a~także obywatelem, członkiem państwa. Państwo w~teorii Smitha nie stanowi, jak to się
czasem próbuje przedstawiać, zła koniecznego, lecz pożądany efekt naturalnego procesu rozwoju ludzkich społeczności
(\cite{smith_lectures_1982} [1763], s.~207).
%\label{ref:RNDLLHqgagP0f}(Smith, 1982 [1763], s.~207).
Jego wartość etyczna dorównuje wcześniejszym formom organizacji
ludzkich społeczności, ale ze względu na swoją ostateczność państwo jest formą najbardziej wzniosłą
(\cite{smith_lectures_1982} [1763], s.~22).
%\label{ref:RNDKABNqSVWe4}(Smith, 1982 [1763], s.~22).
Wartości, na których oparty jest system Smitha, to przede
wszystkim sprawiedliwość, równość i~własność. Z~ich szczegółowym wytłumaczeniem spotkać można się wielokrotnie we
wszystkich dziełach tego autora. Równość stanowi w~filozofii polityki Smitha punkt wyjścia. Pisząc o~początkowej
równości wszystkich ludzi, ma on na myśli nie tylko równość w~sensie moralnym, ale i~podobieństwo przyrodzonych
talentów i~umiejętności: ,,Dwie osoby nie mogą być bardziej odmiennego geniuszu niż filozof i~tragarz, ale nie wydaje
się, aby istniała jakakolwiek naturalna różnica między nimi''
(\cite{smith_lectures_1982} [1763], s.~348).
%\label{ref:RNDlX5tlrl30w}(Smith, 1982 [1763], s.~348).
O~znaczeniu tego fragmentu dla samego Smitha może świadczyć umieszczenie go także na samym początku \textit{Bogactwa
narodów}
(\cite{smith_badania_2007} [1776], t.~1, s.~22).
%\label{ref:RNDeavcgj1oLe}(Smith, 2007 [1776], t. 1, s.~22).

Kluczowa dla zrozumienia teorii prawa Smitha, ekonomisty i~konstruktora nadchodzącego ładu społeczno-gospodarczego,
wydaje się analiza relacji zachodzącej jego zdaniem pomiędzy dwoma najważniejszymi
wartościami -- sprawiedliwością i~własnością. Co może zaskakiwać u~myśliciela do
tego stopnia powiązanego z~wolnorynkową narracją i~swobodą transakcji
ekonomicznych, Smith wyraźnie zaznacza pierwszeństwo sprawiedliwości przed własnością
(\cite{smith_lectures_1982} [1763], s.~5).
%\label{ref:RNDOpWvKBrvEq}(Smith, 1982 [1763], s.~5).
Oczywiście, jak sam autor podkreśla, ochrona przed bezprawnymi naruszeniami własności oraz
przywłaszczaniem sobie cudzych dóbr w~dużym stopniu wchodzi w~zakres najwyższej społecznej wartości, jaką jest
sprawiedliwość. Sprawiedliwość nie gwarantuje jednak absolutnej swobody w~dysponowaniu swoim mieniem, bądź odsuwaniu od
dostępu do niego wszystkich innych. Typowym dla nauk prawnych przykładem jest służebność drogi koniecznej, na którą sam
Smith bezpośrednio się powołuje
(\cite{smith_lectures_1982} [1763], s.~10).
%\label{ref:RNDA439GNDamV}(Smith, 1982 [1763], s.~10).
Osoba odcięta od drogi publicznej
cudzą działką może domagać się przewidywanej przez prawo służebności przekraczania terenu sąsiada. Nadużycie własności
stanowiące zagrożenie dla sprawiedliwości stanowi jeden z~tematów wielokrotnie poruszanych na przestrzeni całej
twórczości Smitha.

Została tu już wspomniana metafora, która najmocniej kojarzy się z~dorobkiem Smitha -- pojawiająca się jednorazowo
zarówno w~\textit{Teorii uczuć moralnych} jak i~\textit{Bogactwie narodów} ,,niewidzialna ręka''
(\cite{smith_teoria_1989} [1759], s.~272; \cite*{smith_badania_2007} [1776], t.~2, s.~40).
%\label{ref:RND3arQWlZpRs}(Smith, 1989 [1759], s.~272, 2007 [1776], t. 2, s.~40).
Gdyby doszukiwać
się w~dziele z~zakresu ekonomii politycznej metafory, która doczekała się choć zbliżonej popularności,
byłaby to bez wątpienia
,,fabryka szpilek'', ilustrująca rozważania na temat podziału pracy
(\cite{smith_badania_2007} [1776], t.~1, s. 10).
%\label{ref:RNDhErhk4QekH}(Smith, 2007 [1776], t.~1, s. 10).
Problem polega na tym, iż te dwa idiomy filozoficzne w~swoim współczesnym rozumieniu prowadzą do sprzecznych
wniosków. Fabryka szpilek służy do opisu malejących kosztów i~rosnących przychodów. Niewidzialna ręka dotyczy rosnących
kosztów i~malejących przychodów''
\parencite[s.~47]{warsh_wiedza_2012}.
%\label{ref:RNDXgp9RVVtIw}(Warsh, 2012, s.~47).
Pierwsza opisuje specjalizację,
która w~efekcie obniżania kosztów na jednostkę prowadzić musi do dominacji rynkowej, druga mechanizm gwarantujący optymalną
alokację zasobów w~ramach procesu rynkowego -- ,,doskonałą konkurencję''. Podczas gdy współcześnie dużo więcej uwagi
poświęca się podkreślaniu skutków tego drugiego procesu, statystyki wyraźnie wskazują, że swoboda ekonomiczna prowadzi
często do rynkowej dominacji
\parencite[s.~125–126]{keen_debunking_2011}.
%\label{ref:RNDf5LDvzVDLR}(Keen, 2011, s.~125–126).
Ta natomiast skutkuje monopolem, który
,,jest wielkim wrogiem dobrej gospodarki''
(\cite{smith_badania_2007} [1776], t. 1, s.~174).
%\label{ref:RNDQi3f6tgrVU}(Smith, 2007 [1776], t. 1, s.~174). 

 Adam Smith zdawał sobie sprawę (może nawet jako pierwszy) z~niedoskonałości konkurencji i~nieuniknionego dążenia
uczestników rynku do zdobycia pozycji monopolistycznej, a~co za tym idzie, do ,,sprzyjania ubóstwu, lub co sprowadza się
do tego samego, niedostępności rzeczy tak bardzo zmonopolizowanych''
(\cite{smith_lectures_1982} [1763], s.~83; \cite*{smith_badania_2007} [1776], t.~1, s.~292).
%\label{ref:RNDSBDgty6ieq}(Smith, 1982 [1763], s. 83, 2007 [1776], t. 1, s.~292).
W~samym tylko \textit{Bogactwie narodów} Smith wielokrotnie wskazuje, że braki w
ustawodawstwie zmuszają posiadających skromny majątek do konkurowania na nierównych zasadach z~już wzbogaconymi.
Zaniedbania takie muszą w~efekcie ,,stworzyć w~każdej gałęzi gospodarki monopol bogaczy''
(\cite{smith_badania_2007} [1776], t.~1, s.~111--112).
%\label{ref:RNDnMNpl04czP}(Smith, 2007 [1776], t. 1, s.~111-112).
Odpowiednie uregulowanie prawne kwestii konkurencji
niepoddającej się naciskom politycznym sił rynkowych stanowi więc dla Smitha jedno z~najważniejszych zadań państwa.

Wartość własności w~poglądach Smitha na temat relacji jednostki i~państwa nie jest tak nienaruszalna, jak mogłoby
się to wydawać. Interes ogółu niejednokrotnie sprawia, że rządzący powinni dokonać interwencji w~sferze gospodarczych
relacji między obywatelami. Jeśli jest coś, co powoduje w~tym zakresie niepokój Smitha, to jest to przeświadczenie, że
rząd jak dotąd zbyt często interweniował w~interesie majętnych, a~nie biednych
\label{ref:RNDKTNonQx9Ln}(Smith, 2007 [1776], t. 1, s.~167).
W~odniesieniu do wcześniejszych epok zdarza mu się nawet posuwać do stwierdzenia (zwykle
kojarzonego raczej ze zwieńczeniem klasycznej ekonomii politycznej niż jej początkami), że dotychczas prawo
niejednokrotnie wykorzystywane było jako narzędzie opresji biedoty
\label{ref:RNDK2BdJNopfX}(Smith, 2007 [1776], t. 1, s.~140).
Poniżej spróbuję wykazać, w~jaki sposób przekonanie Smitha o~możliwości ograniczania
prawa własności i~ingerencji państwa w~rynek wynika z~jego poglądów na rolę państwa i~prawa.

W tym miejscu należy wrócić do kwestii poruszonej już wcześniej -- wynikającej z~opisu historii
państwa w~\textit{Lectures on Jurisprudence} wtórności własności względem zorganizowanej ludzkiej społeczności
(\cite{smith_lectures_1982} [1763], s.~23).
%\label{ref:RNDH9LUKQOk30}(Smith, 1982 [1763], s.~23).
Jak zaznaczono, Smith stwierdza, że wszystkie ludzkie
zbiorowości rozwijają się w~ramach naturalnego procesu postępu. Powstanie współpracy ludzi w~ramach społecznej
organizacji następuje jeszcze przed wykształceniem się prawa własności. Za Tacytem powtarza on, że u~wszystkich ludów
uprawiających na wczesnym etapie rolniczym ziemię, da się dostrzec gospodarkę kolektywną opartą na podziale zbiorów
(\cite{smith_lectures_1982} [1763], s.~22).
%\label{ref:RNDYghmRxhXTi}(Smith, 1982 [1763], s.~22).
Ze względu na wynikający z~instytucji państwa charakter
własności, musi ona podlegać odgórnym regulacjom, których liczba zwiększa się z~kolejnymi etapami postępu
(\cite{smith_lectures_1982} [1763], s.~16).
%\label{ref:RND1yCVEykFX7}(Smith, 1982 [1763], s.~16).
To właśnie nowoczesne państwo stanowi w~teorii Smitha gwarancję
własności. Zostaje ono przeciwstawione organizacji politycznej epoki feudalnej, gdy ,,król i~jego szlachta
przywłaszczali sobie wszystko co tylko mogli''
(\cite{smith_lectures_1982} [1763], s.~23).
%\label{ref:RNDcpxaJZ2IE6}(Smith, 1982 [1763], s.~23).
Jak widać, Smith
odległy jest od wyznaczania wyraźnych i~nienaruszalnych granic własności, nie da się też przypisać mu narracji
opisującej antagonizm rozwiązań rynkowych i~państwowych.

Dbanie o~dobre zaopatrzenie rynków, prowadzące do niskiej ceny wystawionych na nim towarów
(która z~kolei w~ekonomicznym systemie Smitha stanowi źródło dobrobytu
(\cite{smith_lectures_1982} [1763], s.~360)),
%\label{ref:RND3lhiPntOYy}(Smith, 1982 [1763], s.~360))
jest
jednym z~głównych zadań publicznych rządu, wymienionych już na samym początku \textit{Lectures on Jurisprudence}
(\cite{smith_lectures_1982} [1763], s.~6).
%\label{ref:RNDcEz98pxQ1r}(Smith, 1982 [1763], s.~6).
Realizacja tego zadania powinna przebiegać na przykład poprzez
wydatki na budowę i~utrzymanie urządzeń publicznych, takich jak drogi, mosty i~porty
(\cite{smith_badania_2007} [1776], t.~2, s.~380).
%\label{ref:RNDun5237a8bv}(Smith, 2007 [1776], t. 2, s.~380).
W~rzeczywistości poglądy Smitha zbliżone są więc nie do radykalnego leseferyzmu, lecz do
współczesnej ekonomii instytucjonalnej, zgodnie z~którą ,,konkurencyjny rynek -- wcielenie prywatnych instytucji -- jest
sam w~sobie dobrem publicznym'' i~,,żaden rynek nie może długo istnieć bez wsparcia będących jego podstawą instytucji
publicznych''
\parencite[s.~20]{ostrom_dysponowanie_2013}.
%\label{ref:RNDjhM4fjDb9v}(Ostrom, 2013, s.~20).
Smith jest więc prekursorem Ronalda Coase'a, który na
początku XX wieku wskazywał na konieczność uwzględnienia w~analizie ekonomicznej skomplikowanego instytucjonalnego
podłoża rynków
(\cite[s.~8]{coase_firm_1990}; tłum. pol. \cite*{coase_firma_2013}).
%\label{ref:RNDGvBhji2UBe}(Coase, 1990, s.~8, tłum. pol. 2013).

Pomimo uznawania znaczenia własności prywatnej, Smith nie uważa jej jednak za najwyższą istniejącą w~społeczeństwie
wartość. Jej podporę stanowią pierwotne wartości społeczne -- sprawiedliwość i~równość. Własność musi niekiedy przed
nimi ustępować, ze względu na dobrobyt całej społeczności. Tę część rozważań należałoby podsumować cytatem z~końcowych
rozdziałów \textit{Bogactwa narodów}:
\myquote{
W państwie, gdzie nie ma regularnego wymiaru sprawiedliwości, gdzie ludność nie
ma zapewnionej ochrony mienia, a~prawo nie gwarantuje wykonania umów i~gdzie wreszcie władza państwowa nie czuwa nad
tym, by należności od osób, które są w~stanie je uiścić, egzekwowano regularnie, tam handel i~przemysł wkrótce
podupadnie
(\cite{smith_badania_2007} [1776], t.~2, s.~609).
%\label{ref:RNDNF5lVCSa0q}(Smith, 2007 [1776], t. 2, s.~609).
}

\section{Uczestnictwo w~bogactwie i~państwie}
Jak wielokrotnie podkreślano, system stworzony przez Smitha nie miał charakteru leseferystycznego,
jest bliski raczej poglądom myślicieli doby klasycznego liberalizmu, którzy zwracali uwagę na niemożność akceptacji
niektórych międzyludzkich uwarunkowań
\parencite[s.~215]{rawls_wyklady_2010}.
%\label{ref:RNDu40KwlEMET}(Rawls, 2010, s.~215).
Szkocki filozof swój system
filozoficzny oparł na przekonaniu o~istnieniu przysługujących wszystkim ludziom przyrodzonych uprawnień. W~tej końcowej
części pracy wykazane zostanie, że Smith przewidywał dla państwa nie tylko rolę ochrony tych uprawnień,
ale i~zapewniania materialnych warunków, w~których będą mogły one zostać zrealizowane.

Smith w~pełni zdawał sobie sprawę z~mankamentów swojej teorii wolności i~znał dobrze przewiny wcześniejszych
systemów politycznych. W~nowoczesnym państwie pokładał nadzieję na ochronę wartości kluczowych dla jego
systemu -- sprawiedliwości, równości i~własności. W~oparciu o~wcześniej przedstawione zależności, zostanie tu wykazane, jakie
konkretnie nadzieje względem nowoczesnego państwa żywił Smith i~z~jakimi znanymi mu zagrożeniami je konfrontował.

Dzieła pozostawione przez Smitha wielokrotnie przestrzegają przed negatywnymi konsekwencjami, zarówno o~charakterze
ekonomicznym, jak i~politycznym, wynikającymi z~zaszłościowych instytucji feudalizmu. W~ten sposób wyraża swoją
dezaprobatę wobec ciążących na państwie mankamentów dawnego ustroju: ,,Prawa często utrzymują się w~mocy, choć już dawno
zniknęły okoliczności, które przyczyniły się do tego, że prawa te powstały, i~które mogły być ich jedynym
uzasadnieniem''
(\cite{smith_badania_2007} [1776], t.~1, s.~439).
%\label{ref:RNDulI78tI5NW}(Smith, 2007 [1776], t. 1, s.~439).
Smith twierdzi, że w~ramach systemu
feudalnego wolność ludzka została ograniczona ,,według kaprysu ludzi, którzy zmarli, być może, pięćset lat temu''.
Twierdzenie, iż pofeudalnym instytucjom przysługuje jakakolwiek legitymizacja, nazywa ,,założeniem najbardziej
absurdalnym, jakie tylko jest możliwe''
(\cite{smith_badania_2007} [1776], t.~1, s.~440).
%\label{ref:RNDSUPttSQRMo}(Smith, 2007 [1776], t. 1, s.~440).
W~swoich
rozważaniach na temat najlepszego ustroju państwowego posuwa się do stwierdzenia, iż ,,szlachta stanowi największych
przeciwników i~ciemiężycieli ludności, jakich możemy sobie wyobrazić''
(\cite{smith_lectures_1982} [1763], s.~264).
%\label{ref:RND0qdGYLZow2}(Smith, 1982 [1763], s. 264).
Omawiając niegodne poczynania poprzednich królów sprowadzające się do kradzieży i~oszustw wskazuje, że swoboda
króla w~tworzeniu i~opłacaniu urzędów oraz zaciężna armia stanowią jedyne państwowe zagrożenia wolności
(\cite{smith_lectures_1982} [1763], s.~269).
%\label{ref:RNDcctWJa8WzG}(Smith, 1982 [1763], s.~269).
W~tym samym miejscu wyraża aprobatę dla Izby Gmin. Chwali ją nie
tylko jako symbol parlamentaryzmu i~przedstawicielstwa narodu, ale i~ze względu na jej postępujące oddzielenie od
władzy króla, które tworzy przeciwwagę chroniącą wolność obywateli.

Smith okazuje swoje przywiązanie do powstającej na jego oczach idei parlamentaryzmu, gdy skupia się na zagadnieniu
wyboru przedstawicieli. Opisując elekcję parlamentarzystów, wskazuje na konieczność częstych i~regularnych wyborów,
dających ludziom kontrolę nad rządem i~negatywnie odnosi się do faktu, że im dalej do wyborów, tym mniej rządzący
przejmują się głosem ludu
(\cite{smith_lectures_1982} [1763], s.~273).
%\label{ref:RNDGWDyLHaXiI}(Smith, 1982 [1763], s.~273).
Jak pisze wyraźnie: ,,wszelka władza
rządu pochodzi z~ludu'', co czyni go suwerenem, który powinien mieć uprawnienia do kontrolowania swoich przedstawicieli
i~odwoływania ich kiedy i~z~jakiego tylko powodu zechce
(\cite{smith_lectures_1982} [1763], s.~297).
%\label{ref:RND8CG5fqpymD}(Smith, 1982, s.~297).

Naturalnym następstwem uzyskiwania przez ludzi wolności w~ramach nowoczesnego państwa jest ustanowienie w~pełni
niezależnego od ośrodków władzy politycznej sądownictwa. Jak opisuje to Smith -- ,,Gdy jakikolwiek kraj umocni
się w~swojej wolności, a~prawa własności zostaną w~nim ustanowione, muszą zostać wkrótce wskazani sędziowie, aby rozstrzygać
te liczne spory, które muszą się odnośnie praw pojawić''
(\cite{smith_lectures_1982} [1763], s.~313).
%\label{ref:RNDkH7t6CSI0T}(Smith, 1982 [1763], s.~313).
Kluczowe
znaczenie ma dożywotnie pełnienie urzędu i~niezależność od władzy wykonawczej: ,,zabezpieczeniem wolności jest
dożywotnie zajmowanie przez sędziów ich stanowisk oraz całkowita niezależność od króla''
(\cite{smith_lectures_1982} [1763], s.~271).
%\label{ref:RNDQY5dE7difr}(Smith, 1982 [1763], s.~271).
Momentami Smith zrównuje wręcz skuteczność państwa i~wzrost
znaczenia jego roli z~tym, w~jakim stopniu jest ono w~stanie ingerować w~spory indywidualne i~doprowadzać do ich
rozwiązania
(\cite{smith_lectures_1982} [1763], s.~108).
%\label{ref:RNDK9nA5CA9CU}(Smith, 1982 [1763], s.~108).
Jednak i~tu da się dostrzec wyraz jego sceptycyzmu
odnośnie możliwości rozwikłania sporów poprzez samo stworzenie odpowiedniej siatki instytucjonalnej. Cóż z~tego, że
ustanowiono sądy, skoro ,,mimo to liczni nie mogą podołać wydatkom'' związanym z~procesami
(\cite{smith_lectures_1982} [1763], s.~272).
%\label{ref:RNDfgTZLS43gg}(Smith, 1982 [1763], s.~272).
Smith wielokrotnie podaje w~wątpliwość założenie, iż
pozostawienie ludziom swobody stanowi najwyższy możliwy do zapewnienia przez państwo poziom wolności.

Współczesne interpretacje prac Smitha pomijają zagadnienie o~wielkiej wadze, na które on sam wielokrotnie zwracał
uwagę. Ekonomia od początku była nauką o~ludzkich możliwościach i~wyborach. Jednak dominujący współcześnie nurt nie
bierze pod uwagę tych, którzy żadnego wyboru nie mają. Smith wyraźnie zaznacza, że swoboda umów pozostawiona sama sobie
może doprowadzić do ogromnych niesprawiedliwości: ,,człowiek, który nie może egzystować i~jest głodujący, zaakceptuje
jakiekolwiek wynagrodzenie, które pozwoli mu przetrwać, niezależnie od tego, jak skąpe''
(\cite{smith_lectures_1982} [1763], s.~382).
%\label{ref:RNDIiGLDa4lxS}(Smith, 1982 [1763], s.~382).
Świadomość tego zjawiska znajduje także wyraz w~jego poglądach na gospodarkę. Ponownie kojarzą
się one raczej z~końcem niż początkiem epoki klasycznej ekonomii politycznej. Smith krytykuje przyzwolone przez prawo
zmowy mające na celu obniżenie ceny pracy. Zauważa jednocześnie, że przepisy zabraniają tworzenia związków starających
się podnieść wynagrodzenia. Zwraca przy tym uwagę na przewagę dającą nieporównywalnie lepszą pozycję jednej ze stron
sporu: ,,przy wszystkich takich zatargach pracodawcy mogą wytrwać o~wiele dłużej [\mydots] nawet gdyby nie zatrudniali ani
jednego robotnika, mogliby na ogół przeżyć rok lub dwa z~kapitału, który już zdobyli. Wielu zaś robotników nie
potrafiłoby przetrwać bez pracy jednego tygodnia, niewielu przetrwałoby miesiąc, a~chyba żaden nie przetrwałby roku''
(\cite{smith_badania_2007} [1776], t.~1, s.~79).
%\label{ref:RND1j5HGSGWhu}(Smith, 2007 [1776], t. 1, s.~79).

Z wielu wymienionych powyżej powodów nie powinna dziwić podkreślana wyraźnie przez Smitha zależność, że ,,im większa
jest wolność wolnych, tym bardziej nie do zniesienia staje się niewola niewolników''
(\cite{smith_lectures_1982} [1763], s.~185).
%\label{ref:RNDtOthMxkQwu}(Smith, 1982 [1763], s.~185).
Zaznaczyć należy, iż łączy on rozważania na temat niewolników z~tymi dotyczącymi wszystkich
najbardziej opresjonowanych grup społecznych -- żyjących na pofeudalnych majątkach rolników i~ludów ziem odkrytych przez
Europejczyków
(\cite{smith_badania_2007} [1776], t.~1, s.~442--444).
%\label{ref:RND3sDYK8lMK4}(Smith, 2007 [1776], t. 1, s.~442-444).
Alienacja wynikająca z~podziału pracy
stanowi jego zdaniem ogromne zagrożenie dla ludzkiej indywidualności. Zmusza ona pracowników do prymitywnego trybu
życia
(\cite{smith_badania_2007} [1776], t.~2, s.~446--447).
%\label{ref:RNDARpT9oVdt9}(Smith, 2007 [1776], t. 2, s.~446-447).
Stan podległości Smith uważa za dewastujący dla
kondycji ludzkiej. Jak pisze: ,,Nic nie wpływa tak mocno na zepsucie i~osłabienie umysłu, jak zależność i~nic nie daje
wrażeń tak wzniosłych i~obfitych, jak wolność i~niezależność''
(\cite{smith_lectures_1982} [1763], s.~333).
%\label{ref:RNDzgsq8IO3HU}(Smith, 1982 [1763], s.~333). 

Smith dostrzega ważną rolę państwa w~korygowaniu niesprawiedliwych efektów mechanizmów rynkowych i~z~dezaprobatą
wypowiada się o~prawie, które jego zdaniem w~sporach między pracą a~kapitałem niesprawiedliwie opowiada się po stronie
tego drugiego
(\cite{smith_badania_2007} [1776], t.~1, s.~167).
%\label{ref:RNDISiSQpgteG}(Smith, 2007 [1776], t. 1, s.~167).
Analizując uwagi poczynione przez Smitha na
temat stanowiących kluczowy element jego teorii przysługujących jednostkom uprawnień, łatwo dojść do wniosku, że ochrona
ich nie jest jedynym zadaniem państwa. Powinno ono także dokładać starań, aby uprawnienia te były w~rzeczywistości
wykorzystywane, gwarantując materialne warunki ich realizacji.

\section*{Podsumowanie}
Przez cały dorobek intelektualny Smitha przebija przekonanie o~przyrodzonej godności człowieka, skutkujące teorią
uprawnień i~pozytywnie rozumianej wolności. Nie tylko zauważał on liczne ułomności mechanizmu rynkowego,
ale i~wielokrotnie wprost podkreślał jego nieetyczny charakter. Smith był w~pełni świadomy nieuniknionych konfliktów pomiędzy
pracą a~kapitałem i~dostrzegał rolę państwa w~równoważeniu sił negocjacyjnych tych dwóch grup społecznych. Co chyba
najważniejsze, Smith łączył dwa elementy składowe swoich poglądów na temat pozycji jednostki w~państwie -- teorię
uprawnień i~pozytywnie rozumianą wolność -- w~stanowisko, które należałoby uznać za teorię uprawnień efektywnych. Władza
powinna gwarantować obywatelom nie tylko szeroki zakres uprawnień politycznych i~ekonomicznych, ale i~dbać o~materialne
podstawy będące koniecznym warunkiem czerpania korzyści z~uczestniczenia w~państwie. Jest to zarazem konieczne
następstwo przyjmowanych przez Smitha założeń etycznych, jak i~sposób na osiągnięcie w~gospodarce najwyższego możliwego
dobrobytu.

Podsumowując wywód na temat zagadnień prawnych w~systemie Smitha można dojść do następujących wniosków.
Pozostawione przez tego szkockiego filozofa dzieła uznać można za stanowiące w~całości jeden z~kamieni węgielnych oświeceniowego
liberalizmu (także w~politycznym sensie). Adam Smith, podobnie do współczesnych mu myślicieli, których obdarzał
największym szacunkiem, opowiadał się za modelem społeczeństwa dającym jednostce możliwie najszerszą autonomię. Tak jak
Hume, któremu oddawał tytuł najwybitniejszego filozofa swoich czasów, Smith piętnował wypaczenia religii prowadzące do
przemocy i~prześladowań
(\cite{smith_badania_2007} [1776], t.~1, s.~401).
%\label{ref:RNDWdGETBWW27}(Smith, 2007 [1776], t. 1, s.~401).
Za Voltaire'em, którego postawie
nieraz oddawał szacunek, Smith potępiał nietolerancję jako nieznajdującą żadnych sprawiedliwych uzasadnień
(\cite{smith_teoria_1989} [1759], s.~183).
%\label{ref:RNDcOFI1QT8Tr}(Smith, 1989 [1759], s.~183).
Analiza poglądów Smitha rozproszonych w~jego dziełach pozwala
uznać postulowany przez niego rząd cywilny za protoplastę współczesnej demokracji liberalnej -- systemu politycznego
kierowanego przez większość, który za jedno z~pierwszych zadań stawia sobie ochronę mniejszości. Rząd cywilny w~ujęciu
Smitha jest oparty na ideach parlamentaryzmu, przedstawicielstwa i~kadencyjności, które dają mocną podbudowę ciału
ustawodawczemu. Powyższe zabezpieczenia mają łagodzić zakusy zbyt silnej egzekutywy na wolności jednostek. Rząd cywilny
szanuje prawa własności, ale i~wie, kiedy muszą one ustąpić na rzecz sprawiedliwości i~równości stanowiących pierwsze
wartości zorganizowanego społeczeństwa. Co kluczowe, ponieważ w~tym aspekcie wizje z~XVIII wieku wyprzedziły
postępującą przez następne dwieście lat realizację liberalnych idei, Smith w~pełni zdawał sobie sprawę, że bez
materialnego wsparcia zapewniającego obywatelom godne uczestniczenie w~społeczeństwie, nawet najszerszy wachlarz
uprawnień pozostanie nieefektywny.

\end{artplenv}\label{lub-stop}
