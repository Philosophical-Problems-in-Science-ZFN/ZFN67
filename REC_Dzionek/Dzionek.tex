\begin{recplenv}{Joanna Dzionek-Kozłowska}
	{Metarefleksje o~makroekonomii i~polityce makroekonomicznej}
	{Metarefleksje o~makroekonomii i~polityce makroekonomicznej}
	{Tomasz Kwarciński, Agnieszka Wincewicz-Price (red.), \textit{Metaekonomia II. Zagadnienia z~filozofii makroekonomii},
		Kraków, Copernicus Center Press 2019.}





\textls[-10]{Zagadnienia metaekonomiczne} to kwestie tak fundamentalne dla nauk ekonomicznych, że częstokroć jawią się one osobom
parającym się uprawianiem ekonomii jako zbyt oczywiste, przez co niewarte szczególnej uwagi, albo przeciwnie -- na tyle
trudno uchwytne, że w~oczach wielu podważa to zasadność wysiłku włożonego w~pogłębioną nad nimi refleksję. Nieliczni
teoretycy zastanawiają się, czym w~istocie są konstruowane przez nich modele makroekonomiczne. Jaki jest lub powinien
być związek modeli z~realnym życiem gospodarczym? Czy makroekonomia -- tworzone na jej gruncie
modele i~koncepcje -- \mbox{mogą/będą} kiedykolwiek
mogły dostarczyć nam jednoznacznych wskazówek pod adresem polityki gospodarczej? Nieco częściej,
zwłaszcza w~okresach pogorszenia koniunktury, daje się słyszeć pytanie o~to, czy w~ogóle zasadne jest pokładanie wiary
w~teoriach makroekonomicznych jako środku ochrony przed recesją, bezrobociem, inflacją, obniżeniem dobrobytu i~innymi
bolączkami o~charakterze ekonomicznym. Ujmując rzecz z~jeszcze innej perspektywy, można również podnieść kwestię,
czy i~w~jaki sposób pojęcia wykorzystywane w~dyskursie ekonomicznym, stosowane przez ekonomistów narzędzia analityczne,
różnorakie miary i~indeksy -- tworzone, by lepiej poznać badane zjawiska -- kształtują nasz ogląd tychże zjawisk i~czy
mogą wywrzeć wpływ na nie same. Idąc dalej można w~końcu zapytać: czy kształt tych indeksów i~miar w~jakikolwiek sposób
przekłada się na decyzje z~zakresu polityki gospodarczej? 

Autorzy tekstów zebranych w~drugim tomie \textit{Metaekonomii}, wydanego właśnie nakładem krakowskiego Copernicus Center
Press, są dalecy od uznania tak postawionych pytań za łatwe czy trywialne. Co więcej, lektura kolejnych rozdziałów
przekonuje, że namysł nad zagadnieniami z~zakresu filozofii ekonomii powinien być istotny nie tylko dla samych
ekonomistów, ale i~decydentów kształtujących politykę publiczną, a~może nawet nas wszystkich. Nie-ekonomiści są
przecież bezpośrednim, choć nie jedynym adresatem nienapawającego zbytnim optymizmem tekstu Daniela Hausmana, który
odpowiadając na jedno z~przywołanych wyżej pytań konkluduje, że ,,[e]konomia może dostarczyć amunicji dla wsparcia lub
podważenia działań, które popiera część stron biorących udział w~podejmowaniu decyzji, ale nie powinniśmy oczekiwać od
ekonomistów szczegółowych rozwiązań problemów dotyczących kształtowania polityki gospodarczej''.

\enlargethispage{-.5\baselineskip}

Prezentując oddaną do rąk czytelników monografię nie sposób pominąć faktu, iż stanowi ona kontynuację i~poszerzenie
rozważań podjętych w~bardzo dobrze przyjętym pierwszym tomie \textit{Metaekonomii}
%\label{ref:RNDtabn4twYXo}(Gorazda, i~in., 2016).
\parencite{gorazda_metaekonomia:_2016}\footnote{Za szczególnego rodzaju formalny dowód uznania można uznać fakt, iż pierwszy tom zdobył
	I~nagrodę w~konkursie Polskiego Towarzystwa Ekonomicznego
	na najlepszy podręcznik akademicki z~ekonomii wydany w~latach 2014--2016.}.
Do powracających wątków należą m.in. refleksje na temat przyczynowości w~ekonomii, o~której piszą Mariusz
Maziarz i~Robert Mróz, dociekania poświęcone dobrobytowi (konceptualizacji i~pomiarowi tej kategorii)
pogłębiane przez Tomasza
Kwarcińskiego, problematyka modeli ekonomicznych, obecna tym razem w~rozdziałach autorstwa Emilii Tomczyk (modelowanie
oczekiwań) i~Franciszka Chwałczyka (potraktowanie jako modeli stosowanych w~ekonomii miar), a~także, choć w~mniejszym
wymiarze, we wspomnianym już tekście Daniela Hausmana. Jednakże tematem przewodnim, stanowiącym swoiste ,,spoiwo''
szesnastu esejów składających się na tom drugi, są zaanonsowane w~tytule kwestie makroekonomiczne\footnote{Choć niektóre
	wywody -- na przykład rozważania Bartosza Scheuera (rozdz. 3 części pierwszej) czy Franciszka Chwałczyka (rozdz. 5 części
drugiej) -- z~powodzeniem można odnieść do całej ekonomii.}. Specyfika podejścia makroekonomicznego, dla którego
powstania silnym bodźcem było dążenie do dostarczenia wskazówek dla polityki gospodarczej, powoduje zaś, że istotnym
\textit{novum} w~stosunku do tomu pierwszego jest tu zdecydowanie szersza refleksja na temat relacji pomiędzy teorią
ekonomii (głównie makroekonomii) a~polityką gospodarczą. W~sposób bezpośredni zagadnienia te analizowane
są w~dedykowanej im części trzeciej, lecz w~mniejszym bądź większym wymiarze są one obecne we wszystkich rozdziałach. 

Część pierwszą, traktującą o~rozwoju makroekonomii, otwiera esej Wojciecha Gizy, który szukając źródeł makroekonomii,
wskazuje na szereg idei makroekonomicznych stworzonych niekiedy na długo przed powstaniem podejścia makroekonomicznego.
To ostatnie jest zwykle wiązane z~\textit{Ogólną teorią zatrudnienia, procentu i~pieniądza}
(\cite{keynes_ogolna_2003} [1936])
%\label{ref:RND4QEz0dw5rq}(Keynes, i~in., 2003 [I wyd. 1936])
oraz postacią autora tego wpływowego dzieła, Johna
Maynarda Keynesa. Warto może nadmienić, że ta, jak mogłoby się wydawać, dawno rozstrzygnięta kwestia ,,ojcostwa''
makroekonomii okazuje się nie być aż tak jednoznaczna. Poza Keynesem, w~tej roli względnie często obsadzany jest również
Michał Kalecki (traktowany w~ten sposób m.in. przez przybliżającego jego koncepcje w~rozdziale drugim Jerzego
Osiatyńskiego). Interesujące jest, że Autorzy \textit{Metaekonomii} zwracają także uwagę na Knuta Wicksella (Jacek
Wallusch w~rozdziale piątym) oraz -- co jest rzadkością -- przedstawianego zwykle jako jednego z~twórców ekonomii
behawioralnej George'a Katonę (Agnieszka Wincewicz-Price i~Paweł Śliwowski za
\parencite{colander_complexity_2014}).
%\label{ref:RNDFfW8PtQiqA}(Colander, Kupers, 2014)).
Rzecz jasna, waga tej kwestii jest niewspółmiernie mała w~porównaniu do, na przykład, poszukiwania
odpowiedzi na pytania o~problemy przekładania się teorii makroekonomicznej na politykę gospodarczą. Jednak zwrócenie
uwagi na tę wielość udzielanych przez Autorów \textit{Metaekonomii} odpowiedzi stanowi w~istocie egzemplifikację cechy,
która jest obecna w~szeregu innych przypadków. Mam tu mianowicie na myśli to, że w~omawianej monografii często mamy
możliwość odnalezienia różnych, niekiedy krańcowo odmiennych, stanowisk na temat poszczególnych analizowanych na jej
kartach problemów. Od razu zaznaczę, że nie podnoszę tej kwestii uznając ją za wadę. Przeciwnie, uważam, że możliwość
zapoznania się w~jednym tomie z~argumentacją rozwijaną przez różne strony toczonych w~filozofii, metodologii i~historii
ekonomii debat nie tylko daje czytelnikom lepszy wgląd w~stan tych dyskusji, ale i~szanse na wyrobienie sobie własnego
zdania.

Przykładem może tu być zestawienie argumentacji Bartosza Scheuera, który wychodząc od słusznego skądinąd wykazania
słabości wpisywania rozwoju makroekonomii w~schematy wypracowane na gruncie filozofii nauki, proponuje dość
radykalne, oparte na dyskursywnej koncepcji rozwoju wiedzy, stanowisko odnośnie do wyjaśniania ewolucji teorii ekonomii
z~wywodem Jacka Walluscha, który z~kolei zwraca uwagę na związek pomiędzy rozwojem makroekonomii a~procesami
zachodzącymi w~realnym życiu gospodarczym (zmiany dynamiki cen zbieżne ze zmianami popularności dwu głównych tradycji
makroekonomicznych -- keynesowskiej i~neoklasycznej). Jeśli, jak proponuje Scheuer, ,,wszelkie składowe konstrukcji
naukowych mają charakter językowy i~w~tym sensie nie odnoszą się do niczego innego, jak tylko do innych
elementów o~takim samym charakterze'', a~tworzenie wiedzy naukowej jest w~istocie ,,dyskursywnym tworzeniem i~rozwijaniem faktów
naukowych'', to przyjmując taką perspektywę trudno jest mówić o~zmianach, które -- jak się wydaje -- nastąpiły w~rozwoju
ekonomii w~konsekwencji wydarzeń mających miejsce w~gospodarce. Przywołanym już przykładem może być samo zaproponowanie
przez Keynesa podejścia makroekonomicznego w~reakcji na wielki kryzys (niezwykle popularna opinia,
powtórzona w~omawianym zbiorze przez m.in. Osiatyńskiego), wystąpienie stagflacji jako czynnika
ułatwiającego zaakceptowanie rewizji
krzywej Phillipsa dokonanej przez Friedmana i~Phelpsa, kryzys lat siedemdziesiątych XX wieku jako podłoże tzw.
,,kontrrewolucji neoklasycznej'' Lucasa etc. Równie trudno byłoby także wyjaśnić wskazaną wyżej zależność dostrzeżoną
przez Walluscha.

\enlargethispage{.5\baselineskip}

Z podobną sytuacją, tzn. możliwością zaznajomienia się z~argumentacją na rzecz odmiennych stanowisk, mamy też do
czynienia w~przypadku kwestii oceny roli makroekonomii (faktycznej i~potencjalnej) w~rozwiązywaniu problemów z~zakresu
polityki publicznej. Niezwykle krytyczne oceny formułuje w~tym zakresie Hausman, w~którego stanowisku rezonują
argumenty Johna Stuarta Milla i~Alfreda Marshalla (mimo że na tego ostatniego się nie powołuje). Otóż Hausman stwierdza
m.in., że ,,[a]ni teoria, ani badania empiryczne w~ekonomii nie są w~stanie ostatecznie dowieść ani zaprzeczyć
zasadniczym twierdzeniom dotyczącym ogólnego funkcjonowania gospodarki''. Znacznie większy optymizm wykazuje natomiast
Stanisław Mazur omawiając w~swoim rozdziale tzw. politykę publiczną opartą na dowodach (\textit{Evidence-Based Public
Policy}). Stanowisko tego Autora jest tym bardziej ciekawe, że przywołując piętnastopunktowy katalog poważnych wyzwań
stojących przed stosowaniem \textit{Evidence-Based Public Policy} (jednym z~nich jest m.in. potencjalne wykorzystywanie
wsparcia ekspertów jako fasady dla decyzji dyktowanych ideologią lub interesem decydentów), daje świadectwo świadomości
tych zagrożeń. Nie wskazuje jednak, w~jaki sposób można by ich uniknąć.

Wartą podniesienia kwestią, analizowaną w~rozdziałach autorstwa Kwarcińskiego i~Chwałczyka, jest znaczenie stosowanych
przez ekonomistów narzędzi analitycznych -- określonych miar i~indeksów -- dla konstruowania teorii makroekonomicznych.
O ile ciekawe i~czerpiące spoza nauk ekonomicznych rozważania drugiego z~nich mają jednak charakter dość ogólny (co
pozwala odnieść je do ekonomii \textit{sensu largo}), o~tyle w~artykule Kwarcińskiego mamy konkretyzację tych wywodów
opartą na analizie różnorakich mierników dobrobytu. 

Zaletą monografii jest również uwzględnienie podejść wykraczających poza tradycję keynesowską i~neoklasyczną. Robert
Mróz przybliża makroekonomię post-keynesowską i~podejście do zagadnień makroekonomicznych właściwe reprezentantom
szkoły austriackiej (zważywszy na anty-makroekonomiczną orientację szkoły austriackiej, celowo nie używam tu określenia
,,makroekonomia austriacka''). Na uwagę zasługuje również część traktująca o~interesującym i~-- jak się wydaje -- mało
znanym stanowisku George'a L.S. Shackle'a, które, zdaniem Mroza, można uznać za szczególnego rodzaju próbę pogodzenia
elementów ekonomii post-\mbox{-keynesowskiej} i~austriackiej.

Michał Możdżeń pochyla się nad podejściem do polityki makroekonomicznej, bazującym na teorii wyboru publicznego, zaś
Agnieszka Wincewicz-Price i~Paweł Śliwowski przedstawiają wpływ, jaki na makroekonomię wywarła ekonomia behawioralna.
Czytelnik ma również możliwość zapoznania się z~przedstawionym przez Autorów przypadkiem konkretnej aplikacji koncepcji
stworzonych przez reprezentantów tego ostatniego nurtu (wsparcia programów prywatnych oszczędności). Z~kolei Marcin
Gorazda analizuje system fiskalny, koncentrując się na sprawiedliwości podatkowej. Idąc za Amartyą Senem, słusznie
zwraca uwagę, że niemożliwe jest skonstruowanie systemu fiskalnego, który byłby w~całości oparty na jednej teorii
sprawiedliwości. Z~drugiej strony Gorazda podkreśla też, że niemożliwe jest stworzenie takiej teorii sprawiedliwości,
która obejmowałaby cały, nieustanie ewoluujący system instytucji i~ocen odnoszących się do tej kategorii. Stąd postulat
poświęcania baczniejszej uwagi na analizę konkretnych sytuacji, do czego Autor namawia również ekonomistów.

Kończąc omówienie drugiego tomu \textit{Metaekonomii} pozwolę sobie zwrócić uwagę na pewną kwestię, która może wzbudzić
u~czytelników (zwłaszcza tych zaznajomionych z~rozwojem dyskursu makroekonomicznego) pewien niepokój. Rzecz dotyczy
jednej z~najbardziej burzliwych, nadal żywych, a~zarazem znaczących dla rozwoju makroekonomii dyskusji, której
przedmiotem były tzw. mikropodstawy makroekonomii. Wnosząc na podstawie tytułów poszczególnych tekstów, problematyce
tej poświęcony jest rozdział drugi -- \textit{Spory o~mikroekonomiczne podstawy makroekonomii po Kaleckim i~Keynesie}.
Wydaje się jednak, że tytuł ten jest nieco mylący. Nie twierdzę tym samym, że wskazany artykuł całkowicie pomija to
zagadnienie, bo pewne jego fragmenty faktycznie dotyczą wybranych etapów sporu o~mikropodstawy. Niemiej rozdział ten
jest raczej przedstawionym z~perspektywy Kaleckiańskiej wnikliwym omówieniem rozwoju makroekonomii w~drugiej połowie XX
stulecia niż zdaniem relacji z~dyskusji toczącej się wokół mikropodstaw makroekonomii. Kwestia ta jest natomiast obecna
zarówno w~rozdziałach autorstwa Jakuba Janusa i~Krystiana Muchy, Jacka Walluscha, Petera\textit{ }Galbácsa, Roberta
Mroza, Mariusza Maziarza i~Roberta Mroza, jak i~Tomasza Kwarcińskiego (co zresztą można potraktować jako argument
świadczący o~jej znaczeniu dla ewolucji makroekonomii).

Nie mam jednak najmniejszych wątpliwości, że najnowszy owoc kooperacji grona autorów skupionych wokół Polskiej Sieci
Filozofii Ekonomii odegra rolę nie mniej inspirującą do refleksji na temat kondycji współczesnej ekonomii, co pierwsza
odsłona tej współpracy. 







\autorrec{\textls[-30]{Joanna Dzionek-Kozłowska}}


\subsubsection{Bibliografia}\nopagebreak[4]
\end{recplenv}
