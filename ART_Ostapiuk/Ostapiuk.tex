\begin{artplenv}{Aleksander Ostapiuk}
	{Droga ekonomii wolnej od wartościowania do epistemologicznej pychy. Użycie i~nadużycie matematyki przez
		ekonomistów}
	{Droga ekonomii wolnej od wartościowania\ldots}
	{Droga ekonomii wolnej od wartościowania do epistemologicznej pychy. Użycie i~nadużycie matematyki przez
		ekonomistów}
	{Uniwersytet Ekonomiczny we Wrocławiu\label{ost-start}}
	{Value-free economics' road towards epistemological hubris. The use and abuse of mathematics by economists}
	{The goal of the article is to substantiate that despite the criticism the paradigm in economics will not change because of the axiomatic assumptions of value-free economics. How these assumptions work is demonstrated on the example of Becker's economic approach which is analyzed from the perspective of scientific research programme (Lakatos). The author indicates hard cores of economic approach (maximization of utility, instrumental rationality) and the protective belt which makes hard cores immune from any criticism. This immunity leads economists to believe that they are objective scientists and, consequently, it results in epistemological hubris. Due to its tautological nature (and other problems), economic approach is considered to be a~degenerative programme. This conclusion is extended on value-free economics. In spite of these problems, many economists still believe in positive economics and they dismiss normative approaches. It has a~negative influence on people (well-being, choices over time). The conclusion of the article is that thanks to axiomatic assumptions economists do not have objective and ironclad methodology and they should accept normative values in their research.}
	{economic approach, scientific research programme, philosophy of economics, value-free economics.}



\section*{Wstęp}
\lettrine[loversize=0.13,lines=2,lraise=-0.05,nindent=0em,findent=0.2pt]%
{O}{}d pewnego czasu interdyscyplinarność jest bardzo popularnym terminem w~ekonomii (jak i~we wszystkich naukach).
Tradycyjna ekonomia otworzyła się na inne nauki, m.in. psychologię (ekonomia behawioralna), socjologię (nowa ekonomia
instytucjonalna) i~filozofię (filozofia ekonomii). Ekonomia neoklasyczna czerpie z~tych dziedzin, a~proces ten nazwano
,,odwrotnym imperializmem''. Celem artykułu jest udowodnienie, że pomimo otwarcia się ekonomii na inne nauki, nie
doprowadziło i~może nie doprowadzić to do zmiany paradygmatu
\parencite{kuhn_structure_1962}
%\label{ref:RNDiyyPXdcNoM}(Kuhn, 1962)
w~ekonomii. Główne
założenia ekonomii neoklasycznej -- ekonomii wolnej od wartościowania, nie zmieniają się pomimo krytyki ze strony innych
nauk, a~tradycyjna ekonomia w~znacznej mierze działa tak jak wcześniej (\textit{business as usual}). Aktualność tego
problemu jest szczególnie widoczna na przykładzie ekonomii behawioralnej, która w~ostatnich latach zamiast przeobrazić
ekonomię stała się w~pewnym stopniu ekonomią neoklasyczną w~przebraniu psychologii
\parencite{berg_as-if_2010}.
%\label{ref:RNDEwhJ2zP5na}(Berg, Gigerenzer, 2010).

Paradygmat ekonomii się nie zmienia pomimo krytyki i~otwarcia się na nową wiedzę z~powodu tego, w~jaki sposób zbudowane
są założenia ekonomii wolnej od wartościowania. Jedną z~głównych przyczyn utrzymywania się \textit{status quo }jest
myślenie matematyczne wykorzystywane przez ekonomistów. Celem niniejszego artykułu nie jest ponowna ocena procesu
matematyzacji ekonomii, która miała swój szczyt w~latach osiemdziesiątych XX wieku i~została dobrze
opisana w~literaturze przedmiotu
\parencite{beed_what_1991,debreu_mathematization_1991,weintraub_how_2002,mirowski_machine_2002,ostapiuk_matematyzacja_2017}.
%\label{ref:RND8nLTfYOoX6}(Beed, Kane, 1991; Debreu, 1991; Weintraub, 2002; Mirowski, 2002; Ostapiuk, 2017a).
Nie chodzi również o~rozstrzyganie wyższości pomiędzy metodą przyrodniczą
a~socjologiczno-historyczną, z~której nieudaną próbą mieliśmy do czynienia pod koniec XIX wieku podczas
\textit{Methodenstreit} (niem. spór o~metodę). Nie chodzi również o~odpowiedź na pytanie, czy ekonomia powinna być
bliższa rzeczywistości, czy budować uproszczone modele o~dużej sile predykcyjnej\footnote{Bardzo trafną analogię
przedstawił
\parencite[s.~43–44]{rodrik_economics_2015}
%\label{ref:RNDglkG4gXtAZ}(Rodrik, 2015, s.~43–44)
wykorzystując opowiadanie Borgesa \textit{On the
Exactitude of Science} o~społeczeństwie, które budowało mapy. Z~czasem chcieli oni coraz dokładniej opisywać
rzeczywistość (budować idealne mapy) i~w~ostateczności zaczęli tworzyć mapy w~skali 1:1. Rodrik poprzez tę analogię
chce pokazać, że modele zawsze muszą być uproszczone, żeby być użytecznymi. Dlatego zarzuty o~tym, że modele
ekonomiczne nie przedstawiają człowieka w~pełni, są w~znacznej mierze nieuzasadnione.}.

Podejście matematyczne jest w~tym artykule pojmowane w~szerokiej perspektywie epistemologicznej. Oznacza to, że ekonomia
jest zbudowana na pewnych aksjomatycznych założeniach, z~których na podstawie dedukcji dochodzi się do wiedzy. Celem
artykułu jest ukazanie, że te aksjomatyczne założenia wraz z~użytą metodą są przyczynami tego, że ekonomia neoklasyczna
jest w~stanie wchłonąć dowolną krytykę, dzięki czemu paradygmat może pozostawać bez zmian. Wypada w~tym miejscu
uściślić, co dokładnie oznacza ekonomia neoklasyczna, ponieważ jest to pojemny termin, który ma wiele znaczeń
\parencite{colander_death_2000}.
%\label{ref:RNDXvTQFkLHg2}(Colander, 2000).
W~artykule terminy ekonomia neoklasyczna i~ekonomia wolna od wartościowania
są używane zamiennie, ponieważ oba podejścia są w~istotnej mierze oparte na teorii preferencji ujawnionych, w~której
ekonomiści nie są zainteresowani ludzkimi motywacjami i~celami. Zakładają natomiast, że ludzie wybierają rzeczy, które
są dla nich najlepsze, a~racjonalność jest zdefiniowana przez techniczne kryteria, takie jak ciągłość i~kompletność
preferencji. Oczywiście autor zdaje sobie sprawę z~tego, że wyżej wymieniona definicja jest ogólna. Jednak celem
artykułu nie jest klasyfikacja obu podejść, które z~powodu swojej wieloaspektowości nie mają jednoznacznej
definicji w~literaturze. Nie ulega wątpliwości, że ekonomia neoklasyczna jest obszerniejszym pojęciem niż ekonomia wolna od
wartościowania. Jednakże wolność od wartościowania jest podstawową cechą ekonomii neoklasycznej
\parencite{sen_rational_1977,putnam_end_2011,hausman_etyka_2017_ost}
%\label{ref:RNDbADWLKYLjj}(Sen, 1977; Putnam, Walsh, 2011; Hausman, i~in., 2017)
i~ta cecha jest
najważniejsza z~perspektywy artykułu. 

Pierwsza część artykułu traktuje o~teoretycznych podstawach ekonomii wolnej od wartościowania i~powinna ułatwić
czytelnikowi zrozumienie, dlaczego ekonomia wolna od wartościowania jest w~artykule utożsamiana z~ekonomią
neoklasyczną. Główny akcent został położony na analizę wpływu pozytywizmu na myślenie ekonomistów, który doprowadził do
wiary w~obiektywność ekonomii i~odrzucenia normatywnych komponentów teorii. Ta krótka analiza ma za zadanie wykazać
główne aksjologiczne założenia ekonomii wolnej od wartościowania. Dzięki temu można będzie przejść do głównego zadania
artykułu, jakim jest analiza tych założeń na przykładzie podejścia ekonomicznego Beckera, które jest
postrzegane jako ucieleśnienie ekonomii wolnej od wartościowania. Analiza ta będzie
przeprowadzona z~perspektywy naukowych programów badawczych
\parencite{lakatos_methodology_1980,lakatos_pisma_1995}.
%\label{ref:RNDKLXPCWGQoP}(Lakatos, 1980, 1995).
Dzięki tej analizie możliwe
będzie zbadanie, w~jaki sposób ekonomiści przy użyciu swoich aksjomatycznych założeń byli i~są w~stanie odpowiedzieć na
każdą krytykę, pozostawiając paradygmat bez zmian. Zostanie również zaprezentowane, do jakich negatywnych konsekwencji
doprowadza wykorzystanie założeń ekonomii wolnej od wartościowania w~realnym świecie, gdzie człowiek jest pojmowany
jako \textit{homo oeconomicus}.

\section{Filozoficzne korzenie ekonomii wolnej od wartościowania}
Logiczny pozytywizm jest być może najważniejszym czynnikiem, który doprowadził do uformowania się ekonomii wolnej od
wartościowania, a~wpływ tej szkoły filozoficznej nadal jest widoczny w~ekonomii. Niewątpliwie logiczny pozytywizm miał
bardzo duży wpływ na L. Robbinsa, którego można uznać za jednego ze stwórców ekonomii wolnej od wartościowania. Jego
najważniejszy esej
\parencite{robbins_essay_1935},
%\label{ref:RNDTZp6C8otZZ}(Robbins, 1932),
który nadał kształt ekonomii neoklasycznej, obejmował trzy
główne założenia:

\begin{enumerate}
\item Przedmiotem ekonomii jest napięcie pomiędzy rzadkością a~potrzebami (problem środki/cele).
\item Ekonomia jest oparta na aksjomatach (abstrakcjach), które pochodzą z~doświadczenia i~które prowadzą do twierdzeń
na temat rzeczywistości (dlatego ,,naukowa'' natura przedmiotu). 
\item Ekonomia nie jest skupiona na celach, ale na środkach potrzebnych do osiągnięcia tych celów. Dlatego jest wolna od
wartościowania
\parencite[s.~58]{witztum_ethics_2007}.
%\label{ref:RNDpT5zjroqZD}(Witztum, 2007, s.~58).
\end{enumerate}

Założenia te były niezbędne, by osiągnąć trzy cele: 1. Zdefiniować ekonomię jako badanie racjonalnego wyboru
ograniczonego rzadkością. 2. Postawić ekonomię na solidniejszej epistemologicznej podstawie (odejście od
hedonizmu) 3. Przedstawić argument przeciwko interpersonalnemu porównaniu użyteczności
\parencite{hands_effective_2007}.
%\label{ref:RNDFXzNxKz5vU}(Hands, 2007).
Gdyby te trzy cele zostały osiągnięte, wtedy ekonomia marginalistyczna osiągnęłaby wyłączne prawo to tytułu
naukowej ekonomii. Najważniejszym z~perspektywy tego artykułu jest trzecie założenie: ekonomia wolna od wartościowania,
która powoduje, że część ekonomistów nadal wierzy w~przepaść pomiędzy normatywnym a~pozytywnym
podejściem
\parencite{van_dalen_values_2019}.
%\label{ref:RNDzkMImibUG1}(Van Dalen, 2019).

Zanim przejdziemy do analizy poglądów Robbinsa wyjaśnić należy, w~jaki sposób logiczny pozytywizm klasyfikuje sądy.
Dzięki tej wiedzy możliwe będzie zrozumienie różnicy pomiędzy normatywnym a~pozytywnym podejściem, które nadal się
utrzymuje w~ekonomii. Pozytywiści logiczni wprowadzili trójczęściową klasyfikację zdań: 1. Syntetyczne 2. Analityczne
3. Nonsens (za Wittgensteinem)
\parencite[s.~10 i~18]{putnam_collapse_2002}.
%\label{ref:RNDJE9B9EQlD8}(Putnam, 2002, s.~10 i~18).
Zdania syntetyczne są empirycznie
weryfikowalne lub falsyfikowalne. Na przykład możemy stwierdzić, że Rysy to najwyższa góra w~Polsce. Łatwo jest
empirycznie sprawdzić, czy to zdanie jest prawdziwe, czy fałszywe. Z~kolei zdania analityczne nie mogą być zweryfikowane
przez odniesienie do rzeczywistości. Przybierają one formę tautologii. To oznacza, że możemy stwierdzić, czy są one
prawdziwe, czy fałszywe tylko na podstawie zasad logiki. Celem tautologii nie jest wyrażanie opinii na temat
rzeczywistości, ale pokazanie logicznej struktury świata. Często używanym przykładem tautologii jest stwierdzenie, że
wszyscy kawalerowie nie są żonaci. Zdania analityczne stanowią podstawę matematyki. Przykładem może być twierdzenie, że
suma kątów w~trójkącie równa się 180 stopni. Trzeci rodzaj zdań zawiera wszystkie etyczne, metafizyczne i~estetyczne
sądy, które są ,,kognitywnie bezsensowne''
\parencite[s.~10]{putnam_collapse_2002}.
%\label{ref:RNDTuXgWOrMkm}(Putnam, 2002, s.~10).
Logiczni pozytywiści uważają
je za nienaukowe, bo nie da się ich empirycznie zweryfikować. Przez to nie zasługują nawet na miano fałszywych. Takie
pojmowanie trzeciego rodzaju sądów ma konsekwencje dla etyki, w~której normy są wykorzystywane do formułowania
orzekających zdań jak `Zabijanie jest złe'. Powszechnie uznaje się, że takie zdanie musi być albo prawdziwe, albo
fałszywe. Jednakże, logiczna analiza udowadnia, że ze stwierdzenia `Zabijanie jest złem' nie możemy wyciągnąć żadnych
zdań, które powiedzą nam o~przyszłych doświadczeniach. Z~tego powodu, to stwierdzenie, jak i~cała etyka jest
niesprawdzalna empirycznie, a~przez to bezsensowna dla logicznych pozytywistów
\parencite[s.~25]{carnap_philosophy_1935}.
%\label{ref:RNDVUJx0tYu6l}(Carnap, 1935, s.~25). 

Już przed logicznymi pozytywistami w~ekonomii istniał podział na normatywne i~pozytywne podejście. Podziału dokonał John
Neville Keynes. Podział na pozytywne i~normatywne ujęcia był częścią bardziej generalnego podziału Keynesa pomiędzy
nauką pozytywną, normatywną i~sztuką. Nauka pozytywna zajmuje się faktami (to, co jest), nauka normatywna bada
normy i~zasady (to, co powinno być), a~sztuka jest skupiona na aplikacji polityki (to, co można osiągnąć)
\parencite[s.~34–35]{keynes_scope_1917}.
%\label{ref:RNDSjhLavdBZs}(Keynes, 1917, s.~34–35).

Na początku ekonomii neoklasycznej różnica pomiędzy pozytywnym i~normatywnym podejściem była wyraźna, ale ekonomiści nie
podważali potrzeby stosowania normatywnego ujęcia. Uznawali, że etyka jest niezbędna, ale nie powinna być tematem,
którym zajmuje się ekonomia. Jednym z~powodów tego przekonania była specjalizacja. Ekonomiści nie mogli zajmować się
każdym problemem i~dlatego musieli się specjalizować. Do 1920 roku większość ekonomistów wierzyła, że relacja pomiędzy
ekonomią a~etyką jest hierarchiczna. Ekonomia była nauką o~bogactwie (Jevons, Mill, Smith), a~etyka używała wniosków z
ekonomii i~innych nauk społecznych, by wydawać opinie na temat kierunków postępowania, które są etycznie pożądane
\parencite{yuengert_positive-normative_2000}.
%\label{ref:RNDnYREBrpeNq}(Yuengert, 2000). 

Specjalizacja była ważnym czynnikiem, który spowodował, że ekonomia odrzuciła etykę. Ale to Robbins ustanowił dychotomię
pomiędzy ekonomią normatywną i~pozytywną. Bez wątpienia na jego poglądy miał wpływ logiczny pozytywizm, w~którym
utrzymuje się, że etyczne zdania nie reprezentują twierdzeń, a~postawy emocjonalne (emotywizm). Robbins
przedstawił różnicę pomiędzy etyką a~pozytywną ekonomią w~sławnym cytacie:

\myquote{
Ekonomia zajmuje się możliwymi do sprawdzenia faktami, a~etyka -- aktami wartościowania i~zobowiązaniami. Na tych dwóch
polach badań stosowane są zupełnie różne sposoby rozumowania. Logiczna przepaść dzieli uogólnienia z~zakresu badań
pozytywnych i~z~zakresu badań normatywnych
\parencite[s.~148]{robbins_essay_1935}.
%\label{ref:RNDs9I3ynBcW5}(Robbins, 1932, s.~148).
}

Zgodnie z~wnioskami logicznego pozytywizmu wszystkie osądy wartościujące (\textit{value judgments}) zostały włożone do
jednego worka z~nonsensami, o~których nie możemy racjonalnie dyskutować. Z~tego powodu Robbins chciał oddzielić
ekonomię od etyki i~argumentował, że ,,ekonomiczna analiza jest \textit{wertfrei }(wolna od wartości)''
\parencite[s.~91]{robbins_essay_1935}.
%\label{ref:RNDiEPGhmzWD3}(Robbins, 1932, s.~91).
Później według Sena, ekonomia stała się ,,świadomie nieetyczną''
%\label{ref:RNDedpzDnp5gT}(Sen, 1987, s.~2)
\parencite[s.~2]{sen_ethics_1987}\footnote{Podział na normatywną i~pozytywną naukę można znaleźć wcześniej niż
u~logicznych pozytywistów. Gilotyna Hume'a jest najbardziej znanym przykładem. Co istotne, w~ostatnich latach wielu
autorów podważało zasadność istnienia tej dychotomii w~tak wyraźnej wersji, uznając, że przynosi ona więcej szkód niż
korzyści
\parencite{blaug_metodologia_1995,putnam_collapse_2002,mongin_value_2006,czarny_pozytywizm_2010}.
%\label{ref:RNDq8sDWAdtJJ}(Blaug, 1995; Putnam, 2002; Mongin, 2006; Czarny, 2010).
}. 

W ekonomii wolnej od wartościowania ekonomiści zachowują się jak inżynierowie, którzy umieją poradzić sobie z~problemem
w~sposób efektywny, ale nie wskażą celów, które powinniśmy osiągnąć. Ekonomiści pomagają ludziom w~osiąganiu celów,
które ludzie sami sobie wyznaczyli. Ekonomiści są dumni z~odrębności od etyki, bo dzięki temu mogą być przestrzegani
jako obiektywni naukowcy, którzy zajmują się tylko faktami. 

Drugim powodem uformowania się ekonomii wolnej od wartościowania jest proces odchodzenia przez ekonomię od psychologii,
który miał miejsce na początku XX wieku. Ten proces w~literaturze jest nazywany `zwrotem Pareto'. Wielu ekonomistów,
wśród nich Bruni i~Sugden
\parencite*{bruni_road_2007},
%\label{ref:RNDZKFvxecjY0}(2007)
kompleksowo przedstawili, jak do tego doszło. Dlatego w~tym
artykule uwaga zostanie skupiona tylko na kilku najważniejszych czynnikach tego procesu. Na początku warto zwrócić
uwagę na fakt, że marginaliści opierali swoje teorie na kardynalnej użyteczności. Jednakże nie byli oni w
stanie znaleźć przekonującej metody, jak wyliczać użyteczność
\parencite{stigler_development_1950}.
%\label{ref:RNDvzY4TY3ybf}(Stigler, 1950).
Co zrozumiałe,
brak tej metody doprowadził do krytyki. Robbins podkreślał, że marginalna teoria użyteczności była w~dużej mierze
krytykowana z~powodu jej hedonistycznych podstaw: ,,Pogranicza ekonomii są rajem dla umysłów niechętnych precyzyjnej
myśli, i~w~ostatnich latach, w~tych nieokreślonych rejonach niezliczony okres czasu został poświęcony na ataki w~stronę
rzekomych psychologicznych założeń ekonomii''
\parencite[s.~83]{robbins_essay_1935}.
%\label{ref:RNDhUIELbq0LL}(Robbins, 1932, s.~83).
Hedonizm psychologiczny
szybko stracił wiarygodność i~stało się oczywistym dla marginalistów, że muszą zreformować swoje teoretyczne podstawy. 

Ujęcie kardynale wraz z~wiarą, że można wyliczyć użyteczność było nie do utrzymania. Dlatego ekonomiści zaczęli używać
ordynalnego podejścia, w~którym konsument może tylko ustawiać swoje preferencje w~szeregu. Oznacza to, że konsument nie
jest w~stanie wyliczyć, jak wiele użyteczności uzyskuje z~konsumpcji danych dóbr, ale jest w~stanie ocenić, czy
satysfakcja otrzymana z~danego dobra jest taka sama, wyższa lub niższa od zadowolenia z~innego dobra. W~tej
perspektywie konsument jest postawiony przed wieloma kombinacjami różnych dóbr i~jest on/ona w~stanie sklasyfikować je
w~zgodzie z~jego/jej własną skalą preferencji. Niezdolność do wyliczania użyteczności nie tylko doprowadziła do nowego
ordynalnego podejścia, ale doprowadziła również do odejścia od oceniania ludzkich celów. Uznano, że
szczęście i~przyjemność to zbyt nieuchwytne pojęcia i~ekonomia nie powinna tracić czasu na dyskusję o~nich.
Odejście od psychologii i~hedonistycznych podstaw dopełniło się wraz z~Samuelsonem i~jego teorią preferencji ujawnionych. 

Samuelson był zauroczony przejrzystością języka matematycznego i~twierdził, że ,,matematyka to język''
\parencite[s.~52]{samuelson_economic_1952}.
%\label{ref:RND5ihFfT44nj}(Samuelson, 1952, s.~52).
Na uformowanie tego poglądu największy wpływ miał pozytywizm
logiczny. W~tamtym czasie Carnap promował wersję filozofii jako ,,matematyki i~fizyki języka''
\parencite[s.~180]{richardson_geometry_2003}
%\label{ref:RNDhlJo1FS5UX}(Richardson, 2003, s.~180)
i~wydał \textit{Logiczną składnię języka}
\parencite{carnap_logical_1936},
%\label{ref:RNDgMPwleIfm9}(Carnap, 1936),
w~której próbował dostarczyć metajęzyka dla całej nauki. Nie dziwi wybór
języka matematycznego przez pozytywistów logicznych, gdy spojrzymy na to, jak bardzo skrupulatni byli oni w~swoich
analizach i~jak bardzo unikali wieloznaczności.

Samuelson
\parencite*{samuelson_note_1938}
%\label{ref:RNDqzQbQCX59X}(1938)
podążając za pozytywistami logicznymi spróbował zbudować teorię wyboru
konsumenta na ściśle obserwowalnych fundamentach. Jednakże na początku celem Samuelsona nie było `ujawnienie'
preferencji, a~stworzenie wyłącznie operacyjnej teorii wyboru konsumenta, w~której
preferencje i~użyteczność w~ogóle się nie liczą. Samuelson obiecał pozbyć się ostatnich
,,szczątkowych śladów koncepcji użyteczności''
\parencite[s.~61]{samuelson_note_1938}
%\label{ref:RNDyYhBVk6RTo}(Samuelson, 1938, s.~61)
z~teorii wyboru konsumenta. Nie był zainteresowany wyjaśnianiem
(dlaczego ludzie coś wybierają) ani oceną tych wyborów, ale chciał opisać rzeczywistość. By to osiągnąć, potrzebował
metody pomiaru. Dlatego Samuelson oparł swój model na obserwowalnych warunkach i~konsekwencjach. Był
operacjonistą, a~na jego poglądy miał wpływ pozytywizm logiczny. Samuelson nie uznawał, że celem nauki
powinno być znalezienie prawdy.
Dla niego nauka miała bardziej praktyczny wymiar: ,,Ci którzy potrafią, uprawiają naukę;
ci którzy nie potrafią, paplają o~jej metodologii''
\parencite[s.~240]{samuelson_my_1992}.
%\label{ref:RNDYLh4fCY9dq}(Samuelson, 1992, s.~240). 

Główny cel teorii preferencji ujawnionych był filozoficzny: potrzebujemy obserwowalnych i~mierzalnych danych, by
zbudować poprawną naukową teorię. Nie jest jednak możliwe zaobserwowanie samych preferencji. Możemy natomiast
obserwować ludzkie wybory i~na tym założeniu Samuelson zbudował swoją teorię. Generalnie preferencje ujawnione działają
w~następujący sposób. Jeżeli wybiorę opcję $x$ zamiast opcji $y$, wtedy opcja $x$ ujawnia się jako
opcja bardziej preferowana niż opcja $y$. Moje wybory są spójne, jeżeli spełniają one `słaby aksjomat preferencji
ujawnionych', w~skrócie WARP (\textit{Weak Axiom of Revealed Preference}). Oznacza to, że jeżeli $x$ jest jawnie
preferowany w~stosunku do $y$, to wtedy $y$ nie może być jawnie preferowanym w~stosunku do $x$.
Jeżeli wybory spełniają wymagania spójności, to na ich podstawie możemy zbudować pełne, przechodnie i~ciągłe
preferencje ujawnione
\parencite{sen_choice_1971,sen_behaviour_1973}.
%\label{ref:RNDISuHrvwpYX}(Sen, 1971, 1973). 

Z perspektywy artykułu, najważniejszą cechą teorii preferencji ujawnionych jest zakładana przez nią koncepcja racjonalności.
W~tej teorii agenci są racjonalni i~postępują w~zgodzie ze słabym aksjomatem preferencji ujawnionych (WARP).
Racjonalność ma charakter instrumentalny, bo ekonomiści nie znają i~nie chcą znać motywacji ludzi. Ekonomiści
neoklasyczni są wyłącznie zainteresowani rezultatami, a~nie przyczynami zachowań i~ich oceną. Zakładają oni
\textit{a~priori,} że ludzie są zawsze racjonalni, przez co rozumie się, że zawsze maksymalizują swoją użyteczność. Jednakże
bardzo ważna jest świadomość faktu, że użyteczność w~ekonomii neoklasycznej jest różna od użyteczności przyjmowanej
przez Benthama czy marginalistów. Czasami ekonomiści mówią o~ludziach, którzy dążą do maksymalizacji użyteczności.
Jednak nie oznacza to, że użyteczność jest obiektem wyboru lub że jest ona ostatecznym dobrem. Maksymalizator
użyteczności robi tylko to, co on/ona preferuje. Twierdzenie, że ludzie maksymalizują swoją użyteczność nie mówi
nic o~naturze ich preferencji. Łączy to tylko preferencje z~wyborem. Racjonalni agenci ustawiają w~szeregu dostępne
opcje i~wybierają tę, którą preferują najbardziej
\parencite[s.~18]{hausman_inexact_1992}.
%\label{ref:RND42WWYgUnG1}(Hausman, 1992, s.~18).

Ostatecznie jednak Samuelson nie osiągnął tego, co zamierzał. W~roku 1938 postrzegał on teorię preferencji ujawnionych
jako alternatywę do teorii użyteczności. W~1950 roku było jednak wiadome, że jego teoria nie różni się niczym od
ordynalnego podejścia
\parencite{wong_foundations_2006}.
%\label{ref:RNDsRjT4YoK6p}(Wong, 2006, s.~55).
Pomimo tego, że Samuelson nie był w~stanie
zaproponować teorii różniącej się od ordynalnego podejścia, to jego teoria do dziś ma ogromny wpływ na ekonomię
neoklasyczną, która postrzega racjonalność w~instrumentalny sposób, nie interesuje się ludzkimi motywacjami i~zakłada,
że wybór `ujawnia' ludzkie preferencje. Aksjomatyczna teoria preferencji ujawnionych jest samopotwierdzająca, ponieważ
z~góry zakładamy, że ludzie są racjonalni, jeżeli spełnią techniczne warunki teorii. Dzięki temu aksjomatycznemu
systemowi ekonomiści zdobyli pewność i~uwierzyli, że tworzą obiektywną naukę. 

W procesie stawania się ekonomii pozytywną (,,twardą'') nauką miały szczególny udział dwie osoby, które ekonomiści
wykorzystali do swoich celów. Pierwszą z~nich jest Milton Friedman, którego \textit{The Methodology of Positive
Economics} (1953) jest najbardziej znaną pracą z~metodologii ekonomii w~XX wieku. Caldwell określił ją jako
,,marketingowe arcydzieło''
\parencite[s.~173]{caldwell_beyond_1982},
%\label{ref:RNDe0oVnFlou9}(Caldwell, 1982, s.~173),
które jest cytowane w~każdym podręczniku
od ekonomii. Sześćdziesiąt lat po publikacji nadal pozostaje ,,jedynym esejem z~metodologii, którego wielu, być może
większość, ekonomistów przeczytała''
\parencite[s.~162]{hausman_inexact_1992}.
%\label{ref:RNDknbMs3Zj7M}(Hausman, 1992, s.~162).
Najważniejszą rzeczą, która
liczyła się dla Friedmana, była predykcja. Nowe fakty, które nie zostały wcześniej zaobserwowane, traktowane są jako
dowody i~to one przesądzają, czy teoria ekonomiczna jest udana. Friedman pisał: ,,ostatecznym celem pozytywnej nauki jest
rozwinięcie `teorii' lub `hipotezy', która skutkuje prawidłowymi i~istotnymi […] predykcjami zjawisk, które nie były
wcześniej zaobserwowane''
\parencite[s.~7]{friedman_essays_1953}.
%\label{ref:RNDNnJgK14vNS}(Friedman, 1953, s.~7).
Friedman jest postrzegany jako
instrumentalista, ponieważ nie interesuje go realizm założeń
%\label{ref:RNDZCFsCU6A8v}(Boland, 1979; Caldwell, 1992)
\parencite{boland_critique_1979,caldwell_critique_1992}\footnote{%
\parencite{maki_unrealistic_2009}
%Mäki \label{ref:RNDPUiPHL0tkX}(2009b)
argumentuje, że pozycja Friedmana jest bardziej skomplikowana,
zob. także
\parencite{hoyningen-huene_revisiting_2017}.
%\label{ref:RND4xmm9hnovz}(Hoyningen-Huene, 2017).
}. Są one tylko używane do predykcji nowych faktów. Co
więcej, ,,ogólnie mówiąc, im bardziej znacząca teoria, tym bardziej nierealistyczne założenia''
\parencite[s.~14]{friedman_essays_1953}.
%(Friedman 1953, s.~14).
Friedman stosował założenia uproszczające (\textit{as if}\footnote{Słynny przykład bilardzisty
\parencite[s.~13]{friedman_essays_1953}.
%\label{ref:RND99MiHa7xdV}(Friedman, 1953, s.~13).
}), bo dzięki nim siła predykcyjna była większa. 

Drugą postacią mającą znaczny wpływ na metodologię ekonomii był Popper wraz ze swoją koncepcją falsyfikacjonizmu, która
mówi o~tym, że teorie muszą być możliwe do empirycznego obalenia, jeżeli chcą nosić miano naukowych. Koncepcja
falsyfikacjonizmu ma dwa cele: 1. Demarkacja nauki od pseudonauki; 2. Metodologiczne (jak nauka powinna być uprawiana)
\parencite{hands_popper_1993}.
%\label{ref:RNDAEL1b0XmAg}(Hands, 1993).
Popper w~swojej teorii negatywnie odnosił się do pozytywizmu logicznego.
Dlatego może się wydawać dziwnym, że ekonomiści, na których logiczny pozytywizm miał znaczący wpływ, tak ochoczo zaczęli
wykorzystywać koncepcję falsyfikacjonizmu, która miała zapewnić ekonomii pozycję prawdziwej nauki. To zauroczenie
falsyfikacjonizmem było i~nadal jest możliwe, ponieważ ekonomiści nie do końca zdają sobie sprawę z~ograniczeń i~tego
jak ta metoda działa. Istnieją dwa główne problemy z~falsyfikacjonizmem w~ekonomii. Po pierwsze, ekonomiści tak naprawdę
nigdy jej nie praktykowali. Blaug
\parencite*[s.~175]{blaug_metodologia_1995}
%\label{ref:RNDR4Dr8G9mTm}(1995, s.~175)
twierdzi, że ,,współcześni ekonomiści często
głoszą falsyfikacjonizm, jednak rzadko go praktykują. Ich rzeczywistą filozofię nauki określa się trafnie mianem
`nieszkodliwego falsyfikacjonizmu'\footnote{To sformułowanie Blaug zawdzięcza Coddingtonowi
\parencite[s.~542]{coddington_rationale_1975}.
%\label{ref:RNDlsT2nt5lEr}(1975, s.~542).
}''.
Oczywiście, ekonomiści zajmują się badaniami empirycznymi, co powinno
spowodować, że ich teorie są bardziej podatne na falsyfikacjonizm, ale ,,wielka część tych badań przypomina grę w~tenisa
bez siatki. Zamiast próbować obalać testowalne prognozy, współcześni ekonomiści ciągle zbyt często zadowalają się
demonstrowaniem, że realny świat potwierdza ich przepowiednie, zastępując w~ten sposób trudną falsyfikację łatwą
weryfikacją''
\parencite[s.~348]{blaug_metodologia_1995}.
%\label{ref:RND47shNlBNBK}(Blaug, 1995, s.~348).
Dlatego w~opinii Blauga ekonomiści nie falsyfikują swoich
teorii, a~raczej je weryfikują, co uwidocznia niesłabnący wpływ pozytywizmu logicznego na ekonomię\footnote{Jednak
niezastosowanie się do falsyfikacjonizmu nie dotyczy tylko ekonomii, a~wszystkich nauk. Kuhn
\parencite*{kuhn_structure_1962}
%\label{ref:RNDdS2sd97E2o}(1962)
pokazał, dlaczego naukowcy nie stosują się do zasad falsyfikacjonizmu.}. 

Problemy z~falsyfikacjonizmem są jeszcze głębsze. Wielu ekonomistów nie wie o~niedookreśleniu
(\textit{underdetermination})
\parencite{quine_two_1951},
%\label{ref:RNDVdySsnv1TM}(Quine, 1951)
czy o~tym, że obserwacja zawsze jest
uteoretyzowana (\textit{theory-ladenness})\footnote{Oba podejścia pokazują, że w~nauce nie istnieją obiektywne fakty,
dzięki którym możemy poznać absolutną prawdę. Niedookreślenie oznacza, że bez względu na ilość dowodów nie jesteśmy
w~stanie z~pewnością ocenić, która teoria/hipoteza jest prawdziwa. Uteoretyzowanie w~najogólniejszym znaczeniu polega na
tym, że obserwacje czynione przez badaczy są poprzedzone jakąś presupozycją, która ma wpływ na postrzeganie obserwacji.}.
Postrzegają oni falsyfikacjonizm jako specjalny typ empirycznego fundamentalizmu (pozytywizm logiczny), ale bez
problemu indukcji
\parencite[s.~292]{hands_reflection_2001}.
%\label{ref:RND7evhLaRK3E}(Hands, 2001, s.~292).
Popper nigdy nie był fundamentalistą empirycznym,
dlatego ekonomiści są Popperystami ze złych powodów
\parencite[s.~292]{hands_reflection_2001}.
%\label{ref:RNDAkUOWBzLiK}(Hands, 2001, s.~292).
Po drugie, jeżeli
ekonomiści czytaliby Poppera we współczesny sposób, nie byliby Popperystami, ponieważ falsyfikacjonizm nie zapewnia
metodologii, która pokazuje, jak nauka powinna być uprawiana i~nie wyznacza solidnej linii demarkacyjnej pomiędzy nauką
a~pseudonauką. Ostatecznie nie możemy całkowicie uciec od założeń ontologicznych
\parencite{kuhn_structure_1962,feyerabend_against_1975,mccloskey_rhetoric_1998,hands_reflection_2001}.
%\label{ref:RNDOFbhSJJLqP}(Kuhn, 1962; Feyerabend, 1975; McCloskey, 1998; Hands, 2001).
Hands twierdzi, że ,,falsyfikacyjna metodologia odważnych
domysłów i~surowych testów dostarcza reguł dla gry w~naukę bez podania ostatecznego celu grania w~tę grę''
\parencite[s.~293]{hands_reflection_2001}.
%\label{ref:RNDYa0RKrqLjQ}(Hands, 2001, s.~293).
Jednak ten cel musi być podany, co zostało podkreślone przez Lakatosa:
,,Reguły gry, metodologia, stoi na własnych nogach; ale te nogi majtają w~powietrzu bez filozoficznego wsparcia''
\parencite[s.~154]{lakatos_methodology_1980}.
%\label{ref:RNDySWNKAjV40}(Lakatos, 1980, s.~154).

Wykorzystanie i~wychwalanie zarówno Poppera, jak i~Friedmana doprowadza do pewnego rodzaju filozoficznej
schizofrenii. Z~jednej strony ekonomiści są naiwnymi realistami, którzy wierzą w~obiektywne (mierzalne)
dane i~pozytywizm logiczny. Z~drugiej strony używają oni instrumentalizmu,
gdzie prawda nie może być poznana, a~predykcja jest jedyną rzeczą, która
się liczy, dzięki swojej użyteczności. \textit{Ta niekonsekwencja może być postrzegana jako rezultat} braku wiedzy
metodologicznej i~oportunizmu ekonomistów, którzy wykorzystują metodologię, która im odpowiada. Obiektywne dane dają im
twarde, naukowe fundamenty, które odróżniają ekonomię od innych nauk społecznych. Z~kolei instrumentalizm
wraz z~brakiem wiary w~Prawdę, daje ekonomistom przyzwolenie na uproszczone modele matematyczne,
które nie muszą wyjaśniać
rzeczywistości, a~mają tylko siłę predykcyjną. 

Na koniec analizy podstaw metodologicznych ekonomii wolnej od wartościowania warto zwrócić uwagę na
społeczno-historyczny kontekst, dzięki któremu można zaobserwować, w~jaki sposób ekonomia, wykorzystując matematykę,
stała się najefektywniejszą z~nauk społecznych. Lata pięćdziesiąte XX wieku określa się jako czas, w~którym doszło do rewolucji
formalistycznej w~ekonomii
\parencite{blaug_formalist_2003}.
%\label{ref:RNDkjrVNGkWGA}(Blaug, 2003).
W~krótkim czasie uproszczone i~zmatematyzowane
spojrzenie na rzeczywistość i~człowieka stało się dominującym poglądem w~ekonomii, a~formalizm stał się naukowym
esperanto. Po II wojnie światowej nastąpił znaczny wzrost ilości publikacji ekonomicznych wykorzystujących matematykę,
zawdzięczających swoją popularność prostocie, efektywności i~obiektywności
\parencite{debreu_mathematization_1991}.
%\label{ref:RNDWWnA0Uim6D}(Debreu, 1991).
Matematyzacja ekonomii spowodowała upowszechnienie się opinii, że ekonomia jest niemal tak `twarda', jak nauki
ścisłe i~daleka od innych, bardziej `miękkich' nauk społecznych.
Należy również podkreślić ahistoryczność metod ortodoksyjnej
ekonomii, w~której prawa ekonomiczne są jak prawa przyrodnicze: funkcjonują niezależnie od historii, są obiektywne,
odrębne od społecznej rzeczywistości i~można je poznać przez doświadczenie. Co równie istotne, w~ortodoksyjnej ekonomii
realizowany jest przyrodniczy ideał poznania naukowego, rozumiany jako neutralny aksjologicznie, możliwie precyzyjny,
niezawierający elementów wartościujących opis faktualnej rzeczywistości gospodarczej
\parencite[zob.][]{zboron_dyskurs_2013}.
%\label{ref:RNDvUbQLmvKjR}(zob. Zboroń, 2013).
Dzięki przyjęciu tej metodologii upowszechnił się pogląd, że ekonomia osiągnęła najszybszy rozwój ze
wszystkich nauk społecznych, czego rezultatem był imperializm ekonomiczny, w~którym ekonomia wykorzystywała swoją
efektywną metodologię (metoda ekonomiczna Beckera) do wyjaśniania tematów, którymi tradycyjnie zajmowały się inne nauki
społeczne
\parencite{lazear_economic_2000,maki_economics_2009,maki_scientific_2017}.
%\label{ref:RNDJUdE7QGLyN}(Lazear, 2000; Mäki, 2009a; Mäki, i~in., 2017).

\section{Metoda ekonomiczna jako naukowy program badawczy}
Przejdźmy teraz do głównego celu tego artykułu, którym jest analiza metody ekonomicznej Beckera za pomocą koncepcji
naukowych programów badawczych
\parencite{lakatos_methodology_1980}.
%\label{ref:RNDW8Z0IppcZ2}(Lakatos, 1980).
Dzięki tej analizie będzie możliwe omówienie
głównych założeń ekonomii wolnej od wartościowania, które powodują, że ekonomia może wchłonąć wszelką krytykę ze strony
innych nauk i~pozostawić paradygmat bez zmian. 

Najważniejszą ideą Lakatosa jest wskazanie, że podstawą rozwoju nauki nie jest pojedyncza hipoteza, ale naukowy program
badawczy. Składa się on z: (1) twardego rdzenia, (2) pasa ochronnego, (3)~pozytywnej i~(4) negatywnej heurystyki. 

(1) `Twardy rdzeń' zawiera główne metafizyczne założenia programu badawczego. Jest to struktura składająca
się z~hipotez, których elementy są niemożliwe do obalenia przy pomocy empirycznych dowodów. Twardy rdzeń pozostaje bez zmian
pomimo rozwoju samego programu, ponieważ odrzucenie twardego rdzenia powoduje porzucenie samego programu
\parencite[s.~122]{hands_reflection_2001}.
%\label{ref:RNDu0vWhxi7dt}(Hands, 2001, s.~122).
(2) `Pas ochronny' zawiera pomocnicze hipotezy, konwencje
empiryczne i~inne teoretyczne struktury programu, które mogą być sfalsyfikowane. W~tej części z~czasem dochodzi do
zmian w~programie. Pas ochronny jest strefą buforową pomiędzy twardym rdzeniem a~dowodami empirycznymi. Pas ochronny zmienia
się w~rezultacie zmian w~dowodach empirycznych
\parencite[s.~122]{hands_reflection_2001}.
%\label{ref:RNDgx8kLJ2hS4}(Hands, 2001, s.~122).
(3)/(4)
Pozytywne i~negatywne heurystyki dostarczają natomiast wskazówek na temat tego, co powinno (pozytywne) i~co nie powinno (negatywne)
być poszukiwane podczas rozwoju programu. Tworzą one zestaw akceptowanych reguł metodologicznych
\parencite[s.~47]{lakatos_falsification_1970}.
%\label{ref:RNDdZyYtwXc8e}(Lakatos, 1970, s.~47).
`Pozytywna heurystyka' to ,,zestaw sugestii i~wskazówek'', który pomaga
nam w~dopasowywaniu części pasa ochronnego
\parencite[s.~50]{lakatos_falsification_1970}.
%\label{ref:RNDCKDW3y4PX1}(Lakatos, 1970, s.~50).
Z~kolei `negatywna
heurystyka' kieruje krytykę dotyczącą fałszywości lub nieadekwatności teorii w~stronę pasa ochronnego, dzięki czemu
twardy rdzeń pozostaje nie do obalenia
\parencite[s.~48–50]{lakatos_falsification_1970}.
%\label{ref:RNDVNL2fnZR2g}(Lakatos, 1970, s.~48–50).
Lakatos był pod wpływem
teorii Kuhna
\parencite*{kuhn_structure_1962}
%\label{ref:RNDjI8wf6auXU}(1962)
i~jego program jest spójny w~niektórych obszarach z~podejściem do nauki
prezentowanym przez autora \textit{Struktury rewolucji naukowych}. Większość zmian, które mają miejsce w~programie
badawczym zachodzi w~pasie ochronnym, podczas gdy twardy rdzeń pozostaje nienaruszony. Rewolucja naukowa ma miejsce,
kiedy twardy rdzeń zostaje zastąpiony przez nowy. Według Lakatosa rozwój nauki ma miejsce, kiedy degeneracyjny program
badawczy zostaje zastąpiony przez progresywny program, który doprowadza do odkrywania nowych zjawisk.

Jednak z~czasem okazało się, że trudno jest dokładnie orzec, kiedy mamy do czynienia z~progresywnym czy degeneracyjnym
programem
\parencite{marchi_appraising_1991,blaug_ugly_1997,hands_second_1990}.
%\label{ref:RNDcCX6Kl36c1}(Marchi, Blaug, 1991; Hands, 1990).
Dlatego pomimo początkowego zachwytu ekonomistów
naukowym programem badawczym wraz z~czasem stracił on swoją popularność. Wieloznaczność oceny programów nie zmienia
faktu, że nadal jest to bardzo użyteczna metoda, dzięki której można zbadać teoretyczne założenia poszczególnych
teorii. Pomimo szerokiego wykorzystania naukowych programów badawczych do analizowania ekonomicznych
teorii w~literaturze
\parencites[zob.][]{drakopoulos_review_2005}[s.~299–300]{hands_reflection_2001},
%\label{ref:RNDhZwNoTft59}(zob. Drakopoulos, Anastassios, 2005; Hands, 2001, s.~299–300),
nie została
ona w~pełni wykorzystana przy okazji metody ekonomicznej Beckera, która dobrze się do tego nadaje. 

W ujęciu Beckera metoda ekonomiczna składa się z~aksjomatycznych założeń, na podstawie których są dedukowane znaczące
predykcje na temat ludzkiego zachowania. Pisał on: ,,Bezwzględne i~konsekwentne posługiwanie się kombinacją tych założeń
(maksymalizującego charakteru zachowań, równowagi rynkowej i~stałości preferencji) stanowi istotę podejścia
ekonomicznego w~moim rozumieniu''
\parencite[s.~23]{becker_ekonomiczna_1990}.
%\label{ref:RNDB9NQdB96p7}(Becker, 1990, s.~23).
Uważał on, że ,,podejście ekonomiczne
dostarcza nam cennego, jednolitego schematu służącego do zrozumienia wszelkich zachowań ludzkich''
\parencite[s.~38]{becker_ekonomiczna_1990}.
%\label{ref:RNDFMM4S6jgqv}(Becker, 1990, s.~38).
Jednak tutaj rozumienie nie jest tożsame ze zdroworozsądkowym
zrozumieniem. Dla Beckera i~Stiglera
\parencite*{becker_gustibus_1977}
%\label{ref:RNDcL2TMmBvEO}(1977)
ekonomiczna metoda musi być oceniana pod względem
siły predykcyjnej, a~nie pod względem deskryptywnego realizmu jej założeń lub wyjaśnień
\parencite[s.~402–403]{becker_economic_1993}.
%\label{ref:RND5dPX1nhrH4}(Becker, 1993, s.~402–403).
Takie pojmowanie sprawy jest odpowiednie do naukowych programów
badawczych, w~których twardy rdzeń nie musi odzwierciedlać rzeczywistości. Dla Beckera rozumienie ludzkich zachowań
oznacza zdolność tworzenia predykcji i~w~tym miejscu można zaobserwować wpływ eseju Friedmana
\parencite*{friedman_essays_1953}.
%\label{ref:RNDXvN6XWjHkE}(1953).
Robbins był drugim ekonomistą, który miał bardzo duży wpływ na Beckera i~sprawił, że
metoda ekonomiczna jest zbudowana na podobnych fundamentach, co ekonomia wolna od wartościowania. Jego oddziaływanie
można zauważyć przede wszystkim w~dwóch obszarach. Po pierwsze, Becker uznawał, że ekonomia może i~powinna być wolna od
wartościowania. Dlatego ekonomiści nie powinni oceniać ludzkich celów i~decydować, do których rzeczy warto dążyć. Po
drugie, Becker opiera swoją metodę na pojęciu rzadkości\footnote{Odwołanie do niezwykle popularnej definicji ekonomii
zaproponowanej przez Robbinsa: ,,Ekonomia jest nauką, która bada zachowania człowieka jako związku między
celami i~ograniczonymi środkami mogącymi mieć alternatywne zastosowania''
\parencite[s.~15]{robbins_essay_1935}.
%\label{ref:RND2EOx07Klww}(Robbins, 1932, s.~15).
}.
Ludzie zawsze muszą wybierać, ponieważ z~powodu ograniczonego czasu nie możemy robić w~tym samym czasie wszystkich rzeczy,
na które mamy ochotę. Dzięki takiej metodologii ekonomia jest w~stanie analizować każdy ludzki wybór (religia, miłość,
wychowywanie dzieci etc.) i~z~tego powodu ekonomia zaczęła zajmować się tematami, którymi wcześniej zajmowały się inne
nauki społeczne. Między innymi za imperializm ekonomiczny Becker otrzymał Nagrodę Nobla
\parencite[s.~1]{noauthor_royal_1993}.
%\label{ref:RND6m9zMXt0RY}(Royal Swedish Academy of Sciences, press release. The Nobel Memorial Prize in Economics 1992, 1993, s.~1)

\subsection{2.1. Metoda ekonomiczna. Twardy rdzeń i~pas ochronny}
W artykule zostały wskazane dwa twarde rdzenie metody ekonomicznej według Beckera: maksymalizacja
użyteczności i~instrumentalna racjonalność. Oczywiście jest to tylko propozycja, a~inni naukowcy mogą wskazać
inne twarde rdzenie. Ta
dowolność jest uważana jest za jedną z~wad programów Lakatosa
\parencite{marchi_appraising_1991}.
%\label{ref:RND6UxdN1GLbI}(Marchi, Blaug, 1991).

\textbf{(1) Maksymalizacja użyteczności}. Becker nie ukrywał, jak wiele czerpał od utylitaryzmu i~Benthama. Dla
niego ludzie również zawsze dążą do maksymalizacji swojej użyteczności. Becker twierdzi, że:

\myquote{
Każdy zdaje sobie sprawę, że w~podejściu ekonomicznym założenie o~maksymalizującym charakterze badanych
zachowań jest obecnie stosowane jawniej i~w~szerszym zakresie niż w~innych podejściach, niezależnie od tego, co jest
maksymalizowane: funkcja użyteczności czy bogactwa gospodarstwa domowego, przedsiębiorstwa, zrzeszenia czy urzędu
\parencite[s.~22]{becker_ekonomiczna_1990}.
%\label{ref:RNDjqWI61SD4C}(Becker, 1990, s.~22). 
}

Dla Beckera dużo ważniejsze od samego mierzenia użyteczności był sposób, w~jaki użyteczność jest postrzegana. Pojmował
on użyteczność \textit{ad libitum}, co oznacza, że użyteczność może oznaczać dosłownie wszystko. Becker dzięki
poszerzeniu konceptu użyteczności był w~stanie odpowiedzieć na dwa rodzaje krytyki dotyczącej \textit{homo oeconomicus}.

Po pierwsze, uwzględnił on wszystkie altruistyczne zachowania w~procesie maksymalizacji użyteczności. Od samego początku,
a~w~szczególności od czasów rewolucji marginalistycznej, ekonomiści byli oskarżani o~przedstawianie ludzi jako
egoistycznych istot, które nie myślą o~innych. Wielu przeciwników \textit{homo oeconomicus} pytało, dlaczego niektórzy
oddają swoje organy ludziom, których nawet nie znają, dlaczego dają napiwki w~przydrożnej restauracji, czy wręczają
prezenty. W~metodzie ekonomicznej rozszerzono pojęcie własnego interesu/maksymalizacji użyteczności, by wyjaśnić wyżej
wymienione anomalie. Odpowiedzią jest stwierdzenie, że ludzie pomagają innym, bo daje im to użyteczność.
Zgodnie z~metodą ekonomiczną, jeżeli ktoś ofiarowuje swoją nerkę nieznajomemu, to robi to dlatego, że daje to jej
użyteczność. Becker nie utożsamiał maksymalizacji użyteczności z~własnym interesem i~doskonale zdawał sobie
sprawę z~tego, że ludzie mają różne motywacje (np. miłość, zazdrość, lojalność). Twierdził, że ,,jednostki maksymalizują
dobrostan tak jak go postrzegają, nieważne, czy są egoistyczne, altruistyczne, lojalne, złośliwe czy masochistyczne''
\parencite[s.~386]{becker_economic_1993}.
%\label{ref:RNDwvpKCVpn44}(Becker, 1993, s.~386).
Jednak jego metoda nie zwracała uwagi na te różnice i~stwierdzała
tylko, że ludzie maksymalizują swoją użyteczność. To relatywistyczne pojmowanie użyteczności doprowadziło do tego, że
użyteczność stała się swego rodzaju czarną skrzynką, która może zawierać wszystko. Dlatego nie ma znaczenia, czy ludzie
są altruistami czy egoistami, bo w~ostateczności zawsze maksymalizują swoją użyteczność
\parencite{sen_rational_1977}.
%\label{ref:RNDb14Ckvui8P}(Sen, 1977). 

Pod drugie, rozszerzona koncepcja użyteczności powoduje, że nie możemy ustalić żadnych ostatecznych wartości, do których
ludzie chcą dążyć. Nie tylko nie jest możliwym porównanie użyteczności pomiędzy ludźmi (interpersonalne
porównania użyteczności), ale też pomiędzy wyborami dokonanymi przez poszczególną osobę. W~tym miejscu widać, że Becker
wspiera ordynalne podejście i~ekonomię wolną od wartościowania. Sprawą ważniejszą nawet od maksymalizacji użyteczności
jest myślenie konsekwencjalistyczne, wykorzystywane w~metodzie ekonomicznej. Nie ma w~niej żadnych dóbr ani celów, które
mają wartość samą w~sobie. Jedyną rzeczą, która się liczy, jest wybór, a~zgodnie z~założeniami, kiedy wybieramy coś, to
maksymalizujemy naszą użyteczność. Postrzeganie użyteczności \textit{ad libitum }tworzy `pas ochronny'. Jest
oczywistym, że ludzie czasem podejmują nietrafione decyzje i~nie maksymalizują swojej użyteczności (zażywanie
narkotyków, palenie, jazda samochodem po spożyciu alkoholu), czy też zachowują się altruistycznie. Te zachowania są
anomaliami\footnote{Anomalie to określenie używane przez ekonomistów neoklasycznych, którzy w~taki sposób postrzegali
wnioski płynące z~ekonomii behawioralnej. Dzisiaj istnieje ogólna zgoda co do tego, że ludzie z~powodu ograniczeń
poznawczych popełniają błędy, które są systematyczne
\parencite{thaler_misbehaving:_2015}.
%\label{ref:RNDX94q4CismZ}(Thaler, 2015).
}, które stanowią
zagrożenie dla `twardego rdzenia', w~którym ludzie zawsze maksymalizują swoją użyteczność. Dlatego użyteczność jest
pojmowana \textit{ad libitum}. W~ten sposób twardy rdzeń może być ochroniony przed anomaliami, bo z~definicji
niemożliwym staje się zachowywanie, w~którym nie maksymalizujemy swojej użyteczności. 

\textbf{(2) Racjonalność instrumentalna}. Głównie w~wyniku rewolucji formalistycznej ekonomia
neoklasyczna zaczęła postrzegać ludzi jako hiper-racjonalnych agentów, którzy są w~stanie przetwarzać nieskończoną
ilość informacji i~podejmować decyzje, które maksymalizują ich użyteczność. Jednakże, racjonalność doskonała
stała się celem krytyki, która znacznie się nasiliła w~latach siedemdziesiątych i~osiemdziesiątych XX wieku w~dużym
stopniu za sprawą ekonomii behawioralnej
\parencite{thaler_misbehaving:_2015}.
%\label{ref:RNDGzCOayPi2i}(Thaler, 2015).
Dzisiaj ekonomiści neoklasyczni nie
mówią już o~racjonalności doskonałej. Zamiast tego mówią o~racjonalności instrumentalnej. Jest to mniej wymagająca
wersja racjonalności i~jest on stosowana, by ochronić twardy rdzeń metody ekonomicznej. Jest oczywistym, że ludzie nie
są w~stanie przetworzyć wszystkich dostępnych informacji i~na tej bazie podjąć najlepszą decyzję. Często ludzie
zachowują się w~głupi i~irracjonalny sposób. Ekonomiści mają dwa sposoby radzenia sobie z~tymi anomaliami.

Po pierwsze, ekonomiści wierzą w~to, że ludzie dokonywaliby doskonale racjonalnych wyborów, gdyby nie rozmaite
ograniczenia, takie jak ograniczony czas lub ograniczone zdolności kognitywne. Czasami ekonomiści argumentują również,
że pełna racjonalność może być irracjonalna. Ma to miejsce wtedy, kiedy koszty zdobycia i~przetworzenia informacji są
zbyt duże. Jak zauważył \mbox{Knight}: ,,Jest ewidentnym, że racjonalną rzeczą jest bycie irracjonalnym, kiedy
przezorność i~oszacowanie przynoszą wyższe koszty niż są one warte''
(\cite{knight_risk_1921}, s.~67 [pierwsze wydanie 1864]).
%\label{ref:RNDPVtG6fufvQ}(Knight, 1921, s.~67 [pierwsze wydanie 1864]).
Również Hayek doceniał wagę przeczucia, kiedy to podejmujemy natychmiastowe decyzje bez długich
kalkulacji: ,,Gdybyśmy przestali wykonywać czynności, dla których nie znamy przyczyn lub nie mamy uzasadnienia… wkrótce
bylibyśmy martwi''
\parencite[s.~68]{hayek_fatal_1988}.
%\label{ref:RNDywloH9eKxc}(Hayek, 1988, s.~68).
Dzisiaj heurystyki -- uproszczone reguły myślenia,
których ludzie często używają, by formułować sądy i~podejmować decyzję -- nie są postrzegane jako irracjonalne, ponieważ
ludzie nie mieliby wystarczającej ilości czasu, by przetworzyć każdą dostępną informację. Gdybyśmy musieli zastanawiać
się nad każdym naszym wyborem, to nie bylibyśmy w~stanie funkcjonować, bo każdego dnia musimy podejmować tysiące
decyzji. Jest to jedna z~konkluzji zaprezentowanych przez Kahnemana
\parencite*{kahneman_pulapki_2012}.
%\label{ref:RNDKtF2hlSGCP}(2012).
Udowodnił on,
między innymi, że szybki system 1 jest bardziej narażony na błędy niż rozważny system 2, który jest
utożsamiany z~\textit{homo oeconomicus}. Jednakże te dwa systemy są współzależne ze sobą i~nie możemy
polegać tylko na systemie 2, bo
w~większości wypadków potrzebujemy podjąć szybkie decyzje. Najbardziej znanym ekonomistą, który podważył założenia
racjonalności doskonałej, był Simon. Argumentował on, że ludzie nie maksymalizują, a~dokonują wyborów
satysfakcjonujących: ,,Podczas gdy człowiek ekonomiczny maksymalizuje -- wybiera najlepszą z~dostępnych opcji,
administracyjny człowiek satysfakcjonuje -- wybiera opcję, które jest satysfakcjonująca lub dość dobra''
\parencite[s.~XXIX]{simon_administrative_1947}.
%\label{ref:RNDJ92bWCjQ0i}(Simon, 1947, s.~XXIX).
Ludzie nie maksymalizują swojej użyteczności, ale dokonują wyborów
satysfakcjonujących z~powodu ograniczonej racjonalności. Ludzie mają ograniczoną racjonalność m.in. dlatego, że
brakuje im czasu, by przetworzyć wszystkie istotne informacje, a~ich kognitywne zdolności też mają swoje ograniczenia.
Pomimo tej krytyki, idea Simona została potraktowana przez wielu ekonomistów jako próba dopasowania \textit{homo
oeconomicus} do rzeczywistości, a~nie jako próba jego odrzucenia. Ograniczona racjonalność nie odrzuca `twardego rdzenia'
(racjonalności instrumentalnej), bo w~podejściu ekonomicznym racjonalność jest zawsze ograniczona.

Drugim sposobem radzenia sobie z~irracjonalnymi zachowaniami jest zmiana znaczenia samego pojęcia racjonalności.
W~powszechnej opinii racjonalność oznacza mądry i~przemyślany wybór, ale ekonomia neoklasyczna jest wolna od
wartościowania i~dlatego nie dyskutuje na temat ludzkich celów. Zakłada ona, że nie da się racjonalnie ustalić, czego
ludzie powinni chcieć. Właśnie dlatego ekonomiści tylko sprawdzają, czy ludzie spełniają swoje preferencje w~efektywny
sposób. Hume zaproponował najbardziej znaną definicję racjonalności instrumentalnej: ,,rozum sam nie może nigdy być
motywem żadnego aktu woli''
(\cite{hume_traktat_1963}, s.~186 [pierwsze wydanie 1739])
%\label{ref:RNDCm37xZcwYC}(Hume, 1963, s.~186 [pierwsze wydanie 1739])
i~,,rozum
jest i~powinien być tylko niewolnikiem namiętności''
\parencite[s.~188]{hume_traktat_1963}.
%\label{ref:RNDSPXtCyDIC0}(Hume, 1963, s.~188).
Ekonomiści neoklasyczni
potraktowali te słowa bardzo poważnie i~racjonalność w~ekonomii nie dotyczy wyników, a~samego procesu. Nie ma znaczenia
co ludzie wybierają, ale czy proces wyboru spełnia warunki przechodniości i~kompletności. 

\subsection{2.2. Problemy z~racjonalnością instrumentalną i~maksymalizacją użyteczności}
Pas ochronny przyjęty przez Beckera chroni twardy rdzeń metody ekonomicznej. Jednak doprowadza to do różnego rodzaju
problemów. Zacznijmy od racjonalności instrumentalnej, która jest przedstawiona przez Beckera na przykładzie
racjonalnego uzależnienia. W~\textit{Teorii racjonalnego uzależnienia} Becker i~Murphy piszą: 

\myquote{
Jednak, tak jak wskazuje tytuł naszego artykułu, twierdzimy że uzależnienia, nawet te mocne, są zazwyczaj racjonalne
w~sensie maksymalizacji dotyczącej długiego okresu wraz z~stałymi preferencjami. Nasze twierdzenie jest nawet mocniejsze:
racjonalny model pozwala na dogłębne zrozumienie zachowań uzależniających
\parencite[s.~675]{becker_theory_1988}.
%\label{ref:RNDwyBYyBCayi}(Becker, Murphy, 1988, s.~675). 
}

Używki są tematem, w~którym Becker posuwa teorię racjonalnego wyboru do logicznego ekstremum. Robi to, ponieważ chce
sprawdzić, czy jego podejście ekonomiczne może działać. Celem ich artykułu jest udowodnienie, że uzależnieni ludzie nie
są irracjonalni, a~ich zachowania wcale nie są żadnym odstępstwem od racjonalności. Co więcej, Becker i~Murphy
twierdzą, że korzystanie z~używek zwiększa użyteczność uzależnionego. Takie pojmowanie jest możliwe, ponieważ
Becker i~Murphy używają koncepcji racjonalności instrumentalnej: 

\myquote{
Nasz artykuł polega na słabej koncepcji racjonalności, która nie wyklucza silnego dyskontowania przyszłych wydarzeń.
Konsument w~naszym modelu staje się coraz bardziej krótkowzroczny, kiedy preferencje dotyczące teraźniejszości stają
się większe
\parencite[s.~683]{becker_theory_1988}.
%\label{ref:RNDMFgyvOuGqK}(Becker, Murphy, 1988, s.~683).
}
To oznacza, że zażywanie narkotyków jest racjonalne, ponieważ przyjemność z~ich użytkowania jest tak ogromna, że
przeważa nad przyszłymi kosztami. Dlatego branie narkotyków może maksymalizować użyteczność. Może wydawać się dziwnym
stwierdzenie, że uzależniony od heroiny jest racjonalny, jeżeli spojrzymy na skutki uzależnienia. Jednakże nie stanowi
to problemu, ponieważ ekonomia wolna od wartościowania nie zakłada, co jest dobre, a~co złe dla ludzi. Natomiast
zakłada, że ludzie maksymalizują swoją użyteczność w~momencie wyboru. W~przypadku narkotyków ludzie dokonują wyboru
pomiędzy teraźniejszą i~przyszłą użytecznością. Becker zakłada, że słaba wola (\textit{akrasia}) nie występuje i~ludzie
są w~stanie jakoś porównać długookresowe cele (zdrowie) z~krótkookresowymi przyjemnościami (palenie). Jako ekonomiści
nie wiemy, jak ludzie to robią (nie jesteśmy zainteresowani ich motywacjami). Po prostu zakładamy, że potrafią to
zrobić (założenie \textit{as if}). 

W podejściu ekonomicznym przekonanie o~tym, że ludzie maksymalizują swoją użyteczność jest bardzo mocne. Nawet jeżeli
ludzie żałują swojej decyzji, podjętej pod wpływem chwili, to nie znaczy wcale, że nie wybrali tego co chcieli. Becker
i~Murphy twierdzą, że:

\myquote{
Zapewnienia niektórych alkoholików i~palaczy, że chcą, ale nie potrafią zakończyć swoich uzależnień, wydaje się nam nie
różnić od zapewnień singli, którzy chcą, ale nie są w~stanie wziąć ślubu, albo od zapewnień niezorganizowanej osoby,
która chce być bardziej zorganizowana. Te zapewnienia znaczą, że dana osoba dokona pewnych zmian -- dla przykładu,
weźmie ślub lub przestanie palić -- w~momencie, w~którym znajdzie sposób, by zwiększyć długookresowe korzyści ponad
krótkoterminowe koszty zmiany swojego zachowania
\parencite[s.~693]{becker_theory_1988}.
%\label{ref:RNDuYOkuRkqmV}(Becker, Murphy, 1988, s.~693).
}

Dla Beckera deklaracje nic nie znaczą. Jeżeli powstrzymanie się od papierosów da ci więcej użyteczności niż palenie, to
je rzucisz. Problem z~ludzkimi deklaracjami jest widoczny na przykład, kiedy ludzie zapytani o~programy, jakie chcą
oglądać w~telewizji, wskazują spektakle, teatr lub operę, ale kiedy przychodzi do rzeczywistego wyboru, to oglądają
opery mydlane. Ekonomiści nie wierzą ludzkim deklaracjom, a~w~ostateczności liczą się czyny nie słowa.

Problemem z~pojmowaniem racjonalności z~perspektywy Beckera są sytuacje w~którym trudno uznać, by ludzie robili coś, bo
tego naprawdę chcieli. Co prawda, jeżeli zastanowimy się nad tym, jak wiele przyjemności można osiągnąć z~konsumpcji
narkotyków, wtedy możemy się zgodzić z~teorią Beckera, w~której pozornie nieracjonalne zachowanie, jak branie
narkotyków, może maksymalizować użyteczność. Powstaje jednak pytanie, czy takie pojmowanie racjonalności ma jakikolwiek
sens. W~podejściu ekonomicznym były alkoholik, który nie chce pić, ponieważ wie, gdzie go to doprowadzi, jest
racjonalny, jeżeli wróci do picia. Taka osoba nie chce pić i~po popijawie czuje wstręt do swojego zachowania. Czy
jest możliwe, by chwilowa potrzeba, by wrócić do alkoholu, była ważniejsza niż ludzkie cele i~motywacje? Becker
zakłada, że ludzie są w~stanie zintegrować te przeciwstawne motywacje i~sprowadzić je do wspólnego mianownika
użyteczności. Takie pojmowanie racjonalności powoduje, że łatwo dojść do absurdalnych wniosków: 

\myquote{
Wyobraź sobie sytuację, w~której z~powodu katastrofy morskiej człowiek leży na małej łódce pośrodku
oceanu i~myśli o~tym, by napić się wody z~oceanu. On wie, że nie powinien pić tej wody, bo to tylko spowoduje,
że będzie czuł się
gorzej. Pomimo tej wiedzy, ulega pokusie i~pije wodę. Rozchorowuje się i~w~ostateczności umiera. Czy jest rozsądnym
stwierdzenie, że ta osoba zachowała się racjonalnie? Czy impuls, sekundowa użyteczność jest wystarczająca, by nazwać
takie zachowanie racjonalnym?
\parencite[s.~11]{ostapiuk_human_2018}
%\label{ref:RNDeTyr7vEpMG}(Ostapiuk, 2018, s.~11)
}

Powyższe przykłady poruszają sprawę tego, czy powinniśmy rozróżniać pomiędzy racjonalną i~przemyślaną
decyzją a~biologiczną potrzebą, pomiędzy długookresowymi celami a~prostymi przyjemnościami? Oczywiście powyższe przykłady są
ekstremalne, ale pytania z~nich wynikające dotyczą wszystkich ludzkich zachowań. Ludzie często wybierają pomiędzy teraz
a~później: konsumuj teraz lub oszczędzaj na później, miej przyjemność teraz lub pracuj ciężko teraz i~miej więcej
przyjemności później. Becker zakłada, że ludzie zawsze wybierają rzeczy, które są dla nich dobre. Jednak, kiedy
spojrzymy na decyzje w~czasie, trudno przyznać mu rację. 

Przejdźmy teraz do problemów z~maksymalizacją użyteczności. Jak już było wspomniane, ekonomia wolna od wartościowania nie
zajmuje się ludzkimi celami i~pojmuje użyteczność \textit{ad libitum}. To relatywistyczne podejście powoduje, że
ekonomia staje się niewrażliwa na różnice w~ludzkich motywacjach. Jest to widoczne, kiedy altruistyczne i~egoistyczne
zachowania są włożone do tej samej czarnej skrzynki użyteczności. To oznacza, że ekonomiści nie są w~stanie
zauważyć odmienności pomiędzy tymi dwoma różnymi motywacjami (obie mogą maksymalizować użyteczność w~ekonomii
neoklasycznej). Tak szerokie pojmowanie użyteczności powoduje, że ekonomia wolna od wartościowania nie jest w~stanie
dostrzec różnicy pomiędzy żołnierzem, który skacze na granat, by ocalić towarzyszy, od żołnierza, który popycha na
granat innego żołnierza, by ocalić siebie samego. W~obu sytuacjach żołnierz maksymalizuje swoją użyteczność i~to
wszystko co na ten temat może powiedzieć ekonomia. 

Relatywność (wolność od wartościowania) metody ekonomicznej powoduje, że ekonomiści zajmują się tylko konsekwencjami
akcji. Dlatego (jak w~utylitaryzmie) ekonomia neoklasyczna nie uznaje żadnych wartości, które są niezależne od miar
użyteczności i~mają wartość samą w~sobie (deontologia). Jednakże różnica pomiędzy wartościami a~maksymalizacją
użyteczności opartą na własnym interesie jest zasadnicza. Sen, by pokazać te różnice, wprowadza pojęcia
\textit{sympathy}\footnote{U Sena można zauważyć wiele odniesień do Smitha, dla którego uczucie \textit{sympathy} było
jedną z~podstaw \textit{Teorii uczuć moralnych}
(\cite{smith_teoria_1989_ost} [pierwsze wydanie 1759]).
%\label{ref:RNDjAm6Ebftd1}(Smith, 1989 [pierwsze wydanie 1759]).
}
i~\textit{commitment}:

\myquote{
Pierwsze z~nich odwołuje się do sytuacji, w~której nasze zainteresowanie innymi bezpośrednio wpływa na nasz dobrobyt.
Jeżeli wiedza o~tym, że inni są torturowani, powoduje, że czujesz się niedobrze, to jest to przypadek
\textit{sympathy}; jeżeli nie czujesz się gorzej osobiście z~tego powodu, ale uważasz, że to coś złego, i~jesteś
gotowy, by coś zrobić, aby to zakończyć, jest to \textit{commitment}
\parencite[s.~326]{sen_rational_1977}.
%\label{ref:RNDx8dBiYKdvY}(Sen, 1977, s.~326). 
}

Konsekwencjonalizm metody ekonomicznej nie doprowadza tylko do tego, że ekonomia nie jest w~stanie odróżnić
egoistycznego zachowania od altruistycznego. Problem jest nawet większy. Ekonomia nie może być tylko skupiona na akcie
wyboru, ponieważ kontekst wyboru jest równie ważny\footnote{%
\parencite[s.~130]{sen_rationality_2002}
%\label{ref:RND05oEXp3pfn}(Sen, 2002, s.~130)
w~koncepcji
`\textit{menu-dependence}' i~\parencite{kahneman_prospect_1979}
%\label{ref:RNDoaskF9OWJH}(Kahneman, Tversky, 1979)
w~`teorii perspektywy' udowadniają, jak
ważny jest kontekst wyboru.}. Sen
\parencite*[s.~75]{sen_development_1999}
%\label{ref:RNDxD0rEWCw3J}(1999, s.~75)
podaje przykład człowieka, który jest głodny,
przy czym w~jednym przypadku jest głodny z~braku dostępu do jedzenia, a~w~drugim jest głodny z~powodów religijnych
(post). Ekonomia neoklasyczna nie jest w~stanie zauważyć różnicy pomiędzy tymi dwoma sytuacjami, ponieważ nie
interesuje się ludzkimi motywacjami ani wartościami.

Na koniec analizy działania podejścia ekonomicznego warto zwrócić uwagę na pewną problematyczną kwestię dla części
ekonomistów, czy innych przedstawicieli nauk społecznych niechętnych metodzie ekonomicznej. Część z~nich
krytykuje Beckera za nierealistyczność założeń dotyczących \textit{homo oeconomicus}. Według nich, człowiek ekonomiczny
przedstawiony przez Beckera jest absolutnie racjonalny i~egoistyczny, dążąc do maksymalizacji swojej użyteczności, co
nie jest zgodne z~obserwowalnymi faktami. Becker jednak poprzez pojmowanie racjonalności w~perspektywie instrumentalnej
sprawia, że każde zachowanie można uznać za racjonalne. Co więcej, Becker pojmuje samo pojęcie użyteczność \textit{ad
libitum}, co oznacza, że ,,wrzucił altruizm, empatię i~poczucie sprawiedliwości do `kotła bez dna' maksymalizacji''
\parencite[s.~74–75]{ostapiuk_moralna_2017}.
%\label{ref:RND1ueyRVx8jm}(Ostapiuk, 2017b, s.~74–75).
Takie pojmowanie użyteczności i~racjonalności sprawia,
że w~podejściu ekonomicznym, każde zachowanie człowieka spełnia wymogi (przyjęte przez Beckera)
racjonalności i~maksymalizacji użyteczności. Oznacza to, że podejście ekonomiczne jest zbudowane na aksjomatycznych założeniach, które
są tautologiami, dzięki którym model zawsze działa, a~ludzie nie są w~stanie nie maksymalizować swojej użyteczności,
bądź zachowywać się nieracjonalnie. Dlatego nie ma sensu krytykować założeń ustanowionych w~podejściu
ekonomicznym pod względem ich realistyczności deskryptywnej. 

\section{Beckera ucieczka od tautologii i~problemy z~niej wynikające}
Celem analizy metody ekonomicznej było wykazanie, że dzięki aksjomatycznym założeniom ekonomia wolna od wartościowania
jest w~stanie wchłonąć wszelką krytykę ze strony innych nauk społecznych, pozostawiając paradygmat bez zmian. Jednak
rozszerzenie `pasa ochronnego' w~celu obrony `twardego rdzenia' metody ekonomicznej doprowadza do pewnego problemu.
Poprzez rozszerzenie pojmowania koncepcji racjonalności i~użyteczności Becker sprawił, że te pojęcia mogą oznaczać
dosłownie wszystko. Dzięki temu metoda ekonomiczna jest w~stanie odpowiedzieć na każdą krytykę twardego rdzenia, ale za
cenę stania się tautologią. Blaug zauważył, że Becker zawsze był w~stanie wprowadzić dodatkowe, \textit{ad hoc}
założenia (jak instrumentalna racjonalność, czy użyteczność \textit{ad libitum}), które wyjaśniałaby wszelkie
anomalie. Blaug skrytykował to podejście nazywając je ,,adhokerią''
\parencite[s.~325–326]{blaug_metodologia_1995}.
%\label{ref:RNDQJu4oYX8Hr}(Blaug, 1995, s.~325–326).
Z~powodu wcześniej wymienionych problemów można uznać metodę ekonomiczną za program degeneracyjny, ponieważ pomocnicze
założenia nie mają sprawić, by teoria lepiej opisywała nowe fakty, ale istnieją tylko po to, by obronić twardy rdzeń. 

Becker nie chciał, żeby jego teoria była tautologią, dlatego postanowił wprowadzić założenie o~stałych preferencjach.
Jest ono analizowane jako `negatywna heurystyka', która ma za zadanie sprawić, by metoda ekonomiczna funkcjonowała
(takie same postrzeganie w~\parencite[s.~323–324]{blaug_metodologia_1995}).
%\label{ref:RNDJZzV5wRELy}(Blaug, 1995, s.~323–324)).
Becker miał dwa ważne powody, by
wprowadzić założenie o~stałych preferencjach. Po pierwsze, był on spadkobiercą pozytywizmu i~ekonomii wolnej od
wartościowania. Dlatego uznawał on, że nie możemy dyskutować o~gustach i~preferencjach. Tą niechęć do badania ludzkich
motywów można zauważyć, analizując tytuł jednego z~najważniejszych artykułów Beckera \textit{De Gustibus Non Est
Disputandum} (o gustach się nie dyskutuje). W~tym artykule Becker podkreślał wartość stałych preferencji: ,,Osoba nie
spiera się o~gusta z~tego samego powodu, dla którego nie spiera się o~Góry Skaliste -- one tam są, będą
również w~przyszłym roku i~są takie same dla wszystkich ludzi''
\parencite[s.~76]{becker_gustibus_1977}.
%\label{ref:RNDsivURl8HaZ}(Becker, Stigler, 1977, s.~76).
Becker
był świadomy tego, że upraszcza rzeczywistość poprzez założenie o~stałych preferencjach. Jednak to uproszczenie było
niezbędne, ponieważ Becker nie chciał być uwikłany w~niewykonalną analizę ludzkich motywacji. Według niego, preferencje
nie są empirycznie testowalne. Becker twierdził, że ,,żadne istotne zachowanie nie zostało wyjaśnione dzięki
założeniu o~różnicach w~gustach''
\parencite[ s.~89]{becker_gustibus_1977}.
%\label{ref:RNDja1EkIrzUn}(Becker, Stigler, 1977, s.~89).
Z~tych powodów, metoda ekonomiczna nie
potrzebuje żadnych informacji o~gustach, żeby działać. 

Drugim powodem wprowadzenia założenia o~stałych preferencjach była potrzeba siły predykcyjnej. Becker twierdzi, że
,,Założenie stałości preferencji umożliwia przewidywanie reakcji na rozmaite zmiany i~chroni badacza przed pokusą
przyjmowania -- w~celu \guillemotleft wytłumaczenia\guillemotright\ 
ewentualnej rozbieżności między jego prognozami a~biegiem wypadków -- że
nastąpiło odpowiednie przesunięcie preferencji''
\parencite[s.~23]{becker_ekonomiczna_1990}.
%\label{ref:RNDC0sM3Z4Pzd}(Becker, 1990, s.~23).
By udowodnić, że
założenie o~stałych preferencjach jest ważne, Becker analizuje odwrotne założenie -- gusta są zmienne i~podatne na
zmianę. W~przypadku anomalii wystarczyłoby wtedy stwierdzić, że nastąpiła zmiana w~gustach. Kończyłoby to problem, nie
zostawiając żadnej innej ewentualności w~grze. Zmiany w~gustach tłumaczyłyby i~wyjaśniałyby wszystkie anomalie, a~wiemy,
że teoria, która wyjaśnia wszystko, nie wyjaśnia niczego
\parencite{popper_logic_1959}.
%\label{ref:RND8pPmr0J2ry}(Popper, 2002).
Becker nie chciał,
żeby jego podejście ekonomiczne stało się pseudonauką, jak teorie Freuda czy Marksa. Dlatego założył istnienie stałych
preferencji. Według Beckera to założenie rozwiązywało wcześniejsze problemy ekonomii neoklasycznej, w~której wszystko
można było rozwiązać za pomocą różnych cen i~dochodów w~wyjaśnianiu ludzkich zachowań
\parencite{becker_gustibus_1977}.
%\label{ref:RND67wLUGQsl2}(Becker, Stigler, 1977).

Jednak założenie o~stałych preferencjach ma jedną wadę. Zaprzecza ono obserwowalnym faktom. Preferencje zmieniają się
wraz z~czasem i~to powoduje, że podejście ekonomiczne nie jest w~stanie sobie poradzić z~wyborami w~czasie. Za każdym
razem, kiedy ludzie wybierają pomiędzy teraźniejszością a~przyszłością, metoda ekonomiczna nie jest w~stanie
stwierdzić, co ludzie powinni wybrać. Z~powodu hiperbolicznego dyskontowania, według wielu naukowców zasadnym jest
podział człowieka na wiele osobowości. Wynika to z~tego, że ludzie mają zupełnie inne preferencje
teraz, a~inne w~przyszłości i~nie da się ich sprowadzić do jednego mianownika użyteczności.
Dlatego zasadnym wydaje się podział na
,,krótkookresowego człowieka'' i~,,długookresowego człowieka''
\parencites{ostapiuk_human_2018}[więcej na temat koncepcji wielu osobowości][]{ainslie_breakdown_2001}{cowen_self-constraint_1991}{davis_theory_2003}%
{elster_multiple_1986}{frederick_time_2002}{fudenberg_dual-self_2006}{heilmann_rationality_2010-1}%
{loewenstein_out_1996}{schelling_intimate_1980,schelling_self-command_1984,schelling_enforcing_1985,schelling_coping_1996}%
{read_which_2006}{strotz_myopia_1955}{thaler_economic_1981}.
%\label{ref:RND8sohQkGHAy}(Ostapiuk, 2018; więcej na temat koncepcji wielu osobowości Ainslie, 2001;
%Cowen, 1991; Davis, 2003; Elster, 1986; Frederick, i~in., 2002; Fudenberg,
%Levine, 2006; Heilmann, 2010; Loewenstein, 1996; Schelling, 1980, 1984, 1985, 1996; Read, 2006; Strotz, 1955; Thaler,
%Shefrin, 1981).
Ekonomia wolna od wartościowania nie decyduje o~tym, co ludzie powinni cenić i~co daje im największą
użyteczność. Jednakże ta wolność od wartościowania jest tylko pozorna, ponieważ ekonomiści normatywnie zakładają, że
ludzie maksymalizują swoją użyteczność. W~rzeczywistości oznacza to, że ekonomiści wspierają `krótkookresowego
człowieka', ponieważ ludzie podlegają hiperbolicznemu dyskontowaniu i~są zbytnio (irracjonalnie) skupieni na
teraźniejszości i~krótkookresowych przyjemnościach.

Pomimo krytyki, metoda ekonomiczna nie może przyznać, że ludzkie preferencje zmieniają się wraz z~czasem, bo bez
założenia o~stałych preferencjach nie miałaby żadnej siły predykcyjnej. Z~powodu wolności od wartościowania metoda
ekonomiczna nie może powiedzieć nic o~ludzkich preferencjach przed tym, jak wybór zostanie podjęty. Przed wyborem może
tylko stwierdzić, że ludzie maksymalizują swoją użyteczność i~zachowają się racjonalnie. Jednak z~powodu
tautologiczności tych założeń nie oznacza to niczego (albo oznacza wszystko). W~ostateczności, to po prostu bardziej
wyszukany sposób stwierdzenia, że ludzie robią coś, bo robią coś. Bez założenia o~stałych preferencjach metoda
ekonomiczna mogłaby analizować tylko wybory \textit{post factum}. Wtedy możemy tylko stwierdzić, że Marek kupuje
samochód, bo dzięki temu maksymalizuje swoją użyteczność, ale nie jesteśmy w~stanie przewidzieć, co on zrobi. 

W artykule wykazano, że założenia o~stałych preferencjach i~racjonalności instrumentalnej są stworzone, by uratować
twardy rdzeń metody ekonomicznej. Jednak wielu ekonomistów uznawało te techniczne założenia za opis rzeczywistości i~na
ich podstawie wyciągało wnioski dotyczące przyszłości. Thaler argumentował, że:

\myquote{
Ekonomiści rzadko wyznaczają granicę pomiędzy normatywnymi modelami wyboru konsumenta a~deskryptywnymi czy pozytywnymi
modelami. Pomimo tego, że teoria jest normatywna (opisuje, co racjonalni konsumenci powinni robić) ekonomiści
argumentują, że służy ona również jako deskryptywna teoria (przewiduje ona, co konsumenci w~rzeczywistości robią). Ten
artykuł argumentuje, że wyłączne poleganie na normatywnej teorii prowadzi ekonomistów do popełniania systematycznych,
przewidywalnych błędów w~opisywaniu i~przewidywaniu zachowań konsumentów
\parencite[s.~39]{thaler_toward_1980}.
%\label{ref:RND2pIZU6Ltta}(Thaler, 1980, s. 39).
}
Dzięki ekonomii behawioralnej coraz mniej naukowców wierzy w~siłę predykcyjną ekonomii neoklasycznej. Jednak nadal wielu
ekonomistów uznaje założenia ekonomii wolnej od wartościowania, co ma negatywny wpływ na rzeczywistość. 

\section{Negatywny wpływ założeń ekonomii wolnej od wartościowania na ludzi i~społeczeństwo}
Po II wojnie światowej z~powodu aksjomatycznych założeń ekonomii wolnej od
wartościowania i~wiary w~pozytywizm i~obiektywność ekonomii jako nauki,
ekonomiści zaczęli przypominać naukowców z~książki \textit{Gra szklanych paciorków}
Hermanna Hesse, którzy żyją w~wieży z~kości słoniowej i~nie interesują się rzeczywistością. Blaug był jedną z~osób,
która zauważyła ten problem: ,,ekonomia coraz bardziej staje się intelektualną grą wykonywaną dla samej siebie, a~nie dla
praktycznych konsekwencji potrzebnych do zrozumienia ekonomicznego świata''
\parencite[s.~3]{blaug_ugly_1997}.
%\label{ref:RNDBzaEcqW02L}(Blaug, 1997, s.~3).
Ekonomiści tworzą teorie dla tworzenia teorii.
Nie interesuje ich, czy są one prawdziwe i~czy dobrze opisują rzeczywistość.
Jedyną rzeczą, która się liczy, jest stworzenie teorii, które są zamkniętymi aksjomatycznymi
systemami i~logicznie działają\footnote{W ostatnich latach można zaobserwować `zwrot empiryczny'
(\textit{empirical turn}) w~ekonomii.
Jednakże jak wskazują
\parencite{backhouse_age_2017},
%\label{ref:RNDH5LLc3nVg6}(Backhouse, Cherrier, 2017)
nie oznacza to wcale, że ekonomia
pozbyła się teorii.}. Aksjomatyczne założenia, które zawsze działają, doprowadziły do pychy części ekonomistów, którzy
uwierzyli w~swoją naukowość i~dlatego w~dużej mierze odrzucili metodologiczny pluralizm jak i~normatywne ujęcia.

Szczególnie odrzucenie badania normatywnych koncepcji zajmujących się szczęściem, doprowadziło do negatywnych
konsekwencji dla ludzi. Wielu ekonomistów nadal traktuje racjonalność instrumentalną, Optimum Pareto czy koncepcję
efektywności jako pozytywne, a~nie normatywne koncepcje. Rezultatem tego stanu rzeczy jest traktowanie dobrobytu przez
ekonomistów jako pozytywnej koncepcji. Jest to błędne myślenie, ponieważ nie istnieje żadna deskryptywna teoria
dobrostanu (\textit{well-being})\footnote{Dobrostan jest to koncepcja, która w~holistyczny sposób podchodzi do potrzeb
ludzi i~ich zadowolenia z~życia. Jest dużo szerszą koncepcją niż ekonomiczny dobrobyt. }. Nie można dyskutować na temat
dobrostanu bez używania pewnych sądów wartościujących. Wielu ekonomistów tego nie zauważa, ponieważ nie używają oni
substancjalnych, a~formalnych teorii dobrostanu, które nie mówią o~tym, co jest dobre ostatecznie, ale podają metodę,
dzięki której można dowiedzieć się, co jest dobre dla ludzi
\parencite[s.~245]{hausman_etyka_2017_ost}.
%\label{ref:RNDtxYR4MO4C9}(Hausman, i~in., 2017, s.~245).
Według ekonomistów powinniśmy poczekać i~zobaczyć, co ludzie wybiorą (preferencje ujawnione). Co więcej, ekonomiści
zakładają, że ludzie wybierają to, co jest dla nich najlepsze. Z~tego powodu: ,,Ekonomiści nie powinni mieć żadnego
substancjalnego poglądu na temat koncepcji dobra. Ale to oczywiście oznacza wydanie sądu wartościującego i~popieranie
specyficznej teorii dobrostanu: mianowicie, dobrostan jest tym, czego ludzie pragną''
\parencite[s.~214]{reiss_philosophy_2013}.
%\label{ref:RNDdqRB8v3kjQ}(Reiss, 2013, s.~214).
Taki pogląd ma negatywny wpływ na ludzi, ponieważ ekonomia szczęścia demonstruje, że preferencje
ujawnione nie są dobrym wskaźnikiem szczęścia/dobrostan
\parencites{bruni_economics_2005,bruni_handbook_2007}{bruni_capabilities_2008}%
{kahneman_well-being:_1999}{frey_what_2002}{frey_happiness_2010,frey_economics_2018}.
%\label{ref:RNDoK5kQfczvx}(Bruni, Porta, 2005, 2007; Bruni, i~in., 2008; Kahneman, i~in., 1999; Frey, Stutzer, 2002; Frey, 2010, 2018).
Ta teoria ma bardzo duży wpływ na
społeczeństwo, ponieważ dzisiejszy system kapitalistyczny jest zbudowany na założeniu, że wystarczy dać ludziom
możliwości wyboru, a~oni wybiorą dla siebie najlepszą opcję
\parencite[zob.][]{friedman_free_1980}.
%\label{ref:RNDONMcgSYdkf}(zob. Friedman, Friedman, 1980).

Problemy z~aksjomatyczną koncepcją preferencji ujawnionych są szczególnie widoczne podczas wyborów
dokonywanych w~czasie. Tradycyjną koncepcją wykorzystywaną przez ekonomistów neoklasycznych jest model zdyskontowanej użyteczności
\parencite{samuelson_note_1937},
%\label{ref:RND1MrIXZkQv8}(Samuelson, 1937),
w~którym ludzie potrafią porównywać użyteczność w~czasie, podejmując
najlepszą dla siebie decyzję. Te teoretyczne założenia mają potencjalne problemy, które autor
zauważył i~zaadresował w~swoim artykule
\parencite{samuelson_note_1937}.
%\label{ref:RNDR4qzlsq1en}(Samuelson, 1937).
Ekonomiści neoklasyczni nie wgłębiali się jednak w~ograniczenia modelu,
który wkrótce stał się standardowym modelem dotyczącym decyzji w~czasie, mającym wpływ na
kształtowanie się rzeczywistości. Modigliani
\parencite*{modigliani_life_1966}
%\label{ref:RNDOsjUpQ70ZT}(1966)
jest najlepszym przykładem sposobu, w~jaki
ekonomiści wykorzystywali model zdyskontowanej użyteczności w~rzeczywistości. Zbudował on swój model, który nazwał
`hipotezą cyklu życia', na całkowitym dochodzie indywidualnym. W~tej teorii ludzie są racjonalni i~w~młodości tworzą
plan, dzięki któremu mogą `wyrównać' konsumpcję w~perspektywie całego życia. Co więcej, hipoteza cyklu życia nie tylko
zakłada, że ludzie są zdolni, by dokonać wszystkich kalkulacji (z racjonalnymi oczekiwaniami) dotyczących tego, jak
długo będą żyli, jak wiele zarobią etc. Ludzie w~tej teorii również posiadają samokontrolę, która jest niezbędna, by
wprowadzić w~życie optymalny plan. Nie powinno zaskakiwać, że z~czasem ekonomiści zaobserwowali coraz większą liczbę
przypadków, które pokazywały, że ludzie nie potrafią dyskontować swojej użyteczność (hiperboliczne dyskontowanie).

\enlargethispage{.5\baselineskip}

Warto zastanowić się nad tym, dlaczego ekonomiści trzymali się modelu zdyskontowanej użyteczności, pomimo tego, że nie
przystawał on do rzeczywistości i~nie opisywał prawdziwych zachowań ludzi. Ta kwestia jest niezwykle istotna, ponieważ
wielu współczesnych ekonomistów twierdzi, że w~ekonomii zawsze wiedziano o~tym, że człowiek nie jest w~pełni
racjonalny, a~model \textit{homo oeconomicus} był tylko koniecznym uproszczeniem rzeczywistości, a~nie opisem
świata\footnote{Takie wnioski można wysnuć, obserwując seminarium ,,Ekonomia behawioralna a~ekonomia głównego nurtu''
zorganizowane przez Radę Naukową Polskiego Towarzystwa Ekonomicznego wraz z~Komitetem Nauk Ekonomicznych PAN
\parencite[zob.][]{noauthor_stenogram_2018}.
%\label{ref:RNDQQgZxzHokl}(zob. Stenogram z~otwartego  seminarium  Rady  Naukowej  Polskiego  Towarzystwa Ekonomicznego
%zorganizowanego  wspólnie  z~ Komitetem  Nauk  Ekonomicznych  PAN pt.  ,,Ekonomia behawioralna a~ekonomia głównego
%nurtu'', 2018).
}. Nie do końca jest to prawda. Należy zwrócić uwagę na tworzenie wielu modeli przez
ekonomistów, w~którym fundamentem były założenia o~pełnej racjonalności. Co ważniejsze, często nie były to tylko modele, którymi
zajmowali się teoretycy, ale były one wykorzystywane w~realnym świecie. Szczególnie dotyczy to czasów, gdy
neoliberalizm święcił swoje triumfy, a~ludzie właśnie byli przedstawieni jako racjonalni agenci, którzy sami wiedzą, co
jest dla nich najlepsze. Celem artykułu nie jest próba ocenienia, ile winy należy przypisać ekonomistom za takie
postrzeganie świata. Ważniejszy wniosek został przedstawiony przez Sedláčka
\parencite*{sedlacek_ekonomia_2012_ost},
%\label{ref:RNDLlXH8jkVRO}(2012),
który
argumentuje, że wielu ekonomistów zapomniało o~tym, że człowiek ekonomiczny to tylko założenie (wiara), a~zaczęli go
traktować jako rzeczywistość (pewną wiedzę). Abstrakcyjne teorie, które są używane przez ekonomistów (np. \textit{homo
oeconomicus}) kształtują jednak sposób, w~jakim patrzymy na świat. Ekonomiści dopasowują rzeczywistość do swoich teorii,
na co zwrócił uwagę Callon w~swojej koncepcji performatywności ekonomii
\parencites{callon_what_2006}[więcej zob.][]{boldyrev_enacting_2016}.
%\label{ref:RNDpxYYTpuwL3}(Callon, 2006; więcej zob. Boldyrev, Svetlova, 2016).
Teorie ekonomiczne również mają bardzo duży wpływ na ludzi, którzy się z~nimi
zapoznają. Na ten fakt zwrócili uwagę Hausman i~McPherson twierdząc, że ,,Uczenie ekonomii, jak się wydaje, sprawia, że
ludzie są bardziej egoistyczni''
\parencite[s.~674]{hausman_taking_1993}.
%\label{ref:RNDjH1c6RRty0}(Hausman, McPherson, 1993, s.~674).
Mając na względzie
przekonanie do swoich teorii, nie dziwi fakt, że kiedy w~latach osiemdziesiątych XX wieku zaczęła się rozwijać ekonomia
behawioralna, towarzyszyła jej negatywna reakcja ze strony ekonomistów neoklasycznych
\parencite[zob.][]{thaler_misbehaving:_2015}.
%\label{ref:RNDImIRdjN3wn}(zob. Thaler, 2015).
Do dziś istnieją ekonomiści, którzy nieprzychylnie patrzą na heterodoksyjne podejścia, które zagrażają
\textit{status quo}
\parencite[zob.][]{fourcade_superiority_2015,colander_how_2018}.
%\label{ref:RNDdVBmuEaZUL}(zob. Fourcade, i~in., 2015; Colander, Su, 2018).

\section*{Podsumowanie}
Celem artykułu było pokazanie na przykładzie metody ekonomicznej Beckera, że aksjomatyczność ekonomii wolnej od
wartościowania doprowadza do tego, że paradygmat ekonomii się nie zmienia, pomimo nieustającej od wielu lat krytyki.
Wynika to z~rozbudowy pasa ochronnego, w~którym znaczenie koncepcji racjonalności i~użyteczności zostało rozszerzone do
maksimum. Doprowadza to jednak do tautologizacji tych założeń. Pomimo swojej efektywności i~zgodności dedukcyjnej,
ekonomia wolna od wartościowania ma problemy, które są w~głównej mierze widoczne w~przypadku wyborów w~czasie i~analizy
ludzkiego dobrostanu.

Aksjomatyczne założenia same w~sobie nie są problemem. Stanowią one na przykład podstawę matematyki. W~większości
przypadków jednak matematycy akceptują aksjomaty, ponieważ są one użyteczne do wykazania pewnych zależności formalnych.
Jednak to, co działa w~matematyce, niekoniecznie działa w~ekonomii. Problemem jest to, że aksjomatyczne myślenie daje
ekonomistom epistemologiczną pewność siebie, co skutkuje niechęcią do pluralizmu metodologicznego i~normatywnych ujęć.
Co więcej, ekonomiści nie zawsze traktują aksjomatyczne założenia jako założenia techniczne, które mają być tylko
użyteczne. Część z~nich uwierzyła w~prawdziwość tych założeń, co prowadzi do opisanych negatywnych skutków.

Ekonomiści zwykle nie chcą dyskutować o~ludzkich celach i~motywacjach, bo normatywne założenia są traktowane przez wielu
za nienaukowe. Jednak aksjomatyczne założenia (racjonalność instrumentalna, maksymalizacja użyteczności, dobrobyt) są
również założeniami normatywnymi -- nazywane zostały przez Webera
\parencite*{weber_methodology_1949}
%\label{ref:RNDabw0XnkVT4}(1949)
`metodologicznymi
sądami wartościującymi'. W~ostateczności nie da się uciec od tych niefalsyfikowalnych założeń (twardy rdzeń), więc
naukowcy muszą je analizować.

Daleko idącym wnioskiem z~tego artykułu może być propozycja nowego paradygmatu, w~którym ekonomia zajmuje się
normatywnymi koncepcjami (badanie szczęścia i~ludzkich celów). Co prawda, są one bardziej
problematyczne i~ekonomia straci swoją efektywność (imperializm ekonomiczny), ale nowy paradygmat może okazać
się lepszym dla zwykłych
ludzi. Oczywiście, nie chodzi o~to, by wszyscy ekonomiści zajmowali się normatywnymi ujęciami. W~żadnym wypadku autor
artykułu nie chce dać do zrozumienia, że ekonomiści nie powinni zajmować się modelami i~predykcjami. Chodzi tylko o~to,
żeby zauważyć, że ekonomia pozytywna doprowadza do pewnych problemów i~wcale nie stoi na twardych epistemologicznych
podstawach patrząc z~perspektywy filozofii nauki
\parencite{kuhn_structure_1962,feyerabend_against_1975,lakatos_methodology_1980,caldwell_beyond_1982,mccloskey_rhetoric_1998,hands_reflection_2001}.
%\label{ref:RNDLBY1jCcG9N}(Kuhn, 1962; Feyerabend, 1975; Lakatos, 1980; Caldwell, 1982; McCloskey, 1998; Hands, 2001).
Wnioskiem autora nie jest argumentowanie za zmianą
paradygmatu w~ekonomii w~sensie Kuhna, która pasuje bardziej do nauk ścisłych niż społecznych.
Celem artykułu jest wskazanie potrzeby
otwarcia się na pluralizm metodologiczny. Samo wykorzystanie wiedzy z~innych nauk społecznych nie sprawia, że twardy
rdzeń ekonomii neoklasycznej się zmienił
\parencite[zob.][]{dow_foundations_2012}.
%\label{ref:RNDcOKPhFdauo}(zob. Dow, 2012).
Co więcej, może ono nawet
doprowadzić do wzrostu imperializmu ekonomii, co można zaobserwować na przykładzie ekonomii behawioralnej
\parencite{berg_as-if_2010}.
%\label{ref:RND4ocfrdXako}(Berg, Gigerenzer, 2010).

Na końcu wykorzystajmy metaforę Neuratha, w~której wiedza jest porównywana do statku na morzu. Neurath pisze, że
,,jesteśmy jak marynarze na otwartym morzu, którzy muszą zrekonstruować statek, ale nigdy nie są w~stanie zacząć na nowo
od spodu''
\parencite[s.~199]{neurath_anti-spengler_1973}.
%\label{ref:RNDFl7Blorjx2}(Neurath, 1973, s.~199).
Neurathowi chodzi o~to, że nie ma jednej fundamentalnej,
obiektywnej epistemologii, na której można zbudować system, który jest uodporniony na krytykę. Dlatego my, jako
ekonomiści, nie powinniśmy budować takiego fundamentalnego systemu, a~raczej powinniśmy otworzyć się na normatywne
koncepcje i~pluralizm metodologiczny. Oczywiście to już się dzieje i~coraz więcej ekonomistów zajmuje się nurtami
normatywnymi, ale wydaje się, że nadal wielu ekonomistów nie traktuje tych nurtów poważnie
(jako naukowych) i~wierzy w~wyższość ekonomii pozytywnej.
Dlatego tak istotne jest wskazanie, jak działa i~do jakich negatywnych konsekwencji
prowadzi ekonomia wolna od wartościowania, która jest uodporniona na krytykę, dzięki swoim aksjomatycznym założeniom. 

\end{artplenv}\label{ost-stop}