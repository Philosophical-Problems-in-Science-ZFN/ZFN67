\begin{recplenv}{Milena Cygan}
	{Empatia w~służbie ludzkości}
	{Empatia w~służbie ludzkości}
	{Frans de Waal, \textit{Wiek empatii. Jak natura uczy nas życzliwości}, tłum. Ł.~Lamża, Copernicus Center
		Press, Kraków 2019, ss.~380.}




Chociaż wydaje się, że darwinizm społeczny dawno już przeminął i~należy do przeszłości, to jednak okazuje się, że duch
jego trwa nadal, przejawiając swoją obecność w~wielu obszarach ludzkiej działalności. Z~całą pewnością można odnaleźć
go w~sferze liberalizmu gospodarczego, politycznego i~ekonomicznego, czego efektem jest tak dobrze nam znany wyścig
szczurów, oraz teza, zgodnie z~którą jesteśmy tylko zwierzętami napędzanymi przez konkurencję i~rywalizację, a~sukces
należy do tych ,,najlepiej przystosowanych''. Oczywiście trudno zaprzeczyć, jakoby mechanizmy owe nie odgrywały żadnej
roli i~były pozbawione sensu. Czymś oczywistym jest, że w~świecie występuje konkurencja i~ciągła ,,walka o~byt'',
niemniej trudno byłoby żyć i~funkcjonować, ograniczając się tylko do niej. Natura zdołała jednak ,,wyprodukować'' jeszcze
inną siłę, a~mianowicie empatię i~opartą na niej zdolność do współpracy. O~tej, trochę dziś -- w dobie szerzącego się
indywidualizmu -- zapomnianej prawdzie, przypomina nam Frans de Waal w~swojej książce \textit{Wiek empatii. Jak natura
uczy nas życzliwości}, która w~polskim tłumaczeniu ukazała się w~tym roku nakładem Copernicus Center Press.

Autor to światowej sławy prymatolog i~etolog, profesor Uniwersytetu Emory, kierownik Living Links Center w~Yerkes
National Primate Research Center w~Atlancie, członek Amerykańskiej Akademii Nauki oraz Holenderskiej Królewskiej
Akademii Nauk. Jest autorem kilkunastu książek i~kilkudziesięciu artykułów dotyczących zachowań
prospołecznych i~zdolności poznawczych zwierząt, ich altruizmu, empatii, a~także ewolucyjnych korzeni moralności.
W 2007 roku został
uznany za jednego ze stu najbardziej wpływowych ludzi magazynu ,,Time'', a~trzy lata później otrzymał holenderski Order
Lwa -- nagrodę przyznawaną tym, którzy świadczą wyjątkowe usługi na rzecz społeczności. Znany jest polskiemu
czytelnikowi z~kilku pozycji, wydanych również nakładem CCPress: \textit{Małpy i~filozofowie. Skąd pochodzi moralność?}
\parencite*{waal_malpy_2013};
%\label{ref:RNDaacV5N3EhJ}(2013);
\textit{Bonobo i~ateista. W~poszukiwaniu humanizmu wśród naczelnych}
\parencite*{waal_bonobo_2014};
%\label{ref:RND5to1TyLU5Z}(2014);
\textit{Małpa w~każdym z~nas}
\parencite*{waal_malpa_2015};
%\label{ref:RND6qtPzblfNL}(Waal, 2015);
\textit{Bystre zwierzę. Czy jesteśmy dość mądrzy by zrozumieć mądrość zwierząt?}
\parencite*{waal_bystre_2016}.
%\label{ref:RNDJusgwKWrqM}(2016).
Ostatnia z~nich była też recenzowana na łamach ZFN
\parencite{sarosiek_odmiennosc_2018}.
%\label{ref:RNDIrz16H5Io4}(Sarosiek, 2018).

Prezentowana książka \textit{Wiek empatii} zawiera dwa główne, przeplatające się ze sobą w~siedmiu rozdziałach wątki.
Pierwszym jest analiza empatii jako takiej oraz jej różnych poziomów, tak jak przedstawiają się one z~ewolucyjnego
punktu widzenia. De Waal śledzi początki empatii, skupiając się głównie na ssakach. Oczywiście to zawężenie badawcze
nie wynika z~przekonania Autora jakoby inne zwierzęta (np. ptaki) niezdolne były do empatii, lecz dlatego, iż znajdują
się one poza zakresem bezpośrednich zainteresowań etologa. Ponadto badania empatii u~innych zwierząt niż wspomniane
ssaki nie są jeszcze zbyt zaawansowane i~niewiele na ten temat wiadomo. Drugi kluczowy wątek to rola, jaką odgrywa
empatia w~społecznościach zwierzęcych, szczególnie zaś w~społeczeństwie ludzkim, w~obszarach takich jak polityka,
gospodarka, religia czy zwykłe relacje międzyludzkie. Autor zastanawia się, czy nie można by w~tych przestrzeniach
pozwolić sobie na trochę więcej empatii, gdyż chciwość, konkurencja i~wyzysk, stawszy się głównymi siłami napędowymi,
rujnują nie tylko związki między członkami danych społeczności, ale okazują się prowadzić do katastrof
finansowych i~gospodarczych. Książkę otwiera zatem optymistycznie brzmiące zdanie: ,,Era chciwości za nami,
nadchodzi czas empatii''
(s.~7), a~kończą pełne przekonania i~nadziei słowa: ,,Odwołanie się do tej wrodzonej umiejętności może przynieść
społeczeństwu jedynie korzyści'' (s.~313). I~te dwie sentencje wydają się najlepiej streszczać zawartość recenzowanej
pozycji.

Jak zostało wspomniane powyżej, \textit{Wiek empatii} składa się z~siedmiu rozdziałów. W~pierwszych dwóch,
zatytułowanych ,,Biologia z~prawej i~lewej strony'' oraz ,,Ten inny darwinizm'' Frans de Waal zapoznaje
czytelnika z~tym, w~jaki sposób wykorzystywana jest biologia w~polityce i~w~ekonomii (głównie zachodniej,
szczególnie zaś amerykańskiej).
Autor zwraca uwagę, że niezależnie od prezentowanego stronnictwa, politycy, socjologowie czy ekonomiści bardzo chętnie
odwołują się do biologii: ,,Każda dyskusja na temat społeczeństwa i~rządu opiera się na silnych założeniach na temat
natury ludzkiej, zwykle czynionych tak, aby zdawało się, że wypływają wprost z~biologii'' (s.~13). Biologia, jak
twierdzi de Waal, weszła w~szeroko zakrojony dyskurs polityczny, stała się nieodłącznym elementem debat politycznych,
jednak -- zapytuje dalej -- czy rzeczywiście ta biologia jest tą samą biologią, którą zajmują się naukowcy? I~dlaczego
założenia na temat biologicznej strony człowieka są zawsze negatywne? Dlaczego naturę ludzką i~relacje społeczne
opisuje się starym powiedzeniem ,,człowiek człowiekowi wilkiem'' -- ,,[\mydots] wątpliwym stwierdzeniem na temat naszego gatunku
opartym na nieuzasadnionych założeniach na temat innego gatunku'' (s.~13), krzywdzącym człowieka i~niesprawiedliwym dla
wilka? W~toku prezentowanych argumentów Autor nieco przesadnie stwierdza, że badacze prawa, ekonomii i~polityki nie
mają narzędzi, które pozwoliłyby im spojrzeć na społeczeństwo obiektywnie. W~jego opinii rzadko korzystają
oni z~potężnego zasobu wiedzy na temat zachowania ludzkiego i~innych zwierząt żyjących stadnie, jakie zgromadzili
antropolodzy, psychologowie, biolodzy czy neuronaukowcy. W~tej sytuacji postuluje on nowe podejście, swego rodzaju
rewolucję w~poglądzie na naturę ludzką: ,,Zbyt wielu ekonomistów i~polityków -- pisze de Waal -- modeluje społeczeństwo
ludzkie jako wieczną walkę, która ich zdaniem występuje w~przyrodzie -- i~która jest tak naprawdę czczym wymysłem.
Zachowują się jak magik, który najpierw podrzuca swoje założenia ideologiczne do kapelusza natury, aby następnie
wyciągnąć je tryumfalnie za uszy, demonstrując nam, jak bardzo świat przyrody zgadza się z~ich poglądami. To sztuczka,
na którą zbyt długo dawaliśmy się złapać. Jest jasne, że w~świecie istnieje konkurencja, jednak ludzie nie mogą żyć,
jeśli ograniczą się tylko do niej'' (s.~17). Doświadczenia i~obserwacje, które Autor przytacza w~tym rozdziale -- jak
choćby to, gdy szympansy, które na drodze konkurencji zdobyły jedzenie jako pierwsze, dzielą się swoją zdobyczą z~tymi
osobnikami stada, dla których zabrakło pożywienia -- wskazują, że przekonanie o~tym, iż wolność jednostki jest
ważniejsza od wartości społecznych, stanowi tylko kolejny mit.

\enlargethispage{-.5\baselineskip}

Na tym tle Frans de Waal podejmuje się także polemiki z~tzw. darwinizmem społecznym. Ideologię tę spopularyzował w~XIX
wieku Herbert Spencer, tłumacząc prawa natury na język biznesu i~ekonomii. To on ukuł słynną frazę: ,,Przetrwanie
najlepiej przystosowanych''. Zgodnie z~tą ideologią, jeśli w~świecie natury ,,przeżycie'' jest domeną ,,najlepszych'', to
nie ma najmniejszego powodu, żeby owi ,,najlepsi'' pomagali tym ,,najsłabszym''. W~ujęciu spencerowskim społeczeństwo
ludzkie -- oparte na prawach natury -- nie powinno więc przejawiać zbytniej troski o~tych, którzy nie potrafią nadążyć za
jej wymaganiami. De Waal zauważa jednak, że mechanizm konkurencji, do którego nawiązuje darwinizm społeczny, choć
prawdziwy i~opisujący stan rzeczywisty, jest jeszcze niewystarczający do tego, żeby na jego podstawie wnosić o~tym, jak
być powinno. Ponadto to nie wszystko, co natura, zwłaszcza nasza własna, ma nam do zaoferowania. 

,,Prawo pięści'' i~opinia, iż natura ludzka jest ,,czerwona od krwi i~pazurów'', prowadzą do innej wizji społeczeństwa niż
pogląd, że u~źródeł ewolucji człowieka odnajdziemy współpracę i~solidarność. Poglądy, że ,,sukces jest swoim własnym
uzasadnianiem'' a~,,egoizm nie jest wadą'', oparte na rzekomo naturalnych skłonnościach, o~ile nie jest błędny, to na
pewno niepełny. Jako uzasadnienie de Waal przytacza wiele doświadczeń z~udziałem zwierząt i~ludzi, świadczących o~tym,
iż wzajemna pomoc jest znaczniej bardziej ceniona niż postawy samolubne. ,,Jeśli biologia ma być źródłem inspiracji dla
rządu i~społeczeństwa powinniśmy zadbać choćby o~poznanie jej pełnego obrazu -- porzućmy kreskówkową wersję, jaką
oferuje nam darwinizm społeczny, i~spójrzmy, co naprawdę przygotowała dla nas ewolucja. Jakiego rodzaju zwierzętami
jesteśmy? Cechy wykształcone wskutek doboru naturalnego tworzą bogatą i~zróżnicowaną mieszaninę -- [\mydots] Tak naprawdę
jestem przekonany, że biologia jest dla nas największym źródłem nadziei. Ideologie przychodzą i~odchodzą ale natura
ludzka trzyma się twardo'' (s.~70).

W kolejnych rozdziałach książki prymatolog dokonuje analizy empatii. Poszukuje jej elementów składowych i~tego, co jest
potrzebne, by mogła się wykształcić i~rozwijać. Oczywiście warto w~tym miejscu wspomnieć jeszcze, że \textit{Wiek
empatii} nie jest książką, którą moglibyśmy nazwać ,,zbiorem dowodów na empatię''. Autor wydaje się raczej być przekonany
o jej istnieniu tak w~świecie ludzkim, jak i~zwierzęcym, uważając ją za dosyć dobrze udokumentowaną i~udowodnioną. Nie
stara się więc przekonać czytelnika, że jest coś takiego jak empatia. Bardziej interesuje go to, ,,jak'' ona jest. Dlatego
w trzecim rozdziale ,,Ciała rozmawiające z~ciałami'' prymatolog opisuje mechanizm ,,zarażania się emocjami'', wyrażając
swój podziw nad tym, jak łatwo dajemy się ponieść emocjom naszych towarzyszy. Istnieje w~nas głęboko zakorzeniona,
dokonująca się niemal automatycznie, poza poziomem naszej świadomości synchronizacja emocjonalna i~ruchowa względem
drugiego osobnika. Bezwiednie wczuwamy się w~ciała otaczających nas osób, tak, że ruchy i~emocje znajdują w~nas
odzwierciedlenie, jak gdyby były naszymi własnymi. Do wyrażania i~przekazywania informacji o~swoich stanach
emocjonalnych niekoniecznie potrzebujemy języka, ale bezwzględnie potrzebujemy ciała. Empatia domaga się ciała,
szczególnie zaś twarzy. Innymi słowy: empatia jest ucieleśniona: ,,Gdy mimika jest upośledzona -- pisze Autor -- jest taka
również i~empatia, a~interakcja międzyludzka, gdy nie dochodzi w~jej trakcie do zwyczajnego, nieustannego komunikowania
się na poziomie cielesnym, staje się pozbawiona wyrazu'' (s.~121).

Jednakże zdolność do synchronizacji emocjonalnej, naśladowania czy wczuwania się w~drugiego to nie wszystko. W~następnym
rozdziale zatytułowanym ,,W~nie swoich butach'' de Waal poddaje analizie reakcje zwierząt i~ludzi na ,,trudne'' położenie
drugiego. W~tych przypadkach -- zauważa -- dochodzi do zdumiewających sytuacji, gdzie jeden osobnik nie tylko wyczuwa
stan drugiego, ale próbuje go ,,zrozumieć'' i~,,polepszyć''. Jest to pójście o~krok dalej, od empatii do współczucia.
Empatia zdaniem Autora to proces, za pomocą którego zbieramy informacje na czyjś temat (może być więc wykorzystana
niekoniecznie do celów szlachetnych), natomiast współczucie to już wyraz troski o~drugiego i~chęci polepszenia jego
sytuacji. Co jednak skłania nas i~inne zwierzęta do spieszenia na pomoc, wyzwalając chęć ulżenia doli towarzysza?
Motywacje oczywiście mogą być różne, od bardziej interesownych, do tych wydających się nie mieć żadnych wymiernych
korzyści oprócz dobra drugiego. Jednakże de Waala bardziej jeszcze interesuje coś innego, a~mianowicie warunki, jakie
muszą zaistnieć, aby być zdolnym do empatii, czy bardziej zaawansowanego współczucia, a~są to umiejętność przyjmowania
perspektywy innego, wejścia w~stan umysłu drugiego i~samoświadomość.

Tym powyższym zdolnościom przygląda się de Waal bliżej w~rozdziale ,,Słoń w~pokoju''. Empatia potrzebuje nie tylko
poczucia ,,ty'', lecz także bardziej fundamentalnego poczucia ,,ja''. O~tym, że ludzie posiadają świadomość siebie, nie
trzeba specjalnie nikogo przekonywać. A~jak jest u~zwierząt? Autor przytacza szereg eksperymentów ,,testu lustra''
przeprowadzonego na małych dzieciach, małpach, delfinach i~słoniach. Test, a~właściwie reakcje zwierząt na lustro uważa
za względnie nieciekawe i~niewiele wyjaśniające, bo przecież z~tego, że dane zwierzę nie rozpoznaje się w~lustrze, nie
można ostatecznie wnosić, iż nie posiada poczucia siebie. Niemniej tym, co sprawia, iż test lustra wydaje się
interesujący, jest fakt, że dostarcza wiele informacji o~tym, jak jednostka pozycjonuje się względem otoczenia.
Przykładowo wynikiem testu jest obserwacja, że ludzkie dzieci potrafią rozpoznać się w~lustrze dopiero w~wieku około
dwóch lat. W~tym okresie zaczyna wzrastać u~nich świadomość siebie i~da się zauważyć mocniejsze pozycjonowanie się
względem otoczenia, co jest świadectwem tego, iż pewne cechy pojawiają się dopiero na określonym etapie
rozwoju i~występują razem. Taka korelacja konkretnych cech z~etapem rozwoju danej jednostki prowadzi Autora do opowiedzenia się
za hipotezą ,,współwyłaniania się'' (\textit{co-emergence hypothesis}). Jeśli więc u~danego osobnika pojawia się
określona umiejętność, to wraz z~nią występują także inne, nieodłącznie z~nią związane. Tak między innymi jest właśnie
z~empatią. Ona także nie występuje jako umiejętność wyizolowana, ale postuluje posiadanie przez daną jednostkę zestawu
innych, warunkujących ją zdolności. Jeśli więc na przykład zwierzęta potrafią być empatyczne (a Autor jest przekonany
że potrafią), są zdolne także do oddzielenia swojego stanu mentalnego od cudzego. A~o~tym, iż zwierzęta potrafią przyjąć
perspektywę innego, zdając sobie sprawę z~bycia kimś innym niż drugi, dowodzi ,,wskazywanie''. Skoro zwierzę potrafi
wskazać coś innemu osobnikowi albo go czegoś nauczyć, wie ono, że ten drugi nie wie, tego co ono wie. Albo też w~drugą
stronę -- że ten drugi posiada jakąś wiedzę, której ono aktualnie nie posiada, a~chciałoby ją z~takich czy innych
względów nabyć.

\enlargethispage{-.5\baselineskip}

Rozdział przedostatni, ,,Fair play'', jest już nie tylko o~empatii, ale również o~jej przeciwieństwie -- tak zwanej
\textit{Schadenfreude} -- i~sprawiedliwości. De Waal stwierdza, że chodzimy na dwóch nogach: ,,jedna jest
samolubna a~druga prospołeczna'' (s.~224). Jesteśmy stworzeniami złożonymi. Zazwyczaj współczujemy tym, którym się to współczucie
należy, ale bywa i~tak, że widząc nieszczęście innych odczuwamy satysfakcję (\textit{Schadenfreude}), zwłaszcza gdy do
gry wchodzi sprawiedliwość. Człowiek ma w~sobie mocne skłonności do egalitaryzmu. Na przykład uznajemy hierarchię
społeczną, jednak nie ,,traktujemy'' jej aż tak poważnie jak np. szympansy, które raczej nie zdobywają się na podważenie
autorytetu swoich przywódców, przykładowo przez ośmieszanie. My, ludzie, jak podkreśla Autor, jesteśmy urodzonymi
rewolucjonistami i~skłonność do podważania pionowych podziałów nigdy nas nie opuściła. Nie lubimy, gdy ktoś ma od nas
więcej. Zdaniem de Waala badania pokazują, iż z~natury nie jesteśmy też egoistami. Owszem, tacy ludzie też się zdarzają
ale to mniejszość. ,,Większość z~nas jest altruistami, współpracuje, jest wyczulona na sprawiedliwość i~zorientowana na
wspólne cele'' (s.~228). Zasadniczo mamy jednak inne wyobrażenie o~naszej naturze. Postrzegamy przedstawicieli naszego
gatunku jako wyrachowanych oportunistów, co podważa zaufanie do innych i~sprawia, że stajemy się bardziej
zdystansowani, ostrożni i~mniej hojni. Stąd też niekiedy owe ,,wyobrażenia'' sprawiają, że ludzie im ulegający sami są
skłonni przejawiać te mniej szlachetne cechy. Jednak zasadniczo jesteśmy skorzy do współpracowania z~innymi, a~to
dlatego, że zazwyczaj maksymalizuje to nasze zyski. Pokazują to dobrze sytuacje, w~których mamy wybór pomiędzy małą
nagrodą i~indywidualizmem a~dużą nagrodą i~kolektywizmem. ,,Dwóch myśliwych musi zdecydować, czy pójdą polować na
zające, czy też razem wybiorą się na jelenia -- znacznie większy łup, nawet jeśli podzielą się nim na pół'' (s.~229). De
Waal zdaje się twierdzić, choć nie bezkrytycznie, że w~takich okolicznościach, znaczna część wybiera raczej drugą
opcję.

\enlargethispage{-.5\baselineskip}

Perspektywa większego zysku nie jest jednak wystarczająca do tego, aby podjąć kooperację. Ważną rolę odgrywa tu także
wspomniane wyżej zaufanie. I~znów nie jest ono charakterystyczne tylko dla ludzi. Autor przytacza wiele
sytuacji, w~których zwierzęta zdają się tworzyć więzi oparte na zaufaniu. Przykładami mogą być kapucynki i~ich ,,wąchanie ręki'',
czyli wkładanie sobie wzajemnie palców do nozdrzy albo pomiędzy gałkę oczną i~powiekę. Hipotezy na temat tych
,,dziwnych'' zachowań mówią o~sprawdzaniu więzi i~poziomu zaufania. W~końcu trzeba mieć do kogoś duże zaufanie, by
pozwolić mu włożyć palec do swojego oka, czyli wystawić się na dość spore ryzyko, zakładając przy tym, że ów drugi
osobnik nie obróci tej sytuacji na swoją korzyść. De Waal zauważa, że zwierzęta dosyć chętnie godzą się nad tak
niebezpieczne zachowania i~to nie tylko z~przedstawicielami własnego gatunku -- małpy czyszczące zęby hipopotama, czy
wargatki usuwające pasożyty większym rybom-drapieżnikom, nasze codzienne relacje z~domowymi pupilami itd. Przykładów
nie brakuje. 

Jednak naturalne jest także, że tam, gdzie zaufanie, tam trzeba wspomnieć również o~uczciwości. Autor pokazuje,
że i~w~królestwie zwierząt poczucie niesprawiedliwości nie jest czymś niespotykanym. Polemizuje w~tym kontekście z~tezą Adama
Smitha, który twierdził, że zwierzęta nigdy nie handlują i~obce jest im poczucie uczciwości. Swój pogląd ilustruje
kilkoma przykładami badań ,,ekonomii behawioralnej'' zwierząt, w~których manifestują one doznaną krzywdę oraz pomagają
drugiemu osobnikowi, odwdzięczając się za wcześniej okazaną przez niego pomoc. Przytacza też takie
okoliczności, w~których zwierzęta mszczą się za doznaną niesprawiedliwość i~prezentują zachowania i~działania zmierzające do
,,wyrównania rachunków''. Analizując te powyższe skłonności zwierząt, de Waal formułuje przypuszczenie, iż różnica
pomiędzy ludźmi a~innymi gatunkami jest taka, że ci pierwsi wykazują zasadniczo wszystkie tendencje
spotykane w~królestwie zwierząt, tyle że w~nieco większym stopniu niż inne gatunki, dlatego też są zdolni do bardziej złożonej
współpracy i~na większą skalę: ,,Psychika ludzka wyewoluowała w~kierunku, który pozwala na coraz bardziej złożone
polowania na jelenia, wykraczające poza wszystko, co można zaobserwować w~królestwie zwierząt'' (s.~254). Różnice
pomiędzy ludźmi i~innymi zwierzętami, a~raczej posiadanymi przez nie zdolnościami, byłaby raczej natury ilościowej niż
jakościowej. 

,,Krzywe Drewno'' to już ostatni, podsumowujący rozdział, w~którym Autor stwierdza, że prawdą jest, iż historię naszego
gatunku znaczą konflikty, w~dużej mierze napędzane konkurencją i~walką o~przetrwanie, ale to tylko połowa prawdy o~nas
samych. Chcąc świadomie modelować czy kreować społeczeństwo i~wzajemne relacje, byłoby czymś niestosownym ignorować
wszystkie czynniki odpowiadające za społeczną naturę naszego gatunku. Jednym z~nich jest także empatia, o~której nie
często się w~tym kontekście wspomina. Empatia posiada głębokie korzenie ewolucyjne; nie jest niedawnym wynalazkiem,
lecz prastarą zdolnością, z~którą wszyscy się rodzimy i~dzielimy z~innymi niż człowiek gatunkami. Wyłoniła się ona na
drodze milionów lat doboru naturalnego, co oznacza, że została wyjątkowo wypróbowana ze względu na swoją przydatność
dla przeżycia. Empatia zdaniem de Waala to ,,część dziedzictwa tak starego jak linia ewolucyjna ssaków. [\mydots] Zdolność do
niej wyewoluowała dawno temu, wraz z~naśladowaniem ruchu innych i~zarażaniem się emocjami, na co nakładane były kolejno
dalsze warstwy, aż nasi przodkowie stali się zdolni nie tylko do odczuwania tego, co czują inni, ale również
rozumienia, czego chcą i~potrzebują'' (s.~291). Całą umiejętność porównuje de Waal do matrioszki, której rdzeniem jest
automatyczny proces, jaki człowiek dzieli z~wieloma gatunkami, obudowany przez kolejne warstwy, pozwalające na
subtelniejszą kontrolę jego zasięgu i~celów. Nie wszystkie więc gatunki posiadają wszystkie warstwy: ,,Nawet jednak
najbardziej wyrafinowane warstwy matrioszki pozostają zwykle ciasno przytulone do jej pierwotnego rdzenia (s.~291). Jej
głębokie zakorzenienie w~naszej naturze powinno być więc wystarczającym argumentem do tego, by nie pomijać jej przy
szukaniu pomysłów na urządzanie społeczeństwa. To jest ważny postulat de Waala, zwłaszcza, że pomimo dobrze
wykształconej zdolności empatii, ciągle jeszcze ma ona stosunkowo niewielki zakres odziaływania. ,,Empatia skierowana ku
`obcym ludziom' to towar, którego światu brakuje bardziej niż ropy naftowej. Byłoby świetnie, gdyby udało się wytworzyć
jej choćby krztynę''(s.~285) -- takie jest zatem ostateczne przesłanie Autora dla czytelnika i~społeczeństwa.

\enlargethispage{-.5\baselineskip}

\textit{Wiek empatii} to książka napisana stylem prostym, przystępnym, pełnym humoru i~erudycji. Nie jest to pozycja dla
specjalistów, więc czytelnik nie zagubi się w~zaawansowanej, technicznej terminologii ani nie poczuje się przytłoczony
,,suchymi'' sprawozdaniami z~badań i~eksperymentów. Chociaż nie brakuje odwołań do rzetelnie udokumentowanych badań, to
jednak Autor woli posługiwać się bardziej ,,przyswajalnymi'' dla potencjalnego czytelnika anegdotami, ukazującymi
zachowanie zwierząt. Oczywiście, każdy taki przypadek zwierzęcego zachowania tam zawarty, ma swoje
odzwierciedlenie w~badaniach i~eksperymentach, dokumentujących wiele podobnych sytuacji, których referencje Autor umieszcza na końcu
książki. (Warto wspomnieć, że rysunki ilustrujące opisywane historie także są autorstwa de Waala). Dzięki temu
\textit{Wiek empatii} zyskuję tę przewagę, że nie jest męczący dla czytelnika, i~sprawia wrażenie, jakby dawał Autorowi
większą swobodę w~wyrażaniu myśli, ekspresji intuicji i~odczuć.

Zawartość merytoryczna tej publikacji de Waala każe sądzić, iż książka ta jest z~jednej strony
polemiką z~ekonomicznymi i~politycznymi pozostałościami po darwinizmie
społecznym, z~drugiej zaś, studium nad ewolucyjnymi korzeniami empatii.
Autor prowadzi dyskusję ze stereotypowymi poglądami na temat ewolucyjnych mechanizmów ,,walki o~przetrwanie'', która ma
usprawiedliwiać nasze antyspołeczne zachowania: egoizm, samolubstwo, osiąganie własnych celów kosztem innych ludzi oraz
ciągłą konkurencję, będącą motorem napędzającym gospodarkę i~politykę. Frans de Waal ukazuje, że powyższe idee okazały
się być na tyle silne i~atrakcyjne, iż zdołały zagłuszyć w~społeczeństwie zdolność do empatii, wraz z~zorientowaniem na
dobro drugiego. Stanowisko de Waala jest bardziej zrozumiałe, jeśli weźmie się pod uwagę, że tłem jego rozważań jest
społeczeństwo i~świat polityki amerykańskiej, w~którym rzeczywiście idea konkurencji, wolności jednostki oraz
niezależności jest głęboko zakorzeniona. Oczywiście tendencje te nie są złe, ale doprowadzone do skrajności i~podparte
założeniami na temat ich biologicznego uwarunkowania, wymagają zrewidowania i~uzupełnienia. To właśnie czyni
Autor, i~wydaje się, że skorzystać na tym może nie tylko amerykańskie społeczeństwo. Nasz gatunek ma stronę samolubną, ale
również i~stronę społeczną, możliwą dzięki empatii. Odwołanie się więc do tej wrodzonej zdolności może przynieść
społeczeństwu tylko korzyść -- przypomnienie tej prawdy jest zasługą Autora.

Inną niewątpliwą zasługą Fransa de Waala jest zwrócenie uwagi czytelnika na świat zwierząt, nierzadko jeszcze
deprecjonowany, poniżany i~wykorzystywany. Nie tyle chodzi mu o~dostarczenie dowodów na to, że zwierzęta posiadają
uczucia, nawet te, które określamy jako wyższe, lecz raczej o~uzmysłowienie nam, że świat ich emocji jest o~wiele
bogatszy i~złożony niż nam się to wydaje. De Waal stara się pokazać w~swojej książce, że w~aspekcie
zachowania i~czynności poznawczych, nie ma niczego u~innych gatunków, czego nie moglibyśmy zaobserwować także w~naszym ludzkim
świecie, oraz że granice pomiędzy ,,nimi'' a~,,nami'' nie są wcale takie ostre. Autor eksponuje podobieństwa, jakie łączą
nas z~innym przedstawicielami królestwa zwierząt, i~jest to jak najbardziej uzasadnione, biorąc pod uwagę, że przez
wieki podkreślano raczej różnice (jak choćby rozróżnienie pomiędzy racjonalnym człowiekiem, a~sterowanym przez
instynkt, niezdolnym do myślenia i~odczuwania zwierzęciem), które jednak w~toku rozwoju nauki okazały się nie do
utrzymania. Pomimo tego, iż jesteśmy zwierzętami -- jak twierdzi de Waal -- tylko trochę bardziej złożonymi, wydaje się
brakować w~jego podejściu spojrzenia na to, co nas od nich oddziela. Skoro jesteśmy ludźmi, a~nie np. szympansami czy
jakimkolwiek innym gatunkiem, to owo spostrzeżenie automatycznie prowokuje do pytania o~to, co w~omawianym przez Autora
aspekcie wyróżnia nas jako osobny gatunek. Co prawda celem de Waala nie jest zmierzenie się z~tym pytaniem, niemniej
lektura książki rozbudza je w~czytelniku i~zmusza do przemyśleń nad specyfiką naszego gatunku.

\enlargethispage{-.5\baselineskip}

Inne, godne uwagi przemyślenia amerykańskiego naukowca związane są z~jego poglądami na naturę człowieka i~relacje na osi
jednostka -- społeczeństwo. Prezentuje on stanowisko, że człowiek z~natury nie jest zły i~antyspołeczny. Człowiek
pierwotny był o~wiele bardziej szlachetny niż się to zazwyczaj twierdzi. Autor uzasadnia to wynikami badań
antropologicznych nad stanem ludów pierwotnych, zdających się przeczyć tezie o~morderczych i~wojowniczych skłonnościach
zapisanych w~naszym DNA. Człowiek dla de Waala to istota z~natury pokojowa. Co więcej, wydaje się on podkreślać, że te
pierwotne potrzeby, pomimo zmiany warunków i~bardziej złożonych struktur społecznych, w~nas nie wygasły. W~tym kontekście
rozważania de Waala przypominają z~jednej strony nieco podstawy sentymentalizmu J.J. Rousseau i~jego tęsknotę za
,,szlachetnym dzikusem'', którego zepsuła cywilizacja i~wydają się brzmieć nieco anachronicznie. Z~drugiej zaś przywodzą
na myśl A. Schweitzera i~jego dość pesymistyczne spostrzeżenia na człowieka prymitywnego, posiadającego oczywiście
empatyczne struktury w~swojej osobowości, i~pojęcie solidarności, ale przecież nader wąskie, ograniczone swoim
zasięgiem do członków rodziny i~plemienia. Czy poszerzenie zasięgu empatii byłoby jakimś kolejnym etapem ewolucji?

\enlargethispage{-.5\baselineskip}

Interesującym, będącym ,,na czasie'' zagadnieniem, które porusza de Waal, jest wspomniana relacja jednostki do
społeczności i~ich wzajemnych obowiązków względem siebie. Jak wielokrotnie podkreśla Autor, jesteśmy zwierzętami
stadnymi, potrzebujemy społeczności nie tylko do tego, żeby przeżyć w~sensie czysto fizycznym, potrzebujemy także innych,
żeby móc się prawidłowo rozwijać na poziomie psychicznym, emocjonalnym. Potrzebujemy relacji, fizycznego kontaktu, gdyż
ich brak prowadzi do zaburzeń osobowości, a~nawet do śmierci. Dowodzą tego eksperymenty psychologii behawioralnej,
przeprowadzane na wychowywanych w~izolacji małpach oraz dzieciach. Te ostatnie były poddawane takim doświadczeniom na
przykład w~rumuńskich sierocińcach założonych przez Nicolae Ceaușescu. Efekty były zatrważające. Tworzenie więzi
jest więc czymś kluczowym dla naszego gatunku. Potrzebujemy siebie nawzajem, a~więc potrzebujemy społeczeństwa. To zaś
sugeruje, że ono jest częścią naszej tożsamości: społeczeństwo jest częścią nas, ale i~my jego, tworząc jeden organizm.
Ta zależność obliguje do tworzenia systemu wzajemnych praw i~obowiązków pomiędzy społecznością i~jednostką. Jeszcze do
niedawna podkreślano, że dobro wspólne ma nieco większą wartość niż dobro jednostki, a~poświęcenie dla wspólnoty,
państwa, ojczyzny było wręcz moralnym obowiązkiem każdego obywatela. To starali się nam przypominać autorzy klasycznych
tekstów i~traktatów moralno-politycznych. Dobro wspólne stanowiło o~dobru jednostki. Jeśli przyczyniam się do dobra
wspólnego, to dbam również o~dobro swoje i~najbliższych. Dzisiaj pogląd ten jakby stracił na wartości, a~owo dobro
wspólne wydaje się być w~trochę mniejszym poszanowaniu. Tym, co się niekiedy podkreśla jako priorytetowe, to dobro
jednostki, nawet kosztem dobra wspólnego. Dobrze byłoby znaleźć jakiś balans pomiędzy tymi dwoma tendencjami. Do tych
też poszukiwań skłania nas Autor \textit{Wieku empatii}. Książka stanowi więc dobry punkt wyjścia do
interesujących i~inspirujących rozważań i~dyskusji. Etologia w~ujęciu Fransa de Waala prowokuje do stawiania pytań, podejmowania
refleksji na temat naszej moralnej i~społecznej kondycji. To interesujący przykład tego, w~jaki sposób nauka może
pomagać w~rozwijaniu refleksji na temat klasycznych problemów filozofii.



\autorrec{Milena Cygan}


\subsubsection{Bibliografia}\nopagebreak[4]
\end{recplenv}
