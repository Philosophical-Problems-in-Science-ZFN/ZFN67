\begin{artplenv}{Bartosz Paweł Kurkowski}
	{Czy konstruktywiści społeczni mówią nam o~czymś realnym w~ekonomii?}
	{Czy konstruktywiści społeczni mówią nam o~czymś realnym\ldots}
	{Czy konstruktywiści społeczni mówią nam o~czymś realnym\\w~ekonomii?}
	{Uniwersytet Ekonomiczny w Poznaniu}
	{Do social constructivists tell us something real about economics?}
	{Social constructivists use to say that economists cannot study objective reality, absolute truth does not exist, and economic
		knowledge is being constructed, not discovered and it depends on temporary culture. Realists notice and admit social
		conditions which may affect economic theories and unrealistic assumptions of models, but they claim that these models
		describe the real world, at least to some extent. The notions of truth and reality are crucial in both of these concepts. In this papar they are analysed based on Putnam theory developed by Grobler.}
	{realism, social constructivism, intersubjectivity, reality, truth, philosophy of economics.}



\section*{Wstęp}

\lettrine[loversize=0.13,lines=2,lraise=-0.05,nindent=0em,findent=0.2pt]%
{W}{}e współczesnych sporach dotyczących epistemologii ekonomii można wyróżnić stanowiska różnego rodzaju konstruktywizmu
społecznego i~opozycyjne względem nich stanowiska realizmu. W~najogólniejszym sensie zgodnie z~tymi pierwszymi badacz
nie ma możliwości odwzorowania struktury realnego (pozakulturowego) świata, z~powodu braku bezpośredniego dostępu do
niego
\parencite[s.~30]{zboron_teorie_2009}.
%\label{ref:RNDTf9TLKm3Gq}(Zboroń, 2009, s.~30).
Zdaniem realistów natomiast obiektywna struktura realnego świata
może być opisywana przez teorie naukowe i~na mocy adekwatności ich opisu można mówić, że są one prawdziwe lub fałszywe
\parencite[s.~74]{kincaid_realistic_2009}.
%\label{ref:RNDTGZMkWI4Sm}(Mäki, 2009, s.~74).
W~niniejszym eseju zestawione zostaną ze sobą oba te stanowiska i~rozważone
zostanie, czy konstruktywizm społeczny odnosi się do czegoś realnego w~świecie, czy może realizm jest bliższy
powszechnie akceptowanym przekonaniom społecznym oraz czego, w~nawiązaniu do nich, możemy dowiedzieć się z~teorii
ekonomicznych. Celem rozważań jest sprawdzenie, czy można mówić o~realnych zjawiskach i~mechanizmach w~gospodarce,
nawet jeśli nie istnieją one poza ludzkimi umysłami.

\section{Konstruktywizm społeczny w~perspektywie konstruktywizmu społecznego}

Konstruktywiści zaobserwowali, że wszelkie poznanie naukowe jest uwarunkowane, a~zatem niejako zniekształcone
kulturowo. Dzieje się tak między innymi dlatego, że zarówno pojęcia, za pomocą których formułowane są teorie,
jak i~powszechnie akceptowane wartości i~przekonania-oczywistości wywodzą się z~kultury. Czy jednak fakt wpływu kultury na
procesy poznawcze naukowców automatycznie uniemożliwia odwzorowanie struktury świata? Czy kultura -- jako społecznie
żywione przekonania i~zbiór pojęć -- nie może choć częściowo trafnie odwzorowywać rzeczywistości\footnote{Kultura
zdaniem konstruktywistów stanowiąc zbiór powszechnie akceptowanych przekonań i~praktyk rozwija się w~oparciu o~zbiór
powszechnie akceptowanych przekonań i~praktyk. Jednak jeśli jest ona uwarunkowana jedynie samą sobą z~przeszłości, to
pojawia się regres w~nieskończoność, natomiast jeśli warunkuje ją również coś pozakulturowego, to zdaje się, że z~jej
wnętrza można sformułować pewne wnioski związane z~pozakulturowym światem.}? Konstruktywiści zaprzeczają temu,
wskazując na lokalne i~czasowe zróżnicowanie kultur
\parencite[s.~33]{zboron_teorie_2009},
%\label{ref:RNDrzl88lgxF0}(Zboroń, 2009, s.~33),
zaś zgodnie
ze~społecznie akceptowanym relatywizmem kulturowym nie ma możliwości porównywania trafności czy słuszności przekonań
występujących w~różnych kulturach
\parencite[s.~76]{przymenski_socjologia:_2008}.
%\label{ref:RNDmsAB7rfVw4}(Przymeński, 2008, s.~76).
Należy więc rozważyć pytanie, czy
gdyby znalazły się pojęcia, przekonania lub inne obiekty kulturowe wspólne dla wszystkich kultur w~każdym
czasie i~przestrzeni, czy można byłoby je uznać za realnie istniejące poza kulturą, a~zatem istniejące obiektywnie? Takim
obiektem zdaje się być na przykład czas, jako kolejność następujących po sobie zjawisk. Kant zwrócił jednak uwagę, że
nie można utożsamiać świata istniejącego niezależnie od człowieka -- świata rzeczy samych w~sobie -- i~świata
postrzeganego przez człowieka -- świata zjawisk. W~tym kontekście jeśli ,,istniejące poza kulturą'' rozumiemy jako
,,istniejące poza procesem poznawczym człowieka'', to samo istnienie danego obiektu wspólnego dla wszystkich kultur nie
przekłada się na jego istnienie poza kulturą. Warto jednak zwrócić uwagę, że Kant mówił o~pewnych
koniecznych w~procesie poznania kategoriach zawartych w~umyśle (takich jak na przykład czas i~przestrzeń).
Ponadto uważał on, że są
one wspólne wszystkim ludziom, co miało gwarantować powszechność pewnych przekonań odnośnie rzeczywistości, a~zatem
,,intersubiektywność osiąganą w~granicach całego ludzkiego świata''
\parencite[s.~389]{zboron_podzial_2014},
%\label{ref:RNDkqE0VAB20e}(Zboroń, 2014, s.~389),
którą na potrzeby tego eseju będę nazywał \textit{pełną intersubiektywnością}. W~takim ujęciu obiekty istniejące
\textit{w pełni intersubiektywnie} należy wyróżnić na tle innych obiektów kulturowych, a~skoro nie są uwarunkowane
miejscem, czasem i~stopniem rozwoju danej kultury, można je również określić jako istniejące pozakulturowo. Pojawiają
się jednak pewne problemy z~tak zdefiniowanymi obiektami pozakulturowymi. Przede wszystkim wątpliwa jest możliwość ich
jednoznacznego określenia. Co prawda, współcześnie mogą odbywać się różnego rodzaju rozmowy pomiędzy przedstawicielami
różnych kultur, w~celu wzajemnego odszukania obiektów postrzeganych w~ten sam sposób lub chociaż ich fragmentów.
Jednak, jeśli chcielibyśmy się odnieść do kultur, które przeminęły, można zdać się jedynie na pozostawione przez nie
artefakty, które przetrwały do naszych czasów. Jeśli zaś chcielibyśmy się odnieść do kultur, które
nadejdą w~przyszłości, to nie ma żadnej możliwości sprawdzenia, na ile wspólne dotychczas obiekty-przekonania pozostaną dalej
wspólne. Jednak brak możliwości ostatecznego i~pewnego określenia \textit{w~pełni intersubiektywnych} obiektów nie
prowadzi do wniosku o~ich nieistnieniu. Prowadziłoby do niego jedynie ukazanie dwóch kultur, w~których przedstawiciele
nie mogliby się zgodzić w~sprawie żadnego ze swoich przekonań.


Przyjmując tezę, że nie da się skonfrontować treści teorii naukowej z~rzeczywistością, konstruktywiści twierdzą, że
naukowcy nie zajmują się odkrywaniem prawdy obiektywnej w~jej klasycznym rozumieniu, gdyż tak rozumiana prawda nie może
zaistnieć
\parencite[s.~34]{zboron_teorie_2009}.
%\label{ref:RNDO1PHIaYoe5}(Zboroń, 2009, s.~34).
W~tym podejściu przyjęcie lub odrzucenie danej teorii
naukowej nie odbywa się na podstawie jej prawdziwości -- zgodności z~rzeczywistością. Uprawomocnienie danej teorii
odbywa się zaś jedynie na podstawie dyskursu między naukowcami. Podczas dyskursu na uznanie przedstawianych propozycji
wpływają jedynie aktualnie podzielane przez społeczność naukowców przekonania, powstałe na podstawie (1) tradycji
myślowej i~(2) osobistego doświadczenia życiowego. W~ramach tych przekonań pochodzących z~tradycji mieszczą się
podstawowe zasady logiki klasycznej przy konstruowaniu wypowiedzi, na przykład zasada niesprzeczności lub chociaż
unikanie sprzeczności, które prowadzą do przepełnienia systemu. Natomiast z~doświadczenia życiowego, które pełni
istotną rolę w~procesie uprawomocniania teorii naukowej, wywodzone są świadectwa tak zwanego oporu -- odnoszonego
najczęściej do rzeczywistości fizycznej
\parencite[s.~50]{zboron_teorie_2009}
%\label{ref:RNDQ2XUwaEB1h}(Zboroń, 2009, s.~50)
 -- który polega na tym, że ,,nie
da się realizować wszystkiego, co daje się pomyśleć''
\parencite[s.~42]{zboron_teorie_2009}.
%\label{ref:RNDFGGrXVPfBH}(Zboroń, 2009, s.~42).
Można zatem
wywnioskować, że zgodnie z~konstruktywizmem jedynie podstawowe zasady logiki klasycznej i~doznawanie owego oporu
składają się na -- oczywiście jedynie pośredni -- dostęp do rzeczywistości pozakulturowej. Nawiązując do wcześniejszych
rozważań warto tutaj postawić pytanie, czy akceptację logiki i~doświadczanie oporu w~tych samych sytuacjach bez względu
na kulturę należałoby uznać za \textit{w pełni intersubiektywne} zjawiska, na podstawie których  można dojść do
\textit{w pełni intersubiektywnych} przekonań?


Z perspektywy konstruktywizmu społecznego tworzenie teorii naukowych polega na tworzeniu konstruktów myślowych na
podstawie potocznych konstruktów formułowanych w~codziennym życiu również przez osoby spoza społeczności naukowej.
Teorie naukowe są zatem konstruktami konstruktów
\parencites{schutz_potoczna_1984}[zob.][s.~34]{zboron_teorie_2009}.
%\label{ref:RNDC61JBCWzcS}(Schütz, 1984; zob. Zboroń, 2009, s.~34).
Jeśli przyjąć, że ,,teoria dotycząca genezy teorii musi wyjaśniać samą siebie''
\parencite[s.~17]{leszczynski_poszukiwanie_2016},
%\label{ref:RND32FT5qwMX7}(Leszczyński, 2016, s.~17),
to należałoby stwierdzić, że wszelkie teorie odnoszące się do teorii naukowych są konstruktami
stworzonymi w~oparciu o~konstrukty konstruktów, a~zatem konstruktami trzeciego stopnia, natomiast ten esej jest takim
małym konstruktem czwartego stopnia, gdyż odnosi się do tych stopnia trzeciego. Jednak, co istotniejsze, należy
zastanowić się, na czym polega wchodzenie na owe wyższe poziomy konstruowania i, przyjmując terminologię
konstruktywistyczną, czy średnio powinno zbliżać nas to do uprawomocniania danych stwierdzeń, czy oddalać.
Czy z~perspektywy konstruktywizmu jego tezy są na pozycji średnio bardziej czy mniej uprawomocnionej niż tezy pozostałych
nauk takich jak ekonomia? Otóż konstruowanie na podstawie konstruktów niższego poziomu polega na ich
porządkowaniu w~taki sposób, aby zostało to uznane przez jak największą liczbę osób -- lub przynajmniej naukowców danej
dziedziny. A~zatem można powiedzieć, że zarówno każdy kolejny poziom konstruktu, jak i~każdy kolejny czasowo konstrukt
dąży w~kierunki \textit{pełnej intersubiektywności}, uwzględniając również przekonania i~doświadczenia osób z~odległej
historii, przynajmniej na tyle, na ile jest to możliwe na podstawie zachowanych dzieł kultury. Zgodnie z~tym
spojrzeniem ,,wiedza jest historycznym konstruktem na miarę kultury, w~której została wytworzona''
\parencite[s.~29]{zboron_teorie_2009},
%\label{ref:RNDJMAxgf9MvQ}(Zboroń, 2009, s.~29),
ale jednocześnie tezy przekazywane przez konstruktywizm społeczny są
bliższe \textit{pełnemu intersubiektywizmowi} niż tezy tak zwanej naiwnej wersji realizmu, a~zatem więcej osób po
bezstronnym i~wnikliwym rozważeniu stawianych dotychczas w~historii argumentów byłaby skłonna przyjąć konstruktywizm
niż naiwny realizm. Można sobie jednak wyobrazić sytuację, w~której przykładowo w~wyniku katastrofy naturalnej zerwane
zostało połączenie pomiędzy przeszłymi i~przyszłymi myślicielami, a~ci ostatni w~ogromnej większości wyznają ów naiwny
realizm odrzucając sugestie o~wpływie kultury czy też struktury umysłu na formułowane przez siebie teorie.
Zapewne w~takiej sytuacji stanowisko konstruktywizmu społecznego zostałoby odrzucone podczas dyskursu opartego o~powszechnie
respektowane przekonania i~doświadczanie codziennego oporu, jednak czy z~tego powodu nie przekazywałoby ono nic
prawdziwego -- czy choćby domagającego się uprawomocnienia -- o~uprawianiu nauki? Czy konstruktywizm społeczny
mówiłby w~takich okolicznościach o~czymś realnym? Zdaje się, że konsekwentny konstruktywista w~takiej sytuacji odrzuciłby cały
konstruktywizm -- nie mogąc doprowadzić do jego uprawomocnienia -- lub musiałby odrzucić swoje kulturowe
przekonanie o~absolutnym nieistnieniu prawdy jako sądu wartego utrzymywania bez względu na otaczającą kulturę. Owa prawda musiałaby
się pojawić przynajmniej na poziomie metateoretycznym -- jako prawda o~nauce
\parencite[s.~134]{hardt_studia_2013}.
%\label{ref:RNDYgQXC7mKgS}(por. Hardt, 2013, s.~134).
Skoro zaś nauka jest działalnością społeczną, znaczyłoby to, że ów konstruktywista tym samym uznałby
możliwość istnienia prawdy odnośnie działalności społecznej człowieka -- może nie odnośnie całej jego działalności, ale
choćby tego jednego fragmentu, jakim jest działalność naukowa.

\section{Prawda i~realność}

Jak widać w~zagadnieniach związanych z~konstruktywizmem społecznym i~powyżej zarysowanymi problemami z~nim związanymi
kluczowe są pojęcia: \textit{prawdy }oraz\textit{ realności (rzeczywistości)}. Warto je zatem tutaj dokładnie rozważyć.
Nawiązując do konstruktywizmu i~sytuacji konstruktywisty w~świecie naiwnych realistów, prawdę można zacząć
minimalistycznie definiować jako sąd, którego autor lub propagator podziela, oraz chciałby, aby był podzielany przez
inne osoby. Nie jest to oczywiście korespondencyjna definicja prawdy, jednak skłania ona do refleksji nad przesłankami,
na podstawie których można domagać się uznania przez innych własnego sądu. Taką przesłanką powinno być coś wspólnie
doświadczanego w~danym czasie i~przestrzeni. Może nią być przykładowo wspólnie uznana skuteczność
technologiczna i~wtedy mielibyśmy do czynienia z~pragmatyczną koncepcją prawdy,
choć sama ocena zaistnienia skuteczności jest uwikłana w
uprzednie teoretyzowanie
\parencite[s.~264]{grobler_metodologia_2006},
%\label{ref:RNDAZOAoiyJgA}(Grobler, 2006, s.~264),
z~konieczności zatem odnosi się do wspólnego
doświadczenia innego niż skuteczność. Ponadto w~ramach tej koncepcji mogą wystąpić prawdy wzajemnie się wykluczające.
Powoduje to, że można domagać się uznania przez innych swojej prawdy dopóki nie pojawi się równie skuteczna lub
skuteczniejsza prawda, która jest sprzeczna z~tą poprzednią. Przyjęcie pragmatycznej koncepcji prawdy oznacza zatem
uznanie jej względności, czyli zależności od czasu i~miejsca jej wypowiadania. Na podobnej zasadzie koherencyjna
i~konsensualna koncepcja okazuje się przypisać prawdzie względność. Wspólnie doświadczać można również tę samą lub
chociaż taką samą część \textit{realności} -- co dokładne zostanie omówione później -- i~na podstawie tego domagać się od
innych osób uznania sądu o~niej. Warto zwrócić już teraz uwagę, że jeśli owa \textit{realność} ma pewną konieczną
cechę, mianowicie jest wewnętrznie niesprzeczna, to sąd, który byłby o~niej -- byłby z~nią \textit{zgodny} -- stałby się
prawdą bezwzględną i~zgodną z~jej korespondencyjną koncepcją. Dzieje się tak, gdyż wewnętrzna niesprzeczność realności
oznacza tutaj przede wszystkim to, że jeśli coś się w~niej wydarzy w~dany sposób, to nie może z~niej zniknąć, tak jak
może zniknąć przekonanie gremium sędziowskiego, że dany oskarżony zabił daną osobę. Należy tutaj dodać, że
korespondencyjna teoria prawdy często wiąże się z~realizmem
\parencite[s.~391]{gorazda_filozofia_2014},
%\label{ref:RNDWUfAn1YBkZ}(Gorazda, 2014, s.~391),
choć
zależy to w~dużej mierze od definicji owej \textit{realności}.

\subsection{2.1. Bezwzględność prawdy}

Bezwzględność prawdy -- jej niezależność od czasu, miejsca, okoliczności i~osoby ją wyrażającej -- sugeruje, że odnosi się
ona do czegoś bezwzględnego i~wewnętrznie niesprzecznego, jak zarysowana powyżej realność. Zatem uzasadnienie
bezwzględności prawdy wskazuje, że z~przedstawionych koncepcji najbardziej pasuje do niej ta korespondencyjna. W~pracy
Kazimierza Twardowskiego
\parencite*{twardowski_o_1900}
%\label{ref:RNDRjPn6HmkuC}(1900)
\textit{O tak zwanych prawdach względnych} pojawiły się argumenty za
bezwzględnością prawdy, które warto tutaj przytoczyć przy okazji dookreślając rozumienie kluczowych w~tym eseju pojęć:
\textit{prawdy }i \textit{realności}. W~trakcie wywodu będę się również odwoływał do pracy Groblera 
\parencite*{grobler_prawda_2000}
%\label{ref:RNDfbFxh0OKqT}(2000)
\textit{Prawda a~względność}, która stanowi bardziej współczesne spojrzenie na to
zagadnienie.


Jest wiele \textit{zdań}, które zdają się być prawdą i~fałszem w~zależności od okoliczności. Między innymi ,,deszcz
pada'', ,,chińskie produkty są tanie'', ,,zimna kąpiel jest zdrowa'', ,,Pluton jest planetą'', czy ,,to światło świeci na
zielono''. Jednak, zgodnie ze spostrzeżeniem Twardowskiego, pojęcie \textit{prawda }nie odnosi się do prawdziwego
\textit{zdania}, lecz \textit{prawdziwego sądu}, a~zatem zdania już zinterpretowanego
\parencite[por.][s.~27]{grobler_prawda_2000}.
%\label{ref:RNDDhjOUpeyMs}(por. Grobler, 2000, s.~27).
Samo zdanie nie jest w~stanie oddać w~pełni wszystkich elementów precyzujących opisywany sąd.
Jest ono tylko słownym wyrazem \textit{sądu} i~jego prawdziwość jest zrelatywizowana do danego rozumienia używanych
pojęć
\parencite[s.~47]{grobler_prawda_2000}.
%\label{ref:RNDvgSs5CBmMp}(Grobler, 2000, s.~47).
Absolutnie prawdziwe mogą być zatem jedynie sądy. Widać to
szczególnie dobrze na następującym przykładzie. Kiedy mówimy ,,Tak'' w~zależności od sytuacji możemy kłamać lub mówić
prawdę. Lecz czy to znaczy, że ,,Tak'' jest prawdą względną? Nie. Jedynie to samo słowo ,,Tak'' w~różnych sytuacjach wyraża
inny sąd. ,,Tak'' w~odpowiedzi na pytanie ,,czy byłeś dzisiaj w~szkole?'' znaczy coś zupełnie innego niż w~odpowiedzi na
pytanie ,,czy byłeś dziś na wagarach?''. Jak widać, dane słowo, ale i~zdanie, może wyrażać wiele sądów i~intuicyjnie każdy
zdaje sobie z~tego sprawę. Z~tego powodu, gdy jakiś sąd został niedokładnie lub nieodpowiednio wyrażony w~zdaniu,
zadajemy pytanie ,,Co masz na myśli mówiąc to?''. Oczywiście, w~odpowiedzi nie spodziewamy się, że zapytana osoba ukaże
nam wprost zawartość swoich myśli. Oczekujemy jedynie, że -- nadal używając jedynie słów, ewentualnie gestów -- przybliży
lub doprecyzuje swój opisywany sąd. Widać zatem, że pojęcie \textit{prawdy }nie ma sensu bez osoby, która ją podziela,
a~zatem osoby, która kategoryzuje pewien wycinek swojego doświadczenia na indywidua, ich własności i~relacje między
nimi, mając na celu przekazanie lub zachowanie jakiegoś przekonania o~tym wycinku
\parencite[s.~79]{grobler_prawda_2000}.
%\label{ref:RNDqOfcsLjgfT}(Grobler, 2000, s.~79).


Bezwzględność prawdy zdefiniowana została jako jej niezależność od \textit{czasu, miejsca, okoliczności} jej
przedstawiania oraz \textit{osoby}, która ją podziela. A~jednak ,,deszcz pada'' w~pewnym momencie i~miejscu wypowiedziane
jest prawdziwe, a~w~innym fałszywe. Dzieje się tak, gdyż w~wypowiedziach przeważnie ogranicza się do minimum
wypowiadane słowa. Zazwyczaj wszelkie elementy sądu, które są domyślne, nie są wyrażane. Domyślne w~zdaniu o~deszczu
jest choćby ,,teraz'' -- na co wskazuje czas teraźniejszy i~,,tutaj''. I~mimo tego, że zdanie ,,deszcz pada teraz i~tutaj''
nadal pozostaje raz prawdziwe, raz fałszywe, to łatwo zauważyć, że gdy pod ,,teraz'' podstawi się dokładne określenie
czasu -- chwilę wypowiadania zdania, a~pod ,,tutaj'' dokładne określenie miejsca, gdzie zostało ono wypowiedziane, to samo
zdanie może być oceniane jako prawdziwe (lub fałszywe) niezależnie od miejsca i~czasu wypowiadania. Warto zwrócić
jeszcze tutaj uwagę, że nasze ludzkie umysły operują w~czasie i~przestrzeni. Zatem wszelkie zjawiska fizyczne
przedstawiają się nam jako umieszczone w~danym czasie i~przestrzeni, przynajmniej w~makroświecie codziennych
doświadczeń. Zdanie ,,deszcz pada'' bez określenia -- choćby domyślnie -- czasu i~umiejscowienia w~przestrzeni nie ma więc
sensu. Można powiedzieć za to, że ,,deszcz pada zawsze i~wszędzie'' i~to jest oczywisty fałsz, jak również, że ,,gdzieś
i~kiedyś padał deszcz'' i~jest to oczywista prawda.


Przyjrzyjmy się teraz zdaniu ,,chińskie produkty są tanie''. Można wskazać na kilka tanich chińskich
produktów i~powiedzieć, że to prawda lub wskazać na kilka drogich i~powiedzieć, że to fałsz. Słowo ,,chińskie produkty'' to
oznaczenie jakiejś grupy przedmiotów. Zatem gdy o~nich jest mowa, to chodzi o~całą grupę albo jej większą lub mniejszą
część. Zatem w~zdaniu ,,chińskie produkty są tanie'' może chodzić, że ,,wszystkie chińskie produkty'' takie są, co zdaje
się być fałszem lub ,,większość z~nich'', co mogłoby okazać się prawdą. Zjawisko taniości też występuje w~czasie, więc
domyślnie dołącza się również ,,współcześnie'', czy też ,,w przeciągu ostatnich lat''. Jeszcze słowo ,,tani'' można tutaj
rozumieć na różny sposób. Wypowiadającemu to zdanie może chodzić o~,,tańsze niż średnie ceny podobnych produktów na
rynku'', albo też ,,w takich cenach, że za płacę jaką mam, mogę kupić te produkty w~większej ilości niż potrzebuję''.
W~każdym dookreśleniu wydanego sądu przedstawia się on jako bezwzględnie prawdziwy lub fałszywy.


Podobnie jest ze zdaniem ,,zimna kąpiel jest zdrowa''. Można powiedzieć -- dla jednych zdrowa, dla innych nie. Aczkolwiek
sam przymiotnik ,,zdrowa'' sugeruje dopełnienie ,,dla kogo''. ,,Zdrowa'' wypowiadane jest wszak w~ znaczeniu ,,sprzyjająca
zdrowiu danej osoby''. I~znów można rzec ,,zimna kąpiel jest zdrowa dla każdego i~w~każdym wypadku'', co zdaje się być
fałszem. Można też powiedzieć ,,zimna kąpiel bywa zdrowa w~danych przypadkach'' i~je wymienić, zatem określić
okoliczności. Można tutaj powoli zauważać, że stanowisko o~względności prawdy bierze się między innymi z~utożsamienia
prawdy ze zdaniami, w~których okoliczności nie są podane. Jednak jak widać, przy danych sądach brak określenia (choćby
domyślnie) okoliczności sprawia, że nie mają one sensu. Jaki sens ma padanie deszczu bez czasu i~przestrzeni, chińskie
produkty bez określenia, które z~nich (wszystkie, część z~nich lub żadne) albo sprzyjanie zdrowiu bez określenia, czyjemu
zdrowiu? Zwracanie uwagi na te domyślne okoliczności nie jest zatem sprytnym ubezwzględnianiem prawd
względnych, a~wskazaniem, że bez tych okoliczności sądy (prawdy lub fałsze) nie mają wcale znaczenia.


Warto nadmienić jak jest tu rozumiane słowo ,,\textit{domyślne}''. ,,\textit{Domyślne}'' nie oznacza, że każda osoba przy
wypowiadaniu zdaniem danego sądu ma w~myślach wszystkie niewypowiedziane szczegóły świadomie określone. Chodzi
raczej o~to, że aby uzyskać w~ogóle sens danego sądu, należałoby ,,\textit{domyśleć}'' pominięte fragmenty.
I~to \textit{domyślenie}
wskazywałoby jednoznacznie, czy dany sąd jest prawdziwy czy fałszywy, przy czym słowne uszczegóławianie jednego sądu
mogłoby nieraz trwać w~nieskończoność. Brak owego domyślenia lub domyślenie nieprecyzujące, jak ,,gdzieś i~kiedyś''
sprawia, że dany sąd odnosi się do nieskończenie wielu światów możliwych i~albo jeden z~nich jest tym rzeczywistym, albo
żaden z~nich -- jak w~przypadku sądu ,,zawsze i~wszędzie pada deszcz''. Owe domyślanie odnosi się na podobnej zasadzie do
domyślania znaczenia pojęć. Najlepiej zobrazować to nawiązując do paradoksu \textit{grue} (zielbieski) sformułowanego
przez Goodmana, zgodnie z~którym każde doświadczenie potwierdzające, że wszystkie szmaragdy są zielone, potwierdza
jednocześnie nieskończenie wiele innych sprzecznych z~tą właściwością cech, na przykład to, że szmaragdy są zielbieskie,
czyli zielone do roku 2050, a~potem niebieskie
\parencite[s.~68]{grobler_prawda_2000}.
%\label{ref:RNDifYrkWHWLD}(Grobler, 2000, s.~68).
W~tym kontekście
używając pojęcia ,,zielone'' w~odniesieniu do szmaragdów nie trzeba domyślać tego, czy po oderwaniu od nich wzroku
zmieniają one kolor na inny, czy pozostają zielone. W~swoim sądzie o~szmaragdach można odnosić się do ich własności
jedynie w~chwili patrzenia na nie, a~zatem pojęcie ,,zielone'' domyślać jako ,,zielone w~chwili patrzenia na
nie, a~potem nie wiem, nie będę spekulować''. Oczywiście takie ograniczanie używanych pojęć przeszkadzałoby
choćby w~wykonywaniu
przewidywań, dlatego naturalnym jest rozszerzanie ich znaczenia tak, aby informowały też o~przyszłych stanach rzeczy.
Choć dzieje się to kosztem mniejszej pewności co do prawdziwości podzielanych za ich pomocą sądów.


Przyda się tutaj również nawiązanie do koncepcji radiacyjnej struktury pojęć opracowanej przez Eleonorę Rosch,
zgodnie z~którą potoczne pojęcia kategoryzujące świat odnoszą się do ,,klas o~granicach
rozmytych, z~charakterystyczną strukturą
centrową i~peryferyjną''
\parencite[s.~110]{grzegorczykowa_o_1998}.
%\label{ref:RNDTzn5xoonRZ}(Grzegorczykowa, 1998, s.~110).
W~centrum znajdują się prototypy,
pewne wzorce dla danego pojęcia, natomiast na peryferiach znajdują się kategorie, które jedni zaliczyliby do desygnatów
danego pojęcia, a~inni nie. Oczywiście te prototypy i~peryferia zależą w~dużej mierze od kultury. Prototypem pojęcia
\textit{warzywo} jest w~polskiej kulturze zazwyczaj marchew, dlatego nikt w~Polsce nie ma wątpliwości co do tego, że
marchew jest warzywem. Inaczej jest natomiast w~kwestii pomidora czy papryki, dlatego nieraz odbywają się dyskusje, czy
\textit{rzeczywiście} są to warzywa. Z~analogicznych powodów w~filozofii ekonomii odbywają się spory, które zjawiska
uznać za społeczne i~które wśród tych społecznych uznać za gospodarcze, a~zatem co jest \textit{właściwym} przedmiotem
ekonomii. Zdaje się, że dopóki dane pojęcie mające porządkować postrzegane zjawiska powiązane jest jedynie
skojarzeniami z~prototypem zależnym od kultury, to spory o~to, co jest, a~co nie jest \textit{warzywem} lub
\textit{właściwym }przedmiotem ekonomii mogą trwać w~nieskończoność. Można taką sytuację porównać do
obrysowywania w~różny sposób obszarów na tej samej mapie i~mówienia, że dane pojęcie odwołuje się do fragmentu wewnątrz tego
obrysowania. Dopóki określanie pojęć jest tylko tym obrysowaniem, można spierać się używając różnych kryteriów, jakie
obszary powinny być od siebie oddzielone, a~jakie powinny być połączone w~danym pojęciu. Dopiero przy wspólnym
uzgodnieniu na zasadzie umowy jednego kryterium, na przykład własności, która ma być łączona z~tym pojęciem, pojawia się
sens sporu o~resztę zawartości danego pojęcia, a~zatem czy z~daną własnością w~rzeczywistości łączą się zawsze, nigdy
lub czasem inne określone własności. Jeśli zatem na zasadzie umowy ustalimy, że \textit{warzywo} to \textit{jadalna
część rośliny}, to sensowny i~mający rozwiązanie staje się spór o~to, czy jabłko jest warzywem lub też czy warzywem
może być tylko korzeń. Aczkolwiek warto zwrócić uwagę, że definicje również składają się z~pojęć -- tutaj
\textit{jadalność, część, roślina} -- co do których także na zasadzie umowy należałoby uzgodnić jedną własność, aby był
sens się o~spierać o~ich znaczenia. Podobnie przy ustaleniu, że przedmiotem badań ekonomii są zjawiska gospodarcze
należy się następnie pochylić nad pojęciem \textit{gospodarczości}. Takie ujęcie zdaje się prowadzić do
regresu w~nieskończoność przy wspólnym ustalaniu znaczenia pojęć, choć coraz bardziej podstawowe pojęcia
odwołujące się do coraz
prostszych zjawisk stają się też łatwiejsze do ,,obrysowania''. Konieczne jest tutaj zaznaczenie, że przyporządkowanie
pojęć określonym fragmentom rzeczywistości na podstawie umowy nie podważa istnienia prawdy jako prawdziwego sądu
\parencite[s.~145]{grobler_prawda_2000}
%\label{ref:RNDTsj91YQI9z}(Grobler, 2000, s.~145)
bez względu na tę umowę. Ponadto pojęcia mogą być zakreślone w~sposób
mniej lub bardziej użyteczny ze względu na treść sądów, które mają być wyrażane. Z~tego też powodu pojęcia naukowe
zmieniają się mniej lub bardziej wraz z~nowymi odkryciami i~ujęciami teoretycznymi, co zostanie omówione poniżej.


Przemiany następujące w~pojęciach są szczególnie istotne w~związku z~prawdziwością sądów naukowych. Wiąże się to z~tezą
o~niewspółmierności teorii naukowych
\parencite[s.~88]{grobler_metodologia_2006}.
%\label{ref:RND1yueElbIj5}(Grobler, 2006, s.~88).
Przed 2006 rokiem prawdą był sąd
wyrażany zdaniem, że ,,Pluton jest planetą'', a~obecnie tak już nie jest. Wynika to z~tego, że na podstawie coraz
większej liczby obserwacji okazało się, że w~Układzie Słonecznym istnieje wiele podobnych obiektów do Plutona. Pluton
zaś jest bardziej podobny do nich niż do tych ośmiu obiektów, które zachowały status planety w~naszym Układzie. Na tej
podstawie zmieniono pojęcie znajdujące się pod słowem ,,planeta'' i~dodano pojęcie ,,planeta karłowata'', gdyż taki układ
pojęciowy jest bardziej użyteczny ze względu na opisywanie badanych zjawisk. Zatem Pluton był planetą w~rozumieniu tego
pojęcia w~~astronomii sprzed 2006 roku, natomiast nie jest planetą we współczesnym rozumieniu tego pojęcia. Podobnie
budowla na wzgórzu Wawel nad Wisłą w~Krakowie jest zamkiem w~rozumieniu budynku, ale nie jest zamkiem w~rozumieniu
suwaka.


Lecz czy takim sposobem nie można by uznać wszelkiej teorii naukowej, przykładowo fizyki, za w~pełni prawdziwą, w~każdym
stadium rozwoju od fizyki Arystotelesa przez tą Newtona aż po współczesną, twierdząc, że odnosiły się one jedynie do
nieco innych aspektów badanych zjawisk? Sądzę, że nie. Na przykład równania ruchu stosowane w~mechanice
klasycznej w~świetle dzisiejszej wiedzy naprowadzają na fałszywe sądy na temat rzeczywistości, zatem błędnie ją oddają.
Tak samo jak
błędnie oddawały ją w~XIX wieku, choć na ich podstawie można było budować maszyny, które działają, gdyż odchylenia od
rzeczywistości wynikające z~błędu są na poziomie błędów pomiaru -- przynajmniej przy wielkościach, z~którymi mamy na co
dzień styczność. Jednak pojęcie grawitacji -- postrzegając je minimalistycznie, jedynie jako coś, co powoduje spadanie
jabłek\textit{ }i zarazem nieuciekanie planet od Słońca -- w~dzisiejszej wiedzy zostało zachowane, czyli nadal zdaje się
prawdziwe i~jednocześnie jest niezgodne z~fizyką arystotelesowską. Oczywiście skłania to do wniosku, że również
dzisiejsza wiedza w~świetle przyszłej w~wielu aspektach może okazać się fałszywa czy też pojęcia używane w~niej mogą
okazać się zupełnie nieprzystające do pojawiających się doświadczeń, przez co sądy formułowane za ich pomocą nie mają
sensu, tak jak wszystkie sądy o~cechach eteru z~perspektywy współczesności. Można zatem zaryzykować stwierdzenie, że
jedna część sądów naukowych składa się z~prawd -- choćby przy maksymalnym ograniczaniu znaczeniowym stosowanych w~nich
pojęć, druga z~fałszów, a~trzecia i~ostatnia część zawiera tezy z~fałszywymi presupozycjami, jak tezy o~cechach eteru
zawierające presupozycje, że eter istnieje
\parencite[s.~99]{grobler_metodologia_2006}.
%\label{ref:RND5RNmxQdKuz}(Grobler, 2006, s.~99).
Oczywiście zdaje się, że
nigdy nie nadejdzie chwila, w~której będzie można jednoznacznie i~ostatecznie określić, które sądy należą do danej
kategorii. Zgodnie z~koncepcją Quine’a można za to określać, które koniunkcje wszystkich założeń i~hipotez są fałszywe
i~tym samym wyeliminować niektóre wyjaśnienia świata z~nieskończonego zbioru jego możliwych wyjaśnień
\parencite[s.~77]{grobler_metodologia_2006}.
%\label{ref:RND2QpWDqqtYH}(Grobler, 2006, s.~77).
Należy tutaj jednak zwrócić uwagę, że zbiór tych możliwych
wyjaśnień z~każdym usunięciem z~niego tych obalonych wciąż pozostaje nieskończony.


Na podstawie tych rozważań przeciwnik korespondencyjnej koncepcji prawdy mógłby wskazać, że część sądów prawdziwych
jak i~fałszywych jest właśnie pusta, a~wszelkie opierają się na błędnej presupozycji, zgodnie z~którą ludzki umysł nadaje
się do jakiegokolwiek odzwierciedlenia rzeczywistości. Ponadto mógłby podważyć sens mówienia o~\textit{prawdzie}, czyli
odzwierciedleniu realności przez sąd w~czyimś umyśle, skoro nigdy nie dałoby się jej jednoznacznie i~z~pewnością
określić w~żadnym umyśle. Sądzę, że te zastrzeżenia są zasadne, choć można się z~nimi uporać przy odpowiednim,
minimalistycznym rozumieniu \textit{realności}.

\subsection{2.2. Warstwy realności}

Sąd jest zjawiskiem psychicznym, a~różne osoby nie tylko różne pojęcia oznaczają tymi samymi słowami lub różnymi słowami
oznaczają te same pojęcia, ale mogą również różnie odbierać te same bodźce lub tak samo odbierać różne. Na przykład
grupa daltonistów może nie zgodzić się z~sądem opisanym zdaniem ,,to światło świeci na zielono'', nawet przy podaniu
wszelkich szczegółów i~okoliczności zaświecenia danego światła, podczas gdy nie-daltoniści będą przekonani, że ich sąd
jest prawdziwy. Jednak jak wiadomo kolory to zjawiska -- fenomeny, które powstają w~wyniku interpretacji przez umysł
długości fali świetlnej, natomiast nic samo w~sobie zielone nie jest. Zatem zdanie ,,to światło świeci na zielono''
wyraża sąd ,,to światło świeci na zielono dla osób, które daną długość fali postrzegają jako zielony'', co w~skrócie
można wyrazić ,,to światło świeci na zielono dla pewnych osób''. Co nam jednak po \textit{prawdzie}
bezwzględnej, w~której koniecznie trzeba określać, że jest
tak \textit{dla osoby X i~Y}, a~już nie jest tak \textit{dla osoby A~i~B}? W~takiej sytuacji nie można się nią
przecież dzielić, ani wymagać od kogoś, aby była ona, \textit{także dla tego kogoś},
uznawana. Wymagać można jedynie od innych uznania, że dla danych osób dany sąd jest słusznie prawdziwy. I~dzieje się
tak, że gdy wyrażamy zdanie ,,jest zielone światło -- jedź'' to \textit{domyślnie }chodzi o~to, że dana długość fali jawi
się jako zielona dla określonych osób, nie-daltonistów. W~przypadku ,,\textit{zielonego światła}'' osoba mówiąca o~nim
może \textit{domyślać }i sądzić, że \textit{zielone} jest światło samo w~sobie lub że jest to jedynie postrzegane
zjawisko. W~pierwszym wypadku jest w~błędzie, a~w~drugim ma rację. Może też nigdy nie wgłębiać się w~to
zagadnienie i~nie \textit{domyślać }sądu na temat kolorów do tego momentu, aby świadomie to określić. Jednak oczywiście do
codziennego, bezpiecznego poruszania się po drodze nie potrzeba mieć prawdziwego sądu w~pełni domyślonego na temat
światła, a~nawet wystarczy sąd fałszywy. Możliwość określenia fałszywości sądu na tym przykładzie przy jego
skuteczności ukazuje pewne braki w~pragmatycznej koncepcji prawdy w~stosunku do  korespondencyjnej.


Wróćmy jednak do konieczności uzupełniania \textit{prawdy} dopiskiem \textit{dla osoby X.} Widzenie kolorów jest tutaj
dobrym przykładem. Podczas egzaminu na prawo jazdy prawidłową, a~zatem wyrażającą \textit{prawdę}, odpowiedzią na
pytanie ,,Czy tu zapala się zielone światło?'' jest ,,Tak''. Dzieje się tak, ponieważ większość osób w~ten sam sposób
odbiera kolor tego światła i~w~ten sam sposób odbiera kolor trawy, pojawia się tutaj zatem pewien stopień
intersubiektywności. Warto zwrócić uwagę, że nie jest tu istotne czy rzeczywiście dana długość fali skutkuje takimi
samymi qualiami w~danej grupie osób -- czego nie sposób określić -- a~jedynie czy w~obrębie pewnych obiektów postrzegane
jest takie samo podobieństwo kolorów jak wśród pozostałej większości umysłów. Kolor trawy w~procesie wychowywania
nazwano zielonym, więc i~dane światło jest zielone. Bez tej intersubiektywnej powtarzalności, a~zatem w~jakiejś mierze
wspólnych zakresów pojęć ukrytych pod słowami, nie mógłby funkcjonować system transportu drogowego, który jest przecież
międzykulturowy. Prawda o~kolorze świateł nie jest mimo to jedynie prawdą pragmatyczną z~definicji, a~jest to
korespondująca z~rzeczywiście istniejącymi -- intersubiektywnie postrzeganymi zjawiskami prawda, dla której skuteczność
jej użycia w~działaniu międzykulturowym stanowi jedynie kryterium, które ją potwierdza.


Co w~tym przypadku odsłania się jako \textit{realność}? Nie jest to świat rzeczy samych w~sobie, które istnieją poza
działalnością poznawczą człowieka, a~jedynie świat fenomenów -- zjawisk, i~to zjawisk codziennych. Zatem sądy w~umysłach
są tutaj zgodne ze zjawiskami w~umysłach i~wtedy są prawdziwe. Zjawiska są w~pełni zależne od sposobu funkcjonowania
umysłu, a~umysły osób zdają się być oddzielne, więc nieraz mogą być różne. Z~tego wynika niezgoda
daltonistów z~nie–daltonistami i~konieczność dopisku \textit{dla osoby X} (gdy mówimy o~zjawiskach). Aczkolwiek, jak pokazuje
funkcjonowanie systemu dróg i~autostrad, można z~dużą dozą prawdopodobieństwa założyć intersubiektywność zjawisk do
pewnego stopnia, co pozwala już przekazywać sobie prawdy (na temat zjawisk) i~przekonywać się nawzajem w~wielu
istotnych kwestiach. Zatem pozostaje pytanie, kiedy i~do jakiego stopnia zakładać i~wyznaczać intersubiektywność
fenomenalnej warstwy realności?


Zdaje się, że gdy zamiast o~kolorach światła, mówimy o~długości fali światła, to już nie trzeba dodawać \textit{dla
osoby X.} Można więc stwierdzić, że im bardziej badamy fenomeny i~wchodzimy w~głąb rzeczywistości od najbardziej
powierzchownych zjawisk niejako w~kierunku rzeczy samych w~sobie, tym bardziej nasze \textit{sądy} się
absolutyzują w~tym sensie, że zamiast  domyślać do nich \textit{dla osoby X} można domyślać \textit{dla wszystkich osób.}
Mimo to jednak nadal fale elektromagnetyczne są pewnego rodzaju zjawiskami, a~raczej pewnym teoretycznym
uporządkowaniem zjawisk intersubiektywnie doświadczanych przez naukowców. Dobrze to widać na przykładzie zjawiska
temperatury. Absolutną prawdą jest, że ,,jest zimno \textit{danej osobie}'', podczas gdy innej osobie jest ciepło.
Natomiast dla obu z~nich jest około 17 stopni Celsjusza, a~zatem (dla obu) płyn w~termometrze dochodzi do danej kreski.
Za pomocą teorii, która \textit{skonstruowała} pojęcia dobrze pasujące do badanych zjawisk, można to przełożyć
na sąd o~mikrostanie cząsteczek atmosferycznych, które poruszają się z~taką średnią prędkością, która wywołuje
temperaturę 17 stopni Celsjusza. Jednak cząsteczki atmosfery nie są rzeczami samymi w~sobie, a~pewną kategoryzacją
zjawisk, a~nawet -- tutaj użyję tego słowa -- są konstruktem, czyli sposobem postrzegania rzeczywistości na
podstawie współczesnej teorii.
Można jednak \textit{w pełni intersubiektywnie }oceniać adekwatność tego konstruktu do wciąż powiększanej puli
doświadczeń i~adekwatność do potrzeb poznawczych danej dziedziny wiedzy
\parencite[s.~95]{grobler_prawda_2000}
%\label{ref:RNDwpnTa67UZh}(Grobler, 2000, s.~95)
i~przy tym minimalistycznie konstruować używane pojęcia, co przekłada się na mniejszą niepewność konstruowanych za ich
pomocą sądów.


Przykładem jak działają potrzeby poznawcze może być tutaj różny sposób dookreślania pojęcia rzeki. Rzekę można rozumieć
jako ciek wodny wraz z~płynącymi w~nim elementami bez skał wystających z~niego, które nie płyną. Jednak alternatywnie
do pojęcia rzeki można też doliczyć te stojące skały. W~tym drugim wypadku sąd zawarty w~zdaniu ,,(cała) rzeka płynie''
nie jest prawdziwy -- jest bliski prawdzie, bo do zmiany jego wartości logicznej wystarczy delikatnie zmodyfikować
pojęcie rzeki. Zaś w~tym pierwszym rozumieniu pojęcia rzeki owe zdanie jest prawdziwe, zatem do wyrażenia tego sądu to
pierwsze pojęcie lepiej spełnia potrzeby poznawcze. Gdy chcemy natomiast powiedzieć dziecku ,,nie chodź po rzece w~parku
narodowym'', może nam raczej chodzić o~rzekę wraz z~kamieniami z~niej wystającymi. Chodzi zatem o~dwa, nieznacznie
różniące się pojęcia. Równie dobrze można by dać tym pojęciom nazwy odpowiednio ,,rzeka'' i~,,subrzeka'' i~wtedy nie byłoby
nieporozumień, jednak nie jest to potrzebne na co dzień -- przynajmniej w~naszej kulturze. W~kulturze Inuitów na
przykład precyzyjniejsze niż u~nas rozróżnianie rodzajów śniegu okazuje się potrzebne
\parencite[s.~148]{grobler_prawda_2000}.
%\label{ref:RND35uGBnBgaS}(Grobler, 2000, s.~148).
Warto zwrócić tutaj uwagę, że w~nauce, gdzie precyzyjność wyrażeń jest
o~wiele ważniejsza niż w~życiu potocznym, spotyka się właśnie takie pojęcia jak ,,atomy'' i~,,cząstki subatomowe''. Takie
precyzyjne dookreślanie pojęć naukowych powoduje jednocześnie częstszą konieczność ich modyfikacji
\parencite[s.~118–122]{grobler_prawda_2000}.
%\label{ref:RNDP1pytzI4q3}(zob. Grobler, 2000, s.~118–122).


Zgodnie ze współczesnym stanem wiedzy fizycznej cząsteczki są zatem dobrze spełniającym potrzeby poznawcze
\textit{konstruktem} stworzonym na określenie grup atomów, atomy to w~większości pusta przestrzeń i~jądro, a~jądro to
nic innego jak protony i~neutrony, które z~kolei są wynikiem połączenia się kwarków. A~czym są kwarki? Też pewnym
pojęciem opisującym zjawiska -- w~tym wypadku zjawiska zaobserwowane podczas eksperymentów naukowych. Ponadto niektórzy
pod kwarkami upatrują również struny drgające w~co najmniej dziesięciu wymiarach. Jak widać zatem na przykładzie
fizyki, z~jednej strony kontinuum mamy fenomeny silnie zależne od postrzegającego w~danej chwili umysłu, jak
smaki i~kolory, które postrzegamy na co dzień. Badając zaś świat coraz głębiej zmniejsza się zależność od ukształtowania umysłu
danego człowieka, natomiast zwiększa się potrzeba interpretacji tego, co jest postrzegane w~eksperymentach. Nadal
jednak obracamy się w~sferze zjawisk -- nawet mówiąc o~,,kwarkach''. Umysły kosmitów różne od ludzkich umysłów, patrząc na
wyniki tych samych eksperymentów, które przeprowadziliśmy, mogłyby spostrzec coś zupełnie innego i~na tej podstawie
opracować zupełnie odmienne pojęcia niż my. Zdaje się zatem, że do wielu sądów, w~tym naukowych, wypada domyśleć ,,tak
się to jawi intersubiektywnie w~umysłach ludzkich''. Natomiast nauka powinna w~tym kontekście dążyć do \textit{pełnej
intersubiektywności}, jako największego z~możliwych zasięgu obowiązywania odkrywanych prawd.


Po takich rozważaniach można jednak się zastanowić, czy zamiast ,,tak się to jawi intersubiektywnie w~umysłach ludzkich''
nie powinno być dopisywane ,,tak się to jawi w~danej kulturze w~umysłach ludzkich''. Gdyby to drugie było
największym z~możliwych rozszerzeń intersubiektywności, to mielibyśmy do czynienia właśnie z~konstruktywizmem kulturowym
i~mówiłby on o~czymś realnym. Choć oczywiście sąd ,,kultura determinuje poznanie -- tak się to jawi w~pełnej intersubiektywności''
również nie mógłby się pojawić i~pozostałby sąd: ,,kultura determinuje poznanie -- tak się to jawi w~naszej kulturze''.
Pewną zasadność konstruktywizmu można zobrazować nawiązując do eksperymentu Putnama o~bliźniaczej Ziemi, która jest
niemal identyczna z~tą, którą znamy. Jedyną różnicą w~niej jest to, że płyn, który zachowuje się tam jak u~nas woda,
wypełnia te same miejsca i~nadaje się do takiego samego użytku, ale nie ma struktury chemicznej H\textsubscript{2}O,
a~jakąś bardziej skomplikowaną -- nazwijmy ją XYZ. Zdaniem Putnama w~takiej sytuacji płyn o~budowie chemicznej XYZ
należałoby nazwać inaczej niż woda
\parencite[s.~61]{grobler_prawda_2000}.
%\label{ref:RNDlEIwsbm0x8}(Grobler, 2000, s.~61).
W~toku dalszych badań może się
jednak okazać, że na poziomie kwarków lub głębszym struktura H\textsubscript{2}O jest tożsama z~XYZ. W~takim wypadku
naukowe w~pewnym okresie odróżnianie H\textsubscript{2}O od XYZ wystąpiło jedynie ze względu na historię i~sposób
prowadzenia badań w~danym czasie, a~zatem zależało od kultury. Nawiązując do historii nauki analogiczna sytuacja
wystąpiła odnośnie rozróżniania zjawisk w~sferze podksiężycowej i~nadksiężycowej. Zatem w~kulturze dzisiejszej,
uwzględniającej historię rozwoju nauki, może zdawać się zasadna teza o~pełnym kulturowym zdeterminowaniu sądów
naukowych. Sądzę jednak, że również historyczny fakt możliwości uzgadniania niektórych spraw międzykulturowych -- jak
światowy system transportu drogowego -- oraz fakt częściowego rozumienia odległych historycznie kultur stanowi silny
argument za istnieniem \textit{pełnej intersubiektywności} przynajmniej w~obrębie niektórych zjawisk.


Istotne jest tutaj ponowne odwołanie się do tego, jak mały zakres informacji mogą zakreślać i~często zakreślają używane
pojęcia, tym samym będąc w~pełni odporne na zmianę kultury lub teoretycznych ujęć. Dobrze obrazuje to sytuacja ze
zjawiskiem padania deszczu. Można wszak powiedzieć, że przecież deszcz nie \textit{pada}, gdyż padać to bezwładnie
lecieć \textit{z góry na dół}, natomiast \textit{góra i~dół }przecież nie istnieją. Tak naprawdę deszcz jest
przyciągany przez grawitację! Czy jednak osoba wyrażająca zdanie ,,deszcz pada'' ma na myśli, że spada on z~obiektywnej
góry na obiektywny dół? Zazwyczaj nie. Po angielsku zdanie oznaczające \textit{ten sam }sąd można wyrazić ,,It’s
raining'', a~zatem coś na kształt neologizmu ,,deszczy się''. Zatem słowem ,,padanie'' w~odniesieniu do deszczu nie określa
się w~pełni i~dokładnie procesu związanego ze zjawiskiem, a~jedynie wyraża zaistnienie wrażenia deszczu -- w~tym wypadku
samo zjawisko deszczu jest już jego padaniem. Dlatego też jeśli po wielu badaniach okaże się, że tak naprawdę deszcz
(sam) nie pada, a~jest np. wyciskany z~gąbek przez anioły, to sądy o~padaniu deszczu nie staną się fałszywe, a~zatem
już dziś można bez większych wątpliwości określać ich bezwzględną, \textit{w~pełni intersubiektywną} prawdziwość.


Rzeczywistość, o~której mowa w~większości analizowanych powyżej sądów, można nazwać \textit{rzeczywistością zjawisk
codziennych}. Te \textit{zjawiska codzienne} są już pełnoprawnym odniesieniem dla sądów na ich temat, czyli już te
zjawiska \textit{same w~sobie} pozwalają mówić o~\textit{prawdzie} jako o~sądzie zgodnym z~nimi -- z~rzeczywistością.
Owe zjawiska codzienne -- potoczne -- są bardzo istotne nie tylko dlatego, że stanowią punkt wyjścia rozważań nad tak
zwanymi \textit{głębszymi warstwami rzeczywistości} omówionymi wcześniej, ale również ich nazwy stanowią źródło metafor
dla języka zajmującego się pozostałymi zjawiskami, w~tym języka naukowego. Zatem język nie może być ostatecznie
\textit{określający}, a~jedynie \textit{wskazujący }i~\textit{sugerujący}. Wskazujący przez wiele odniesień do zjawisk
codziennych i~sugerujący pozostałe zjawiska na ich podstawie. Natomiast \textit{sąd} może być ostatecznie
\textit{określający}, mimo że nie zawsze jest do końca pomyślany. Język nie jest konieczny dla jego zaistnienia, choć
wspiera wyrażanie i~analizę sądów.


Na koniec omawiania warstw rzeczywistości warto pochylić się nad pojęciem \textit{obiektywności} i~tym samym
\textit{realności obiektywnej} w~odróżnieniu od \textit{realności w~pełni intersubiektywnej}. Patrząc na \textit{Krzyk} Muncha
można wyrazić prawdziwy sąd, że ,,postać na tym obrazie krzyczy'', odnosząc się do rzeczywistości obrazu. Jednak
już w~ramach \textit{rzeczywistości zjawisk codziennych} należałoby stwierdzić, że ,,kolory układają się w~kształty
przypominające krzyczącą postać''. Zaś z~perspektywy rzeczywistości niektórych teorii fizycznych należałoby
wejść w~zagadnienia długości fali świetlnej. Wszystkie te \textit{różne }rzeczywistości są intersubiektywne i~dla większości
ludzi zrozumiałe. Widać też, że są one w~pewien sposób od siebie zależne. Rzeczywistość długości fal świetlnych mogłaby
istnieć nienaruszona bez rzeczywistości interpretacji danego obrazu, natomiast odwrotnie już nie. Choć zatem Popper był
zdania, że rzeczywistości na różnych poziomach są równie realne
\parencite[s.~265]{grobler_metodologia_2006},
%\label{ref:RNDPFZg8cPhbT}(Grobler, 2006, s.~265),
to
zgodnie z~tym rozumowaniem \textit{rzeczywistość fal świetlnych} zdaje się być bardziej \textit{realną rzeczywistością}
niż \textit{rzeczywistość obrazu}. Jeśli tak to określić, to znaczyłoby, że najbardziej realna
rzeczywistość -- a~zatem\textit{ realność obiektywna} -- to ta, która mogłaby istnieć bez pozostałych, a~na której opierałyby się
pozostałe. Owa \textit{realność }niekoniecznie musiałaby być jedna, gdyż na przykład w~perspektywie dualizmu
antropologicznego takie realności mogłyby się niejako równolegle przenikać
\parencite[por.][]{judycki_swiadomosc_2004}.
%\label{ref:RND4mQxYRnPeH}(por. Judycki, 2004).

\section{Realne problemy realizmu}

Po dookreśleniu pojęć \textit{prawdy i~realności }można przejść do rozważenia stanowiska realizmu w~ekonomii.
Przedstawicielom tego stanowiska zarzuca się nieuwzględnianie kulturowego zdeterminowania procesu tworzenia wiedzy
\parencite[s.~396]{zboron_podzial_2014}.
%\label{ref:RND4gEoEgP0rf}(Zboroń, 2014, s.~396).
Nie jest jednak tak, że nie zwracają oni uwagi na wpływ kultury na
badania naukowe. Sądzą jedynie, że ów wpływ nie uniemożliwia dostępu do pozakulturowych, realnych faktów
\parencite[s.~72]{kincaid_realistic_2009}.
%\label{ref:RND5PVTJKeItB}(Mäki, 2009, s.~72).
Jednak w~związku z~ekonomią można znaleźć wiele innych, pozakulturowych
przeszkód w~dostępie do \textit{w pełni intersubiektywnej} realności. Zgodnie z~tym, co twierdzi Mäki -- czołowy
przedstawiciel realizmu -- tworzenie prostych modeli jest najlepszym sposobem dostępu do realności
\parencite[s.~18]{kincaid_realistic_2009},
%\label{ref:RNDAEWrNNvzgu}(Mäki, 2009, s.~18),
a~modele w~ekonomii, które są eksperymentami myślowymi
\parencite[s.~16]{kincaid_realistic_2009},
%\label{ref:RNDOziK5D8KLR}(Mäki, 2009, s.~16),
można porównać do fizycznych eksperymentów
przeprowadzanych w~laboratoriach. Jak w~fizyce izoluje się badane obiekty od otoczenia,
tak w~ekonomii stosuje się różnego rodzaju
idealizacje i~izolacje, zostawiając te właściwości obiektów, które są istotne dla badanych zjawisk
\parencite[s.~14]{kincaid_realistic_2009}.
%\label{ref:RNDjPm2vR6GSw}(Mäki, 2009, s.~14).
Pojawia się tutaj jednak znaczna różnica pomiędzy tymi dwoma naukami,
przeszkadzająca ekonomistom w~dostępie do rzeczywistości. Otóż intuicje co do zależności pomiędzy wyidealizowanymi
obiektami mogą być niezgodne z~realnością. Przykładowo z~eksperymentów myślowych Arystotelesa na temat spadania
ciał w~świecie podksiężycowym wynikało, że im cięższy jest dany obiekt, tym szybciej dąży w~kierunku Ziemi \textit{ceteris
paribus}. Dopiero fizyczne eksperymenty Galileusza sfalsyfikowały tę hipotezę. Nasuwają się trzy rozwiązania tego
problemu. Można prowadzić jedynie ścisłe wnioskowanie dedukcyjne i~aparat matematyczny unikając wszelkich komponentów,
do których nie nadają się te metody. Jednak takie podejście może prowadzić do omijania istotnych ontologicznie
czynników sprawczych. Można też oceniać, czy modele trafnie oddają realne mechanizmy na podstawie trafności predykcji,
które z~nich wynikają. Jednak w~modelach często występują założenia określające zakres ich stosowania
(\textit{applicability assumptions}), który jest na tyle wąski, że uniemożliwia obserwację przewidywań danego modelu
\parencite[s.~264]{gorazda_filozofia_2014}.
%\label{ref:RND4xP2eHEyv0}(Gorazda, 2014, s.~264).
Poza tym brak trafnej predykcji może się zdarzyć także wtedy, gdy
modelowany mechanizm dobrze odzwierciedla strukturę rzeczywistości, ale w~tak złożonym środowisku jakim jest
gospodarka, na obserwacje wpływają za każdym razem inne losowe czynniki. Te przykładowe losowe czynniki mogą nie być
ontologicznie istotne dla danego mechanizmu, gdyż żaden z~nich nie uczestniczy w~nim na tyle często, aby był istotny,
natomiast prawie zawsze przy obserwacji jeden z~nich się uaktywnia. Podobny problem z~brakiem możliwości dobrej
predykcji po idealizacji pojawiałby się w~sytuacji tak zwanych ,,chemicznych'' interferencji czynników sprawczych
\parencite[s.~89]{hardt_studia_2013}.
%\label{ref:RNDWpUEzsysKD}(Hardt, 2013, s.~89).
W~takim wypadku zupełnie nieistotne oddzielnie czynniki sprawcze mogłyby
razem przekładać się na ten najistotniejszy czynnik. Innym problemem związanym ze stosowaniem predykcji do falsyfikacji
wyjaśnień jest to, że zmiany technologiczne, społeczne i~tym podobne powodują, że badane mechanizmy mogą powtarzać się
nieregularnie i~w~ograniczonym czasie, a~następnie zupełnie zaniknąć
\parencites[s.~116]{hardt_studia_2013}[s.~271]{gorazda_filozofia_2014}.
%\label{ref:RNDvHGcp92OgB}(Hardt, 2013, s.~116; Gorazda, 2014, s.~271).
Trzecim rozwiązaniem problemu myślowych eksperymentów jest stosowanie zamiast nich
laboratoryjnych eksperymentów w~związku z~tymi zależnościami, dla których jest to możliwe, chodzi więc między
innymi o~metody ekonomii behawioralnej. Ma ona jednak również swoje ograniczenia -- między innymi samo środowisko laboratorium
może wpływać na sposób podejmowania decyzji przez badanych, ponadto jest wiele zależności istotnych z~punktu widzenia
ekonomistów, których nie da się przenieść do laboratorium. Aczkolwiek myśląc futurologicznie można wyobrazić sobie
eksperymenty polegające na wprowadzaniu badanych osób w~symulacje komputerowe odpowiednio modelujące różne stany
gospodarek, w~których na czas eksperymentu te osoby zapominałyby, że jest to jedynie symulacja i~eksperyment.


Innym ważnym przedstawicielem realizmu jest Lawson, który mówi o~trzech poziomach rzeczywistości -- empirycznym,
faktycznym i~realnym. Ponownie nawiązując do fizyki i~przedstawianych tam warstw rzeczywistości warto się zastanowić,
gdzie w~tej koncepcji leży granica pomiędzy poziomem faktycznym i~realnym, skoro fakt zamarzania wody można opisać na
poziomie cząsteczek H\textsubscript{2}O, ale jednocześnie na poziomie protonów, neutronów i~elektronów, lub jeszcze
niżej -- kwarków. Czy zatem jeśli ,,nie istnieje wyjaśnienie, które nie potrzebuje dalszych wyjaśnień''
\parencite[s.~235]{popper_cel_2002},
%\label{ref:RNDlRbZlFRlJj}(Popper, 2002, s.~235),
to poziom realny przesuwa się w~głąb wraz z~najgłębszym przedstawionym
ostatnio wyjaśnieniem, czy zaczyna się on już od poziomu H\textsubscript{2}O? Jeśli przesuwa się wraz z~trafnie
przedstawianymi wyjaśnieniami to jest zmienny i~zależy od historii, a~jeśli zaczyna się od H\textsubscript{2}O to można
wyobrazić sobie cywilizację i~umysły ludzkie, które badając przyrodę od razu zaczęły wyjaśnianie od połączeń
protonów i~neutronów, przez co poziom realny zaczynałby się u~nich gdzie indziej.


Jeszcze jednym problemem związanym z~realizmem w~ekonomii jest ogromna złożoność jej przedmiotu i~wiele współzależności
w~nim występujących. Nawiązując do Klausa Mainzera, ekonomię można porównać do meteorologii, gdyż bada duże skupiska
jednostek, które nawzajem oddziałują na siebie, a~drobne zmiany w~przyczynach w~jednym miejscu mogą prowadzić do
ogromnych różnic w~skutkach w~całym układzie
\parencite[s.~278–280]{gorazda_filozofia_2014}.
%\label{ref:RNDwT1Ozcud5S}(Gorazda, 2014, s.~278–280).
Jednak, jak
wynika z~powyższych rozważań, ekonomia jest w~sytuacji nieco gorszej, a~mianowicie takiej jakby meteorologia była bez
podstaw z~fizyki dotyczących prawa grawitacji i~tym podobnych. W~takim ujęciu fizykę przypominałaby bardziej psychologia niż
ekonomia, aczkolwiek zdaje się, że relacja psychologii do ekonomii nie jest tak bliska jak fizyki do meteorologii. Na
korzyść realizmu i~przeciwko instrumentalizmowi wypada tutaj dopowiedzieć, że w~meteorologii właśnie o~wiele łatwiej
jest tworzyć poprawne wyjaśnienia niż długoterminowe i~sprawdzające się predykcje, co wskazuje ponownie na to, że
teoria bez dobrych predykcji może poprawnie wyjaśniać mechanizmy w~gospodarce. Pozostaje jednak problem, na jakiej
podstawie rozpoznać to zgodne z~realnością wyjaśnienie.

\section{Realność jako przedmiot badań ekonomii}

Problemy realizmu, na które zwróciłem uwagę, nie wykluczają jednak jego kluczowej tezy, zgodnie z~którą ,,istnieje świat
niezależny od obserwatora i~może on być przez niego poznany''
\parencite[s.~16]{hardt_studia_2013}.
%\label{ref:RNDk8xpmUa8wg}(Hardt, 2013, s.~16).
Warto się
jednak zastanowić, co owa \textit{niezależność} miałaby znaczyć w~kontekście ekonomii. Przy założeniu istnienia wolnej
woli ludzi wiele mechanizmów w~gospodarce, takich jak prawo popytu, zależy wszak od poszczególnych ludzkich decyzji.
Patrząc z~tej perspektywy można sobie wyobrazić, że wszystkie zależności w~ekonomii, których komponentem jest ludzka
decyzja, mogłyby zachodzić zupełnie różnie w~zależności od umowy między rozważanymi ludźmi. Na szczęście dla ekonomii
takie globalne umowy nie są podejmowane na poziomie świadomym, a~jedynie socjalizacja, kultura masowa i~propagowanie
różnych postaw stopniowo modyfikują sposób podejmowania przez ludzi decyzji. Dobrym przykładem są tutaj kwestie na
styku ekonomii i~ekologii, gdzie teoretycznie można by się umówić ze wszystkimi choćby w~danym kraju, aby zużywać mniej
surowców i~produkować mniej zanieczyszczeń, ale pewien \textit{opór} ludzki sprawia, że i~w~zagadnieniach społecznych
,,nie da się realizować wszystkiego, co daje się pomyśleć''
\parencite[s.~42]{zboron_teorie_2009},
%\label{ref:RNDlAKpEAAPgk}(Zboroń, 2009, s.~42),
a~przynajmniej nie natychmiast. Widać to również wyraźnie w~historii i~zagadnieniach związanych z~inżynierią społeczną.
Choć pewne schematy ludzkich działań -- instytucje -- zmieniają się w~sposób ciągły wraz z~każdą wolną jednostką, która
modyfikuje swój światopogląd, to zdaje się, że właśnie ten brak pełnej koordynacji społecznej pozwala na
trafny i~przybliżony opis mechanizmów działających w~danym czasie
\parencite[s.~21]{hardt_studia_2013}.
%\label{ref:RND8f3KaiWAg4}(por. Hardt, 2013, s.~21).
Należy
tutaj zaznaczyć, że twierdzenie o~obecności danego mechanizmu w~gospodarce w~zależności od czasu i~kultury nadal jest
realizmem, tak jak \textit{w pełni intersubiektywną} i~absolutną prawdą jest twierdzenie o~systemie prawnym istniejącym
w~Cesarstwie Rzymskim, choć już nigdzie on nie obowiązuje. W~takim ujęciu zdaje się, że i~konstruktywiści społeczni
skłonni byliby uznać, że istniały różne społeczności i~ich kultura zupełnie \textit{niezależnie} od
ekonomisty z~przyszłości, który stara się je badać. Czy jest możliwe to badanie, to już realizm epistemologiczny, %co?
który będzie
omawiany poniżej. Podobnie ten sam ekonomista badając współczesne społeczności, choć jest oczywiście wśród współtwórców
jednej z~nich, stanowi jej pomijalny element, o~ile nie jest kimś sławnym -- przykładowo noblistą w~dziedzinie ekonomii
(wówczas działanie społeczeństwa może być w~sposób znaczny uzależnione od niego).


Poznawalność \textit{niezależnej} realności w~ekonomii, czyli realizm epistemologiczny, jest znacznie trudniejszą do
obrony tezą niż realizm ontologiczny. Warto w~tym temacie przywołać omawianą powyżej kategorię \textit{rzeczywistości
zjawisk codziennych}. Jak zauważył Mäki, pojęcia używane w~ekonomii są bardzo mocno osadzone w~tak zwanych
\textit{commonsensibles}, co można tłumaczyć jako zdroworozsądkowe, codzienne pojęcia, a~zatem pojęcia odnoszące się do
\textit{rzeczywistości zjawisk codziennych}
\parencite[s.~87]{kincaid_realistic_2009}.
%\label{ref:RNDNG485CuHM7}(Mäki, 2009, s.~87).
Podobnie Lawson potwierdza,
że poziom empiryczny, czy faktyczny, w~których mieszczą się te zdroworozsądkowe obiekty, również jest częścią
realności. Zatem samo stwierdzenie ,,istnieją pieniądze'' jest prawdziwe i~odnosi się do jakiejś realności -- do
\textit{rzeczywistości zjawisk codziennych}, mimo że poza umową między umysłami ludzkimi w~konkretnym czasie i
przestrzeni pieniądze zdają się nie istnieć
\parencite[s.~269]{grobler_metodologia_2006}.
%\label{ref:RNDMaFsLdOLtr}(Grobler, 2006, s.~269).
Bardziej precyzyjne -- domyślane -- zdanie wyrażające
sąd o~istnieniu pieniędzy brzmiałoby więc ,,istnieją pieniądze w~ramach danych kultur
\textit{w pełni intersubiektywnie}''. Nawet jeśli za paręset lat ziściłyby się marzenia niektórych komunistów o
społeczeństwie funkcjonującym bez pieniędzy, to nikt z~tamtejszych obywateli nie mógłby słusznie zaprzeczyć istnieniu
pieniędzy w~poprzednich kulturach -- oczywiście jako użytecznych konstruktach tychże kultur. Prawdą jest zatem
stwierdzenie przywoływane na początku tego eseju, że ekonomiści tworzą konstrukty w~oparciu o~potoczne konstrukty,
jednak te pierwsze odnosząc się do tych ostatnich mogą odwzorowywać je prawdziwie bądź fałszywie lub chociaż trafniej
bądź mniej trafnie.


Sama możliwość stwierdzania o~istnieniu pewnych konstruktów potocznych nie jest jednak jeszcze realizmem
epistemologicznym, w~którym chodzi o~odwoływanie się do nieobserwowalnych mechanizmów i~wyjaśnianie na ich podstawie
tych potocznych zjawisk
\parencite[s.~261]{gorazda_filozofia_2014}.
%\label{ref:RNDRlmRjinKLB}(Gorazda, 2014, s.~261).
Jednakże dużą
część -- jeśli nie całość -- w~gospodarce stanowią te zależności, których ważnym komponentem jest ludzka decyzja czy intencja
\parencite[s.~214]{gorazda_filozofia_2014}.
%\label{ref:RNDcoiHd3hnK5}(Gorazda, 2014, s.~214).
Natomiast możliwość porozumiewania się -- i~przez to wychwycenia
intencji lub chociaż procesu racjonalizowania sobie nieświadomych procesów decyzyjnych -- pozwala w~przybliżeniu
określić, w~jakim kierunku w~zależności od warunków decyzje ludzkie zwykle zmierzają, przynajmniej w~perspektywie
\textit{pełnej intersubiektywności}. Warto tutaj przywołać prawo popytu: kiedy cena rośnie to popyt maleje
\textit{ceteris paribus}. Dla większej precyzji należałoby do tego dodać jeszcze \textit{zazwyczaj} ze względu na takie
anomalie jak paradoks Giffena, czy efekt snobizmu. Łatwo zauważyć, że \textit{cena} jest rzeczywiście konstruktem
potocznym używanym przy wymianie handlowej pomiędzy ludźmi, natomiast \textit{popyt }odnosi się do decyzji
podejmowanych w~umyśle na podstawie wysokości tej \textit{ceny}. \textit{W pełni intersubiektywnie }można zatem orzec,
że w~naszej obecnej kulturze ludzie najczęściej wybierają niższe ceny i~nie ma w~tym ani jednego odwołania do bytów
pozaumysłowych, a~jedynie pewien konstrukt porządkujący konstrukty potoczne, wyciągający z~nich wspólny schemat, czyli
kluczowe właściwości odarte z~przygodnych zjawisk towarzyszących.


Nauki społeczne różnią się więc istotnie od przyrodniczych. Te ostatnie, zwłaszcza fizykę, można sobie wyobrazić jako
wielki kawałek materiału, który jest kładziony na niewidzialne góry, w~celu poznania ich kształtu
\parencite[s.~127]{grobler_prawda_2000}.
%\label{ref:RNDFerfDjYAQy}(por. Grobler, 2000, s.~127).
Z~konieczności materiał ten, choć oddający ich ogólny
zarys, w~zależności od naciągnięcia i~faktury lepiej oddaje raz jeden, raz drugi obszar badanych gór. Ponadto materiał ten przy
mocnym naciągnięciu gniecie trawę rosnącą na tych niewidzialnych górach, co też zniekształca ich obraz. W~przypadku
zjawisk fizycznych takich jak deszcz czy atomy, zdaje się, że istnieje coś poza umysłami, co jest z~tym
powiązane i~w~tym kierunku podążają nauki przyrodnicze -- w~kierunku pozamateriałowego kształtu niewidzialnych gór. Ekonomia, jako
nauka społeczna, bada zaś zjawiska społeczne, a~zatem coś, co swój początek bierze w~umysłach. Nawiązując do metafory
niewidzialnych gór, w~naukach społecznych bada się same fałdy materiału, pod którymi nic nie ma. To powoduje, że
ułożenie tych fałd z~czasem może się diametralnie zmieniać. Poza \textit{pełną intersubiektywnością} ludzką nie ma
niczego, do czego można by odnieść takie pojęcia jak \textit{cena}, czy \textit{dochody}. Powoduje to tym samym, że
\textit{pełna intersubiektywność} danych zjawisk społecznych staje się niejako ich obiektywnością, gdyż obiektywne jest
to, że \textit{w pełni intersubiektywnie} w~danym czasie i~miejscu wystąpił dany mechanizm w~gospodarce. Wszelkie
pytania o~powiązanie tego z~procesami w~mózgu lub poruszeniami duszy i~tym podobne nie stanowią już przedmiotu
zainteresowania samej ekonomii, a~co najwyżej badań interdyscyplinarnych łączących ekonomię z~kognitywistyką lub
psychologią.

\section*{Zakończenie}

Konstruktywizm społeczny zwraca uwagę na realne zniekształcenia i~utrudnienia występujące w~procesie poznania,
jednak w~swojej radykalnej formie zdaje się wikłać w~pewne trudności. Co więcej, przy jego pełnej akceptacji niemożliwa by była
żadna rozsądna reforma społeczna, gdyż jej powodzenie mogłoby zajść jedynie przez przypadek. Podobnie
walka z~ewentualnymi nierównościami w~społeczeństwie, w~wyniku akceptacji konstruktywizmu, mogłaby polegać jedynie na zmianie
powszechnie obowiązujących przekonań w~związku z~tym, co powinno być uznawane za nierówność przy zachowaniu
różnic w~majątkach lub dochodach na tym samym poziomie.  Koncepcję prawdy można osłabiać, aż po konieczność dodawania do niej za
każdym razem ,,jawi się tak jedynie w~danej kulturze'', aczkolwiek bez niej trudno by było się porozumieć i~podejmować
realne wyzwania społeczne. Ponadto doświadczenie dialogu międzykulturowego wskazuje na to, że istnieją pewne
ponadkulturowe i~wspólne wszystkim ludziom schematy poznawcze, dzięki którym można też mówić o~\textit{pełnej
intersubiektywności} w~stosunku do pewnego wycinka doświadczenia ludzkiego. Te wspólne wszystkim kulturom
elementy w~procesie poznawczym nie muszą być uświadomione, a~jedynie jakby na zasadzie anamnezy uświadamiane po zetknięciu z~inną
kulturą i~przyjęte jako coś oczywistego. Realność międzykulturowa jawi się tutaj zatem jedynie jako część wspólna
wszystkich światów subiektywnie i~oddzielnie przeżywanych przez każdą osobę. Przy tak minimalistycznie zdefiniowanej
realności, stanowisko realizmu w~ekonomii wydaje się zasadne lub przynajmniej niepostulujące nazbyt wiele. Można by
jednak powiedzieć, że nadal jest to konstruktywizm tylko innego rodzaju, gdzie zamiast nieprzezwyciężalnych uwarunkowań
kulturowych pojawiają się uwarunkowania aparatu poznawczego człowieka, skoro nie ma sposobu wyjścia poza \textit{pełną
intersubiektywność}. O~ile w~przypadku nauk przyrodniczych, jeśli mają one ambicje stawiać tezy ontologiczne, zarzut
ten uznaję za zasadny, to w~przypadku ekonomii nie, gdyż nie ma sensu mówić o~bytach społecznych poza \textit{pełną
intersubiektywnością} społeczeństwa, które tworzy ludzkość. Pomimo to z~pewnością nadal bardzo trudnym
zagadnieniem z~perspektywy realizmu pozostaje określenie stopnia prawdziwości konstruowanych wyjaśnień.


Prezentowane w~niniejszym opracowaniu stanowisko można umieścić na styku realizmu i~konstruktywnego empiryzmu van
Fraasena, aczkolwiek nie przedkłada ono szczegółowości opisu ponad trafność schematu przy danym celu badacza. Tym samym
dopuszcza możliwość prawdziwych wyjaśnień zjawisk społecznych -- w~obrębie
\textit{pełnej intersubiektywności} -- jednocześnie zwracając uwagę na duże problemy
związane z~ich osiągnięciem i~oceną. Podstawowym wnioskiem jednak jest
to, że \textit{pełna intersubiektywność }zjawisk społecznych staje się niejako ich obiektywnością, a~zatem można mówić
o~realnych zjawiskach i~mechanizmach w~gospodarce, nawet jeśli nie istnieją one poza ludzkimi umysłami.


\end{artplenv}
