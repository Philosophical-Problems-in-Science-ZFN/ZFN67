\begin{artplenv}{Krzysztof M. Turek}
	{W poszukiwaniu racjonalności ekonomicznej w~dziełach Adama Smitha}
	{W poszukiwaniu racjonalności ekonomicznej\ldots}
	{W poszukiwaniu racjonalności ekonomicznej w~dziełach\\Adama Smitha}
	{Stowarzyszenie Studentów i~Absolwentów Wydziałowej Indywidualnej Ścieżki Edukacyjnej -- ,,WISE''\label{turek-start}}
	{In search of the assumption of economic rationality in Adam Smith’s thought}
	{In Adam Smith's \textit{Inquiry into the Nature and Causes of the Wealth of Nations} the term `rational' occurs only twice. Neither of these uses assigns the property of rationality either to human beings or to economic agents. Despite that, Smith is widely recognised as the founder of modern mainstream economics, a~science which is defined by the assumption of the rationality of an economic agent. This paper aims to locate and discuss the notion of rationality which is implied by Smith's work. To achieve that a~double-track approach is taken. First, the current paper reconstructs Smith's general view on human nature which was presented mainly in his earlier book \textit{The Theory of Moral Sentiments}. This is followed by a~discussion of Smith's theses on selected mechanisms which drive the economy as presented in \textit{The Wealth of Nations}.}
	{rationality, Adam Smith, optimization, self-love, sympathy, invisible hand.}
	


\section*{Wstęp}

Adam Smith uważany jest za twórcę pierwszego naukowego systemu ekonomii, a~przez to za ojca tej dyscypliny. Data
pierwszego wydania \textit{Badań nad naturą i~przyczynami bogactwa narodów} (1776) wyznacza natomiast symboliczny
początek ekonomii
\parencite[s.~25]{samuelson_ekonomia_2000}.
%\label{ref:RNDSNN7u3epsg}(Samuelson, Nordhaus, 2000, s.~25).
Ekonomię definiować można jako
naukę o~gospodarowaniu
\parencite[s.~17]{milewski_elementarne_2003}
%\label{ref:RNDQYWRez9Ts3}(Milewski, 2003, s.~17)
rozumianym jako produkcja, dystrybucja,
konsumpcja i~handel środkami służącymi do zaspokojenia ludzkich potrzeb lub też jako naukę o~tym, jak decydujemy o~wykorzystaniu
rzadkich i~mających alternatywne zastosowanie zasobów
\parencite{black_slownik_2008,samuelson_ekonomia_2000}.
%\label{ref:RNDrlJSmsFU9i}(Black, 2008; Samuelson, Nordhaus, 2000).
Niezależnie od tego, w~swoim przedmiocie badań nauka ta odwołuje się do pewnego typu decyzji i~działań
ludzi. Z~tego powodu ekonomiści przyjmują pewien zestaw założeń na temat tego, jaki jest i~jak działa podmiot w~szeroko
rozumianym obszarze działalności gospodarczej.

Zgodnie z~najpowszechniej przyjmowaną i~najbardziej wpływową w~historii ekonomii i~filozoficznych rozważań na temat
ekonomii wizją podmiotu gospodarującego, przyjmuje się, że jest on przede wszystkim istotą racjonalną. Rozważając
kwestię założenia o~racjonalności podmiotu gospodarującego, możemy wyróżnić dwa zagadnienia. Po pierwsze, jaki jest
jego teoretyczny status, czyli czy występuje jako abstrakcyjne założenie, czy też jako realistyczny opis rzeczywistości.
Po drugie, jaka jest jego treść, czyli jakie cechy składają się na racjonalność ekonomiczną. W~toku rozwoju ekonomii,
znaleźć możemy teorie w~różny sposób odnoszące się do obu postawionych pytań.

Pierwszych prób sformułowania koncepcji ,,człowieka ekonomicznego'', które rozwiązują oba te problemy, szukać
należy w~dziele Johna Stuarta Milla
\parencite[s.~222]{persky_retrospectives:_1995}.
%\label{ref:RNDrRthwaZDDy}(Persky, 1995, s.~222).
W~eseju \textit{O definicji ekonomii
politycznej i~o~właściwej jej metodzie dociekań} Mill -- w~zgodzie ze swoim apriorycznym podejściem -- pisze następujące
słowa
\parencite[s.~101]{mill_essays_2000}:
%\label{ref:RND2llwJNg1pe}(Mill, 2000, s.~101): 

\myquote{
Geometria zakłada arbitralną definicję prostej, ,,która ma długość, ale nie szerokość''. W~ten sam sposób ekonomia
polityczna zakłada arbitralną definicję człowieka jako bytu, który niezmiennie czyni to, co pozwala mu osiągnąć
największą ilość dóbr niezbędnych do życia, dla wygody lub dla luksusu, przy zaangażowaniu najmniejszej ilości
pracy i~wysiłku fizycznego. 
}
W innym miejscu Mill
\parencite*[s.~97]{mill_essays_2000}
%\label{ref:RNDONtG67AgWn}(Mill, 2000, s.~97)
mówi że ekonomia polityczna postrzega człowieka jako
istotę, która

\myquote{
pragnie posiadać bogactwo i~która zdolna jest oceniać porównywalną skuteczność środków, prowadzących do tego celu. 
}

W ekonomii Milla założenia dotyczące podmiotu są upraszczającymi idealizacjami, które mają umożliwić rozwój
teorii. W~intencji autora nie jest to pełny opis ludzkiej natury. Składają się na niego tylko dwa elementy: motywem działań
podmiotu ekonomicznego jest poprawianie własnej sytuacji materialnej oraz zdolność do optymalizacji swych działań
prowadzących do tego celu. Te dwie cechy określiły charakter człowieka ekonomicznego (\textit{homo oeconomicus}), choć
sam termin \textit{economic man} został użyty nie przez Milla, a~przez jego krytyków
\parencite[s.~222]{persky_retrospectives:_1995}.
%\label{ref:RNDyl7WIXEIyP}(Persky, 1995, s.~222).

W okresie rozwoju ekonomii neoklasycznej (druga połowa XIX wieku), obserwować możemy inne podejście reprezentowane przez
tak wpływowe dla rozwoju ekonomii postacie, jak Francis Edgeworth i~Alfred Marshall. Ten pierwszy
\parencite[s.~16]{edgeworth_mathematical_1881}
%\label{ref:RNDHzWuro9C4A}(Edgeworth, 1881, s.~16)
stwierdził, iż 

\myquote{
pierwszą zasadą ekonomii jest to, że podmiot jest powodowany jedynie dbałością o~własny interes.
}
I mimo że twierdzenie to nie obejmuje natury człowieka w~całości, właściwe jest dla dwóch obszarów życia:
wojny i~kontraktów
\parencite[s.~52]{edgeworth_mathematical_1881}.
%\label{ref:RND3sM2hyXcXJ}(Edgeworth, 1881, s.~52).
Z~kolei Marshall, który prawa ekonomiczne traktował jego
wskazanie typowych tendencji
\parencite[s.~94-95]{marshall_zasady_1925},
%\label{ref:RNDDcgj1qE1XU}(Marshall, 1996, s.~31),
dodaje, że w~wypadku ekonomii
chodzi o~te zachowania, których motywy można mierzyć ceną pieniężną. I~pomimo iż przyznaje on, że pomiędzy jednostkami mogą
zachodzić różnice, to:

\myquote{
W ogólnych wynikach tych badań różnorodność i~niestałość działania indywidualnego znika we względnie prawidłowym zespole
działań wielkiej ilości ludzi.
}

Na takim gruncie powstało wiele uznawanych do dziś za podstawowe koncepcji ekonomicznych. Ekonomiści okresu
neoklasycznego, zachowując treść założenia o~podmiocie ekonomicznym, jaki nadał mu Mill, starali się uczynić je
realistycznym. Z~abstrakcyjnego założenia, podobnego do geometrycznych definicji, staje się twierdzeniem oddającym
naturę rzeczywistości gospodarczej. 

Kolejnym ważnym krokiem w~rozwoju koncepcji racjonalności ekonomicznej jest pojawienie się w~XX wieku tak zwanej teorii
racjonalnego wyboru. W~miejsce klasycznego i~neoklasycznego pożądania bogactwa i~mierzenia motywów ceną pieniężną,
wprowadza bardziej ogólną kategorię: ,,preferencje''
\parencite[s.~119]{hausman_etyka_2017}.
%\label{ref:RNDgwtrR2Gwpt}(Hausman, i~in., 2017, s.~119).
Podmiot
działa racjonalnie, jeśli jego zbiór preferencji spełnia warunki racjonalności oraz jeśli spośród dostępnych alternatyw
wybiera taką, którą preferuje najbardziej. Preferencje podmiotu są racjonalne, jeśli ich zbiór spełnia dwa warunki
formalne: przechodniości i~kompletności. Preferencje podmiotu A~są przechodnie wtedy i~tylko wtedy, gdy dla wszystkich
alternatyw (przedmiotów wyboru bądź oceny) $x$, $y$ i~$z$, jeśli A~preferuje $x$ nad
$y$ oraz \textit{y }nad $z$, to preferuje $x$ nad $z$. Warunek kompletności preferencji
jest spełniony wtedy, gdy dla wszystkich możliwych alternatyw $x$ i~$y$, przed jakimi stoi podmiot A,
preferuje on albo $x$ albo $y$ lub jest obojętny między $x$ i~$y$. Innymi słowy, nie
istnieje para możliwości, co do których podmiot nie ma preferencji. Preferencje podmiotu są podstawą do przypisania mu
funkcji użyteczności, czyli dopasowania wartości użyteczności otrzymanych z~realizacji poszczególnych
preferencji. W~zależności od wersji teorii racjonalnego wyboru, możemy mówić o~porządkowej lub kardynalnej funkcji użyteczności.
Pierwsza wersja oddaje tylko hierarchię preferencji jednostki, podczas gdy w~wypadku drugiej znaczące są również absolutne
wartości użyteczności. Podmioty w~swych decyzjach ekonomicznych dążą do maksymalizacji swojej użyteczności.

Treść tego założenia -- podobnie jak w~poprzednich wersjach -- określa, że podmiot cechuje dbałość o~własny interes
(własne preferencje) oraz zdolność do optymalnych zachowań. W~wymiarze tym teoria racjonalnego wyboru podobna jest do
koncepcji Milla, Edgewortha czy Marshalla. Istotną zmianą jest wprowadzenie pojęcia preferencji, które nie musi
oznaczać zdobywania rzeczy materialnych ani spełniania potrzeb własnych podmiotu. Ktoś może nad bogactwo preferować
dobre relacje międzyludzkie lub może preferować spełnienie potrzeb członka rodziny bardziej niż swoich własnych.
Warunki racjonalności w~tym wydaniu są formalne i~ujęte w~języku teorii mnogości. Nie odwołują się do treści
poszczególnych preferencji. Odnośnie do statusu tego założenia w~teorii racjonalnego wyboru, wydaje się, że należy
uznać, że ma ono realistycznie oddawać podstawowy mechanizm podejmowania decyzji
\parencite[s.~120]{hausman_etyka_2017}:
%\label{ref:RNDGtUa2eKVBv}(Hausman, i~in., 2017, s.~120): 

\myquote{
Przyjmując iż ,,preferencje'' obejmują wszystko, co odnosi się do wyboru tj. iż są zupełnymi porównawczymi ocenami
(\textit{total comparative evaluations}), przyjmujemy jednocześnie punkt widzenia ekonomistów. Traktują oni wybory jako
działania wynikające z~ograniczeń, preferencji i~oczekiwań (albo przekonań).
}
Wskazuje na to również fakt, że najgłośniejsza krytyka koncepcji opartych na teorii racjonalnego wyboru płynie ze strony
ekonomii behawioralnej, a~więc dziedziny bazującej na empirycznych badaniach, sprawdzających ludzkie mechanizmy
podejmowania decyzji. Realizmu tej teorii można łatwo bronić ze względu na dużą pojemność pojęcia preferencji.

W historii rozwoju koncepcji racjonalności ekonomicznej wskazać możemy różne sformułowania treści, które wiążą się z~tym
założeniem. W~tym obszarze nastąpiło przejście od interpretacji działań ekonomicznych jako motywowanych wyłącznie
zdobyciem bogactwa do odwoływania się do preferencji. Zawsze jednak jednostka ujmowana była jako
działająca -- w~rozumianym w~pewien sposób -- interesie własnym (zwiększanie bogactwa, spełnianie potrzeb,
realizacja preferencji).
Ponadto, już od sformułowania Milla, racjonalność ekonomiczna wiązała się ze zdolnością porównywania środków do
osiągnięcia swojego celu i~skłonnością do wyboru tych najbardziej optymalnych.

Niniejszy tekst stanowi próbę odpowiedzi na pytania, czy u~,,ojca ekonomii'' podmiotom ekonomicznym przypisywane są cechy,
które później rozwiną się w~świadomie formułowaną i~dyskutowaną koncepcję racjonalności ekonomicznej oraz jaki był
teoretyczny status poglądów Smitha na naturę działań ekonomicznych.

Smith nie sformułował założenia o~racjonalności podmiotu gospodarującego bezpośrednio. W~jego głównym ekonomicznym
dziele -- \textit{Bogactwie narodów} termin ,,racjonalny'' (\textit{rational}) pada
jedynie dwukrotnie
\parencite[s.~603 i~612]{smith_inquiry_2007}.
%\label{ref:RNDb7ncrGke2n}(Smith, 2007a, s.~603 i~612).
Żadne z~tych użyć nie odnosi się natomiast do
własności podmiotu. Pierwszy raz Smith pisze o~,,racjonalnej religii'', czyli o~pewnej dyskutowanej w~jego czasach
ideologii. Za drugim razem wspomina o~,,racjonalnej rozmowie'', która zgodnie z~jego argumentacją za publicznym
finansowaniem edukacji, miała nie być możliwa pomiędzy osobami bez podstaw wykształcenia. Smith nie wyraził również
\textit{explicite} teorii racjonalności ekonomicznej. Dlatego trzeba będzie poszukiwać przesłanek zawartych w~wielkim
dziele Adama Smitha \textit{implicite}. W~tym celu przeanalizowane zostaną poglądy Smitha na człowieka i~naturę jego
działań, które najpełniej zawarte zostały w~drugim z~wielkich dzieł Smitha -- \textit{Teorii uczuć moralnych}.
Przedmiotem analiz będą również opisane w~\textit{Bogactwie narodów} główne mechanizmy życia gospodarczego.

\section{\textit{Das Adam Smith Problem} i~koncepcja natury ludzkiej}

Pomimo iż od pierwszego zdania \textit{Teorii uczuć moralnych} Smith
\parencite*[s.~5]{smith_teoria_1989}
%\label{ref:RND4EkSnneCPj}(1989, s.~5)
odwołuje się do natury
człowieka, nie należy wnosić, że uznawał on istnienie istoty człowieka w~znaczeniu
przyjmowanym w~sporze o~uniwersalia przez realistów. Jego nominalistyczne podejście
badawcze można określić jako empiryzm i~realizm, w~sensie
stronienia od niepopartych doświadczalną rzeczywistością spekulacji. Był on również przedstawicielem
orientacji w~filozofii brytyjskiej, która jako maksymę przyjmowała trzymanie się zdrowego rozsądku.
Filozofia \textit{common sense} stawiała sobie za zadanie dokładne i~krytyczne
rozwijanie oraz badanie poglądów zawartych w~naszych
codziennych i~prostych intuicjach. Podejście Smitha motywowane powyżej
wymienionymi poglądami umożliwiło dokładne, jak na ówczesne
czasy, a~i~po dziś dzień imponujące, zbadanie gospodarki, która jest sferą mocno przywiązaną do konkretów i~aktywności
dnia codziennego\footnote{Dokładniejsze omówienie metodologii Adama Smitha odnaleźć
można np. w~\parencites[rozdz.~I–III]{fleischacker_adam_2005}[rozdz.~8 i~9]{myers_soul_1983}.
%\label{ref:RND8sLQzQDKf8}(Fleischacker, 2005, rozdz. I–III; Myers, 1983, rozdz. 8 i~9).
}. Analizując otaczające go
zjawiska, Smith dostrzegał pewne tendencje zachowań i~powtarzające się motywy, jakimi ludzie się kierują. W~ten sposób
należy rozumieć ,,naturę'' człowieka w~jego ujęciu, choć Smith daleki jest również od poglądu, jakoby człowiek był prosty
i jednowymiarowy
\parencite{lubinski_rola_2019}.
%\label{ref:RNDfMqOaeK2TG}(Lubiński, 2019).

W drugiej połowie XIX wieku w~Niemczech został sformułowany interpretacyjny problem znany jako \textit{Das Adam Smith
Problem}. Polegał on na rzekomej sprzeczności obrazu prospołecznej i~altruistycznej istoty, jaki wyłania się z~treści
\textit{Teorii uczuć moralnych} a~samolubnym, egoistycznym obrazem, który daje się odnaleźć w~\textit{Bogactwie narodów}.
Jednak współcześnie badacze i~komentatorzy Smitha prawie powszechnie zgadzają się, że ów problem bierze
się z~niedokładnego zrozumienia intencji szkockiego filozofa, nie zaś z~faktycznej sprzeczności prezentowanych przez niego
treści
\parencite{wilson_adam_2006}.
%\label{ref:RNDix0vJk9GR0}(Wilson, Dixon, 2006).
Pierwsze wydanie \textit{Teorii uczuć moralnych} pochodzi z~1759
roku. \textit{Bogactwo narodów} pierwszy raz wydane zostało w~1776 roku. Jednak
najpóźniejsza wydana za życia Smitha edycja \textit{Teorii uczuć moralnych} jest z~roku 1790, a~\textit{Bogactwa narodów} z 1789 roku. Tym
bardziej wiarygodna jest teza, że pomiędzy \textit{Bogactwem narodów} i~\textit{Teorią uczuć moralnych} nie ma sprzeczności. Wręcz przeciwnie,
osią zawartej analizy będzie identyfikacja bliskich relacji pomiędzy pozornie przeciwnymi uczuciami miłości
własnej i~współczucia\footnote{Adam Smith korzysta z~angielskiego terminu \textit{sympathy},
który ma inne znaczenie niż polskie
słowo ,,sympatia'', a~zdecydowanie bliżej jest mu do polskiego ,,współczucia''.}.

W \textit{Teorii uczuć moralnych} przedstawiona zostaje koncepcja człowieka jako istoty wyposażonej we współczucie,
które Smith opisuje jako przyrodzoną zdolność i~skłonność do dzielenia odczuć innych. Coś, co czyni osoby wokół nas
smutnymi, również i~nas zasmuca, a~jeśli coś osoby wokół nas cieszy, i~nam jest radośniej. Współczucie bazuje na akcie
wyobraźni, dzięki której jesteśmy w~stanie poczuć, jak to jest być kimś innym w~danej sytuacji. Nasze uczucia
prawdopodobnie nie będą tej samej mocy -- z~racji tego, że odczuwamy je pośrednio -- ale będą tego samego rodzaju. Nasza
skłonność do współczucia jest dodatkowo wzmacniana tym, że wzajemne odczuwanie jest po prostu przyjemne. Adam Smith
\parencite*[s.~9]{smith_theory_2005}
%\label{ref:RNDF0nryYpd5H}(2005, s.~9)
pisze, że ,,nic nie przynosi więcej przyjemności jak obserwować w~innym człowieku
współodczuwanie wszystkich emocji, które wypełniają naszą własną pierś''. Te wrodzone mechanizmy emotywne mają
konsekwencje w~postaci tego, że człowiek ma tendencję do brania pod uwagę, jak inne osoby czują się w związku z jego zachowaniem.
I~nie chodzi tylko o~tych, którzy są pod bezpośrednim wpływem naszych działań, ale również osoby trzecie,
które obserwują, jak zachowujemy się względem innych. W~tych poglądach Smitha, z~których wyłania się obraz działającego
prospołecznie i~biorącego pod uwagę interesy innych człowieka, należy podkreślić dostrzeżony przez niego wpływ czynnika
bliskości lub dystansu pomiędzy osobami. Im bardziej się z~kimś utożsamiamy, tym mocniej współodczuwamy i~na
odwrót. O~wiele mocniej nasze współczucie działa względem własnego dziecka niż mieszkańca drugiej strony globu, którego nigdy
nawet nie widzieliśmy.

Ten ostatni argument zwraca naszą uwagę na fakt, że wpływu współczucia nie należy przeceniać a~mocy tego
uczucia w~motywowaniu naszych działań nie należy absolutyzować.
%Jest to pewna tendencja, która może realizować się w~większym
%bądź mniejszym stopniu.
Ludzie mają tendencję do działań opartych na współczuciu, jednak może się ona realizować w~większym bądź mniejszym stopniu.
Muszą więc występować inne czynniki motywujące ludzkie postępowanie -- Smith dostrzega
je w~miłości własnej. Ta motywacja do działań mocniej podkreślana jest w~\textit{Bogactwie
narodów}, jednak już w~\textit{Teorii uczuć moralnych} zostaje dostrzeżona i~omówiona jej rola.
Miłość własna -- podobnie jak współczucie -- jest silnie zakorzenioną, ale nie absolutną motywacją ludzkich działań. Ronald Coase
\parencite*[s.~9]{coase_adam_1976}
%\label{ref:RNDgMfJqiv148}(1976, s.~9)
dodatkowo zwraca uwagę na fakt, że opisane powyżej mechanizmy przejawiania
się i~funkcjonowania współczucia w~jednostkowych wyborach i~życiu społecznym są uzasadniane również przez miłość
własną. O~innych dbamy, ponieważ czujemy się tak, jak oni oraz przyjemność wzbudza w~nas współodczuwanie, ponadto chcemy wypadać
na godnych pochwały i~podziwu. Adam Smith
\parencite*[s.~69]{smith_theory_2005}
%\label{ref:RND1FWZVGlg4X}(Smith, 2005, s.~69)
pisał: 

\myquote{
Natura [\mydots] nie potraktowała nas na tyle nieżyczliwie, by obdarzyć nas jakąś cechą, która jest w~każdym aspekcie zła,
lub taką, która by była bezwzględnie obiektem godnym pochwały i~podziwu.
}
Miłość własna może przyczyniać się do dobrych i~prospołecznych czynów. Jest to szczególnie ważne z~punktu widzenia
ekonomii. 

\myquote{
Wzgląd na nasze prywatne szczęście i~własny interes w~wielu sytuacjach występuje pod postacią bardzo chwalebnych reguł
postępowania. Zwyczaje związane z~handlem, przemysłem, poufnością, uważnością czy aplikacją pomysłów są generalnie
rzecz biorąc wyrobione przez motyw dbania o~własny interes, a~mimo to ich posiadanie jest jednocześnie uważane za
jakość godną pochwały, która zasługuje na szacunek i~aprobatę wszystkich
\parencite[s.~277]{smith_theory_2005}.
%\label{ref:RNDqsz23ORvb0}(Smith, 2005, s.~277).
}

Miłość własna rozumiana była przez Smitha jako jeden z~czynników motywujących ludzkie czyny, który
wchodzi w~relacje z~innymi, również altruistycznymi pobudkami, i~przejawia się jasno w~działaniach
gospodarczych. Działania te z~jednej
strony mają zapewnić osobom spełnienie w~jak największym stopniu ich potrzeb związanych z~przeżyciem, z~drugiej, przez
gromadzenie bogactwa zaspokajają ludzką ambicję i~próżność. Pisząc o~ambicji Adam Smith twierdził, że odegrała ona
kluczową rolę w~rozwoju cywilizacji oraz gospodarki
\parencite[s.~164]{smith_theory_1969}.
%\label{ref:RNDxyay4Tpgwb}(Smith, 1969, s.~164).

Według Smitha nasze działania, w~tym te gospodarcze, powodowane są motywami irracjonalnymi, w~tym sensie, iż są to
wrodzone emocje: współczucie i~miłość własna. Każda z~nich przejawia się w~różnych formach, takich jak dobroć, chęć
pomocy, poświęcenie, chciwość, próżność, ambicja, pracowitość, oszczędność. W~różnych sytuacjach poszczególne z~tych
czynników grają różne role i~są mniej lub bardziej decydujące. Zależy to m.in. od osobistego ukształtowania jednostki
oraz tego, czy pozostałe zaangażowane w~danej sytuacji osoby są bliskie czy obce. Adam Smith jest dodatkowo zdania, że
wszelkie te rządzące człowiekiem pasje mają wspólne ugruntowanie, jakim jest chęć przetrwania i~zachowania gatunku
\parencite[s.~69]{smith_theory_1969}.
%\label{ref:RNDrWbPAvq3tn}(Smith, 1969, s.~69).
Cechuje to wszystkie istoty żywe, w~tym człowieka. Dbanie o~realizację
tego głównego celu nie zostało jednak, jak pisze Szkot, pozostawione ,,wolnym i~niepewnym dociekaniom naszego rozumu''
\parencite[s.~69]{smith_theory_1969},
%\label{ref:RNDghYEqCSviM}(Smith, 1969, s.~69),
tylko naturalnemu usposobieniu, w~którym na pozór sprzeczne motywy
harmonijnie się uzupełniają. 

\section{Podmiot działań gospodarczych w~\textit{Badaniach nad naturą i~przyczynami bogactwa narodów}}

Aby wyciągnąć wnioski dotyczące podmiotu ekonomicznego w~teorii Smitha, należy poddać analizie jego poglądy na temat
głównych mechanizmów odpowiadających za dynamikę wydarzeń gospodarczych. W~badaniu gospodarki, Smitha przede wszystkim
interesowała perspektywa, obejmująca całość systemu gospodarczego. Sam tytuł \textit{Bogactwa narodów}
wskazuje na to, w~jaki sposób postrzegał on swoją rolę jako badacza oraz jak postrzegał tworzącą się
nową naukę. Pytanie o~przyczyny i~naturę bogactwa całego narodu, czyli rozwój gospodarczy całego systemu, jest przez
niego umieszczany w~centrum pola badawczego. W~zgodzie z~późniejszą terminologią, można więc powiedzieć, że Smith w~pewnym
sensie za pierwotny przedmiot zainteresowania brał makroekonomię. Odróżniamy ją od mikroekonomii, która swą uwagę
ogniskuje na badaniu działań pojedynczych podmiotów -- osób lub przedsiębiorstw. Perspektywa makroekonomiczna Adama
Smitha znajdzie wyraz również w~uzyskanych w~tej pracy wnioskach.

Mówiąc o~ekonomii Adama Smitha, często przywołuje się poniższe stwierdzenie
\parencite[s.~20]{smith_badania_2007}:
%\label{ref:RNDRavQPMiRDt}(Smith, 2007b, s.~20):

\myquote{
Nie od przychylności rzeźnika, piwowara czy piekarza oczekujemy naszego obiadu, lecz od ich dbałości o~własny interes.
Zwracamy się nie do ich humanitarności, lecz do egoizmu i~nie mówimy im o~naszych własnych potrzebach, lecz o~ich
korzyściach.
}
Zdanie to interpretować można co najmniej na dwa sposoby. Z~jednej strony można twierdzić, że oznacza ono,
iż w~działaniach gospodarczych ludzie są istotami w~zupełności samolubnymi. Jest to jednak nazbyt powierzchowna
interpretacja. Cytowane zdanie można interpretować jako wyraz tezy, że relacje gospodarcze nie wymagają koniecznie
dobroci i~współczucia, aby mogły zachodzić. Jest to zdecydowanie słabsza teza, która nie prowokuje
sprzeczności z~tezami \textit{Teorii uczuć moralnych}. Gospodarując, podmioty wchodzą w~relacje z~ogromną liczbą osób, również
nieznajomych, w~wypadku których współczucie działa zdecydowanie słabiej. Przetrwanie piekarza, u~którego kupujemy
bułki, niekoniecznie postrzegamy jako istotną część naszego własnego losu i~naszego przetrwania, jak ma się
to w~wypadku rodziny i~znajomych. Dlatego badając gospodarkę właściwym jest zakładać, że ludzie na dwubiegunowej
płaszczyźnie motywacji współczucie-miłość własna, będą zbliżali się w~kierunku krańca samolubnego. Ekonomia powinna
brać to pod uwagę. Ta lekcja przeniknęła do świadomości ekonomistów i~weszła w~skład założenia o~racjonalności podmiotu
gospodarującego. Wydaje się wręcz, że czasami oddziaływała za mocno, gubiąc humanistyczne tło, które towarzyszyło
Smithowi, spłaszczając tym samym i~wulgaryzując do jednego wymiaru zakładaną koncepcję człowieka.

Kolejny wątek choć może na pierwszy rzut oka wydawać się banalny, to wchodzi jednak w~skład Smithowskich poglądów na
interesujące nas kwestie. Rozwój gospodarki możliwy jest dzięki temu, że człowiek wyposażony jest w~rozum, który
umożliwia mu kategoryzowanie świata oraz wykształcenie złożonych form komunikacji. Rozwój produkcji i~handlu wymaga
określonego stopnia inteligencji. Tylko ludzie mają wystarczająco rozumu, by móc technicznie obsługiwać działania
rynkowe -- pojmować i~komunikować fakty i~zależności, kalkulować, ustanawiać i~przestrzegać prawo własności, projektować
wiele scenariuszy alternatywnych oraz zawierać umowy. Samuel Fleischacker
\parencite*[s.~19]{fleischacker_adam_2005}
%\label{ref:RNDPEHbJWGzte}(2005, s.~19)
zwraca uwagę, że z~\textit{Bogactwa narodów} wysnuć można
wniosek, iż skłonność do wymiany i~handlu jest czymś podstawowym dla człowieka. Twierdzi nawet, że wysoce
prawdopodobnym jest, iż to konieczne następstwo faktu, że człowiek wyposażony jest w~rozum oraz mowę.

Według Adama Smitha kluczowym czynnikiem rozwoju jakości i~wzrostu ilości środków, które potrzebne nam są do
przeżycia i~zaspokajania wyższych potrzeb, jest podział pracy. Innymi słowy, to dzięki podziałowi pracy możliwe
jest powstanie i~wzrost bogactwa narodów.
Współcześnie uważa się, że od Smitha pochodzi ,,prawo podziału'', zgodnie z~którym ,,jeżeli powtarzająca się praca jakiejś
jednostki lub zespołu zostanie podzielona w~ten sposób, że każdą czynność składową oddzielimy i~wykonywać będziemy
seriami na określonym poziomie specjalizacji, to ogólny nakład pracy i~środków wytwórczych zmniejszy się''
\parencite{bielski_organizacje:_1997}.
%\label{ref:RNDow4dP28BAr}(Bielski, 1997).
Pojawiające się w~pierwszym rozdziale \textit{Bogactwa narodów} argumenty wskazują na
trzy główne korzyści, dzięki którym jest to możliwe. Po pierwsze, pracownicy wykonujący tylko określoną, wąską grupę
zadań, specjalizują się, stają się mistrzami w~tym, co robią. Specjalizacja pozwala wykonywać czynność
szybciej i~lepiej. Druga zaleta podziału pracy wynika z~braku konieczności ciągłych zmian narzędzi i~stanowiska pracy. Daje to
znaczną oszczędność czasu. Po trzecie, dzięki specjalistycznej wiedzy, która może się kumulować w~wyniku podziału
pracy, może następować postęp technologiczny. Trzeba bardzo dobrze na czymś się znać, aby móc skonstruować maszynę,
usprawniającą pracę.

Aby przedstawić kolejne charakterystyki podmiotu działań gospodarczych w~myśli Smitha,
wprowadzony zostanie współczesny podział pojęciowy na racjonalność instrumentalną i~racjonalność teleologiczną
%Współczesne rozważania nad racjonalnością
\parencite[s.~12]{bochenek_problem_1999}.
%\label{ref:RNDUP10yBIFve}(Bochenek, 1999, s.~12)
%wprowadzają podział na
%racjonalność instrumentalną i~racjonalność teleologiczną.
Pierwsza związana jest z~doborem i~stosowaniem adekwatnych
środków i~działań do osiągania posiadanego celu, abstrahując od tego, jaki on jest. Racjonalność instrumentalna jest
więc relatywistyczna, uzależniona od dobranych poza jej obrębem celów. Racjonalność teleologiczna występuje na poziomie
doboru celów -- można wybrać to, co chcemy osiągnąć irracjonalnie lub racjonalnie. Pojęcie to zakłada w~pewien sposób
obiektywną hierarchię celów, z~których każdy ewaluowany jest przez jego stopień racjonalności lub
irracjonalności w~określonej sytuacji. Racjonalność teleologiczna nie jest więc pojęciem relatywnym.
Działający podmiot adekwatnie lub
nie rozpoznaje porządek, który nie jest przez niego skonstruowany. Jeśli wybiera cel stojący wysoko w~tym porządku, działa
racjonalnie. Powyżej wskazano, że na płaszczyźnie celów -- według Adama Smitha -- człowieka determinują wrodzone
mechanizmy emotywne, a~nie rozumowa ocena możliwości. Można więc stwierdzić, że według Smitha człowieka nie cechuje
racjonalność teleologiczna.

Czytając \textit{Bogactwo narodów} widać natomiast, że Adam Smith przyjmował,
iż człowieka charakteryzuje cecha, którą uznać możemy za racjonalność instrumentalną. Zwróćmy dla przykładu uwagę na
następujący fragment
\parencite[s.~423]{smith_badania_2007}:
%\label{ref:RNDY8QtNECwqY}(Smith, 2007b, s.~423):
,,W każdym zawodzie wysiłek większości tych, którzy
ten zawód wykonują, jest zawsze proporcjonalny do konieczności zmuszającej ich do wykonywania tego wysiłku''. Po tym
stwierdzeniu Smith rozwija wątek sytuacji, kiedy konkurencja zmusza do podwyższania doskonałości w~wykonywaniu własnego
zawodu, aby utrzymać wymagany do życia poziom zysków. W~tym fragmencie widać, że analizując pewne zjawiska z~obszaru
gospodarczego, Adam Smith przyjmuje, że podmioty reagują na warunki otoczenia w~pewien regularny sposób. To, ile
wysiłku ktoś włoży w~pracę, nie jest zależne od ślepego losu, ale od konkretnej sytuacji i~tego, jak postrzega ją
działający podmiot. Osoba wykonująca jakiś zawód kalkuluje, ile wysiłku musi wykonać i~adekwatnie do tego podejmuje
pewne akcje. Nie należy przypuszczać, że podmiot będzie marnować swoje zasoby w~postaci energii na coś, co nie pomoże
jej spełnić celu, jakim jest utrzymanie się na rynku. Racjonalne kalkulacje pojawiają się na niższym poziomie
determinant ludzkiego działania. Są mniej zasadnicze niż chęć przetrwania i~zachowania gatunku oraz wynikające z~niej
instynkty i~wbudowane mechanizmy emotywne, są im instrumentalnie podporządkowane.

Kolejne twierdzenia Smitha, które dotyczą generalnych mechanizmów funkcjonowania gospodarki, związane są z~pojęciami
ceny naturalnej i~ceny rynkowej oraz popytu efektywnego. Pisząc o~cenie naturalnej danego dobra lub usługi, Smith
twierdzi, że na cenę w~ogóle składają się trzy elementy: 1)~koszt pracy, która była potrzebna do wyprodukowania dobra;
2)~zysk z~zainwestowanego kapitału, czyli zwrot z~inwestycji osoby dostarczającej niezbędne materiały i~warunki do
produkcji lub kupca, który nabywa towar, aby go sprzedać oraz 3) renta gruntowa, czyli koszt podstawowego dla
działalności gospodarczej zasobu -- ziemi. Twierdzi on, że dla danego miejsca i~czasu koszt pracy, zysk z~kapitału
oraz renta gruntowa mają pewną określoną wartość naturalną
\parencite[s.~66]{smith_badania_2007}.
%\label{ref:RND8xYWlBvyBT}(Smith, 2007b, s.~66).
Czyli gdy
nie ma żadnych wyjątkowych okoliczności, dla przykładu praca dowolnego piekarza w~mieście potrzebna do wyprodukowania
bochenka chleba ma taką samą wartość równą $x$. Popyt efektywny Smith definiuje jako zgłaszany na dane dobro
popyt z~realnym pokryciem finansowym na poziomie ceny naturalnej. Odróżnia go od popytu absolutnego, który może być
zgłaszany w~oderwaniu od realiów gospodarczych. Dla przykładu, osoba niezamożna może chcieć posiadać zdobioną złotem
karocę i~być w~stanie zapłacić za to jedną monetę. Jednak jej cena naturalna wynosi 20000 monet. Takiego popytu nie
zaliczamy do popytu efektywnego, ponieważ nie jest on tożsamy z~gotowością zakupu towaru dostarczonego w~cenie na
poziomie naturalnym, a~więc nie jest w~stanie sprawić, aby ten towar pojawił się na rynku.

Po wyjaśnieniu tych pojęć, w~rozdziale siódmym \textit{Bogactwa narodów}
(,,O naturalnej i~rynkowej cenie towarów'') Smith opisuje, jak
według niego działa mechanizm rynkowy, czyli jak ustala się ilość włożonych w~pewne procesy środków wytwórczych i~jak
kształtuje się poziom ceny, po której ostatecznie dokonywane są transakcje. Zależność, którą obserwuje, jest
prosta i~opiera się na stosunku ilości dostarczonych na rynek towarów do wielkości popytu efektywnego. Jeśli ilość towaru jest
mniejsza niż wielkość popytu efektywnego, który może pokryć naturalne stopy zysku z~kapitału, renty
gruntowej i~wynagrodzenia za pracę, nie każdy zgłaszający popyt efektywny jest w~stanie nabyć dobro. Wtedy pomiędzy kupującymi
zaczyna się współzawodnictwo o~dobra, które prowadzi do tego, że niektórzy gotowi są zapłacić więcej, niż wynosi cena
naturalna. W~takim wypadku cena rynkowa dobra rośnie. Dodatkowo, posiadacze kapitału, ziemi lub zasobów pracy,
widząc w~danej gałęzi zyski przewyższające poziom naturalny, przenoszą tam swoje zasoby. Powoduje to zniwelowanie niedoboru na
rynku lub wręcz powstanie nadwyżki. Jeśli ilość dostarczonego na rynek towaru jest większa od poziomu, na jakim
zgłaszany jest popyt efektywny, to do współzawodnictwa zmuszeni są producenci. Aby sprzedać swoje towary, muszą obniżyć
cenę poniżej naturalnej, ponieważ sam popyt efektywny nie wystarcza do skonsumowania podaży. Taka sytuacja wiąże
się z~odpływem kapitału, ziemi i~pracy do gałęzi, w~których stopy zwrotu, płace lub renty są wyższe. Zależnie od stosunku
zrealizowanej podaży do popytu efektywnego, cena rynkowa może różnić się od ceny naturalnej i~być większa lub mniejsza.
Jednak stan odchylenia inicjuje mechanizmy, które powodują niwelowanie się różnicy. Dlatego Smith pisze, że cena
naturalna stanowi ,,\textit{cenę centralną}'' czyli taką, ,,ku której ustawicznie ciążą ceny wszystkich towarów''
\parencite[s.~70]{smith_badania_2007}.
%\label{ref:RND7vkcEWPr30}(Smith, 2007b, s.~70). 

Adam Smith w~ten sposób tłumaczył dwa kluczowe dla ekonomii problemy: w~jaki sposób kształtuje się cena danego dobra
oraz co determinuje ilość dóbr dostarczonych na rynek. W~tych rozważaniach możemy zrekonstruować założenia, jakie
przyjmuje on na temat podmiotu  działań rynkowych. Należy zauważyć, iż Smith musiał założyć, że podmioty
biorące udział w~działalności rynkowej,
posiadają pewną wiedzę na temat warunków na rynku. Podmioty charakteryzuje także wyraźne dążenie
do optymalizowania swoich zachowań rynkowych, czyli wykorzystywania zasobów w~jak najbardziej efektywny sposób. To
ważne założenia, które wejdą w~skład ekonomii głównego nurtu i~odegrają w~nim ogromną rolę.

Oprócz przytoczonego już podziału na racjonalność teleologiczną i~racjonalność instrumentalną, w~literaturze funkcjonują
również pojęcia racjonalności metodologicznej i~racjonalności rzeczowej. Kryterium podziału jest tutaj charakter wiedzy
posiadanej przez podmiot (a nie status celu, jak to miało miejsce wcześniej). Dodatkowo, ważne jest, że poziom wiedzy
wpływa na to, czy działania przyniosą zaplanowany efekt. Zgodnie z~racjonalnością metodologiczną, działanie podmiotu
jest racjonalne, jeśli odpowiada posiadanej przez podmiot wiedzy, nawet jeśli ta rozmija się z~rzeczywistością. Dlatego
działanie można uznać za racjonalne metodologicznie nawet wtedy, gdy nie przynosi zamierzonego efektu. Racjonalność
rzeczową można stwierdzić wówczas, ,,gdy dobór środków odpowiada prawdziwej, obiektywnie istniejącej sytuacji, tj.
istniejącym rzeczywiście faktom, prawom i~stosunkom''
\parencite[s.~140]{lange_ekonomia_1978}.
%\label{ref:RND7go8XevWmZ}(Lange, 1978, s.~140).
W~tym wypadku
podmiotowi przypisuje się pełną i~obiektywną wiedzę (w interesującym nas wypadku na temat rynku). Dobierając
odpowiednie środki do osiągnięcia celu, gdy nasza wiedza jest pełna, efekt jest zgodny z~oczekiwaniami. Można
powiedzieć, że racjonalność metodologiczna jest składową racjonalności rzeczowej. Jak ujmuje to Lange
\parencite*[s.~140]{lange_ekonomia_1978},
%\label{ref:RNDijM6wZysyn}(1978, s.~140),
,,racjonalność metodologiczna jest właściwością działania jako sposobu
postępowania; racjonalność rzeczowa jest sprawą adekwatności wiedzy, na której działanie się opiera''. Choć opisane
teorie nie wystarczą, aby podmiotowi w~ekonomii Smitha przypisać\textit{ }wiedzę pełną, na pewno cechuje go jakieś
zrozumienie sytuacji i~mechanizmów gospodarczych, na bazie których podejmuje swoje optymalizacyjne decyzje. Adam Smith
położył fundamenty pod podejście badawcze w~ekonomii, które przyjmuje racjonalność rzeczową podmiotów gospodarujących
jako założenie pozwalające formułować teorie opisujące mechanizmy gospodarcze.

\section{Podmiot ekonomiczny w~teorii Adama Smitha}

Tomasz Kwarciński
\parencite*{klosinski_racjonalnosc_2009}
%\label{ref:RNDxgiFQA5vku}(2009)
wyróżnia
dwa poziomy racjonalności ekonomicznej. Po pierwsze, mamy do czynienia z~racjonalnością mikro, czyli zestawem możliwych
do wskazania cech pojedynczego podmiotu gospodarującego. Koncepcję takiej racjonalności można odnaleźć w~dziele Adama
Smitha, przy czym nie została ona wyrażona \textit{explicite}. Można wskazać cztery następujące jej cechy:

\begin{enumerate}
\item Gospodarcze wybory podmiotów nie są zdominowane przez motywy altruistyczne, ponieważ w~relacjach gospodarczych
zazwyczaj nie mają do czynienia z~bliskimi osobami. 
\item Podmioty pojmują sytuację, w~której się znajdują -- mają wiedzę na temat sytuacji na rynku oraz różnego rodzaju
stosunków, które na nim występują. Wiedzą na przykład, ile zasobów muszą wykorzystać, by przetrwać w~swojej branży oraz
jak zarządzać swoimi zasobami, aby zwiększać efekt ich wykorzystania w~określonych warunkach popytowo-podażowych.
\item Potrafią dokonać operacji, którą późniejsza ekonomia opisze dokładniej i~nazwie optymalizacją, czyli dążą do
maksymalizacji uzyskanych efektów przy danym nakładzie zasobów lub minimalizacji ilości zasobów wykorzystanych do
osiągnięcia określonego efektu.
\item Mają wystarczające zasoby intelektualne, by móc obsługiwać działania rynkowe -- komunikować i~pojmować fakty oraz
zależności; kalkulować i~projektować scenariusze alternatywne.
\end{enumerate}

W swej ekonomii Adam Smith zakłada podmiot gospodarujący, który cechują racjonalność instrumentalna i~pewna wiedza oraz
zrozumienie rynku, a~także który nie kieruje swych działań opierając się na altruizmie. W~ten sposób kładzie on
fundamenty pod opisany w~wiekach późniejszych model \textit{homo oeconomicus} oraz teorię racjonalności ekonomicznej.

Oprócz racjonalności w~sensie mikro, można mówić również o~racjonalności w~sensie makro. Kwarciński
\parencite*[s.~148]{klosinski_racjonalnosc_2009}
%\label{ref:RNDBnXpGSTNMf}(2009, s.~148)
tłumaczy to pojęcie, pisząc że ,,miano racjonalności można również przypisać
systemowi społeczno-gospodarczemu, który powstał w~wyniku swobodnej działalności tych podmiotów [tj. podmiotów
gospodarujących -- K.T.]''. Według Adama Smitha, wolny rynek wymiany dóbr i~usług niewątpliwie posiada taką cechę. Rynek
jest rozwiązaniem racjonalnym między innymi dlatego, że stanowi kanał wykorzystania miłości własnej w~sposób
prospołeczny. Ponadto, wolne kształtowanie się cen i~rozkładu wykorzystania zasobów powoduje coraz efektywniejsze ich
wykorzystywanie. Rynek jest więc racjonalnym ustrojem życia gospodarczego. Racjonalność podmiotu
gospodarującego u~Adama Smitha przejawia się natomiast na poziomie konkretnych,
jednostkowych wyborów rynkowych, jednak fundamentalne
jest to, że ludzie w~ogóle skłonni są do działań gospodarczych o~charakterze rynkowym. 

Do rozstrzygnięcia pozostaje pytanie o~rolę założenia o~racjonalności w~teorii ekonomicznej Smitha. Na pewno nie jest to
holistyczny pogląd na naturę człowieka w~ujęciu Smitha. Wręcz przeciwnie, pokazuje on człowieka jako istotę w~dużym
stopniu altruistyczną. Jego rozważania w~obszarze działań gospodarczych, aby były w~pełni zrozumiane, muszą
wchodzić w~skład szerszego kontekstu jego poglądów na temat człowieka. Z~tego powodu Deirdre McCloskey
twierdzi, że Adama Smitha można
wskazać jako ojca sposobu uprawiania ekonomii, który nazywa ,,humanomics''
\parencite[s.~20]{mccloskey_bourgeois_2016}.
%\label{ref:RND8i6dxgp5sd}(McCloskey, 2016, s.~20).
U~Smitha rozważania ekonomiczne są częścią o~wiele szerzej zakrojonych badań nad człowiekiem, jego życiem
jednostkowym i~społecznym. Późniejsza ekonomia stała się natomiast dziedziną bardziej
techniczną i~ograniczoną w~doborze problemów. Przyczyniło się to do jej sukcesu i~prestiżu nauki,
którą często postrzega się jako mogącą udzielać
kwantyfikowalnych, ostatecznych rozwiązań. Sam Smith w~dużej mierze był jednak filozofem. U~niego racjonalność widzieć
należy w~ogólniejszej teorii człowieka i~dlatego \textit{Bogactwo narodów} powinno się
interpretować w~świetle \textit{Teorii uczuć moralnych}.

Nie sposób wyczerpać interesującego nas tematu bez dodania komentarza do bodaj najsłynniejszego pojęcia naszego autora.
Adam Smith
\parencite*[s.~46]{smith_badania_2007}
%\label{ref:RNDmWGfdNtvia}(Smith, 2007b, s.~46)
pisał:

\myquote{
gdy [człowiek] kieruje wytwórczością tak, aby jej produkt posiadał możliwie najwyższą wartość, myśli tylko o~swym
własnym zarobku, a~jednak w~tym, jak i~w~wielu innych przypadkach, jakaś niewidzialna ręka kieruje nim tak, by zdążał
do celu, którego wcale nie zamierzał osiągnąć. Społeczeństwo zaś, które wcale w~tym nie bierze udziału, nie zawsze na
tym źle wychodzi. Mając na celu swój własny interes, człowiek często popiera interesy społeczeństwa skuteczniej niż
wtedy, gdy zamierza służyć im rzeczywiście.
}
Ważnym kluczem do interpretacji tego, czym właściwie jest słynna ,,niewidzialna ręka\textit{''}, jest główna teza książki
\textit{The Soul of Modern Economic Man} autorstwa Miltona L. Myersa. We wstępie pisze on następujące słowa
\parencite[s.~2]{myers_soul_1983}:
%\label{ref:RND9RhFQIVHli}(Myers, 1983, s.~2): 

\myquote{
Przekonamy się, że ekonomia klasyczna powstała jako rozwiązanie problemu, który trawił niektóre z~największych umysłów
filozofii moralnej przez prawie wiek przed Adamem Smithem. Ten klasyczny problem odnosił się do relacji pomiędzy
egoizmem (\textit{self-interest}) a~publicznym dobrobytem (\textit{public welfare}). Zobaczymy, że ekonomia klasyczna
powstała w~wyniku rozwiązania problemu, który był intensywnie badany przez filozoficznych poprzedników Adama Smitha.
}

Smith w~dziejach nauki kończy etap filozoficznych rozważań i~rozpoczyna nowy etap -- budowanie bogatych i~złożonych
wyjaśnień ekonomicznych. Jest on dziedzicem dorobku XVII i~XVIII-\mbox{-wiecznych} brytyjskich myślicieli,
który -- zgodnie z~Myersem -- rozpoczyna się od refleksji Thomasa Hobbesa. Metafora ,,niewidzialnej ręki''
ma pokazać, że to, jaką istotą
jest człowiek, sprawia, iż w~naturalny sposób jesteśmy w~stanie zagwarantować spełnienie naszych podstawowych celów:
przeżycia i~dobrostanu jednostkowego oraz gatunkowego (ze szczególnym naciskiem na grono bliskich). Dzieje się to na
drodze podziału pracy i~wymiany rynkowej. Rynek, aby działał w~taki sposób, musi jednak opierać się na uczciwości,
rozumianej jako wymiana oczekiwanych korzyści pomiędzy uczestnikami transakcji oraz jako dostęp do
informacji o~sytuacji rynkowej dla wszystkich uczestników. Tylko w~tych okolicznościach ogólna efektywność wykorzystania zasobów
zwiększa się do najwyższego możliwego poziomu, a~opisywane przez Smitha mechanizmy gospodarcze działają w~modelowany
przez niego sposób. Dobitnie pokazał to ostatni wielki kryzys gospodarczy.

Ta sama ,,niewidzialna ręka'' sprawia, że ludzie w~ogóle łączą się w~funkcjonalne społeczeństwo. \textit{Teorię uczuć
moralnych} rozpoczyna charakterystyczne stwierdzenie
\parencite[s.~5]{smith_teoria_1989}:
%\label{ref:RNDPPANJSiE8h}(Smith, 1989, s.~5): 

\myquote{
Jakkolwiek samolubnym miałby być człowiek, są niewątpliwie w~jego naturze jakieś pierwiastki, które powodują, iż
interesuje się losem innych ludzi, i~sprawiają, że ich szczęście jest dla niego nieodzowne, choć jedyna przyjemność,
jaką może stąd czerpać, to przyjemność oglądania tego. 
}

Opisane w~\textit{Bogactwie narodów} mechanizmy życia gospodarczego rozwijają tę tezę
i~pokazują, w~jaki sposób realizuje się ona w~tym podstawowym obszarze życia społecznego. Koncepcja racjonalności
ekonomicznej u~Smitha stanowi próbę realistycznego opisu natury ludzkiej i~rzeczywistości gospodarczej. Jej treść różni
się od tego, co przyjmowało wielu późniejszych ekonomistów i~właściwie zrozumiana może być tylko w~kontekście całej
filozofii Adama Smitha.

\section*{Zakończenie}

Jak zaznaczono we wstępie, założenie o~racjonalności podmiotu gospodarującego odegrało kluczową
rolę w~rozwoju ekonomii. Jednak jego kluczowa rola nie ogranicza się tylko do aspektów teoretycznych.
Szeroko rozpowszechniona
idea, że społeczeństwo kapitalistyczne formułowane jest przez racjonalne i~mające na celu swój własny interes
jednostki, stanowiła jedną z~kluczowych przesłanek w~procesie kształtowania się ogólnej kultury rozwiniętego
kapitalizmu. Francis Fukuyama
\parencite*[s.~4]{fukuyama_end_1989},
%\label{ref:RNDuNVN9rLhEK}(1989, s.~4),
analizując
stan społeczeństw późnego XX wieku, pisze: 

\myquote{
W rzeczy samej, również na prawicy istnieje fenomen, który nazwać możemy szkołą materializmu deterministycznego spod
znaku \textit{Wall Street Journal}, która w~małym stopniu ceni wagę ideologii i~kultury, a~która człowieka postrzega
jako z~gruntu racjonalną, maksymalizującą zysk jednostkę. Takiego typu jednostka oraz jej pościg za wartościami
materialnymi w~podręcznikach ekonomii przyjmowana jest za podstawę życia ekonomicznego w~ogóle. 
}
Tego typu orientacja, nazwana przez Fukuyamę imieniem organu najważniejszego wówczas centrum finansowego na świecie,
ujmuje życie gospodarcze jako strefę wolną od czynników etycznych. Oferuje wręcz swoją własną moralność, obojętną na
to, co w~zgodzie z~potocznym użyciem tego słowa nazwać możemy dobrym. W~słynnym cytacie z~filmu Olivera Stone'a
\textit{Wall Street} (1987) jeden z~głównych bohaterów, bogaty finansista Gordon Gekko, formułuje \textit{credo} tej
moralności: 

\myquote{
Moim przesłaniem dla Państwa jest, że chciwość, z~braku bardziej odpowiedniego słowa, jest dobra. Chciwość jest słuszna,
chciwość działa. Chciwość klaruje, przenika i~ujmuje istotę postępowego ducha. Chciwość, we wszystkich swoich formach:
chciwość życia, pieniądza, miłości, wiedzy odbiła swoje piętno na wszelkim rozwojowym ruchu ludzkości.
}

Okazuje się, że u~Adama Smitha, ojca teorii dającej zaplecze dla rozwoju kapitalizmu i~wielkiego zwolennika gospodarki
rynkowej, szukać możemy inspiracji do sprzeciwu wobec kultury chciwości. Aby w~pełni zrozumieć jego dorobek, potrzebny
jest szerszy, humanistyczny kontekst jego teorii. Odnaleziony u~niego model racjonalności jednostki oparty jest na
zupełnie innym rozumieniu moralności w~życiu gospodarczym
\parencite[zob. np.][]{przybyla_adam_2006}.
%\label{ref:RNDnzf29lTbB0}(zob. np. Przybyła, 2006).

Można powiedzieć, że u~podstaw kulturowego pojmowania kapitalizmu, o~którym pisał Fukuyama i~które dało podstawy do
usprawiedliwiania postawy moralnej podsumowanej w~maksymie \textit{greed is good}, leży neoklasyczne
myślenie o~podmiocie gospodarującym jako maksymalizatorze zysków, najpełniej wyrażone w~przywołanych we wstępie
tezach Edgewortha. Wynikający z~poglądów Smitha obraz podmiotu ekonomicznego nie sugeruje,
jakoby w~sferze kontraktów postępował on jak na
wojnie. Wręcz przeciwnie, życie gospodarcze i~wszystkie zjawiska związane z~podziałem pracy (zarówno w~skali wspólnoty,
przedsiębiorstwa, jak i~całego społeczeństwa) są wynikiem tych samych cech natury ludzkiej, które leżą u~podstaw
naszego życia społecznego. W~człowieku działają dwie potężne siły: miłość własna i~współczucie. Choć na pozór są one
przeciwne, pewna ,,niewidzialna ręka'' zaprojektowała je w~taki sposób, że tworzą one harmonię i~idealnie się
uzupełniają. Gdy dbamy o~to, aby nam było dobrze, sprawiamy radość współczującym z~nami osobom wokół. Robiąc coś dla
innych, robimy też coś dla siebie, ponieważ dzielimy ich radość i~przyjemność. Na tej zasadzie oparte jest życie
społeczne. Dzięki temu gatunek ludzki jest w~stanie przetrwać i~rozwijać cywilizację, a~poszczególne jednostki mogą
osiągnąć dobrostan. Życie gospodarcze -- jako kluczowa składowa życia społecznego -- jest idealną egzemplifikacją tych
mechanizmów. Kluczową siłą rynku jest to, że jednocześnie przynosi korzyść jednostkom, jak i~całemu społeczeństwu.

Wracając jeszcze do wyjściowego tematu racjonalności, można powiedzieć, że natura uczyniła człowieka bytem, który pomimo
tego, że na bazowym poziomie motywacji dla swych działań jest istotą emotywną, to może realizować racjonalne działania.
Myers
\parencite*[s.~95]{myers_soul_1983}
%\label{ref:RNDEjkEHkizH0}(1983, s.~95)
ujmuje to następująco:

\myquote{
Pragnienie aby handlować, wypływa z~głębi ludzkiego charakteru. Powstaje ono na skutek silnych i~bezpośrednich
impulsów i~jest właściwe całemu gatunkowi ludzkiemu. 
}
To ,,pragnienie'', ,,impulsy'', współczucie oraz miłość własna powodują, że w~sytuacjach rynkowych ludzie postępują w~sposób
zgodny z~racjonalnością w~sensie mikro i~że nasze społeczeństwo rozwinęło się w~kierunku stanu, realizującego
racjonalność w~sensie makro.

\end{artplenv}\label{turek-stop}
