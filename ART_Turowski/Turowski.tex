\begin{artplenv}{Krzysztof Turowski}
	{Mikropodstawy prawdziwe i~fałszywe}
	{Mikropodstawy prawdziwe i~fałszywe}
	{Mikropodstawy prawdziwe i~fałszywe}
	{Uniwersytet Gdański}
	{Microfoundations: true and false}
	{This article presents an overview and critique of the two leading macroeconomic approaches from the last 70 years:
		reasoning using high-level aggregates detached from individuals and their choices, and modeling using so-called
		microfoundations. We judge the validity of both methods, showing their inherent limits and deficiencies as explanatory
		and predictive tools of economics. We also underline several vital improvements, which are required if the models are
		supposed to guide policy decisions -- even if this means a~more modest and less conceited approach.}
	{methodology of economics, history of economic thought, macroeconomics, microfoundations, DSGE.}



\section*{Wstęp}
\lettrine[loversize=0.13,lines=2,lraise=-0.05,nindent=0em,findent=0.2pt]%
{J}{}akkolwiek filozofowie mogą się spierać o~dokładną definicję ekonomii, jej metod i~jej przedmiotu, tak w~codziennej
praktyce i~potocznym postrzeganiu jest to nauka dość dobrze określona i~wyodrębniona spośród nauk społecznych. Widać to
chociażby w~jej strukturze instytucjonalnej: odrębnych wydziałach i~katedrach, własnych specjalistycznych czasopismach
lub konferencjach oraz nagrodach, z~Nagrodą Banku Szwecji im. Alfreda Nobla na czele. W~dużym uproszczeniu można
powiedzieć, że ekonomia zajmuje się wszystkimi zagadnieniami związanymi z~gospodarką oraz gospodarowaniem.

Istnieje szeroka zgoda wśród ekonomistów, odzwierciedlona m.in. w~układzie większości współczesnych podręczników, co do
fundamentalnego podziału dziedziny na mikroekonomię i~makroekonomię. Mikroekonomia zajmuje się
zjawiskami z~indywidualnego punktu widzenia, m.in. rozpatruje determinanty oraz skutki decyzji konsumentów i~producentów, wyjaśnia
zależności cen, podaży i~popytu czy też bada firmę w~jej danym otoczeniu gospodarczym. Makroekonomia traktuje natomiast
o~fenomenach dotyczących całej gospodarki np. o~dochodzie narodowym, wzroście gospodarczym, kryzysach, inflacji czy
bezrobociu, jak również o~handlu międzynarodowym
\parencite{samuelson_ekonomia_2003}.
%\label{ref:RNDAXNmHXMliV}(Samuelson, Nordhaus, 2003).

Związek między obydwoma sferami wydaje się intuicyjnie oczywisty: bez indywidualnych wyborów jednostek nie istniałaby
gospodarka jako całość. To, co istnieje na poziomie całej gospodarki z~konieczności jest nadbudowane na decyzjach
jednostek, zatem teoria mikroekonomiczna i~makroekonomiczna powinny pozostawać w~ścisłym systematycznym związku. Co
więcej, wielu ekonomistów przynajmniej deklaratywnie przyjmuje metodologiczny indywidualizm, czyli tezę o~potrzebie
wyjaśniania zjawisk ekonomicznych przez odwołanie do subiektywnych ocen i~wartościowań jednostek.  W~tym ujęciu
makroekonomia powinna być raczej pewnym rozwinięciem i~uzupełnieniem mikroekonomii o~zależności wynikające z~agregacji
działań w~całej gospodarce oraz interakcji między różnymi grupami lub sektorami.

Z drugiej strony, istnieje w~ekonomii długa tradycja wyjaśniania zjawisk makroekonomicznych na poziomie abstrakcyjnych
agregatów, nieprzekładalnych wprost na poszczególne procesy na poziomie jednostkowym. Przykładami tego mogą być
rozważania merkantylistów dot. czynników gospodarczych determinujących bogactwo narodów, fizjokratyczne \textit{Tableau
economique} pokazujące przepływy dochodów między poszczególnymi grupami w~gospodarce, a~z~bardziej współczesnych np.
model AS-AD, wiążący poziom cen i~produkcji w~gospodarce ze zagregowanym popytem i~podażą.

Celem niniejszego artykułu jest przedstawienie oraz krytyka dwóch popularnych w~ostatnich 70 latach podejść
makroekonomicznych: nieco wcześniejszego, ujęcia zjawisk według bardzo ogólnych kategorii, oderwanych od poziomu
jednostek i~ich działań, oraz dominującego od lat osiemdziesiątych ubiegłego wieku tzw. modelowania z~mikropodstawami.
Przedstawiona zostanie krytyka obu metod w~kontekście ich zasadności jako narzędzi w~teorii ekonomii oraz wskazane
zostaną ograniczenia ich stosowalności do oceny i~doboru polityki gospodarczej państwa. Dodatkowo, zostaną
przedstawione pewne ogólne postulaty, których uwzględnienie mogłoby znacząco wspomóc lub uzupełnić obecnie stosowane
metody -- a~które to postulaty stanowiły fundament wielu wartościowych wglądów dokonanych przez
ekonomistów w~przeszłości, nakierowanych na zachodzące realnie procesy gospodarcze.

\section{Makroekonomia bez mikropodstaw}
W pierwszych dekadach po II wojnie światowej makroekonomia została zdominowana przez podejście skoncentrowane na
szukaniu relacji między zmiennymi obejmującymi całość gospodarki, takimi jak zagregowany popyt i~podaż, całkowita
produkcja, bezrobocie, poziom cen czy inflacja. Wśród sztandarowych przykładów tego typu podejścia należy wymienić tzw.
krzyż keynesowski (wiążący zagregowany popyt i~realną produkcję), model IS-LM (wyrażający zależność stopy
procentowej i~realnej produkcji), a~także krzywą Philipsa (opisującą postulowaną zależność między inflacją a~bezrobociem).
Modele gospodarcze składały się typowo z~kilku zmiennych, a~zadaniem ekonomistów była analiza interakcji tych zmiennych
%\label{ref:RNDN3QnGZQIJq}(Lachmann, 1973)
\parencite{lachmann_macro-economic_1973}\footnote{Typowym przykładem takiego podejścia jest \textit{Money, Interest,
and Prices}
\parencite{patinkin_money_1956}.
%\label{ref:RNDe8wVA5c7uE}(Patinkin, 1956).
}.

Przyczyn takiego stanu rzeczy można wskazać kilka: przede wszystkim istotny był proces recepcji \textit{Ogólnej teorii
zatrudnienia, produkcji i} \textit{pieniądza} Johna Maynarda Keynesa
\parencite*{keynes_general_1936},
%\label{ref:RND5MranCHmpU}(1936),
która szybko
zdobyła popularność wśród młodych, zdolnych ekonomistów m.in. Johna Hicksa, Franco Modiglianiego czy Paula Samuelsona
\parencite{moggridge_diffusion_1995}.
%\label{ref:RNDnBqm7Ivc9s}(Moggridge, 1995).
To właśnie oni dokonali formalizacji i~matematyzacji ujęć
keynesowskich w~ramach wyżej wymienionych modeli\footnote{Mimo uznania modelu IS-LM przez Keynesa za właściwą
interpretację \textit{Ogólnej teorii}
\parencite{king_history_2003},
%\label{ref:RND0kaWblMGjk}(King, 2003),
wielu ekonomistów wskazuje, że u~Keynesa
istnieją również ważne elementy mikroekonomiczne np. dotyczące oczekiwań i~niepewności, które są sprzeczne z~tzw.
syntezą neoklasyczną
\parencite{leijonhufvud_keynes_1969}.
%\label{ref:RNDs0gQMcriAO}(Leijonhufvud, 1969).
Joan Robinson wręcz nazywała powojenną
podręcznikową interpretację Keynesa ,,zwulgaryzowaną'' i~,,bękarcią''
\parencite{robinson_what_1974}.
%\label{ref:RNDsTNNR3eOoV}(Robinson, 1974).
}. Co
ważniejsze, olbrzymi sukces podręcznika Samuelsona upowszechnił te ujęcia, tworząc podstawy nauczania nowej ortodoksji,
w~której zagadnienia makroekonomiczne rozważano w~zasadniczym oddzieleniu od teorii mikroekonomii, przede wszystkim
traktującej o~alokacji dóbr przez racjonalnych maksymalizatorów w~danym otoczeniu i~systemie cenowym
%\label{ref:RNDA51vqkdiYx}(Colander, Landreth, 1996)
\parencite{colander_coming_1996}\footnote{Weintraub
\parencite*{weintraub_microfoundations:_1979}
%\label{ref:RNDiHw3ZZgTLu}(1979)
wskazuje na to,
że perspektywa mikroekonomiczna wychodzi od badania wyborów jednostek przy danych ograniczeniach, natomiast perspektywa
makroekonomiczna niejawnie zaprzecza, że dezagregacja jest jakkolwiek przydatna w~kontekście predykcji.}.

Źródeł opisywanych zmian zainteresowań ekonomistów można także szukać na zewnątrz. Z~jednej strony, lata międzywojenne
były okresem niestabilności gospodarczej, co naturalnie orientowało zainteresowania ekonomistów na problem kryzysów
gospodarczych oraz ich zapobiegania i~przezwyciężania. Z~drugiej strony, powstanie ZSRR, wdrażającego w~życie
pełnoskalową gospodarkę planową, rozbudziło dyskusje dotyczące wzrostu gospodarczego oraz właściwej roli polityki
gospodarczej w~tym procesie.

Równolegle można było zaobserwować powstanie systematycznego podejścia do badań empirycznych nad gospodarką.
Symbolicznym początkiem stało się ustanowienie w~roku 1930 Econometric Society oraz założenie w~1932 roku Komisji
Cowlesa (pod hasłem \textit{Science is measurement}), zorientowanej na podejście ilościowo-statystyczne. Zaczęły
powstawać coraz bardziej skomplikowane ilościowe modele gospodarcze autorstwa Jana Tinbergena, Lawrence’a Kleina oraz
wielu innych, starające się zasypać przepaść między ekonometrią a~teorią ekonomii oraz mające umożliwiać zarówno
przewidywanie zmian aktywności gospodarczej, jak i~porównywanie alternatywnych rozwiązań politycznych, a~zatem
zapewniać naukowe uzasadnienie dla państwowego nadzoru nad gospodarką
\parencite{de_vroey_keynesian_2012}.
%\label{ref:RNDQfzjK0pDtC}(De Vroey, Malgrange, 2012).

Nałożyło się na to ogólne fizykalistyczno-inżynierskie podejście do zagadnień ekonomicznych, mające swoje źródło jeszcze
w~XIX w.
\parencite{mirowski_more_1999}.
%\label{ref:RNDcTF11CDYlQ}(Mirowski, 1999).
Przykładowo, Irving Fisher uzasadniając głoszoną przez siebie
ilościową teorię pieniądza wprost pisał, że istnieje pełna analogia między prawem Boyle’a a~równaniem wymiany, w~którym
rolę cząsteczek gazu spełniają działające jednostki
%(Fisher, 1922)
\parencite{fisher_purchasing_1922}\footnote{Na marginesie warto dodać, że promotorem
doktoratu Fishera w~Yale był fizyk, specjalista od termodynamiki, Willard Gibbs.}. To podejście można zresztą
zaobserwować do dziś np. w~twierdzeniu, że modelowanie ekonomiczne nie różni się fundamentalnie od przewidywań
meteorologicznych lub epidemiologicznych, w~których nie analizuje się konkretnych jednostkowych interakcji
np. atomów, ale efekty globalne
\parencite{buchanan_forecast:_2013}.
%\label{ref:RNDZuwuTK5nIV}(Buchanan, 2013).

Istnieje jednak poważna różnica: podczas gdy przedmiotem ekonomii na poziomie mikro są ludzie -- znacząco
heterogeniczni, mający różne plany i~cele, nawet w~obiektywnie identycznych okolicznościach, atomy czy nawet
bakterie takiej właściwości -- zgodnie z~naszą wiedzą -- nie mają. Pozwala to sądzić, że analogia nie jest
do końca trafna
\parencite{penrose_biological_1952}.
%\label{ref:RND4t79oENLHg}(Penrose, 1952).
Co więcej, można przywołać w~tym miejscu
modyfikację popperowskiego argumentu przeciw historycyzmowi: ludzie są zdolni do uczenia się, modyfikacji swojego
zachowania, a~więc nie można traktować ich jako stałych i~niezmiennych przedmiotów działania wielkich sił, czy to
dziejowych, czy gospodarczych
\parencite{popper_poverty_1957,hoppe_economic_1995}.
%\label{ref:RNDLB8WeG6ZxD}(Popper, 1957; Hoppe, 1995).

W gruncie rzeczy, stosując takie podejście i~ignorując wiedzę o~poziomie mikroekonomicznym dokonujemy świadomej
rezygnacji z~wiedzy o~różnorodności i~zmienności zachowań jednostek. Zamiast jednak dokonywać abstrakcji
nieprecyzującej  --  a~więc uznać, że skutki tej heterogeniczności są nieznane i~zająć się badaniem tego, co od
nich niezależne  --  dokonywana jest raczej asercja jawnie fałszywej zasady głoszącej, że struktura zachowań
jednostek nie ma znaczącego wpływu na relacje między agregatami makroekonomicznymi
\parencite{long_realism_2006}.
%\label{ref:RND1jtnbEDYR6}(Long, 2006).

Ujęcie makroekonomiczne oparte jedynie o~wysokopoziomowe agregaty napotyka na inny podstawowy problem
metodologiczny: w~przytoczonym przypadku gazów posiadamy dobre operacyjne ujęcie temperatury czy objętości
gazu, a~więc jego właściwości
makroskopowych. Wręcz to te cechy są pierwotne w~porządku poznania wobec teoretycznego modelu atomowego, stanowiącego
wyjaśnienie na poziomie mikro. Natomiast w~ekonomii mamy do czynienia z~sytuacją odwrotną: bezpośrednio obserwujemy
poszczególne wymiany dóbr i~usług w~gospodarce pieniężnej, konkretne zmiany zatrudnienia, ale już treści pojęć ,,podaż
pieniądza'' czy ,,bezrobocie'' są znacznie trudniej uchwytne. Dość powiedzieć, że szeroko stosowane jest aż pięć
miar agregatowej podaży pieniądza (M\textsubscript{0}{}-M\textsubscript{3} oraz MZM)\footnote{Zob. Federal Reserve Bank
of St. Louis: \url{https://fred.stlouisfed.org/categories/24}.} oraz sześć miar bezrobocia\footnote{Zob. Bureau of Labor
Statistic: \url{https://www.bls.gov/news.release/empsit.t15.htm}.}, z~których żadna nie może być niearbitralnie
wyróżniona jako ta ,,właściwa''. Joseph Schumpeter
\parencite*{schumpeter_nature_2010}
%\label{ref:RNDBDWQZcLJxs}(2010)
ujmował to dosadnie:

\myquote{
Nie będziemy więcej zainteresowani nimi [pojęciami ,,dochodu narodowego'', ,,bogactwa narodowego'',
,,kapitału społecznego'' -- przyp. autora]. Gdybyśmy jednak to zrobili,
to dostrzeglibyśmy jak wielkie są ich niejasności i~trudności, i~że są blisko
związane z~rozlicznymi fałszywymi poglądami, nie prowadząc do ani jednego naprawdę ważnego aksjomatu\footnote{``[\mydots] we
will not even deal with them any more; if we would do it, it would become clear the amount of vagueness and problems
attached to it, that they are closely related with a~lot of skewed opinions, without leading to just one really
important axiom'' (jeśli nie zaznaczono inaczej wszystkie tłumaczenia są własne).}.
}

Oskar Morgenstern wskazywał na inną znaczącą różnicę: obserwacje makroekonomiczne są obciążone dokładnie tymi samymi
błędami, co obserwacje w~naukach przyrodniczych, ale również mają szereg własnych problemów, wynikających z~braku
zaplanowanych eksperymentów, unikalności zjawisk, ukrytych zależności czy problematyczności wiarygodności
kwestionariuszy. W~rezultacie, poprawna praktyka badawcza sugerowałaby konieczność zdecydowanie bardziej
ostrożnego i~wymagającego podejścia do zdobywania danych gospodarczych. Jednocześnie, nie sposób nie zauważyć,
że ekonomia, w~odróżnieniu od nauk przyrodniczych nie wypracowała aż tak rygorystycznego podejścia do błędów obserwacji. Niższe
standardy pomiarowe znajdują swoje odzwierciedlenie w~jakości zdobywanych informacji, które mimo tego są przedstawiane
jako pełnowartościowe w~debacie publicznej oraz stanowią materiał do modelowania dla innych ekonomistów, będących tylko
konsumentami pracy pomiarowej
\parencite{morgenstern_accuracy_1963}.
%\label{ref:RNDO4EkQydA8d}(Morgenstern, 1963).

Jak wskazywał Ludwig Lachmann, część miar agregatowych jest konstruowana przy założeniu, że gospodarka znajduje
się w~stanie równowagi ogólnej. Produkt Krajowy Brutto, aby adekwatnie móc agregować nominalną wartość produktów, wymaga,
żeby istniała pełna zgodność planów producentów i~konsumentów w~strukturze kapitałowej. Jeśli pewien proces produkcji
zostanie przerwany na jakimś etapie z~uwagi na brak popytu na dobro końcowe, to należy raczej mówić o~błędnej alokacji
kapitału niż o~czymś, co należy zaliczyć do wartościowej całkowitej produkcji. Co więcej, PKB jest miarodajne tylko
przy założeniu równowagi w~systemie elastycznych cen, dostosowujących się do planów działających jednostek. Jeśli
natomiast w~gospodarce istnieje szeroka kontrola cen (w postaci cen minimalnych/maksymalnych czy płac minimalnych), to
w~zasadzie dokonywana jest agregacja dość arbitralnych liczb, a~więc trudno mówić o~miarodajnym oddaniu poziomu
zasobności społeczeństwa
\parencite{lachmann_macro-economic_1973}.
%\label{ref:RNDL1VjrxjgN6}(Lachmann, 1973).
Dobrą ilustracją tego problemu są dane z~USA z~lat
czterdziestych ubiegłego wieku: w~latach 1941--1945 realne PKB wzrosło o~ponad 50\%, natomiast w~samym roku 1946 spadło
aż o~19\%. Jednak trudno traktować to inaczej jako artefakt gospodarki wojennej, czyli systemu opartego o~arbitralnie
ustalane ceny, w~którym znaczącą część wynosiły wydatki rządowe na przemysł zbrojeniowy
\parencite{vedder_great_1991}.
%\label{ref:RNDGHzLSK9MM6}(Vedder, Gallaway, 1991).
Realne PKB jest szczególnie złą miarą również pod innym względem:
ponieważ eksport i~import się znoszą w~równaniu, to zachodzi systematyczne niedoszacowanie zmiany dochodu realnego
spowodowanej zmianami warunków handlu międzynarodowego
\parencite{kohli_real_2004}.
%\label{ref:RNDYDEMp3ypUa}(Kohli, 2004).

Pokrewny problem, na jaki natrafiają teorie makroekonomiczne oderwane od mikroekonomicznych podstaw to gubienie
wewnętrznej struktury przyjętych zmiennych. Jednym z~czołowych ekonomistów podkreślających ten problem był Friedrich
von Hayek, który w~swojej recenzji \textit{Treatise on Money} Keynesa pisał, że ,,agregaty Pana Keynesa ukrywają
najbardziej fundamentalne mechanizmy zmiany''
%\label{ref:RNDWk8vGqK1SV}(Hayek, 1931b, s. 277)
\parencite[s.~277]{hayek_reflections_1931}\footnote{,,Mr.~Keynes'
aggregates~conceal~the most fundamental~mechanisms of change''.}. Zwracał on również uwagę, że statystyczne uogólnienia,
którymi posługuje się teoria co prawda mogą być wartościowe jako orientacyjne wskaźniki zachowania gospodarki,
ale w~ścisłym sensie są daleko mniej naukowe niż teoria mikroekonomiczna
\parencite{hayek_competition_2002}.
%\label{ref:RNDoWWxWk2jdt}(Hayek, 2002).
Rzeczywiście, w~świecie realnym mamy do czynienia z~bardzo złożoną strukturą zależności produkcyjnych, podzieloną na
wiele etapów, następujących po sobie w~czasie, oraz operujących wielością dóbr kapitałowych i~czynników produkcji.
Jakkolwiek modele oparte o~jedno dobro kapitałowe (np. model Solowa) lub przyjmujące \textit{implicite} jednookresową
strukturę produkcji (np. model mnożnika-akceleratora Samuelsona-Hansena) mogą jak najbardziej być zasadne jako
narzędzia dydaktyczne, to przykładanie ich do realnych danych gospodarczych wymagałoby uzasadnienia, że podejście to
jest przynajmniej dobrym pierwszym przybliżeniem, zamiast zwykłych asercji, że stan stacjonarny jest normalnym stanem
rzeczy w~rozwiniętych gospodarkach
\parencite{solow_growth_1970}.
%\label{ref:RNDKtUMGcbYXg}(Solow, 1970).
Co więcej, nawet jeśli takie uproszczenie
byłoby w~pewnej mierze trafne, to mogłoby być mylące z~zupełnie innych względów: przykładowo można pokazać, że modele z~jednym
dobrem kapitałowym sugerują poprawność naiwnej produktywnościowej teorii procentu, obalonej przez Eugena von
Bohm-Bawerka jeszcze w~XIX~w.
\parencite{murphy_dangers_2005}.
%\label{ref:RNDpZ5TnXyYlX}(Murphy, 2005).

Kenneth Arrow zauważał w~podobnym duchu, że makroekonomia uprawiana \textit{in abstracto} zupełnie ,,ignoruje
fundamentalną kwestię mikroekonomiczną, czyli kwestię heterogeniczności, w~szczególności heterogeniczności oczekiwań''
\parencite{colander_changing_2004}.
%\label{ref:RNDHKLqawzgu1}(Colander, i~in., 2004).
Uwaga Arrowa jest ściśle powiązana z~inną problematyczną cechą
krytykowanych tu teorii makroekonomicznych: ich ,,hydraulicznym'' charakterem, opartym o~założenie istnienia stałych
relacji na poziomie agregatów. Najsłynniejszym przykładem jest tu oczywiście krzywa Philipsa, wprowadzona przez
Samuelsona i~Solowa jako stabilna zależność stopy inflacji i~bezrobocia
\parencite{samuelson_analytical_1960}.
%\label{ref:RNDLFotpNZxf0}(Samuelson, Solow, 1960).
Podobna właściwość charakteryzuje doskonale znany model IS-LM, zgodnie z~którym przy braku zależności stopy
procentowej od zmian wydatków istnieje dodatnia zależność między wzrostem wydatków a~wzrostem produkcji
\parencite{hicks_mr._1937}.
%\label{ref:RNDnySLD72dtV}(Hicks, 1937).
Jak zauważa Coddington (autor pojęcia ,,hydrauliczny Keynesizm''), w~takim
podejściu istnieje tylko jedna siła sprawcza, czyli państwo
\parencite{coddington_keynesian_1976}.
%\label{ref:RNDFrhGvuJSYj}(Coddington, 1976).
Relacje
makroekonomiczne stają się prostymi narzędziami manipulacji w~ramach polityki gospodarczej: wystarczy wpływać na jedną
wartość, aby druga dążyła do odpowiedniego poziomu. Modele takie obiecywały, że większe zatrudnienie  -- 
powszechny cel polityki gospodarczej państw po II wojnie światowej
\parencite{robinson_second_1972}
%\label{ref:RNDhAvylCL6X3}(Robinson, 1972)
 -- jest możliwe do osiągnięcia bardzo prostymi środkami kosztem jedynie wyższego poziomu inflacji.

Powszechnie uznaje się, że ważnym punktem zwrotnym w~dziejach makroekonomii była krytyka przeprowadzona przez Roberta
Lucasa, który zauważył, że aby podejście hydrauliczne było poprawne i~dawało wiarygodne narzędzie oceny alternatywnych
propozycji polityki gospodarczej, to należy założyć stabilność relacji między parametrami modelu. Jednak, jak
wskazywał, już sama zmiana polityki może spowodować zmianę oczekiwań i~zachowań jednostek w~gospodarce, wpływając na
zmianę relacji, a~więc wywołując skutki odmienne niż przewidziane przez model. W~konsekwencji, wszelkie próby
,,racjonalnego'' wykorzystania zależności wskazywanej przez model mogą zawieść
\parencite{lucas_econometric_1976}.
%\label{ref:RND7vUlq0ckuL}(Lucas, 1976).
Jak zauważył Charles Goodhart
\parencite*[s.~116]{goodhart_problems_1984}:
%\label{ref:RNDz8Bw6io9rS}(1984, s. 116):

\myquote{
Każda obserwowana statystyczna regularność będzie dążyła do zaniknięcia, gdy zostanie poddana naciskowi dla celów
kontroli\footnote{,,Any observed statistical regularity will tend to collapse once pressure is placed upon it for
control purposes''.}.
}

Chociaż krytyka ta wywołała szeroki oddźwięk i~wpłynęła znacząco na dalszy rozwój dyscypliny, to należy zauważyć, że nie
był on w~tym względzie pionierem. Podobny argument o~znaczeniu zmienności liczbowych wartości parametrów
makroekonomicznych w~zależności od zmian struktury mikroekonomicznej  --  i~wynikająca z~tego problematyczność
stosowania metod matematycznych oraz ujęcia ilościowego  --  wysuwał już John Maynard Keynes w~sporze z~Janem
Tinbergenem pod koniec lat trzydziestych XX wieku
\parencite{keynes_professor_1939}.
%\label{ref:RNDncDOgZbWH5}(Keynes, 1939).
W~bardziej radykalnej formie
przedstawili tę linię krytyki przedstawiciele szkoły austriackiej, podkreślający brak istnienia stałych relacji między
obserwowalnymi zmiennymi gospodarczymi, z~uwagi na konieczność zapośredniczenia przez kategorię ludzkiego wyboru,
będącą, zgodnie nawet z~dzisiejszym stanem wiedzy, ,,ostateczną daną'' dla ekonomistów
\parencite{mises_ludzkie_2007,rothbard_praxeology:_1976}.
%\label{ref:RNDeqT6vNSc4s}(Mises, 2007; Rothbard, 1976).

Sukces krytyki Lucasa należy ściśle powiązać z~faktem, że coraz bardziej złożone modele makroekonomiczne mimo
wieloletniego zbierania danych i~wprowadzania kolejnych zależności okazywały się nadal dostarczać przewidywania dalekie
od zadowalających. W~szczególności, ujawniły one swoją bezradność wobec zjawiska stagflacji lat siedemdziesiątych
XX~w., wykluczonej z~góry na gruncie teoretycznym
\parencite{kydland_econometrics_1991}.
%\label{ref:RNDWFylmxKR9k}(Kydland, Prescott, 1991).

Dla lepszego zobrazowania większości wyżej wymienionych zagadnień przeanalizujmy poruszone kwestie na konkretnym
przykładzie równania wymiany $MV = PY$, gdzie $M$~oznacza ilość pieniądza w~obiegu, $V$~szybkość biegu, $P$~poziom cen,
$Y$~wolumen produkcji. Przede wszystkim należy zauważyć, że właściwie żadna ze zmiennych nie
jest precyzyjnie określona, chociaż mamy do czynienia z~różnym stopniem braku precyzji.
Wcale nie jest oczywiste, który agregat pieniężny powinien być
uwzględniany: $M0$~(czyli baza monetarna -- raczej nie, ponieważ w~systemie z~rezerwą cząstkową zachodzi
również niezależna kreacja w~ramach systemu banków komercyjnych), $M1$ (baza monetarna oraz pieniądz gotówkowy), a~może jeszcze
inny.

Podobnie rzecz się ma z~poziomem cen i~produkcją realną: stosowanie jakiegokolwiek indeksu opartego o~koszyk dóbr rodzi
problem doboru koszyka oraz odpowiedniego ważenia jego elementów. Typowe praktycznie stosowane indeksy cenowe (CPI,
PPI, deflator PKB) mają ewidentne ograniczenia jeśli chodzi o~zakres uwzględnianych przez nie dóbr. Co więcej, jak
wskazuje Murray Rothbard, w~rzeczywistości produkcja realna nie ma żadnej naturalnej jednostki, więc mamy do czynienia
raczej z~ekwiwokacją iloczynu ,,ogólnej produkcji'' z~,,ogólnym poziomem cen'' oraz sumy wszystkich poszczególnych
transakcji kupna/sprzedaży dóbr w~gospodarce. Ta ostatnia liczba jest jednak tylko drugą stroną łącznej sumy księgowej
wartości pieniężnej transakcji  --  i~jako taka niewiele wnosi do wyjaśnienia czy nastąpił np. wzrost
produkcji w~gospodarce
\parencite{rothbard_man_1962}.
%\label{ref:RNDtuoh9VpzA9}(Rothbard, 1962).

Prędkość obiegu pieniądza $V$~nie jest natomiast w~żaden sposób, nawet przybliżona, obserwowana
bezpośrednio, a~jedynie wnioskowana na podstawie wartości pozostałych trzech zmiennych. Popularne założenie
monetarystyczne o~stabilności wydatków, a~zatem o~stałości $V$~wydaje się dość arbitralne. Co gorsza, jeśli $M$~jest uznane
za podaż pieniądza w~konkretnym punkcie czasu, to pojawia się problem z~interpretacją $V$, ponieważ nie można
wtedy ujmować jej jako szybkości obrotu pieniądza w~jednostce czasu tj. jako zmiennej typu \textit{flow}.

Jeśli zastosujemy krytykę Lucasa, to od razu widać, że zmiany polityki monetarnej (kontrolującej bezpośrednio jedynie
$M0$) mogą wpłynąć w~niejednoznaczny sposób na inne agregaty pieniężne, zależnie od
konkretnych decyzji banków komercyjnych i~innych instytucji kreujących pieniądz. Podobnie, może to wywołać zmiany
oczekiwań inwestorów i~konsumentów przekładające się na inne decyzje gospodarcze, powodujące zmiany $V$~albo
wręcz na zmiany struktury kapitałowej, nieznajdujące swojego odzwierciedlenia w~agregatach $P$~i~$Y$.
Sam Milton Friedman zauważał, że wpływ zmiany pojedynczej zmiennej na dostosowania pozostałych podlega
opóźnieniom o~różnej wielkości i~czasie trwania, niemniej nie podał żadnej systematycznej reguły, która miałaby określać jak
zachodzi ten proces  --  stwierdzał jedynie, że reguła stałego przyrostu podaży pieniądza zredukuje czas
dostosowania do minimum
\parencite{friedman_counter-revolution_1996}.
%\label{ref:RND9kZZCVdmhF}(Friedman, 1996).

Warto dodać również, że równanie wymiany z~założenia pomija zależności cen względnych. Tymczasem, jak wskazuje
powstawanie baniek na rynkach aktywów, ceny różnych dóbr i~usług w~gospodarce nie rosną równomiernie, wbrew modelowi
,,pieniądza z~helikoptera''. Proces rozchodzenia się nowego pieniądza w~gospodarce zachodzi nierównomiernie i~wywołuje efekty
redystrybucyjne tzw. efekty Cantillona, zgodnie z~kolejnością odbiorców: od ostatnich do pierwszych. W~rezultacie nawet
twierdzenie o~tym, że w~długim okresie wzrost podaży pieniądza nie zmienia proporcji cen wydaje się dyskusyjne. Co
więcej, można zasadnie upatrywać przyczyny zaburzeń strukturalnych i~cykli koniunktury w~gospodarce
właśnie w~zaburzeniu struktury cen względnych
\parencite{sieron_efekt_2017}.
%\label{ref:RND6B9MeZMMt0}(Sieroń, 2017).

Jedyna interpretacja, w~której równanie wymiany byłoby powszechnie akceptowane przez wszystkich ekonomistów uznaje je za
tautologię, czysto księgowe równanie \textit{ex post}, w~którym łączna suma wydatków pieniężnych jest równa sumie cen
sprzedanych towarów
\parencite{yeager_tautologies_1994}.
%\label{ref:RNDDbW9o6gDBe}(Yeager, 1994).
Należy jednak zauważyć, że nie jest to sens, jaki nadawali
mu jego monetarystyczni proponenci, z~Miltonem Friedmanem na czele  --  szczególnie, że taka interpretacja nie
nadaje się w~żaden sposób jako kryterium dobrej polityki pieniężnej.

\section{Współczesne modele makroekonomiczne z~mikropodstawami}
Rozczarowanie stylem teoretyzowania makroekonomicznego na poziomie agregatów oderwanych od decyzji jednostek
przyszło w~latach siedemdziesiątych XX wieku, czego głośnym wyrazem była wspomniana już krytyka Lucasa. Sam Lucas, oprócz
wyrażenia swojej dezaprobaty dla dominującego podejścia, wzywał do budowy modeli opartych o~rzeczywiście stałe,
głębokie parametry, najlepiej odniesione do wyborów i~oczekiwań jednostek
\parencite{lucas_econometric_1976}.
%\label{ref:RNDHfFbgxSxuO}(Lucas, 1976).

Podstawą nowego podejścia został model gospodarki zbudowany na bazie optymalizujących agentów, racjonalnych
oczekiwań i~czyszczenia się rynku
\parencite{kydland_time_1982}.
%\label{ref:RND75bn9tCDJK}(Kydland, Prescott, 1982).
To podejście do modelowania gospodarki,
rozwijane było przez następne lata a~współcześnie znane jest pod ogólną nazwą DSGE (\textit{Dynamic Stochastic General
Equilibrium}). Firmy i~gospodarstwa domowe modelowane są w nim przez zadane funkcje użyteczności, maksymalizujące odpowiednio
zyski pieniężne i~użyteczność, natomiast decyzje innych agentów oraz zmiany instytucjonalne są włączane w~postaci
równań-ograniczeń
\parencite{woodford_interest_2011,gali_macroeconomic_2007}.
%\label{ref:RNDsFpeCscraM}(Woodford, 2011; Galí, Gertler, 2007).


Stopniowo podejście to zyskiwało na popularności, jednak oczywisty brak realizmu podstawowych mechanizmów
leżących u~podłoża modelu Kydlanda-Prescotta doprowadziło nowych keynesistów do włączania elementów, mających zbliżyć model do
zachowania świata rzeczywistego. Współcześnie kanoniczne modele zawierają m.in. konkurencję monopolistyczną zamiast
doskonałej, wprowadzają pieniądz, jak również uwzględniają rolę władzy monetarnej jako instytucji wprowadzającej do
gospodarki szoki nominalne
\parencite{fernandez-villaverde_econometrics_2010}.
%\label{ref:RNDV3sfo7IQ4E}(Fernández-Villaverde, 2010).

Modele DSGE są obecnie szeroko stosowane przez banki centralne np. Bank Anglii, EBC (Europejski Bank Centralny) oraz FED (Bank Rezerwy Federalnej)
\parencite{smets_dsge_2010,tovar_dsge_2009}.
%\label{ref:RNDoHzQQGxXO2}(Smets, i~in., 2010; Tovar, 2009).
Nie dziwi zatem, że z~tym podejściem wiązane są duże
nadzieje, co można zauważyć również w~deklaracjach o~skoku niczym ,,od braci Wright do Airbusa 380 w~jednym pokoleniu''
\parencite{fernandez-villaverde_econometrics_2010}.
%\label{ref:RNDKT0sLwvgSN}(Fernández-Villaverde, 2010).
Nie znaczy to jednak, że podejście to nie spotyka się z~krytyką
 --  wręcz przeciwnie, wobec tego podejścia formułowano szereg zarzutów dotyczących niemal wszystkich aspektów
modelowania DSGE. Co zrozumiałe, fala krytyki nasiliła się po kryzysie 2008 roku, który
niewątpliwie nastąpił wbrew przewidywaniom wynikającym ze stosowanych przez banki centralne modeli.

Po pierwsze, kanoniczne wersje modelu DSGE przyjmują, że gospodarka składa się z~wielu identycznych gospodarstw domowych
(oraz firm), którym odpowiada ,,reprezentatywne'' gospodarstwo, mające ,,typowe'' cechy. Stosowanie reprezentatywnych
podmiotów gospodarujących motywowane jest prostotą modelu, jak również wskazaniem na istnienie możliwości agregacji heterogenicznych
podmiotów gospodarujących bez wpływu na ceny równowagowe, o~ile są spełnione tzw. warunki Gormana
\parencite{eichenbaum_estimating_1990}.
%\label{ref:RNDOw9twxJ9GM}(Eichenbaum, Hansen, 1990).
Problem w~tym, że warunki te, co było przeoczane przez kolejne pokolenia ekonomistów, wymagają
restrykcyjnych kryteriów dopuszczalności indywidualnych funkcji użyteczności  --  warunki te nie są
spełnione przez żadną typową funkcję przyjmowaną w~rzeczywistych modelach DSGE
\parencite{jackson_non-existence_2017}.
%\label{ref:RNDIwhIj4mcwm}(Jackson, Yariv, 2017).
Słowem, konstrukt reprezentatywnego podmiotu gospodarującego nie daje się przełożyć na żaden układ preferencji jednostkowych
podmiotów gospodarujących, którzy byliby opisywani przez jakiekolwiek nieliniowe funkcje użyteczności, w~tym typowo
przyjmowane w~literaturze funkcje CARA czy CRRA. Co więcej, jak pokazują ci sami autorzy, każda heterogeniczność preferencji
czasowych powoduje, że ich agregacja staje się niespójna w~czasie  --  a~więc nie może być opisywana przez
funkcje z~wykładniczym dyskontowaniem preferencji, tak jak zwykle jest dokonywane
\parencite{jackson_collective_2015}.
%\label{ref:RNDrV24f8QxiJ}(Jackson, Yariv, 2015).
Podejście oparte o~reprezentatywnych podmiotów gospodarujących jest przykładem skrajnego redukcjonizmu pojęciowego,
utożsamiającego pojęcia z~zakresu mikro i~makro. W~rezultacie wykluczone z~góry zostają wzorce interakcji podmiotów gospodarujących,
przykładowo wskazujące na istnienie systematycznego ryzyka lub problemów koordynacji
\parencite{colander_financial_2009}.
%\label{ref:RNDdPJP62FWpi}(Colander, i~in., 2009).

Po drugie, pouczający jest sam sposób doboru oraz obrony założeń dotyczących dodatkowych mechanizmów przyjmowanych w~wielu
modelach DSGE. Przykładowo, bardzo popularnym sposobem modelowania zmian cen w~nowokeynesowskich modelach DSGE jest
tzw. \textit{Calvo pricing}, polegające na tym, że poszczególne firmy zmieniają cenę niezależnie z~pewnym ustalonym
prawdopodobieństwem, wspólnym dla wszystkich
\parencite{calvo_staggered_1983}.
%\label{ref:RND5WO8K8aB8K}(Calvo, 1983).
Uzasadniane to bywa brakiem
potrzeby jawnego śledzenia rozkładu cen między firmami oraz jednoczesnym zachowaniem nominalnych opóźnień, które są
przez nie wywołane
\parencite{christiano_nominal_2005}.
%\label{ref:RNDnBnXe5xul6}(Christiano, i~in., 2005).
Problemem jest jednak to, że próby oszacowania
liczbowej wartości prawdopodobieństwa opierają się na bardzo zgrubnym szacowaniu kosztu krańcowego przez udział
pracy w~wartości produktu  --  co jednak wymaga kolejnych założeń: o~tym, że funkcja produkcji jest funkcją
Cobba-Douglasa, a~także że rynek pracy jest doskonale konkurencyjny
\parencite{wolman_sticky_1999}.
%\label{ref:RNDmsX4gGreHH}(Wolman, 1999).

Co ciekawe, zwolennicy wyceny według metody Calvo broniąc się przed zarzutami zwracają uwagę na to, że inne często
stosowane modele przyjmują równie nierealistyczne założenia lub wskazując na to, że sprzeciw wobec tego podejścia
wynika z~jego konkluzji, uzasadniających interwencję państwa. Rzeczywiście, alternatywa w~postaci np. modelu
Lagosa-Wrighta, popularnego wśród nowych monetarystów, zakłada wymiany między anonimowymi stronami, które więcej nigdy
się nie spotykają
\parencite{lagos_unified_2005}.
%\label{ref:RNDLMUhDoGVrb}(Lagos, Wright, 2005).
Niemniej, jakkolwiek twierdzenia o~wadliwości innych
podejść lub częstej ocenie modeli według wypływających z~nich wniosków są zasadne, tak trudno uznać to za dobre
uzasadnienie obranego podejścia.

Podobnie rzecz się ma z~innymi popularnymi elementami modeli DSGE np. z~modelem konkurencji monopolistycznej
Dixita-Stiglitza
\parencite{blanchard_monopolistic_1987}.
%\label{ref:RND2Ofggd7tsC}(Blanchard, Kiyotaki, 1987).
Pozostaje on w~sprzeczności z~mikroekonomicznym
twierdzeniem o~zależności elastyczności cenowej popytu na dobro od zmian liczby konsumentów raczej niż od zmian ilości
kupowanej przez każdego konsumenta, ponieważ zakłada, że każdy konsument kupuje u~każdego producenta. Co
więcej, z~modelu wynika bardzo ścisły związek między elastycznością popytu i~narzutem ponad koszt krańcowy, również
niezgodny z~badaniami empirycznymi
\parencite{yun_reconsidering_2011}.
%\label{ref:RNDE4t6zhsvH5}(Yun, Levin, 2011).

Również ujęcie pieniądza w~modelach pozostawia wiele do życzenia: początkowo był on zwyczajnie nieobecny, potem został
wprowadzony jako argument funkcji użyteczności. Jak przyznają nawet zwolennicy tego podejścia, ten sposób jest tyleż
niezbyt elegancki, co w~gruncie rzeczy sprowadza się do przyznania, że nie istnieje dobry model pieniądza, który można
zestawić z~danymi oraz zastosować do oceny polityki gospodarczej
\parencite{fernandez-villaverde_econometrics_2010}.
%\label{ref:RNDhGQ3GkCYBL}(Fernández-Villaverde, 2010).
Ponadto, pieniądz odgrywa rolę w~gospodarce poprzez szereg różnych kanałów zależności i~można rozsądnie postulować, że
np. wpływ inflacji na dobrobyt czy zależności między bazą monetarną a~kreacją pieniądza w~systemie bankowym są
czynnikami, które mogą odgrywać istotną rolę w~decyzjach jednostek i~ich skutkach na poziomie gospodarczym
\parencite{wallace_whither_2001}.
%\label{ref:RNDP00NsOgvwD}(Wallace, 2001).
W~szczególności, problem z~zamodelowaniem sektora finansowego prowadzi do
jego niedocenienia, a~zarazem przeceniania polityki monetarnej oraz szoków realnych
\parencite{tovar_dsge_2009}.
%\label{ref:RNDDWwolbo7gO}(Tovar, 2009).

Ważnym problemem modeli DSGE podkreślanym przez ich krytyków jest także zagadnienie odpowiedniego doboru parametrów.
Przede wszystkim, sami proponenci doskonale zdawali sobie sprawę z~problemów związanych z~ekonometrycznymi testami
własnych modeli. Jak wspominał jeden z~głównych twórców omawianej tu rewolucji makroekonomicznej (i również noblista)
Thomas Sargent
\parencite[s.~567–568]{evans_interview_2005}:
%\label{ref:RNDATmilKwNn4}(Evans, Honkapohja, 2005, s. 567–568):

\myquote{
Bob Lucas i~Ed Prescott początkowo byli bardzo entuzjastycznie nastawieni do ekonometrii racjonalnych oczekiwań. W~końcu
chodziło o~sprostanie tym samym wysokim standardom, których brak wytykaliśmy keynesistom.
Jednak po około pięciu latach testowania współczynników wiarygodności na modelach racjonalnych oczekiwań,
jak sobie przypominam, obaj mówili mi, że te testy odrzucały zbyt wiele dobrych modeli\footnote{``Bob Lucas and Ed Prescott
were initially very enthusiastic about rational expectations econometrics. After all, it simply involved imposing on
ourselves the same high standards we had criticized the Keynesians for failing to live up to. But after about five
years of doing likelihood ratio tests on rational expectations models, I~recall Bob Lucas and Ed Prescott both telling
me that those tests were rejecting too many good models''.}.
}

Konsekwencją tego było wypracowanie podejścia według tzw. kalibracji, czyli zakładania z~góry pewnych parametrów,
odzwierciedlających rzeczywiste cechy gospodarki
\parencite{kydland_econometrics_1991}.
%\label{ref:RNDKMbnj4ne7i}(Kydland, Prescott, 1991).
Wśród typowo
podawanych parametrów można znaleźć m.in. udział pracy w~wynagrodzeniach całkowitej produkcji lub najróżniejsze
założenia z~badań mikroekonomicznych dotyczące np. zachowań gospodarstw domowych. Jednak należy podkreślić, że
ilościowa estymacja parametrów mikroekonomicznych również jest daleka od jednoznaczności i~z~pewnością nie można
postrzegać oszacowań jako czarnych skrzynek, gotowych do włączenia w~model makroekonomiczny z~pominięciem
kontekstu, w~jakim zostały opracowane
\parencite{hansen_empirical_1996}.
%\label{ref:RNDB0nmBH5Oyu}(Hansen, Heckman, 1996).
Co więcej, to podejście,
wbrew postulowanej
ścisłości, nie posiada żadnego odniesienia do jakichkolwiek kryteriów dobrej lub złej zgodności z~danymi, a~więc
istnieje fundamentalna trudność porównania zasadności alternatywnych modeli wobec siebie
\parencite{sims_macroeconomics_1996}.
%\label{ref:RNDQWy9GqcUB9}(Sims, 1996).

Inny problem modeli DSGE wynika z~ich złożoności. Typowo modele te nie posiadają jawnych rozwiązań, więc konieczne jest
numeryczne szacowanie rozwiązań. Powszechną praktyką jest linearyzacja równań w~modelu tj. przekształcanie ich do
postaci równań liniowych przez rozwinięcie w~szereg Taylora i~pominięcie zależności wyższego rzędu, a~dopiero potem
dopasowywanie do danych. Jednak prowadzi to do dość oczywistego błędu polegającego na rozbieżności
miary wiarygodności (\textit{likelihood}) modelu pierwotnego i~zlinearyzowanego. Błąd ten rośnie wraz ze zwiększeniem
rozmiaru próbki danych i~nie można go pomijać w~rzeczywistych modelach
\parencite{fernandezvillaverde_convergence_2006}.
%\label{ref:RND38YrsQ2P1S}(Fernández-Villaverde, i~in., 2006).
Dodatkowo, linearyzacja powoduje szereg innych problemów, takich jak gubienie istotnej dynamiki
gospodarczej, przejawiającej się dopiero w~zachowaniu nieliniowym np. związanym z~premią za ryzyko
\parencite{dou_macroeconomic_2017}.
%\label{ref:RNDde2kWOEhpS}(Dou, i~in., 2017).

Makroekonomiści pracujący z~modelami DSGE podkreślają ich dopasowywanie do danych makroekonomicznych w~postaci
wieloletnich szeregów czasowych. Doskonałe dopasowanie do danych nie musi być jednak wartością samą w~sobie
i~jest jak najbardziej możliwe, że gorszy model prowadziłby do lepszych zaleceń na przyszłość
\parencite{kocherlakota_model_2007}.
%\label{ref:RNDfQAYx4QK1U}(Kocherlakota, 2007).
Przykładowo, ocena wpływu zmiany opodatkowania na podaż pracy zależy od
elastyczności podaży pracy, ale przeszłe dane nie pozwalają na odróżnienie zależności przesunięć krzywej podaży od
przeszłych zmian podatkowych. Jak wskazywano już w~latach osiemdziesiątych XX wieku, dużo ważniejsze dla modelu są
właściwy dobór i~identyfikacja parametrów
\parencite{sims_macroeconomics_1980}.
%\label{ref:RNDwPLq0mt1GI}(Sims, 1980).

Jak się okazuje, modele DSGE wbrew pozorom nie są w~pełni odporne na krytykę Lucasa, ponieważ ich analiza na
długoterminowych szeregach czasowych pokazuje, że rzekomo ,,stałe'' parametry faktycznie wykazują
dryft w~czasie. W~rezultacie można pokazać, że zastosowanie ich w~latach siedemdziesiątych dawałoby
właściwie podobne wyniki co ówczesne
modele oparte o~krzywą Phillipsa, sugerujące istnienie zależności między bezrobociem a~inflacją  --  znikającej
jednak przy próbie jej wykorzystania
\parencite{hurtado_dsge_2014}.
%\label{ref:RND229XO0pmvC}(Hurtado, 2014).
Niezależnie od tego, czy jest to wynik
błędnej specyfikacji modelu, problemów z~identyfikacją i~estymacją parametrów, czy też faktycznej zmienności parametrów
strukturalnych w~świecie realnym, podważa to zasadność wykorzystania modeli jako narzędzi kierowania polityki gospodarczej.

Podsumowując, właściwie trudno nazwać to podejście budowaniem modelu od mikropodstaw, ale raczej przekładaniem
pożądanych makroekonomicznych założeń, wymaganych przez teorię i~posiadane dane, na samą konstrukcję stochastycznego
mikroekonomicznego podmiotu gospodarującego, nie mającego absolutnie nic wspólnego z~faktycznym zachowaniem ludzi na rynku
\parencite{machaj_money_2017}.
%\label{ref:RNDMCIGyExrbG}(Machaj, 2017).
Wiele dobieranych parametrów (np. elastyczności podaży pracy,
nawykowości w~funkcji użyteczności) wydaje się mieć nierealistyczny rząd wielkości, wynikający ze specjalnego dopasowania ich pod
okresy kryzysu
\parencite{korinek_thoughts_2017}.
%\label{ref:RNDgAteS9Gr4U}(Korinek, 2017).
Wydaje się, że dość trafnie istotę całego tego podejścia ujął
Robert Solow
\parencite[s.~241]{solow_state_2008}:
%\label{ref:RND3QW6iALHi7}(2008, s. 241):

\myquote{
[\mydots] dodawanie pewnych realistycznych tarć nie czyni bardziej
wiarygodnym, że obserwowana gospodarka działa zgodnie z~pragnieniami pojedynczej, spójnej, patrzącej w~przyszłość
inteligencji\footnote{``My point is precisely that attaching a~realistic or behavioral deviation to the Ramsey model
does not confer microfoundational legitimacy on the combination. [\mydots] adding some realistic frictions does not make it
any more plausible that an observed economy is acting out the desires of a~single, consistent, forward-looking
intelligence''.}.
}

Konsekwencją tego jest wykluczenie możliwości jednostkowych i~kolektywnych błędów, niewiedzy, a~także niezgodności
planów, prowadzących do przechodzenia przez stany nierównowagowe.

Nic dziwnego, że przy okazji kryzysu 2008~r. na autorów modeli, które spektakularnie zawiodły, spadła lawina krytyki,
zarówno ze strony kolegów po fachu, jak również polityków czy dziennikarzy. Warto zwrócić uwagę na podejście części
makroekonomistów do wysuwanych zarzutów. Lucas, jeden z~ojców omawianego tu podejścia, odpowiadał następująco na
zarzut, że jego podejście zawiodło nie przewidując kryzysu finansowego 2008 roku
\parencite{lucas_defence_2009}:
%\label{ref:RNDEYFvhxvVHN}(Lucas, 2009):

\myquote{
Wiadomo od ponad 40 lat i~jest to jeden z~głównych wniosków z~,,hipotezy rynków efektywnych'' Eugene’a Famy, która
stwierdza, że cena aktywa finansowego odzwierciedla wszystkie istotne, ogólnie dostępne informacje. Gdyby ekonomista
miał formułę, która potrafi wiarygodnie przewidzieć kryzys, powiedzmy, z~tygodniowym wyprzedzeniem, to ta formuła
stałaby się częścią ogólnie dostępnej informacji i~ceny spadłyby tydzień wcześniej\footnote{ ``It has been known for
more than 40 years and is one of the main implications of Eugene Fama's ``efficient-market hypothesis'' (EMH), which
states that the price of a~financial asset reflects all relevant, generally available information. If an economist had
a formula that could reliably forecast crises a~week in advance, say, then that formula would become part of generally
available information and prices would fall a~week earlier''.}.
}

W podobnym duchu wypowiadał się William Easterly
\parencite*{easterly_idiots_2009}:
%\label{ref:RNDmVpUuBkkg4}(2009):

\myquote{
Ekonomiści zrobili coś lepszego niż przewidzenie kryzysu. Prawidłowo przewidzieliśmy, że nie będziemy w~stanie tego
przewidzieć. Najważniejszą częścią mocno znienawidzonej hipotezy rynków efektywnych jest to, że nikt nie może
systematycznie być lepszym niż giełda. Co oznacza, że nikt nie jest w~stanie przewidzieć krachu na rynku, ponieważ
gdyby mógł, to oczywiście pokonałby rynek\footnote{ ``[\mydots] economists did something even better than predict the crisis.
We correctly predicted that we would not be able to predict it. The most important part of the much-maligned Efficient
Markets Hypothesis (EMH) is that nobody can systematically beat the stock market. Which implies nobody can predict a
market crash, because if you could, then you would obviously beat the market''.}.
}

Słowem, czołowi zwolennicy modelowania makroekonomicznego na bazie mikropodstaw sami przyznają, że modele się sprawdzają
wtedy, gdy się sprawdzają, a~wtedy gdy byłyby najbardziej potrzebne, to sprawdzać się nie mogą  --  co w~zasadzie
podaje w~wątpliwość sensowność samego modelowania.

Z jednej strony, ignoruje to fakt, że jak najbardziej istnieli ekonomiści (raczej wywodzący się z~nurtów
heterodoksyjnych), którzy ostrzegali przed nadciągającym kryzysem roku 2008\footnote{Przykładowo: Steve Keen, Michael
Mussa, Ann Pettifor, Raghuram Rajan, Nouriel Roubini czy Mark Thornton.}. Oczywiście, można zasadnie twierdzić, że
generalnie czasy kryzysu powodują wzrost popularności ekonomicznych szarlatanów
\parencite{robinson_second_1972}
%\label{ref:RND4qV3bcm9nv}(Robinson, 1972)
 --  ale też trzeba zwrócić uwagę, że wielu podawało wyjaśnienia mające rzeczowe uzasadnienia, zakorzenione
w obserwowanej rzeczywistości gospodarczej. W~szczególności podkreślano problemy strukturalne systemu finansowego,
powstanie baniek na rynkach aktywów finansowych i~nieruchomości oraz politykę sztucznego zaniżania stóp procentowych
prowadzącą do kumulacji błędnych inwestycji kapitału. Jednak ponieważ te teorie były przedstawione w~formie werbalnej,
czasem prostych diagramów i~modeli, a~nie w~postaci sformalizowanych modeli wraz ze standardową procedurą
kalibracji-estymacji, to nie dziwi, że nie spotkały się z~zauważalną reakcją wśród zwolenników dominującego podejścia
makroekonomicznego.

Z drugiej strony, podaje to w~wątpliwość szumne twierdzenia samego Lucasa
\parencite[s.~1]{lucas_macroeconomic_2003},
%\label{ref:RNDVkrbMcdrLP}(2003, s. 1),
obwieszczającego, że ,,makroekonomia w~pierwotnym znaczeniu odniosła sukces: jej główny problem zapobiegania kryzysom
został rozwiązany''\footnote{``[\mydots] macroeconomics in this original sense has succeeded: Its central problem of depression
prevention has been solved''.}. Paradoksalnie w~ten sposób Lucas powtórzył niemal dokładnie przekonanie Arthura Okuna
\parencite*{okun_political_1970},
%\label{ref:RNDzbEp8W9IuF}(1970),
piszącego na przełomie lat sześćdziesiątych i~siedemdziesiątych XX wieku:

\myquote{
Bardziej energiczne i~konsekwentne stosowanie narzędzi polityki gospodarczej przyczyniło się do przestarzałości modelu
cyklu koniunkturalnego i~obalenia mitów o~stagnacji\footnote{ ``More vigorous and more consistent application of the
tools of economic policy contributed to the obsolescence of the business cycle pattern and the refutation of the
stagnation myths''.}.
}

Jak wykazał okres stagflacji w~latach siedemdziesiątych, Okun nie mógł być dalszy od prawdy.

\section{Ku adekwatnym mikropodstawom}
Samo wskazanie wad zarówno podejścia otwarcie ignorującego potrzebę mikropodstaw, jak i~oparcia
makroekonomii o~neoklasyczne postulaty wzbogacone o~wkład rewolucji racjonalnych oczekiwań nie rozstrzyga zasadności postulowanych
zmian. Należy ponadto zadać pytanie: jakie twierdzenia ze sfery jednostkowych działań oraz interakcji zachowują istotne
znaczenie na poziomie całej gospodarki.

Chociaż istnieje wiele różnic co do interpretacji postulatu metodologicznego indywidualizmu, to nawet w~najsłabszej
wersji uznaje on konieczność odwołania do jednostkowych decyzji jako koniecznego elementu wyjaśnienia
\parencite{udehn_methodological_2001}.
%\label{ref:RNDqgKJhzI31q}(Udehn, 2001).
To w~naturalny sposób uzasadnia podejmowanie problematyki mikropodstaw jako
czynnika sprawczego w~makroekonomii. Zdaniem Larsa Udehna, nacisk na mikroekonomiczne ugruntowanie makroekonomii widoczny
jest przede wszystkim u~teoretyków równowagi ogólnej w~duchu Arrowa-Debreu, jak również u~przedstawicieli szkoły
austriackiej
\parencite{udehn_changing_2002}.
%\label{ref:RNDY1PGnbf2uu}(Udehn, 2002).
Co ważne, do uznania istotności roli mikropodstaw w~teorii
makroekonomicznej wystarczy jedynie odrzucenie tezy o~jednoznacznym determinowaniu działań jednostek przez zmienne
makroekonomiczne i~instytucje -- nawet jeśli uznajemy, że wartościowania jednostek nie są \textit{sui generis}, ale
podlegają znacznym uwarunkowaniom instytucjonalnym i~historycznym. Wówczas nawet przy pominięciu innych potencjalnych
problemów (np. mierzalności) nie możemy ani formułować uniwersalnych praw ekonomicznych oderwanych od poziomu mikro,
ani nawet nie możemy ocenić możliwego zakresu generalizacji wyprowadzonych z~zaobserwowanych korelacji agregatów.

Nie należy się spodziewać, że argumentacja wychodząca od ,,pierwszych zasad'' ma szanse znaleźć posłuch wśród ekonomistów
głęboko zanurzonych w~swoich praktykach, niezależnie od słuszności wysuwanych propozycji. Na szczęście nawet powyższa
krytyka obu przytoczonych stanowisk sugeruje, że istnieją pewne fakty i~zjawiska w~rzeczywistości gospodarczej, które
pozwalają uniknąć przynajmniej części pułapek czyhających w~meandrach analizy makroekonomicznej niekoniecznie wymagając
przy tym (być może niezbędnego, ale trudnego psychologicznie) całkowitego przeorientowania myślenia.

\textbf{Równowaga jest konstruktem pomocniczym, nie celem samym w~sobie}. Praktycznie wszystkie krytykowane powyżej
teorie i~modele makroekonomiczne obu podejść (co widoczne już w~samej nazwie DSGE) oparte są o~równowagę ogólną.
Tymczasem istnienie równowagi ogólnej, nawet rozumianej jako równowagi dynamicznej ze sztywnościami cenowymi, asymetrią
informacji itp., w~świecie realnym nie jest wcale oczywiste. Właściwie nie wiadomo dlaczego należy traktować równowagę
jako na tyle dobre przybliżenie rzeczywistości, żeby była czymś więcej niż konstruktem matematycznym i~nabrała
odniesienia do realnego, konkurencyjnego procesu rynkowego. A~co za tym idzie, brak dobrych powodów, żeby opierać na
niej praktyczne modele gospodarki i~oceny polityki gospodarczej
\parencite{blaug_formalist_2003_tur}.
%\label{ref:RNDDP3xpSLDzs}(Blaug, 2003).

\textbf{Heterogeniczność jest ważna}. Ujęcie kapitału jako jednolitej masy np. w~funkcjach produkcji w~typowych modelach
było już krytykowane wielokrotnie, przede wszystkim w~tzw. debacie dwóch Cambridge w~latach sześćdziesiątych XX wieku,
niemniej nawet współcześnie można spotkać prace rehabilitujące klasyczny model wzrostu Solowa
\parencite{mankiw_contribution_1992}.
%\label{ref:RNDzrxxAmYm7Z}(Mankiw, i~in., 1992).
Jednak struktura dóbr kapitałowych wynikająca z~wcześniej podjętych
decyzji inwestycyjnych ma znaczenie dla aktualnych możliwości i~przyszłego rozwoju gospodarki, zatem abstrahowanie od
niej musi z~konieczności prowadzić do błędów
\parencite{skousen_structure_2007}.
%\label{ref:RNDiGbwYG5Ewk}(Skousen, 2007).

W związku z~heterogenicznością dóbr należy podkreślić istotną rolę cen względnych dóbr. Friedrich von Hayek wskazywał,
że to one z~jednej strony pokazują względną rzadkość poszczególnych dóbr, z~drugiej strony, to ich relacje (ściślej:
oczekiwane relacje) kierują decyzjami inwestycyjnymi jednostek. Ewentualne zaburzenie cen względnych prowadzi do
błędnej alokacji czynników produkcji, co pozornie może wywoływać złudzenie boomu gospodarczego, ale czego negatywne
skutki ujawniają się dopiero po pewnym czasie
\parencite{hayek_prices_1931,mises_ludzkie_2007}.
%\label{ref:RNDMNNAOktdgr}(Hayek, 1931a; Mises, 2007). 

Dodatkowo, należy uznać nie tylko heterogeniczność dóbr, ale również gospodarujących podmiotów: ich celów,
zdolności i~informacji, ponieważ wymiana jako kluczowy element gospodarki wymaga z~samej istoty różnic między agentami
\parencite{colander_financial_2009}.
%\label{ref:RNDe3rzn14JJe}(Colander, i~in., 2009).
Ten punkt wiąże się ściśle z~dwoma następnymi
postulatami dotyczącymi niepewności i~oczekiwań.

\textbf{Niepewność istnieje}. W~tym miejscu chodzi o~niepewność rozumianą w~duchu Franka Knighta, jako niemierzalną
cechę ludzkiego działania, mającą swoje źródło w~ludzkiej kreatywności i~wolnej woli. Jak wskazuje sam Knight, tak
rozumianą niepewność należy odróżnić od ryzyka, dotyczącego wielkości losowych, ale posiadających ustalony i~znany
rozkład prawdopodobieństwa
\parencite{knight_risk_1921_tur}.
%\label{ref:RNDox3nb9igW8}(Knight, 1921).
Istnienie niepewności w~świecie ma istotne
znaczenie dla opisu gospodarczego: z~jednej strony implikuje ono przedsiębiorczość jako fundamentalnie
nierównowagową siłę, przynoszącą odrębny rodzaj dochodu: zysk (lub stratę), w~odróżnieniu od klasycznych kategorii
renty, płac i~procentu z~kapitału, z~drugiej strony natomiast zakreśla granice dla możliwości przewidywania rozwoju
gospodarczego. W~szczególności, w~takim ujęciu firma jest ośrodkiem realizacji funkcji przedsiębiorczej: łączenia
czynników produkcji w~ramach spekulatywnego planu  --  i~dyskusyjne jest ujmowanie jej jako ilościowo określonych
funkcji produkcji w~warunkach równowagowych
\parencite{baumol_entrepreneurship_1968}.
%\label{ref:RNDasUJSrKZaC}(Baumol, 1968).
Podobnie wątpliwe jest ilościowe
modelowanie postępu technicznego lub nawet samo włączanie tych zjawisk w~modele równowagowe, ponieważ nawet jeśli
uznamy założenie o~fundamentalnej tendencji gospodarczej do równowagi, to i~tak trzeba przyjąć, że poszczególne
działania przedsiębiorcze są twórcze, a~więc nieustannie zmieniają stan równowagi
\parencite{schumpeter_creative_1947}.
%\label{ref:RND9pgP03YmaJ}(Schumpeter, 1947).
Nieprzypadkowo literatura na temat ekonomii organizacji, ekonomii innowacji lub zarządzania przedsiębiorczego
powołuje się na ekonomistów (m.in. Josepha Schumpetera, Israela Kirznera, Williama Baumola), którzy otwarcie wskazywali
na to, że od czasów syntezy neoklasycznej ekonomia błędnie pomija ten kluczowy aspekt rzeczywistości gospodarczej
\parencite{foss_organizing_2012}.
%\label{ref:RNDiXEybdJHJ9}(Foss, Klein, 2012).

\textbf{Oczekiwania mają znaczenie}  --  i~niekoniecznie muszą być racjonalne. Jak pisze Sargent przywołując cytat
z~Abrahama Lincolna: ,,Można oszukać część ludzi przez cały czas lub wszystkich ludzi przez pewien czas, ale nie
wszystkich przez cały czas''
\parencite{henderson_rational_2008}.
%\label{ref:RND5bYvQq0Mx0}(Sargent, 2008).
Ta pierwotna motywacja stojąca za ideą
racjonalnych oczekiwań wydaje się być intuicyjna, jednak trudno uznać, że realistycznym modelem odpowiadającym temu
wglądowi jest założenie braku systematycznych błędów przewidywań cen lub uniwersalizacja wyników prostych doświadczeń
laboratoryjnych
\parencite{colander_financial_2009}.
%\label{ref:RNDq03MmuYhqo}(Colander, i~in., 2009).
W~rzeczywistości ma raczej miejsce najróżniejsza
reakcja na zjawiska gospodarcze, czasem rzeczywiście ludzie mogą ulegać systematycznemu złudzeniu, które po prostu po
pewnym czasie zanika. Co ważniejsze, często liczy się nie tyle fakt, że błędy z~obu stron się znoszą, ale jaki
charakter ma cały rozkład oczekiwań
\parencite{lachmann_macro-economic_1973}.
%\label{ref:RNDGIRGwGRewh}(Lachmann, 1973).

Ściśle związana jest z~tym konieczność uwzględnienia niezamierzonych skutków polityki gospodarczej. Przykładowo,
ekonomista analizujący wpływ znaczącego zwiększenia długu publicznego w~walucie krajowej powinien uwzględnić możliwe
reakcje jednostek np. powstanie obawy przed inflacją powoduje ucieczkę w~dolaryzację kontraktów
\parencite{palley_money_2015}.
%\label{ref:RNDCTEKp5p2zc}(Palley, 2015).
Podobnie zmiany podaży pieniądza rozchodzą się nierównomiernie w~gospodarce,
przez co mogą powodować np. systematyczną redystrybucję na rzecz pierwszych odbiorców nowo emitowanego
pieniądza, czyli zjawisko znane pod nazwą efektów Cantillona
\parencite{sieron_efekt_2017}.
%\label{ref:RNDBRBqQqiXME}(Sieroń, 2017).

Trudno także uznać za realistyczną hipotezę rynków efektywnych, zgodnie z~którą ceny zawierają całą przeszłą
informację.
Zwolennicy tej hipotezy zapewniają, że jest ona bardzo dobrze potwierdzona empirycznie.
Wbrew temu można wskazać, że
w~potocznie rozumianej formie hipoteza ta z~założenia wyklucza istnienie baniek na jakichkolwiek aktywach  --  a~więc
jest właściwie niefalsyfikowalna (\textit{vide} uwagi Lucasa i~Easterly’ego w~punkcie drugim niniejszej pracy) lub zwyczajnie
błędna (przy potocznym rozumieniu baniek).

\textbf{Pieniądz jest dobrem} takim jak inne i~tak samo podlega subiektywnej wycenie przez gospodarujące jednostki.
Oczywiście oprócz tego posiada pewne szczególne charakterystyki wynikające z~bycia powszechnym środkiem wymiany: można
zasadnie twierdzić, że nie ma własnego rynku lub, co równoważne, że jego rynek obejmuje praktycznie całą gospodarkę
\parencite{horwitz_microfoundations_2000}.
%\label{ref:RNDzdwQB57bCx}(Horwitz, 2000).
Pieniądz fiducjarny (\textit{fiat money}) nie ma zastosowania jako dobro
konsumpcyjne czy produkcyjne (tak jak historycznie miał pieniądz towarowy), ale za to nadal jest najbardziej zbywalnym
dobrem w~gospodarce. Przez to posiada wartość użytkową dla jego posiadaczy jako efektywny środek wymiany, środek
przechowywania wartości i~jednostka rozliczeniowa
\parencite{jevons_money_1876}.
%\label{ref:RNDMGZCLERnSt}(Jevons, 1876).
Ponadto pieniądz pełni
istotną, lecz często niedocenianą rolę narzędzia kalkulacji ekonomicznej tj. jednostki służącej do porównania
oczekiwanej zyskowności alternatywnych projektów
\parencite{mises_kalkulacja_2011}.
%\label{ref:RNDjQZkmntf3o}(Mises, 2011).

Warto również dostrzec, że stopa procentowa jest odzwierciedleniem preferencji czasowej wśród członków społeczeństwa
\parencite{rothbard_man_1962}.
%\label{ref:RNDIBc7SKwMCf}(Rothbard, 1962).
Jako taka jest ona ściśle związana z~międzyokresową alokacją dóbr, a~zatem
ze strukturą kapitałową gospodarki. Nie można więc jej uznawać li tylko za ,,dźwignię'' na rynku finansowym, którą można
w prosty sposób manipulować, osiągając proste przełożenie na inne parametry makroekonomiczne.

Na koniec należy podkreślić, że \textbf{instytucje rządzą}. Nieprzypadkowo od lat, w~zasadzie zupełnie autonomicznie
wobec formalnego modelowania gospodarczego rozwijają się wielotorowo badania nad rolą instytucji w~procesach w~skali
całych gospodarek. Jak zauważał James Buchanan, analiza zjawisk gospodarczych w~duchu matematyczno-inżynieryjnym jest
zawodna, ponieważ ignoruje uwarunkowania instytucjonalne oraz ich wpływ na interakcję i~wymianę. Ten wpływ zdecydowanie
wykracza poza to, co jest ujmowane w~równaniach jako warunki ograniczające, ale faktycznie zmieniające
bodźce i~oczekiwania ludzi
\parencite{buchanan_economists_2009}.
%\label{ref:RNDP0LLbAbjnM}(Buchanan, 2009).
Dokładnie ten sam błąd popełniali ekonomiści neoklasyczni
w~latach trzydziestych XX wieku podczas debaty o~możliwości rachunku ekonomicznego w~socjalizmie, gdy  --  opierając się o~formalną
poprawność modelu równowagi ogólnej Walrasa  --  dowodzili, że nie istnieje systematyczna różnica między procesem
wyceny w~ustroju kapitalistycznym i~socjalistycznym. Przeoczali bowiem właśnie zarówno wpływ różnic instytucjonalnych
na same motywacje ludzi, ale przede wszystkim gubili instytucjonalną rolę pieniądza jako narzędzia kalkulacji
ekonomicznej
\parencite{mises_ludzkie_2007,mises_kalkulacja_2011}.
%\label{ref:RNDQEgKDSXE16}(Mises, 2007, 2011).

Co więcej, samo przekonanie o~skuteczności modeli oraz stosowanie ich przez instytucje kreujące politykę gospodarczą
powoduje istotne zmiany w~strukturze instytucjonalnej. Zawierzenie modelom stochastycznym prowadzi do systematycznego
zwiększania kruchości gospodarki na zdarzenia z~ogonów rozkładu tzw. ,,czarne łabędzie'' (tj. bardzo rzadkie,
niespodziewane zdarzenia o~dużym wpływie)
\parencite{taleb_antykruchosc:_2013}.
%\label{ref:RNDHP8OAREZ9J}(Taleb, 2013).
Rzeczywiście, w~latach poprzedzających
kryzys można wprost wskazać na zauważalne tendencje do coraz większego wzrostu zależności poszczególnych części
sektora finansowego (prywatnego oraz państwowego) i~ich wpływu na strukturę gospodarki realnej
\parencite{jablecki_financial_2016}.
%\label{ref:RNDyhiwNtHUxt}(Jabłecki, 2016).

\section*{Zakończenie}
Chociaż wiele argumentów z~krytyk wysuwanych przez ostatnie lata trafiało celnie w~niedostatki i~problemy dominujących
metod, modeli i~teorii, to reakcja czołowych makroekonomistów wywoływała wrażenie, że zupełnie nie rozumieją oni
stawianych zarzutów, makroekonomia przeżywa złoty wiek rozkwitu, natomiast jej krytykami są sami ignoranci
\parencite{lucas_defence_2009,sargent_interview_2010}.
%\label{ref:RNDrSaVub7XCz}(Lucas, 2009; Sargent, 2010).
Bardziej krytyczni zwolennicy modelowania twierdzili, że nie
istnieje żadna wiarygodna alternatywa dla analizy zgodnie z~ugruntowanymi paradygmatami
\parencite{blanchard_tienen_2016,korinek_thoughts_2017}.
%\label{ref:RND9txPvUNPJF}(Blanchard, 2016; Korinek, 2017).
Podkreślane jest zarazem, że popularne podejścia są na tyle
sprawne, że można bez problemu je rozbudować o~kolejne elementy np. heterogeniczność podmiotów gospodarujących, rynki finansowe czy
dalsze ograniczenia behawioralne.

Niemniej, powstaje pytanie: dlaczego wszystkie te ulepszenia są uwzględniane dopiero w~sytuacji, gdy modele zostały
wdrożone, oparto na nich politykę, a~potem przekonano się, że zawiodły? Kolejne dekady makroekonomii stanowią raczej
konsekwentne potwierdzenie tezy, że scjentystyczne projekty naukowe nie są dobrymi narzędziami polityki gospodarczej.

Jasne jest, że ekonomiści zajmujący się modelowaniem mają tendencję do zachłyśnięcia się dobrze rokującym, nowatorskim
podejściem ilościowym przy jednoczesnym niedocenianiu jego niedostatków oraz przecenianiu jakości dostępnych danych
gospodarczych. Z~drugiej strony istnieje naturalne zapotrzebowanie polityczne (ze strony państw, banków centralnych czy
instytucji międzynarodowych) na jak najbardziej systematyczne i~naukowe podejście do problemów, w~miarę możliwości
zapewniające koniunkturę i~stabilność gospodarczą. Niemniej, połączenie obu oczekiwań\footnote{Należy podkreślić,
że w~tym miejscu abstrahujemy od konkretnej agendy politycznej lub doszukiwania się psychologicznych motywów politycznego
uwikłania ekonomistów.} powoduje powstanie bodźców, skutkujących w~praktyce tym, że zamiast z~racjonalną polityką
gospodarczą mamy do czynienia z~wielkimi eksperymentami społecznymi. Na nieszczęście słabo kontrolowanymi, więc
stosunkowo niekonkluzywnymi  --  zawsze istnieje ryzyko zrzucenia niepowodzenia na nieprzewidziany szok
zewnętrzny  --  a~co gorsza przeprowadzanymi na żywych społeczeństwach.

Jak wskazuje John Kay, makroekonomia stała się ofiarą sprzeczności: z~jednej strony jest dziedziną rozwijającą modele
jako wartości same w~sobie, z~drugiej natomiast w~jawny sposób deklarującą służebność teorii wobec zastosowania
praktycznego. W~rezultacie pierwsze dążenie prowadzi do nacisku na aksjomatyzację modeli oraz przedstawianie
ich w~formie akceptowalnej dla symulacji komputerowej
\parencite{kay_map_2012}.
%\label{ref:RNDcWSYLEcSGp}(Kay, 2012).
Założenie z~góry możliwości
redukcji ludzkiego zachowania do pewnego zestawu równań może być dla wielu naukowców dobrym punktem wyjścia dla badań
teoretycznych, natomiast bez dobrego zrozumienia, dlaczego taki opis nie daje jedynie przygodnych korelacji, trudno
uzasadnić próby wykorzystania go w~praktyce. A~w~szczególności nie należy się dziwić, gdy rzeczywistość społeczna nie
podąża za  --  w~gruncie rzeczy ciasnym lub wręcz błędnym  --  schematem, w~który próbowano ją ująć,
ponieważ złożoność rzeczywistości społecznej sprawia, że o~pozorne regularności łatwo\footnote{Zob.
\url{http://www.tylervigen.com/spurious-correlations}.}.

Należy dopuścić możliwość, że rację mogą mieć ekonomiści post-keynesowscy wskazujący na istnienie radykalnej
niepewności, uniemożliwiającej modelowanie i~predykcję
\parencite{davidson_keynes_2009},
%\label{ref:RNDY7Suzer839}(Davidson, 2009),
lub przedstawiciele
szkoły austriackiej, podkreślający szczególny jakościowy charakter wszystkich stałych relacji społecznych, w~tym
gospodarczych
\parencite{mises_ludzkie_2007}.
%\label{ref:RND5DFL1sUsyU}(Mises, 2007).
Być może rzeczywiście gospodarka jest systemem złożonym, w~którym
naukowe predykcje ilościowe są po prostu niemożliwe
\parencite{hayek_theory_1964}.
%\label{ref:RNDAnaIjupcOS}(Hayek, 1964).
Możliwe również, że stosowane modele mają po
prostu zbyt wiele stopni swobody, a~więc (zgodnie z~powiedzeniem przypisywanym przez Enrico
Fermiego Johnowi von Neumannowi):

\myquote{
Dzięki czterem parametrom mogę dopasować słonia, a~z~pięcioma mogę sprawić że będzie poruszał swoją trąbą
%\label{ref:RNDpkhSfMwvgw}(Dyson, 2004)
\parencite{dyson_meeting_2004}\footnote{ ,,[\mydots] with four parameters I~can fit an elephant, and with five I~can
make him wiggle his trunk''.}.
}

W każdym razie wydaje się, że fascynacja możliwościami formalnych modeli powinna ustąpić zdrowej dozie sceptycyzmu
związanej z~ich fundamentalnymi ograniczeniami oraz prowadzić do przynajmniej częściowego powrotu krytycznej i~cierpliwej
dyskusji teoretycznej, tak jak była ona prowadzona przed II wojną światową
\parencite{blaug_formalist_2003_tur}.
%\label{ref:RNDg5ZDL7GA9a}(Blaug, 2003).

Nietrudno wyjaśnić przyczynę obecnego stanu. Historyczny splot okoliczności powodujących popularność pewnych podejść
skutkuje również powstaniem zależności od ścieżki (\textit{path-dependence}). Szczególnie nie należy mieć złudzeń co do
wpływu metodologii ekonomii. Jak wskazują George Stigler czy Paul Samuelson, korzyści wynikające ze specjalizacji oraz
koszty alternatywne zainteresowania metodologią ekonomii lub nawet historią myśli ekonomicznej, wydają się skutecznym
bodźcem hamującym ewentualny wpływ dyskusji o~metodzie
%\label{ref:RNDrdJsrThBRy}(Samuelson, 1987; Stigler, 1969)
\parencite{samuelson_out_1987,stigler_does_1969}\footnote{Co ciekawe, obaj wspomniani ekonomiści osobiście przejawiali wieloletnie zainteresowanie historią myśli
ekonomicznej i~zdecydowanie nie stronili od refleksji na temat metodologii ekonomii.}. Z~drugiej strony, jak wskazywał
jeden z~ojców rewolucji marginalistycznej przed ponad stu laty
\parencite{menger_investigations_1996}:
%\label{ref:RNDqhTWAnzUxE}(Menger, 1996): 

\myquote{
Tylko w~jednym przypadku, z~pewnością, badania metodologiczne są tym, co najważniejsze i~najpilniejsze, co należy zrobić
dla rozwoju nauki. [\mydots] Błędne zasady metodologiczne wspierane przez potężne szkoły całkowicie zwyciężają a
jednostronność staje się standardem wszystkich badań. Słowem, postęp nauki jest zablokowany, ponieważ dominują błędne
zasady metodologiczne. W~takim przypadku, rozjaśnienie problemów metodologicznych to warunek wszelkiego dalszego
postępu\footnote{ ``Only in one case, to be sure, do methodological investigations appear to be the most important, the
most immediate and the most urgent thing that can be done for the development of a~science. [\mydots] Erroneous
methodological principles supported by powerful schools prevail completely and onesidedness judges all efforts in a
field of knowledge. In a~word, the progress of a~science is blocked because erroneous methodological principles
prevail. In this case, to be sure, clarification of methodological problems is the condition of any further
progress''.}.
}

Kwestia znaczenia podstaw zjawisk makroekonomicznych wydaje się fundamentalna jeśli chodzi o~zrozumienie procesów
gospodarczych czy dalsze przełożenie na prawidłową politykę gospodarczą. Lepsza metodologia powinna zatem owocować
spójniejszą i~bardziej wyczerpującą teorią, zdolną do pełniejszego wyjaśnienia zjawisk będących w~orbicie zainteresowań
ekonomistów. Wobec tego, już samo uwzględnienie (w gruncie rzeczy mało odkrywczych, bardzo skromnych i~dość
intuicyjnych nawet dla ludzi bez głębszej świadomości metodologicznej) postulatów z~punktu 3 niniejszego artykułu
powinno prowadzić do lepszej, przynajmniej pod pewnymi względami, teorii.

\end{artplenv}
