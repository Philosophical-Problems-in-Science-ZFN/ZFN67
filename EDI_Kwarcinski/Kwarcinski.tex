\begin{editorial}{Tomasz Kwarciński}
	{Filozofia ekonomii -- szkoła pluralizmu i~pokory}
	{Filozofia ekonomii -- szkoła pluralizmu i~pokory}
	{Filozofia ekonomii -- szkoła pluralizmu i~pokory}
	%	{Copernicus Center for Interdisciplinary Studies}
	{Philosophy of economics -- a~school of pluralism and humility}
	%	{Abstrakt lorem ipsum}
	%	{słowo, słowo.}




\lettrine[loversize=0.13,lines=2,lraise=-0.05,nindent=0em,findent=0.2pt]%
{P}{}o raz pierwszy w~historii \textit{Zagadnień Filozoficznych w~Nauce} prezentujemy Czytelnikom numer w~całości poświęcony
filozoficznej refleksji nad ekonomią.  Tematyka poruszana w~poszczególnych artykułach daje wyraz potrzebie pluralizmu
dociekań ekonomicznych, który domaga się wyjścia poza ramy metodologii ekonomii głównego nurtu, uwzględnienia założeń
wartościujących w~modelach ekonomicznych, wzięcia pod uwagę dorobku ekonomicznej heterodoksji oraz poddawania
przeświadczeń ekonomistów krytycznej refleksji. Wszystko to w~czasie, w~którym z~jednej strony część profesji
ekonomicznej oraz studentów ekonomii po doświadczeniu negatywnych skutków kryzysu finansowego 2008 r. doszła do
wniosku, że nadziei na uniknięcie podobnych wydarzeń w~przyszłości trzeba szukać w~większym otwarciu badań i~standardów
edukacji ekonomicznej, uwzględniającym pluralizm poglądów ekonomicznych. Z~drugiej zaś strony, w~okresie, w~którym
wciąż, zarówno na świecie, jak i~w~Polsce na uczelniach ograniczana jest liczba godzin lub likwidowane są kursy
metodologii ekonomii bądź historii myśli ekonomicznej. Tym bardziej cieszy więc fakt, że wśród młodych
ekonomistów i~filozofów znajdują się ludzie, którzy poświęcają swój czas i~energię na metarefleksję nad ekonomią. 

W niniejszym numerze ZFN znajdziemy zarówno teksty osób, które są na wczesnym etapie rozwoju naukowego, jak i~recenzje
prac uznanych naukowców. Znajdziemy tu zarówno teksty dotyczące zagadnień metodologicznych w~ekonomii,
jak i~interpretacji historycznych klasyków ekonomii, a~także prace poświęcone problematyce normatywnych założeń teorii
ekonomicznych. Do pierwszej grupy należy tekst Krzysztofa Turowskiego ,,Mikropodstawy prawdziwe i~fałszywe'', w~którym
przedstawiona została krytyka dwóch podejść w~modelowaniu makroekonomicznym, podejścia skoncentrowanego na agregatach
ekonomicznych typu: PKB, wskaźniki inflacji, agregaty pieniężne, które charakteryzuje się oderwaniem analizy
ekonomicznej od poziomu działań indywidualnych ludzi oraz podejścia poszukującego mikropodstaw zjawisk
makroekonomicznych. Autor nie tylko przedstawia krytykę obu stanowisk lecz także formułuje postulaty, które spełniać
powinny teorie z~adekwatnymi mikropodstawami. Na koniec, wyraża przekonanie, że w~poszukiwaniu mikropodstaw modeli
ekonomicznych może pomóc uwzględnianie postulatów szkół heterodoksyjnych (np. post-keynesismu, szkoły austriackiej),
a~także krytyczna refleksja metodologiczna i~historyczna.

Kolejną fundamentalną kwestię metodologiczną porusza tekst Bartosza Kurkowskiego ,,Czy konstruktywiści społeczni mówią
nam o~czymś realnym w~ekonomii?''. Tytułowe pytanie wskazuje na spór toczony między zwolennikami dwóch stanowisk
epistemologicznych w~ekonomii, czyli konstruktywizmu i~realizmu, które operują odmiennymi koncepcjami
prawdy i~rzeczywistości. Autor stawia sobie za cel sprawdzenie, ,,czy można mówić o~realnych
zjawiskach i~mechanizmach w~gospodarce, nawet jeśli nie istnieją one poza ludzkimi umysłami''.
Opowiadając się ostatecznie za stanowiskiem pełnej
intersubiektywności stwierdza, że w~odniesieniu do zjawisk ekonomicznych staje się ona ich obiektywnością. To z~kolei
prowadzi do uznania, że można mówić o~realnych zjawiskach i~mechanizmach gospodarczych, nawet jeśli nie istnieją one
poza ludzkimi umysłami. 

Krzysztof M. Turek osadza dyskusję na temat kluczowego założenia teorii ekonomicznych, czyli założenia o~racjonalności,
w~kontekście historycznym. W~artykule zatytułowanym ,,W poszukiwaniu racjonalności ekonomicznej w~dziełach Adama Smitha''
stara się odpowiedzieć na pytanie, czy w~pracach ,,ojca ekonomii'' znajdują się już zalążki koncepcji rozwijanej
współcześnie w~ramach teorii racjonalnego wyboru, która może służyć za teoretyczną podbudowę tzw. kultury chciwości,
zgodnej z~maksymą \textit{greed is good}. W~tym celu sięga do głównych dzieł Smitha, czyli \textit{Teorii uczuć
moralnych} oraz \textit{Badań nad naturą i~przyczynami bogactwa narodów}. Autor dochodzi do wniosku, że w~pracach
Smitha możemy się raczej doszukiwać inspiracji do formułowania sprzeciwu wobec kultury chciwości. 

Poglądy Adama Smitha stały się również przedmiotem analiz Filipa Lubińskiego, który w~tekście ,,Rola
państwa i~prawa w~systemie Adama Smitha'' stawia sobie za cel rekonstrukcję poglądów na temat systemu społecznego,
za którym opowiadał się
szkocki filozof. Poza najbardziej znanymi dziełami Smitha dotyczącymi etyki i~ekonomii Autor sięga do rzadziej
analizowanej pracy, którą jest \textit{Lectures on Jurisprudence}. Dochodzi przy tym do wniosku, że ,,ojciec ekonomii''
nie tylko nie był zwolennikiem państwa minimum, określanego mianem ,,stróża nocnego'', lecz na bazie jego prac daje się
zrekonstruować teorię uprawnień efektywnych, zgodnie z~którą, władza powinna gwarantować obywatelom szeroki wachlarz
uprawnień politycznych i~ekonomicznych oraz dbać o~materialne podstawy umożliwiające im realizację tych uprawnień.
Teoria uprawnień efektywnych Smitha opiera się na przyjmowanych przez niego założeniach etycznych, a~także służy
osiąganiu w~gospodarce najwyższego możliwego dobrobytu. 

Kolejne dwa teksty rozwijają kwestie założeń wartościujących (w tym etycznych) w~ekonomii oraz interpretacji i~pomiaru
dobrobytu. W~artykule ,,Droga ekonomii wolnej od wartościowania do epistemologicznej pychy. Użycie i~nadużycie
matematyki przez ekonomistów'' Aleksander Ostapiuk dowodzi, że pomimo krytyki dominujący paradygmat ekonomii
neoklasycznej nie ulega zmianie, gdyż opiera się na aksjomatycznych założeniach teorii wolnej od wartościowania. Autor
dokonuje analizy tych założeń na przykładzie podejścia ekonomicznego Gary’ego Beckera interpretując je przez pryzmat
koncepcji programów badawczych Imre Lakatosa. Na bazie tych rozważań Autor formułuje również postulat otwarcia się
ekonomii na koncepcje normatywne oraz pluralizm metodologiczny.

Bez wątpienia jedną z~najistotniejszych normatywnych koncepcji w~ekonomii jest koncepcja dobrobytu. W~tekście ,,Problem
istoty i~pomiaru dobrobytu'' Wojciech Rybka dokonuje przeglądu najważniejszych filozoficznych i~ekonomicznych koncepcji
dobrobytu, wskazując jednocześnie trudności towarzyszące każdej z~nich. Autor wyróżnia dobrobyt ogólny, dobrobyt
ekonomiczny (materialny) oraz dobrobyt subiektywny. Na tle pozostałych artykułów pracę Rybki wyróżnia fakt, że nie 
poprzestaje on na teoretycznych rozważaniach, i~w~odniesieniu do dobrobytu przechodzi od kwestii pojęciowych do sposobów
operacjonalizacji tego pojęcia poprzez odpowiednie wskaźniki (HDI dla dobrobytu ogólnego, PKB dla dobrobytu
ekonomicznego oraz SWB dla dobrobytu subiektywnego). Następnie, na przykładzie wybranych krajów sprawdza jak bardzo
wyróżnione wskaźniki są ze sobą skorelowane oraz bada ich dynamikę w~czasie. 

Tym co łączy wszystkie prezentowane teksty jest nie tylko przekonanie o~ważności pluralizmu ekonomicznego, niestronienie
od kontekstu historycznego lecz także zwracanie uwagi na konieczność ostrożnego formułowania wniosków na podstawie
teorii i~modeli ekonomicznych, swoistej ekonomicznej pokory. Teksty te łączy również fakt, że zostały one nadesłane w
konkursie na esej metaekonomiczny, który w~2018 roku został zorganizowany przez Polską Sieć Filozofii Ekonomii oraz
Polski Instytut Ekonomiczny, przy wsparciu wydawnictwa Copernicus Center Press. 

Dopełnienie prezentowanych w~niniejszym numerze \textit{Zagadnień} artykułów poświęconych filozofii ekonomii stanowią
recenzje książek. Dwie z~nich dotyczą wprost pozycji z~zakresu filozofii ekonomii,
trzecia jest poświęcona popularnonaukowej pracy z~dziedziny etologii,
podejmującej jednak zagadnienia inspirujące tak dla ekonomistów, jak i~filozofów.
Recenzja Marcina Gorazdy, ,,Believable world of economic models'' prezentuje
monografię Łukasza Hardta zatytułowaną \textit{Economics Without Laws. Towards a~New Philosophy of Economics}. Nie
stroniąc od uwag krytycznych wymierzonych w~główną tezę książki, zgodnie z~którą w~ekonomii nie ma praw naukowych,
rozumianych jako aczasowe, uniwersalne regularności, Gorazda zwraca uwagę, że praca stanowi wartościową lekturę zarówno
ze względu na jej bogatą treść, jak i~inspirujące wnioski. Dowodem tego są uwagi polemiczne sformułowane przez Autora
recenzji. 

Drugą recenzowaną publikacją jest praca zbiorowa \textit{Metaekonomia II. Zagadnienia z~filozofii makroekonomii}, której
redaktorom, Tomaszowi Kwarcińskiemu oraz Agnieszce Wincewicz-Price udało się zaprosić grono uznanych ekonomistów i
filozofów z~kraju i~zagranicy (wśród nich Daniel Hausman, Peter Galbács, Jerzy Osiatyński, Marcin Gorazda, Emilia Tomczyk)
do podjęcia
kwestii metodologicznych oraz etycznych związanych z~makroekonomią. Joanna Dzionek-Kozłowska, w~swojej recenzji zwraca
uwagę na
wielość stanowisk, odnoszących się do poruszanych w~pracy zagadnień i~debat filozoficznych,
metodologicznych i~historycznych. Podkreśla jednocześnie, że daje to czytelnikowi
możliwość nie tylko zapoznania się z~bogactwem poglądów z~obszaru filozofii ekonomii, ale i~szansę wyrobienia sobie własnego zdania.

Ostatnią z~recenzowanych prac jest książka Fransa de Waala \textit{Wiek
empatii. Jak natura uczy nas życzliwości}, w~której z~ewolucyjnej
perspektywy analizuje on różne poziomy empatii oraz rolę, jaką ona
odgrywa w~rozmaitych społecznościach zwierzęcych, w~szczególności w~społecznościach
ludzkich. W~recenzji tej pracy Milena Cygan zwraca
uwagę, że jedną z~najważniejszych tez amerykańskiego naukowca jest
przekonanie, iż człowiek z~natury nie jest zły ani antyspołeczny, a~wciąż
występujące w~niektórych kręgach naukowych pozostałości po
darwinizmie społecznym nie wytrzymują konfrontacji z~badaniami
etologicznymi. Autorka kończy recenzję uwagą, że prezentowana
przez nią książka stanowi przykład tego jak nauka może pomagać w~rozwijaniu
refleksji na temat klasycznych problemów filozoficznych, do
których bez wątpienia należy spór o~naturę ludzką. Dodajmy jedynie, iż
spór ten ma swoje miejsce również w~filozofii ekonomii.

\end{editorial}